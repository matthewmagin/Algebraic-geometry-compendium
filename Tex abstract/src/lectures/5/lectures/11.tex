    \section{Дивизоры}
    
	\subsection{Дивизоры Вейля}

	Начнём с такого примера: 

	\begin{example}
		Несложно показать, что все нормирования на $\bk(t) = \bk(\PP^1)$, тривиальные на $\bk$ соотвествуют $t - \alpha$ и $\infty$. Если говорить конкретнее, то 
		\[
			f(t) = C \cdot \frac{(t - \alpha_1)^{k_1} \cdot \ldots \cdot (t - \alpha_m)^{k_m}}{(t - \beta_1)^{s_1} \cdot \ldots (t - \beta_n)^{s_n}}
		\]
		и тогда мы имеем 
		\[
			\v_{t - \gamma}(f) = \begin{cases} 0, & \gamma \neq \alpha_j, \beta_j \\ k_i & \gamma = \alpha_i  \\ s_j, & \gamma = \beta_j \end{cases}, \ \v_{\infty} = s_1 + \ldots + s_n - k_1 - \ldots - k_m.
		\]
		В частности мы видим, что 
		\[
			\forall f \in \bk\lr*{\PP^1} \quad \sum_{P \in \PP^1} \v_P(f) = 0.
		\]

		Это наблюдение легко обобщить на произвольную неособую проективную кривую $X$. 
	\end{example}

	Далее пусть $X$~--- неособая проективная кривая (и, в дальнейшем всегда так). 
	\begin{definition} 
		\emph{Дивизор на $X$}~--- это формальная целочисленная линейная комбинация $\sum_{P \in X} n_P \cdot P$, где $n_P \in \Z$ и почти все из них равны нулю. 

		Иными словами, группа дивизоров $\DIV(X)$ на кривой $X$~--- это свободная абелева группа, порожденная точками кривой. 
	\end{definition}

	Рассмотрим произвольную рациональную функцию $f \in \bk(X)^*$, тогда в кольце $\cO_P$ она представим в виде $f = z_P^{\v_P(f)} \cdot f_0$, где $(z_P) = \fm_P$~--- локальный параметр, а $f_0 \in \cO_P^{*}$ (так как кривая неособая и локальное кольцо каждой точки регуляно). 


	\begin{definition} 
		Пусть $f \in \bk(X)^*$~--- ненулевая рациональная функция на кривой $X$. Тогда её \emph{дивизором} называют 
		\[
			\Div(f) = \sum_{p \in X} \v_p(f) \cdot p 
		\]
		А  \emph{дивизором нулей} и \emph{дивизором полюсов} называют соотвественно 
		\[
			\Div(f)_0 = \sum_{p \in X, \ \v_p(x) > 0} \v_p(f) \cdot p, \quad \Div(f)_{\infty} = - \sum_{p \in X, \v_p(f) < 0} \v_p(f) \cdot p.
		\]
	\end{definition}

	\begin{remark}
		Нетрудно видеть, что $\Div(f)_{\infty} = \Div(1/f)_{0}$ и $\Div(f) = \Div(f)_0 - \Div(f)_{\infty}$.
	\end{remark}

	\begin{example}
		Пусть $X$~--- неособая неприводимая проективная кривая, $f \in \bk(X)^*$. Рассмотрим морфизм полей $\bk(t) \to \bk(X), \ t \mapsto f$. Ему соответствует некоторое доминантное рациональное отображение $X \to \PP^1$. Но, мы доказывали, что это отображение будет регулярным во всех точках. Рассмотрим некоторую точку $P \in \PP^1$ и $Q_i \in X$ такие, что $Q_i \mapsto P$. Тогда $\v_{Q_i}(f) = e_i \v_P(t) = e_i$. В то же время, степени инерции равны единице (так как поля вычетов алгебраически замкнуты). Тогда по ранее доказанному: 
		\[
			\sum_{i} e_i = [\bk(X) : \bk(f)] = n.
		\]
		Но, с другой стороны, $\{ Q_i \} = \{ Q \in X \ \vert \ \v_Q(f) > 0 \}$, то есть дивизор нулей функции $f$ имеет вид 
		\[
			\sum \v_{Q_i}(f) \cdot Q_i,
		\]
		откуда мы в частности получаем, что $\deg(\Div(f)_0) = [\bk(X) : \bk(f)] = n$. 
	\end{example}

	Пусть $X$~--- неособое неприводимое многообразие, $C \subset X$~--- неособое подмногообразие коразмерности 1. Тогда мы можем определить локальное кольцо по отношению к $C$ (по аналогии с локальным кольцом точки):

	\[
		\cO_{C} \subset \bk(X), \quad \cO_{C} = \left\{ \frac{f }{g}  \ \vert \ g\vert_{C} \not\equiv 0 \right\}. 
	\]

	Тут мы подразумеваем, что $g$ не обращается \bf{тождественно} в 0 на $C$. Видно, что в случае $C = P$ это определение совпадает с определением локального кольца точки. 

	Предположим, что $X$~--- аффинное многообразие с аффинной алгеброй $A = A(X)$. Тогда подмногообразие $C$ соотвествует некоторому (минимальному) простому идеалу $\fp \subset A$ высоты 1.  Тогда $\cO_{C} = A_{\fp}$ (а тот факт, что знаменатель не обращатся тождественно в 0 на $C$ означает как раз, что мы не попали в идеал $\fp$).  

	Теперь возьмём $\fp \subset \fm \subset A$, тогда $\fm$ соотвествует какой-то точке $C$ и тогда ясно, что $A_{\fp}$ получается локализацией кольца $A_{\fm}$, которое регулярно, так как многообразие неособое. Идеал $\fp A_{\fm}$~--- идеал высоты 1. Регулярное локальное кольцо факториально, а в факториальном локальном кольце простой идеал высоты 1 является главным (как мы уже видели). Значит, $\fp A_{\fm}$~--- главный идеал, тогда $\fp A_{\fp}$~--- главный, а это говорит нам, что кольцо $A_{\fp}$ является дискретно нормированным.\footnote{Формально, тут есть некоторая тонкость, см. Шафаревич. } 

	\begin{definition} 
		Пусть $X$~--- неособое многообразие. Тогда \emph{группа дивизоров} $\DIV(X)$~--- свободная абелева группа, образующими которой являются неособые подмногообразия размерности 1.   

		Пусть $D \in \DIV(X)$~--- дивизор, $D = \sum_{Z \subset X} n_{Z} \cdot Z$. Его \emph{носителем} мы будем называть 
		\[
			\supp{D} = \bigcup_{Z \subset X \colon n_{Z} \neq 0} Z. 
		\]
	\end{definition}

	\begin{definition} 
		Пусть $f \in \bk(X)^*$. Тогда её \emph{дифизором} мы будем называть 
		\[
			\Div(f) = \sum_{C \subset X, \ \codim{C} = 1} \v_{C}(f) \cdot C
		\]

		где $C$ неприводимо. Дивизоры такого вида мы будем называть \emph{главными}. Нетрудно заметить, что они образуют подгруппу в $\DIV(X)$, её мы обозначим через $\PDIV(X)$. 
	\end{definition}

	\begin{remark}
		Это определение корректно, так как для заданной функции $f \in \bk(X)$ существует лишь конечное число неособых неприводимых подмногообразий $C$ коразмерности таких, что $\v_{C}(f) > 0$. 

		Рассмотрим сначала случай, когда $X$ аффинно и $f \in A(X)$. В таком случае, просто по определению, если $C$ не является компонентой $Z(f)$, то $\v_{C}(f) = 0$. Покажем, что таких $C$, что $\v_{C}(f) > 0$ конечное число. Пусть $C$ соотвествует идеалу $\fp$ высоты $1$, тогда 
		\[
			\v_{C}(f) > 0 \iff f \in \fp.
		\]
		Рассмотрим $\fp/(f) \subset A/(f)$. заметим, что если $\fp, \fq \lei A$, причём $f \in \fq, \fp$ и $\fq \neq \fp$, то $\fp/(f) \neq \fq/(f)$. Тогда нам достаточно доказать, что в $A/(f)$ конечное число минимальных простых идеалов (а это мы знаем). 


		Если же $X$ всё еще аффинно, но $f \in \bk(X)$, $f = g/h \in A(X)$, то мы видим, что $\v_{C}(f)$, если $C$ не является компонентой $Z(f)$ или $Z(g)$. 

		В произвольном случае мы покроем $X = \bigcup U_i$ аффинными (конечным числом) и тогда  любое $C$ пересекается хоть с одним из $U_i$, поэтому $\v_{C}(f) \neq 0 $ только для тех $C$, которые являются замыканиями таких неприводимых подмногообразий $\widetilde{C} \subset U_i$, что $\v_{\widetilde{C}}(f) \neq 0$. Так как таких $U_i$ конечно, а также $\widetilde{C}$ конечно, определение корректно.
	\end{remark}

	\begin{definition} 
		\emph{Группой классов дивизоров} мы будем называть группу $\Cl(X) = \DIV(X)/\PDIV(X)$.
	\end{definition}

	\begin{definition} 
		Пусть $X$~--- неособая проективная кривая, а $L \in \DIV(X)$. \emph{Степенью} $\deg{L}$ дивизора $L$ называется сумма его кратностей.  
	\end{definition}

	\begin{statement} 
		Степень главного дивизора равна нулю, то есть $\deg(\Div{f}) = 0$.
	\end{statement}

	\begin{proof}
		Поле рациональных функций $\bk(X)$~--- это конечно порожденное поле над $\bk$ степени трансцендентности 1. Рассмотрим подполе $\bk(f) \subset \bk(X)$, тогда $\bk(X)/\bk(f)$~--- конечное расширение полей. Более того, оно соответствует морфизму 
		\[
	 		X \to \PP^1, \quad f \mapsto t \rightsquigarrow \bk(\PP^1) = \bk(t) \to \bk(X). 
	 	\] 	

	 	Тогда, как мы обсуждали выше
	 	\[
	 		\deg(\Div(f)_0) = [\bk(X) : \bk(f)], \quad \deg(\Div(f)_{\infty})  = \deg(\Div(1/f)_0) = [\bk(X) : \bk(1/f)] = [\bk(X) : \bk(f)],
	 	\]
	 	так как $\bk(f) = \bk(1/f)$. Тогда мы получили, что 
	 	\[
	 		\deg(\Div(f)) = \deg(\Div(f)_0) - \deg(\Div(f)_{\infty}) = 0.
	 	\]

	\end{proof}

	\begin{definition} 
		Пусть $Z$~--- неприводимое неособое подмногообразие в $X$ коразмерности 1. Тогда ему соответствует дивизор $1 \cdot Z$. \emph{Простыми} мы будем называть дивизоры такого вида. 
	\end{definition}

	Пусть $U \subset X$~--- открытое подмножество, $Z = X \setminus U$.  Тогда мы можем определить отображение 
	\[
		\DIV(X) \to \DIV(U).
	\] 
	Зададим его на образующих: пусть $T \subset X$~--- простой дивизор, тогда если $T \cap U = \varnothing$, отправим его в 0, а если $T \cap U \neq \varnothing$, то $T \cap U$~--- простой дивизор в $U$ и мы отправим $T$ в  $T \cap U$. Отметим также, что с главными дивизорами при этом отображении происходит также понятная вещь: 
	\[
		\Div(f) \mapsto \Div(f\vert_U).
	\]
	Значит, мы получили корректно определённое отображение $\Cl(X) \to \Cl(U)$. Заметим теперь, что так как отображение $\DIV(X) \twoheadrightarrow \DIV(U)$ сюръективно, отображение $\Cl(X) \twoheadrightarrow \Cl(U)$ также сюръективно. Вычислим его ядро. Предположим, что $\sum n_i Z_i \in \Ker\lr*{\Cl(X) \to \Cl(U)}$, это означает, что он перешел в $\Div(f)$ для некоторой $f$. Тогда $\sum n_i Z_i - \Div(f) \mapsto 0$. Но, если $Z_1 \neq Z_2$, то $Z_1 \cap U \neq Z_2 \cap U$ (если $Z_i \cap U \neq \varnothing$ для $i = 1, 2$). Отсюда следует, что ядро состоит из тех неприводимых подмногообразий коразмерности 1, которые не пересекаются с $U$. А это в точности компоненты $Z = X \setminus U$ коразмерности 1 (в $X$). 

	Таким образом, $\Ker\lr*{\Cl(X) \twoheadrightarrow \Cl(U)} = \Z^m$, где $m$~--- количество неприводимых компонент $Z$ коразмерности 1 в $X$. В частности, у нас есть точная последовательность 
	\[
		\Z^m \to \Cl(X) \to \Cl(U) \to 0. 
	\]

	\begin{example}
		Это наблюдение уже позволяет вычислить группу классов дивизоров для чего-нибудь. 

		\begin{enumerate}
			\item Рассмотрим $X = \AA^n$ и простой дивизор $T \subset X$. По одной из доказанных ранее теорем любое неприводимое подмногообразие коразмерности 1 в $\AA^n$ задаётся одним уравнением: $T = Z(f)$, но тогда $T = \Div(f)$ и $T$ главный. Значит, все простые дивизоры главные, откуда следуте, что $\Cl(\AA^n) = 0$.

			\item Теперь рассмотрим $X = \PP^n$ и $U = \AA^{n}$, Тогда, так как $X \setminus U = \PP^{n - 1}$, мы получаем 
			\[
				\Z \hookrightarrow \Cl{\PP^n} \to \Cl{\AA^n} \to 0,
			\]
			а так как левое отображение инъективно, 
			\[
				0 \to \Z \to \Cl{\PP^n} \to \underbrace{\Cl{\AA^n}}_{ = 0} \to 0,
			\]
			мы имеем $\Cl{\PP^n} \cong \Z$. 
		\end{enumerate}
	\end{example}

	\begin{definition} 
		Дивизор называется \emph{эффективным}, если все его кратности неотрицательны. В таком случае мы пишем $D \ge 0$.
	\end{definition}

	\begin{remark}
		Заметим, что если $f$ регулярна, то $\Div(f) \ge 0$. 
	\end{remark}

	Оказывается, верно и обратное.

    \textcolor{red}{Еще небольшой кусок появится тут несколько позже}

	\subsection{Дивизоры форм}

	Пусть $X \subset \PP^N$~--- многооборазие, $F$~--- форма (многочлен). Определим дивизор формы $F$.

	Рассмотрим неприводимое подмногообразие $C \subset X$ коразмерности 1. Выберем форму $G$ той же степени, что и $F$ так, чтоб $G\vert_{C} \neq 0$ (т.е. не обращается в 0 полностью). Рассмотрим функцию $F/G$ (так как это частное двух форм, это функция) и положим 
	\[
		\v_{C}(F) \eqdef \v_{C}\lr*{\frac{F}{G}}.
	\]

	\begin{remark}
		Покажем, что это определение корректно. Возьмём две формы $G_1$ и $G_2$, удовлетворяющие этим условиям, тогда 
		\[
			\v_{C}\lr*{\frac{F}{G_2}} =  \v_C\lr*{\frac{F}{G_1}} + \v_C\lr*{\frac{G_1}{G_2}}, \text{ но } \v_C\lr*{\frac{G_1}{G_2}} = 0, 
		\]
		откуда мы получаем нужное. 
	\end{remark}

	Тогда определим \emph{дивизор формы $F$} как
	\[
		\Div(F) = \sum_{C \subset X} \v_C(F) \cdot C,
	\]
	где сумма как и ранее берётся по всем неприводимым подмногообразиям коразмерности 1. 

	Пусть теперь $X$ кривая. Тогда мы можем определить \emph{степень дивизора} формы $F$ следующим образом: 
	\[
		\deg{\Div(F)} = \sum_{C \subset X} \v_{C}(F). 
	\]

	Рассмотрим теперь формы $F_1, F_2$ такие, что $\deg{F_1} = \deg{F_2}$. У нас есть очевидное равенство 
	\[
		\Div(F_1) = \Div(F_2) + \Div\lr*{\frac{F_1}{F_2}}	,
	\]
	и применяя степень мы получаем, что 
	\[
		\deg{\Div(F_1)} = \deg{\Div(F_2)}. 
	\]

	Для иллюстрации вычислим дивизор \bf{линейной формы на эллиптической кривой}. 

	Рассмотрим эллиптическую кривую 
	\[
		y^2 = x^3 + ax + b, \quad a, b \neq 0, \ 4a^3 + 27b^3 \neq 0.
	\]

	Рассмотрим точку $P(x_1, y_1) \in C$ и вычислим локальный параметр для кольца $\cO_{P_1}$ (т.е. образующую максимального идеала $\fm_P$). Рассмотрим два случая 

	\begin{enumerate}
		\item Пусть $y_1 \neq 0$. Тогда покажем, что $x - x_1$~--- локальный параметр для кольца $\cO_P$. Ясно, что $\fm_P = (x - x_1, y - y_1)$. Попробуем получить одну образующую вместо двух. 

		\[
		 	\begin{cases} y^2 = x^3 + ax + b \\ y_1^2 = x_1^3 + ax_1 + b \end{cases} \implies (y - y_1)(y + y_1) = (x - x_1)(x^2 + x x_1 + x_1^2  + a)
		 \] 
		 \[
		 	y - y_1 = \frac{(x - x_1)(x^2 + x x_1 + x_1^2  + a)}{y + y_1}.
		 \]
		 Тогда достаточно показать, что $y + y_1 \notin \fm_P$. Действительно, если $y + y_1 \in \fm_{P}$, но тогда $y_1 \in \fm_P$, а это возможно тогда и только тогда, когда $y_1 = 0$ (а мы предположили, что это не так). 

		 \item Пусть $y_1 = 0$, тогда локальным параметром будет $y$. 

		 \[
		 	x_1^3 + a x_1 + b = 0 \implies y^2 = (x - x_1)(x^2 + x x_1 + x_1^2 + a) \implies x - x_1 = \frac{y^2}{x^2 + x x_1 + x_1^2 + a}.
		 \]
		 Покажем, что $x^2 + x x_1 + x_1^2 + a \notin \fm_P$. 

		 \[
		 	x^2 + x x_1 + x_1^2 + a \equiv 3 x_1^2 + a \pmod{\fm_P}.
		 \]
		  Тогда $3x_1^2 + a = 0, \ y_1 = 0$. Но это противоречит тому, что точка $P(x_1, y_1)$ неособая\footnote{Это проверяется вручную}. 
	\end{enumerate}

	Теперь, зная локальный параметр, мы можем считать нормирование от любой рациональной функции. 

	Вычислим степень дивизора линейной формы. Самое простое, что можно сделать с эллиптической кривой~--- это пересечь её с какой-то прямой. Пусть $L$~--- это прямая в $\AA^2$. Проективизуем всё, добавляя бесконечно удалённую точку: 
	\[
		X \colon y^2 z = x^3 + ax z^2 + bz^3.
	\]
	Рассмотрим прямую $L\colon z = 0$. Видно, что $X \cap L$ состоит из одной (бесконечно удалённой) точки. С другой стороны, когда мы пересекаем прямую с эллиптической кривой, точки лежащие в пересечении~--- корни многочлена третьей степени (так что вообще говоря их должно быть 3). Тут дело всё в том, что у этой бесконечно удалённой точки кратность 3: пусть $P_0 = (0 : 1 : 0)$. Покажем, что 
	\[
		\Div(L) = 3 P_0.
	\]
	Деля на $y^3$, перейдём в аффинные координаты:
	\[
	 	\frac{z}{y} = \lr*{\frac{x}{y}}^3 + a \frac{x}{y} \lr*{\frac{z}{y}}^2 + b \lr*{\frac{z}{y}}^3.
	 \]
	 Обозначим $x/y = x_1, \ z/y = z_1$. Тогда 
	 \[
	  	z_1 = x_1^3 + a x_1 z_1^2 + b z_1^3,
	  \] 
	  а $P_0$ имеет координаты $(0, 0)$. Перепишем это уравнение в немного другом виде:
	  \[
	  	z_1 - b z_1^3 = x_1^3 + a x_1 z_1^2 \implies z_1 = \frac{x_1^3 + a x_1 z_1^2}{1 - b z_1^2}.
	  \]
	  Так как $z_1 \in \fm_{P_0}$, знаменатель обратим, тогда, так как $\fm_{P_0} = (x_1)$
	  \[
	  	\v_{P_0}(x_1^3) = 3 \implies \v_{P_0}(z_1) \ge 3,
	  \]
	  но, в то же время, видно, что равенство достигается. Итак, 
	  \[
	  		\Div(L) = 3 P_0.
	  \]

	  Это подводит нас гипотезе о том, что дивизор любой линейной формы имеет степень 3.  

	  Действительно, рассмотрим линейную форму $L\colon y = \alpha x + \beta$, тогда 
	  \[
	  		x^3 + (ax + b) - (\alpha x + \beta)^2 = 0.
	  \]
	  Это уравнение имеет три корня (возможно, с кратностью) и эти корни определяют точки пересечения прямой $L$ с эллиптической кривой. Кратности корней будут соответствовать как раз кратностям точек пересечения. 

	  \[
	  	x^3 + (ax + b) - (\alpha x + \beta)^2 = (x - x_1)(x - x_2)(x - x_3).
	  \]
	  Тогда у нас три точки пересечения: $P_1 = (x_1, y_1)$, $P_2 = (x_2, y_2)$, $P_3 = (x_3, y_3)$. Тогда 
	  \[
	  	\Div(L) = i_1 P_1 + i_2 P_2 + i_3 P_3, \text{ где } i_j\text{~--- кратности соответствующих корней. }
	  \]

  	\begin{exercise}
  		Докажите это.  
  	\end{exercise}

  	Отметим, что мы рассматривали не вертикальную прямую. Вертикальная прямая будет пересекать эллиптическую кривую в двух конечных и бесконечно удалённой точке: 

  	\[
  		\begin{cases}
  			x = x_1 z \\ y^2 z = x^3 + a x z^2 + b z^3
  		\end{cases}
  	\]
  	и очевидно, что $(0 : 1 : 0)$ лежит в пересечении. Также отметим, что вместо  прямой $z = 0$ мы могли бы брать горизонтальную прямую $y = 0$ (и соответственно линейную форму $y$) и рассматривать её пересечение с такой кривой. В этом случае, когда  
  	\[
  		\Div(y) = P_1 + P_2 + P_3
  	\]

   \subsection{Групповой закон для точек эллиптической кривой}

  	Пусть $X$~--- неособая проективная кривая. Мы знаем, что в этом случае любой главный дивизор имеет степень 0, а значит, мы можем рассмотреть отображение 
  	\[
  		\Cl(X) \xrightarrow{\deg} \Z \to 0
  	\]
  	и, обозначая его ядро через $\Cl^0(X)$ написать короткую точную последовательность 
  	\[
  		0 \to \Cl^0(X) \to \Cl(X) \xrightarrow{\deg} \Z \to 0
  	\]

  	Пусть $X$~--- эллиптическая кривая $y^2 z = x^3 + ax z^2 + bz^3$, рассмотрим отображение 
  	\[
  		\varphi\colon X \to \Cl^0(X), \quad P \mapsto [P] - [P_0],
  	\]
  	где $P_0$~--- это бесконечно удалённая точка. Это поможет нам определить закон сложения точек на эллиптической кривой. Вообще говоря, точки на эллиптической кривой можно складывать так: рассмотрим $P_1, P_2 \in X$, рассмотрим прямую, проходящую через них. Эта прямая пересечёт эллиптическую кривую в третьей точке, её мы симметрично отразим относительно оси абсцисс и объявим результатом сложения то, что получилось. Минус этой конструкции в том, что ассоциативность проверить весьма трудно. В нашем же случае все свойства групповых операций будут унаследованы с группы $\Cl^0(X)$.  

  	\begin{theorem} 
  		Отображение $\varphi\colon X \to \Cl^0(X), \ P \mapsto [P] - [P_0]$ взаимно однозначно. 
  	\end{theorem}
  	\begin{proof}
  		Докажем сначала вот такую лемму: 
  		\begin{lemma} 
  			Пусть $X$~--- неособая проективная кривая, на которой есть две различные точки $P \neq Q$ такие, что $P - Q$~--- главный дивизор. Тогда $X = \PP^1$.
  		\end{lemma}
  		\begin{proof}
  			Пусть $P - Q = \Div(f)$. Но тогда $P = \Div(f)_0, \ \Div(f)_{\infty} = Q$. Рассмотрим отображение $X \to \PP^1$, которое даёт нам расширение полей $t \mapsto f$, $\bk(X)/\bk(f)$ и при этом $[\bk(X) : \bk(f)] = \deg{\Div(f)_0} = 1$. Но тогда $\bk(X) = \bk(f) = \bk(t)$. 
  		\end{proof}

  		Так как мы доказывали, что эллиптическая кривая не изоморфна $\PP^1$, из этого следует инъективность отображения. 

  		\noindent\bf{Инъективность.} Предположим, что $[P] - [P_0] = [Q] - [P_0]$ в $\Cl^{0}(X)$, тогда $[P] - [Q] = 0$, то есть $P - Q$~--- главный дивизор. Тогда по доказанной лемме $X \cong \PP^1$ и мы пришли к противоречию. 

  		\noindent\bf{Сюръективность. } Рассмотрим $\sum n_{i} [P_i] \in \Cl^0(X)$. Так как сумма коэффициентов равна нулю, 
  		\[
  			\sum n_{i} [P_i] = \sum n_{i} ([P_i] - [P_0]) 
  		\]
  		Теперь надо доказать, что для любых $n_i \in \Z$ 
  		\[
  			\sum n_{i} ([P_i] - [P_0])  = [S] - [P_0] \in \Cl^0(X)	
  		\]
  		для некоторой $S \in X$. Рассмотрим сначала случай, чтогда все $n_i > 0$. Рассмотрим 
  		\begin{equation}
  			P_1 - P_0 + P_2 - P_0 \label{sum_of_points}
  		\end{equation}

  		Теперь заметим, что если мы рассмотрим две произвольные точки $P_1$ и $P_2$, проведём через них прямую и получим точку $P_3 \in X$, то $P_1 + P_2 + P_3$ является дивизором линейной формы, а тогда, так как формы одинаковой степени линейно эквивалены (их дивизоры отличаются на дивизор функции), 
  		\[
  		 	P_1 + P_2 + P_3 = 3 P_0 \text{ в } \Cl(X)
   		 \] 
  		Рассмотрим точку $P_3'$, симметричную относительно оси абсцисс точке $P_3$. Проведём вертикальную прямую через $P_3$ и $P_3'$, она пересечёт $X$ еще в бесконечно удалённой точке отсюда получим: 
  		\[
  		 	P_3 + P_3' + P_0 = 3P_0  \text{ в } \Cl(X).
  		 \] 
  		 Из двух этих равенств мы получаем, что~\eqref{sum_of_points} мы можем переписать вот так: 
  		 \[
  		 	P_1 - P_0 + P_2 - P_0 = P_3' - P_0 \in \Cl(x).
  		 \]
  		 Значит, сумму $\sum n_{i} ([P_i] - [P_0])$ мы можем свести к одному равенству. Если же не все коэффициенты положительны, то их мы можем поменять искусственно: 
  		 \[
  		  	P_i - P_0 = - (P_i' - P_0) \in \Cl(X).
  		  \] 
  		  Итак, мы доказали сюръективность. 
  	\end{proof}




		


	





	

		


	

