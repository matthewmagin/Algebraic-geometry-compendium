	\begin{proof}[Доказательство теоремы~\ref{closed_image}]
		Отображение $f$ мы можем разложить в композицию дух: 
		\[
			X \xrightarrow{\Gamma_{f}} X \times Y \xrightarrow{\mathrm{pr_2}} Y, \quad x \mapsto (x, f(x)) \mapsto f(x).
		\]

		Тогда нам достаточно доказать, что 
		\begin{enumerate}
			\item $\Gamma_{f}(X)$ замкнут в $X \times Y$,
			\item $\pr_{2}$ переводит замкнутые множества в замкнутые. 
		\end{enumerate}

		Докажем сначала \bf{первое}. Выделим это в отдельную лемму:  

		\begin{lemma}[О замкнутом графике] 
			Пусть $f\colon X \to Y$~--- морфизм, тогда $\Gamma_{f}(X)$ замкнут в $X \times Y$.
		\end{lemma}
		\begin{proof}[Доказательство леммы]
			Рассмотрим диагональный морфизм $\Delta\colon Y \to Y \times Y$. Тогда нам достаточно проверить, что $\Delta(Y)$ замкнут в $Y \times Y$, так как если мы это докажем, то можно рассмотреть 
			\[
				X \times Y \xrightarrow{(f, \id)} Y \times Y
			\]
			и тогда, так как $\Gamma_{f}(X) = (f, \id)^{-1}(\Delta(Y))$, из замкнутости $\Delta(Y$ будет следовать замкнутость графика (просто по непрерывности). 

			Предположим сначала, что $Y$ аффинное. Тогда всё просто: $\Delta(Y) = \Delta(\AA^n) \cap (Y \times Y)$ и оба пересекаемых множества очевидно замкнуты. 

		 	Если же $Y$ произвольное, покроем его аффинными: $Y \subset \bigcup U_i$, тогда $Y \times Y \subset \bigcup U_i \times U_i$ и тогда чтоб показать, что $\Delta(Y)$ замкнуто, нам достаточно показать, что $\Delta(Y) \cap (U_i \times U_i)$ замкнуто для всех $i$. Но это очевидно, так как 
		 	\[
		 		\Delta(Y) \cap (U_i \times U_i) = \Delta(U_i),
		 	\]
		 	а $\Delta(U_i)$ замкнуто по первому шагу доказательства.  
		\end{proof}

		Теперь покажем \bf{второе}. 
		
		Прежде всего, можно считать, что $X = \PP^n$, так как для произвольного $X$ можно рассматривать композицию
		\[
			X \times Y \hookrightarrow \PP^n \times Y \to Y, 
		\]
		применить теорему для неё и из этого всё будет следовать. 

		Покрывая $Y$ аффинными, мы понимаем, что достаточно доказать теорему для случая
		 \[
		 	\PP^n \times \AA^m \to \AA^m.
		 \]

		 А в этом случае работать существенно проще, так как (из теоремы~\ref{closed_in_product}) мы знаем полное описание замкнутых множеств. Все они имеют вид 
		 \[
		 	T = \{ (u, y) \ \vert \ g_i(u, y) = 0, \ 1 \le i \le t \} \rightsquigarrow \mathrm{pr}(T) = \{ y \in \AA^m \ \vert \ \exists u \in \PP^n \ g_i(u, y) = 0 \ 1 \le i \le t \},
		 \]
		 где $g_i$~--- однородный многочлен по $u$. 

		 Пусть $I_{s} = (u_0, \ldots, u_n)^s$. Тогда 
		 \[
		 	y \in \pr_2(T) \iff \forall s \quad (g_{1}(u, y_{0}), \ldots, g_{t}(u, y_{0})) \not\supset I_{s},
		 \]

		 Тогда проекцию мы можем задать, как 
		 \begin{equation}
		 	\mathrm{pr}(T) = \bigcap_{s} \{ y \in \AA^m \ \vert \ I_{s} \not\subset (g_{1}(u, y), \ldots, g_{t}(u, y)) \}. \label{closed_pr}
		 \end{equation}

		 Значит, нам достаточно доказать, что каждое множество из пересечения замкнуто. Соответственно, по этому поводу зафиксируем $s$. Пусть $k_i = \deg_{u_i}(g_{i})$, $\{ M^{(\alpha)}\}_{\alpha \in \N^{n + 1}, \ \sum \alpha_j = s}$~--- все мономы степени $s$\footnote{Например, при $n =2$ есть такой моном: $M^{(2, 3, 1)} = u_0^2 u_1^3 u_2$}. Посмотрим, что означает условие, противоположное к условию~\eqref{closed_pr}:
		 \[
		 	I_{s} \subset (g_{1}(u, y_{0}), \ldots, g_{t}(u, y_{0})) \Leftrightarrow \forall \alpha \  M^{(\alpha)} = \sum g_i(u, y_0) F_{i, \alpha}(u) \quad \quad \quad \quad  (*)
		 \]
		 а $F_i$ однородные по переменным $u_i$. Кроме того, если $s \ge k_i$, то $\deg{F_{i, \alpha}} = s  - k_i$, а если $s < k_i$, то ясно, что $F_{i, \alpha} = 0$. Теперь рассмотрим $\{ N_i^{(\beta)} \}_{\beta}$~--- все мономы (от переменных $u_i$) степени $s - k_i$.  Тогда условие $(*)$ означает, что все $M^{(\alpha)}$~--- всевозможные линейные комбинации $g_i(u, y_0) N_i^{(\beta)}$. Это, в свою очередь, равносильно тому, что 
		 \[
		 	S = \Span\{ g_{i}(u, y_0)N_i^{(\beta)} \},
		 \]
		 где $S$~--- пространство однородных многочленов степени $\alpha$. Тогда ясно, что  
		 \[
		 	I_{s} \not\subset (g_{1}(u, y_0), \ldots, g_{t}(u, y_0)) \Leftrightarrow S \neq \Span\{ g_{i}(u, y_0)N_i^{(\beta)} \} \iff \rank{A} < \dim{S},
		 \]

		 где $A$~--- матрица, состоящая из коэффициентов $g_i(u, y_0) N_i(\beta)$. Тогда ясно, что при фиксированном $y$ это полиномиальное условие (обнуление определителя), так что мы показали замкнутость. 
	\end{proof}

	\begin{corollary}
		Пусть $X \in \Project, \ Y \in \qProj$, а $f \colon X \to Y$~--- морфизм. Пусть $Z \subset X$~--- замкнутое подмножество. Тогда $f(Z)$ замкнуто в $Y$.
	\end{corollary}

	\subsection{Рациональные отображения многообразий }

	\begin{lemma}\label{rat_cor} 
		Пусть $X, Y$~--- неприводимые многообразия, а $\varphi, \psi$~--- морфизмы из $X$ в $Y$. Предположим, что существует такое непустое открытое множество $U \subset X$, что $\varphi\vert_{U} = \psi\vert_{U}$. Тогда $\varphi = \psi$.
	\end{lemma}
	\begin{proof}
		Пусть $Y \subset \PP^n$ для некоторого $n$. Беря композиции морфизмов $\varphi$ и $\psi$ с вложением $Y \to \PP^n$, мы сводим всё к случаю $Y = \PP^n$. Рассмотрим $\PP^n \times \PP^n$ со структурой проективного многообразия, определяемой вложением Сегре (см.~\ref{dir_prod}). Тогда $\varphi$ и $\psi$ определяют морфизм 
		\[
			X \xrightarrow{\varphi \times \psi} \PP^n \times \PP^n.
		\]
		Рассмотрим диагональ $\Delta \subset \PP^n \times \PP^n$. Оно (как и в аффинном случае) замкнуто, так как представляется уравнениями 
		\[
			x_i y_j = x_j y_i , \quad i, j = 0, \ldots, n.
		\]
		По предположению $(\varphi \times \psi)(U) \subset \Delta$, но $U$ плотно в $X$ (так как $X$ неприводимо), а $\Delta$ замкнуто, откуда мы имеем $(\varphi \times \psi)(X) \subset \Delta$, откуда $\psi = \varphi$. 
	\end{proof}

	\begin{definition} 
		Пусть $X, Y$~--- многообразия, $X$ неприводимо. Рассмотрим множество пар $(U, f)$, где $U \subset X$~--- открытое, а $f\colon U \to Y$~--- морфизм. На этом множестве мы можем ввести такое отношение эквивалентности: 
		\[
			(U_1, f_1) \sim (U_2, f_2) \Leftrightarrow f_1\vert_{U_1 \cap U_2} = f_2\vert_{U_1 \cap U_2}.
		\]

		Класс эквивалентности по этому отношению называется \emph{рациональным} отображением и обозначается, как $f\colon X \dashrightarrow Y$.

		Из всех пар $(U, f)$ мы можем выбрать такую, для которой открытое множество $U$ максимально. Это множество $U$ мы будем называть \emph{областью регулярности} рационального отображения. 
	\end{definition}

	\begin{remark}
		То, что описанное выше отношение~--- отношение эквивалентности, следует из леммы~\ref{rat_cor}. 
	\end{remark}

	\begin{definition} 
		Рациональное отображение $f\colon X \dashrightarrow Y$ называется \emph{доминатным}, для некоторой пары $(U, f)$ $f(U)$ плотно в $Y$. 
	\end{definition}

	\begin{remark}
		Определение выше корректно, так как если образ плотен для какой-то пары, то это так и для всех (тоже по лемме~\ref{rat_cor}). 
	\end{remark}

	
	Композицию рациональных отображений определить не всегда возможно (по понятным) причинам, а вот с доминатными рациональными отображениями дело обстоит лучше. 
	
	Пусть у нас есть доминантные рациональные отображения $X \dashrightarrow Y \dashrightarrow Z$ и они представляются морфизмами $f\colon U \to Y$ и $g\colon V \to Z$. Так как $f$ доминантно, $f(U) \cap V \neq \varnothing$, откуда $W = f^{-1}(V) \cap U \neq \varnothing$. Тогда определим $g \circ f \colon X \dashrightarrow Z$ как класс эквивалентности пары $(W, g \circ f\vert_{W})$. 

	 Отметим также, что композиция доминантных отображений является доминантным. Действительно, предположим противное, а именно, что образ $W$ под действием композиции попадает в некоторое замкнутое $T \subsetneq Z$. Но тогда $f(W) \subsetneq g^{-1}(T) \subsetneq Y$, а это противоречит доминантности. 

	 Значит, квазипроективные многообразия с доминатными рациональными морфизмами образуют категорию. 

	Кроме того, доминантное рациональное отображение $\varphi\colon X \dashrightarrow Y$ индуцирует гомоморфизм полей рациональных функций 
	\[
		\varphi^*\colon \bk(Y) \to \bk(X) 
	\]
    Действительно, пусть $\varphi$ представлено парой $(U, \varphi_U)$ и пусть $f \in \bk(Y)$~--- рациональная функция, представленная парой $(V, f)$, где $f$ регулярна на $V$. Тогда, посколько $\varphi_U(U)$ плотно в $Y$, оно пересекается с $V$ и $\varphi^{-1}(V)$~--- непустое открытое подмножество $X$, так что $f \circ \varphi_U$~--- регулярная функция на $\varphi_{U}^{-1}(V)$ (а регуляна она, так как $\varphi_U$~--- морфизм). Она представляет некоторую рациональную функцию на $X$. 
	 \[
	 	\varphi^{-1}(V) \xrightarrow{\varphi} V \xrightarrow{f} \Bbbk.
	 \]
	 Таким образом, мы построили отображение  
	 \[
	 	\Bbbk(Y) \to \Bbbk(X), \quad f \mapsto \varphi^{*}(f).
	 \] 

   Пусть $\mathsf{C}$~--- категория неприводимых многообразий с доминантными рациональными отображениями, а $\mathsf{D}$~--- категория конечно порожденных расширений поля $\bk$.

	 \begin{theorem}\label{antieq_cat_2}
	 	Для любых неприводимых многообрзий $X$, $Y$ приведённая выше конструкция осуществляет биективное соотвествие между 
	 	\begin{itemize}
	 		\item множеством доминантных рациональных отображений $X \to Y$
	 		\item множеством гомоморфизмов $\bk$-адгебр $\bk(Y) \to \bk(X)$.
	 	\end{itemize}

	 	Более этого, это соотвествие осуществляет антиэквивалентность категорий $\mathsf{C}$ и $\mathsf{D}$: 
	 	\[
	 		\cF\colon \mathsf{C} \to \mathsf{D}, \ X \mapsto \bk(X).
	 	\]
	 \end{theorem}
	 \begin{proof}
	 	Построим отображение обратное тому, что было приведено ранее. 

	 	Пусть $\theta\colon \bk(Y) \to \bk(X)$~---- гомоморфизм $\bk$-алгебр. Нам надо построить соотвествующее ему доминантное рациональное отображение $X \to Y$. 

	 	Так как $Y$ можно покрыть аффинными многообразиям, можно полагать, что $Y$ аффинное. Пусть $A(Y)$~--- его аффинное координатное кольцо, а $y_1, \ldots, y_n$~--- его образующие, как $\bk$-алгебры. Тогда $\theta(y_1), \ldots, \theta(y_n)$ являются рациональными функциями на $X$. Выберем открытое множество $U \subset X$ так, чтобы все функции $\theta(y_i)$ были регулярными на $U$. В таком случае $\theta$ определяет инъективный гомоморфизм $\bk$-алгебр 
	 	\[
	 		A(Y) \to \cO(U). 
	 	\]

	 	По теореме~\ref{antieq_cat_1} ему соотвествут морфизм $\varphi\colon U \to Y$, который определяет доминантное\footnote{Оно будет доминантным, так как инчае отображение $A(Y) \to \cO(U)$ не инъективно. } рациональное отображение $X \to Y$. 

	 	Теперь убедимся, что мы действительно построили антиэквивалентность категорий. Нам надо проверить, что для любого неприводимого многообразия $Y$ поле рациональных функций $\bk(Y)$ конечно порождено над $\bk$ и обратно, что всякое конечно порожденное расщирение $K/\bk$ является полем рациональных функций $K = \bk(Y)$ некоторого неприводимого многообразия $Y$.

	 	Пусть $Y$~--- неприводимое многообразие, тогда $\bk(Y) = \bk(U)$ для любого открытого подмножества $U \subset Y$, так что опять же можно полагать $Y$ аффинным.  Тогда $\bk(Y) \cong \Frac{A(Y)}$ и, как следствие, оно являетя конечно порожденным расширением поля $\bk$ степени трансцендентности $\dom{Y}$. 

	 	С другой стороны, пусть $K$~--- конечно порожденное расширение поля $\bk$, а $y_1, \ldots, y_n \in K$ ~--- система образующих. Пусть. $B$~--- подалгебра в $K$, порожденная $y_1, \ldots, y_n$ над $\bk$. Тогда $B$ является фактором кольца многочленов $\bk[x_1, \ldots, x_n]$ по некоторому идеалу $I$, так что $B \cong A(Y)$ для $Y = Z(I) \subset \AA^n$. $Y$ будет неприводимым, так как $A(Y)$ целостное. Значит, $K \cong \bk(Y)$ и теорема доказана. 
	 \end{proof}

    Переводя это на существенно менее изысканный язык, мы получаем такое следствие 

     \begin{corollary}
         Неприводимые многообразия $X$ и $Y$ бирационально эквивалентны тогда и только тогда, когда их поля рациональных функций $\bk(X)$ и $\bk(Y)$  изоморфны как $\bk$-алгебры. 
     \end{corollary}

	 \subsection{Бирациональная эквивалентность}

	 \begin{definition} 
	 	\emph{Бирациональным отображением } $\varphi\colon X \to Y$ называется рациональное отображение, которое обладает обратным, т.е. таким рациональным отображением $\psi\colon Y \to X$, что $\psi \circ \varphi = \id_{X}$, $\varphi \circ \psi = \id_{Y}$. Многообразия $X$ и $Y$ называются \emph{бирационально эквивалентными}, если существует хотя бы одно бирациональное отображение $X \to Y$. 
	 \end{definition}

	 \begin{corollary}
	 	Для любых двух неприводимых многообразий $X$ и $Y$ следующие условия эквивалентны: 

	 	\begin{enumerate}
	 		\item $X$ и $Y$ бирационально эквивалентны, 
	 		\item существуют открытые подмножества $U \subset X$ и $V \subset Y$ такие, что $U$ изоморфно $V$,
	 		\item $\bk(X) \cong \bk(Y)$ в категории $\bk$-алгебр. 
	 	\end{enumerate}
	 \end{corollary}
	 \begin{proof}
	 	Сначала докажем $(\bf{1}) \implies (\bf{2})$. Пусть $\varphi \colon X \to Y$ и $\psi \colon Y \to X$~-- бирациональные отображения. Пусть $\varphi$ представлено парой $(U, \varphi)$, а $\psi$~--- парой $(V, \psi)$. Тогда отображение $\psi \circ \varphi$ представляется парой $(\varphi^{-1}(V), \psi \circ \varphi)$, а так как $\psi \circ \varphi = \id_{X}$ как раицональное отображение, $\psi \circ \varphi$ тождественно на $\varphi^{-1}(V)$. Аналогично, $\varphi \circ \psi$ тождественно на $\psi^{-1}(U)$. Тогда у нас есть $\varphi^{-1}(\psi^{-1}(U)) \subset X$ и $\psi^{-1}(\varphi^{-1}(U)) \subset Y$~--- изоморфные открытые подмножества (изоморфизм осуществляется посредством отображений $\varphi$ и $\psi$). 

	 	Утверждение $(\bf{2}) \implies (\bf{3})$ следует из опредления полей функций:
	 	\[
	 		\bk(U) \cong \bk(X), \quad \bk(V) \cong \bk(Y), \quad \bk(U) \cong \bk(V) \implies \bk(X) \cong \bk(Y).
	 	\]

	 	Утверждение $(\bf{3}) \implies (\bf{1})$ напрямую следует из теоремы~\ref{antieq_cat_2}.
	 \end{proof}

	 Теперь докажем какой-нибудь результат про бирациональную эквивалентность. Напомним несколько фактов из алгебры: 

	 \begin{theorem}[О примитивном элементе]\label{primitive} 
	 	Пусть $L$~--- конечное сепарабельное расширение поля $K$. Тогда существует элемент $\alpha \in L$, порождающий поле $L$, как расширение над $K$. Более того, если $\beta_1, \ldots, \beta_n$~--- произвольная система образующих $L/K$ и  поле $K$ бесконечно, то $\alpha$ можно выбрать $\alpha$ в виде $\alpha = c_1 \beta_1 + \ldots + c_n\beta_n$ элементов $\beta_i$ с коэффициентами $c_i \in K$.
	 \end{theorem}

	 \begin{definition} 
	 	Расширение $K/\bk$ называется \emph{сепарабельно порожденным}, если существует такой базис трансцендентности $\{ x_i \}$ в $K/\bk$, что поле $K$ является сепарабельным алгебраическим расширением поля $\bk(\{ x_i \})$. В таком случае $\{ x_i \}$ называется \emph{сепарабельным базисом трансцендентности}.
	 \end{definition}

	 \begin{theorem} 
	 	Пусть $K/\bk$~--- конечно порожденное и сепарабельно порожденное расширение поля $\bk$. Тогда всякое множество образующих расширения $K/\bk$ содержит подмножество, являющееся сепарабельным базисом трансцендентности. 
	 \end{theorem}

	 \begin{theorem}\label{alg_close_sep_gen} 
	 	Пусть $\bk$~--- ал{}гебраически замкнутое поле. Тогда любое конечно порожденное расширение $K/\bk$ является сепарабельно порожденным. 
	 \end{theorem}


	 \begin{statement} 
	 	Всякие неприводимое многообразие $X$ размерности $r$ бирационально эквивалентно гиперповерхности $Y \subset \PP^{r + 1}$. 
	 \end{statement}
	 \begin{proof}
	 	Начнём с того, что поле функций $\bk(X)$ является конечно порожденным расширением поля $\bk$. Тогда по теореме~\ref{alg_close_sep_gen} оно сепарабельно порождено над $\bk$. Значит, существует базис трансцендентности $x_1, \ldots, x_r \in \bk(X)$ такой, что $\bk(X)$~--- конечное сепарабельное расширение $\bk(x_1, \ldots, x_{r})$. Тогда по теореме о примитивном элементе~\ref{primitive} существует $y \in \bk(X)$ такой, что $K = \bk(y, x_1, \ldots, x_r)$. Элемент $y$ алгебраичен над $\bk$, то есть удовлетворяяет некоторому полиномиальному уравнению с коэффициентами из поля рациональных функций от переменнвых $\bk(x_1, \ldots, x_r)$. Домножая на знаменатели, мы получим 
	 	\[
	 		f(y, x_1, \ldots, x_r) = 0,
	 	\]
	 	где $f$~--- некоторый неприводимый многочлен. Теперь легко видеть, что он определяет гиперповерхность в $\AA^{r + 1}$ с полем функций $\bk(X)$, а отсюда, по теореме~\ref{antieq_cat_2}, она бирационально эквивалентна $X$. Проективное замыкание этой гиперповрехности и есть требуема гиперповерхность $Y \subset \PP^{r + 1}$. 
	 \end{proof}

	 \subsection{Рациональные многообразия}

	 \begin{definition} 
	 	\emph{Рациональным многообразием} мы будем называть многообразие, изоморфное в категории $\mathsf{C}$ проективному пространству. Эквивалентно (по теореме~\ref{antieq_cat_2}), можно говорить, что это многообразие, поле функций которого изоморфно $\bk(t_1, \ldots, t_n)$. 
	 \end{definition}

	 \begin{example}
	 	Например, окружность $x^2 + y^2 = 1$ является рациональным многообразием. Действительно, так как поле алгебраически замкнуто, 
	 	\[
	 		x^2 + y^2 = (x + iy)(x - iy) = st
	 	\]
	 	и окружность задаётся как $st = 1$. Тогда видно, что $\Bbbk(X) \cong \Bbbk(t)$.
	 \end{example}

	 А знаем ли мы многообразия, которые не являются рациональными? Рассмотрим \emph{эллиптическую кривую}
	 \[
	 	y^2 = x^3 + ax + b,
	 \]
	 с условием, что $x^3 + ax + b$ не имеет кратных корней и $\Char{\Bbbk} \neq 2, 3$. Условие про кратные корни гарантированно, например, тем, что $4a^3 + 27b^2 \neq 0$ и $a, b \neq 0$.

	 Делая линейные замены переменных, мы можем свести ситуацию к 
	 \[
	 	y^2 = x(x - 1)(x - \alpha), \quad \alpha \neq 0.
	 \]

	 Покажем, что эта кривая не является рациональной. Посмотрим на проективное замыкание этой кривой, для этого нужно взять гомогенизацию этого многочлена: 
	 \[
	 	y^2z = x^3 + axz^2 + bz^3.
	 \]

	 Несложно видеть, что преоктивное замыкание от самой кривой отличается лишь на бесконечно удалённую точку: если $z \neq 0$, то мы получаем все аффинные точки, а если $z = 0$, то мы как раз получаем бесконечно удалённую точку $(0 : 1 : 0)$. 

	 Так как аффинная кривая содержится в проективной, как открытое подмножество, поля функций у них совпадают. Из этого в частности следует, что если мы докажем, что у аффинная кривая не рациональна, то мы получим, что её проективизация не изоморфна $\PP^1$.

	 Теперь докажем, что аффинная кривая не рациональна. Предположим, что $\Bbbk(X) \cong \Bbbk(t)$, и
	 \[
	 	y \mapsto \frac{p_1(t)}{p_2(t)}, \quad x \mapsto \frac{q_1(t)}{q_2(t)}.
	 \]

	 Не умаляя общности, $(p_1, p_2) = (q_1, q_2) = 1$. Тогда должно быть выполнено соотношение 
	 \[
	 	\frac{p_1^2}{p_2^2} = \frac{q_1}{q_2}\lr*{\frac{q_1}{q_2} - 1}\lr*{\frac{q_1}{q_2} - \alpha} \rightsquigarrow p_1^2 \cdot q_2^3 = p_2^2 q_1(q_1 - q_2)(q_1 - \alpha q_2). 
	 \]
	 Отсюда $p_2^2 \divby q_2^3$  и $q_2^3 \divby p_2^2$, откуда они пропорциональны, то есть $q_2^3 / p_2^2 = c \in \Bbbk$, то мы получим 
	 \[
	 	c p_1^2 = q_1(q_1 - q_2)(q_1 - \alpha q_2)
	 \]
	 и не умаляя общности, $c$ здесь мы можем просто не писать. Отсюда $q_2$~--- квадрат, тогда $  q_1, q_2, \ q_1 - q_2, \ q_1 - \alpha q_2$~--- квадраты. 

	 \begin{statement} 
	 	Пусть $Q_1, Q_2$~--- два взаимнопростых многочлена. Тогда в пространстве $\Span_{\Bbbk}(Q_1, Q_2)$ нет четврёх полных квадратов, каждые два из которых непропорциональны. 
	 \end{statement}
	 \begin{proof}
	 	Пусть $R_1, R_2 \in \Span_{\Bbbk}(Q_1, Q_2)$ непропорциональные квадраты. Так как они пропорциональны, $\Span_{\Bbbk}(R_1, R_2) = \Span_{\Bbbk}(Q_1, Q_2)$. Тогда оставшиеся два квадрата можно записать в виде $\alpha_1 R_1 + \alpha_2 R_2$ и $\beta_1 R_1 + \beta_2 R_2$. Пусть $R_1 = S_1^2, \ R_2 = S_2^2$ и  
	 	\[
	 		\alpha_1 R_1 + \alpha_2 R_2 = (\sqrt{\alpha_1} S_1 + \sqrt{\alpha_2}S_2)(\sqrt{\alpha_1} S_1 - \sqrt{-\alpha_2}S_2), \quad \beta_1 R_1 + \beta_2 R_2 = (\sqrt{\beta_1} S_1 + \sqrt{\beta_2}S_2)(\sqrt{\beta_1} S_1 - \sqrt{-\beta_2}S_2)
	 	\]

	 	Так как $(S_1, S_2) = 1$, значит взаимно просты и правые части равенств, а отсюда $\sqrt{\alpha_1} S_1 + \sqrt{\alpha_2}S_2, \sqrt{\alpha_1} S_1 - \sqrt{-\alpha_2}S_2$, $\sqrt{\beta_1} S_1 + \sqrt{\beta_2}S_2, \sqrt{\beta_1} S_1 - \sqrt{-\beta_2}S_2$~--- квадраты. 

	 	Выберем изначально $Q_1, Q_2$ с наименьшим максимумом степеней. Тогда мы только что смогли спуститься. 
	 \end{proof}

