		\subsection{Применения произведения проективных многообразий}

	\begin{theorem} 
		Пусть $f\colon X \to Y$~--- морфизм, $X$ проективно, $Z \subset X$~--- замкнуто. Тогда $f(Z)$ замкнуто в $Y$.
	\end{theorem}

	\begin{proof}
		Рассмотрим композицию отображений  
		\[
			X \xrightarrow{\Gamma_{f}} X \times Y \xrightarrow{\mathrm{pr}} Y, \quad x \mapsto (x, f(x)) \mapsto f(x).
		\]

		Тогда нам достаточно доказать, что  $\Gamma_{f}(Z)$ замкнуто, а также, что если $T \subset X \times Y$ замкнуто, то $\mathrm{pr}_{Y}(T)$ замкнуто. 

		Докажем сначала певрое. 

		\begin{statement} 
			Пусть $f\colon X \to Y$~--- морфизм, тогда $\Gamma_{f}(X)$ замкнут в $X \times Y$.
		\end{statement}
		\begin{proof}[Доказательство предложения]
			Рассмотрим диагональный морфизм $\Delta\colon Y \to Y \times Y$. Тогда нам достаточно проверить, что $\Delta(Y)$ замкнут в $Y \times Y$, так как если мы это докажем, то можно рассмотреть 
			\[
				X \times X \xrightarrow{(f, \id)} Y \times Y
			\]
			и тогда $\Gamma_{f}(X) = (f, \id)^{-1}(\Delta(Y))$, откуда будет следовать, что $\Gamma_{f}(X)$ замкнут. Почему же замкнута диагональ? 

			Предположим сначала, что $Y$ аффинное. В этом случае $\Delta(Y)$ задаётся понятной системой уравнений и всё очевидно. Если же $Y$ произвольное, то его можно покрыть аффинными и всё получится. В самом деле, пусть $Y = \bigcup U_{i}$ , где $U_i$ аффинные и открытые в $X$. Тогда 
			\[
				\begin{cases} \Delta(Y) \cap (U_i \times U_i) = \Delta(U_i) \\ \Delta(Y) \subset \bigcup U_i \times U_i \end{cases} \implies \Delta(Y) \text{~--- замнкуто. }
			\]

			
		\end{proof}
		
		Прежде всего, можно считать, что $X = \PP^n$, так как для произвольного $X$ можно рассматривать композицию
		\[
			X \times Y \hookrightarrow \PP^n \times Y \to Y, 
		\]
		применять теорему для неё и из этого всё будет следовать. 

		Пусть $Y = \bigcup U_i$, где $U_i$ открытые и аффинные. Тогда утверждение достаточно доказывать для каждого $U_i$, то есть для отображения 
		\[
		 	\PP^n \times U_i \to U_i
		 \] 
		 Действителньо, из этого будет следовать локальная замкнутость, а из этого уже глобальная. Тогда ясно, что достаточно рассматривать ситуацию
		 \[
		 	\PP^n \times \AA^m \to \AA^m.
		 \]

		 А в этом случае работать существенно проще, так как мы знаем полное описание замкнутых множеств. Пусть $T \subset \PP^n \times \AA^n$ замкнутое, тогда 
		 \[
		 	T = \{ (u, y) \ \vert \ g_i(u, y) = 0, \ 1 \le i \le t \} \rightsquigarrow \mathrm{pr}(T) = \{ y \in \AA^m \ \vert \ \exists u \in \PP^n \ g_i(u, y) = 0 \ 1 \le i \le t \}. 
		 \]

		 Пусть $I_{s} = (u_0, \ldots, u_n)^s$. Тогда при каждом конкретном  $y = y_{0}$ условие в задании проекции означает, что 
		 \[
		 	\forall s \quad (g_{1}(u, y_{0}), \ldots, g_{t}(u, y_{0})) \not\supset I_{s}.
		 \]

		 Тогда проекцию мы можем задать, как 
		 \[
		 	\mathrm{pr}(T) = \bigcap_{s} \{ y \in \AA^m \ \vert \ I_{s} \not\subset (g_{1}(u, y), \ldots, g_{t}(u, y)) \}.
		 \]
		 Значит, нам достаточно доказать, что каждое множество из пересечения замкнуто. 	Соовтественно, по этому поводу зафиксируем $s$. Пусть $k_i = \deg_{u_i}(g_{i})$, $\{ M^{(\alpha)}\}_{\alpha \in \N^{n + 1}, \ \sum \alpha_j = s}$~--- все мономы степени $s$\footnote{Например, при $n =2$ есть такой моном: $M^{(2, 3, 1)} = u_0^2 u_1^3 u_2$}. Посмотрим, что означает условие, противоположное условию выше. 

		 \[
		 	I_{s} \subset (g_{1}(u, y_{0}), \ldots, g_{t}(u, y_{0})) \Leftrightarrow \forall \alpha \  M^{(\alpha)} = \sum g_i(u, y_0) F_{i, \alpha}(u) \rightsquigarrow (*)
		 \]
		 а $F_i$ однородные по переменным $u_i$. Кроме того, если $s \ge k_i$, то $\deg{F_{i, \alpha}} = s  - k_i$, а если $s < k_i$, то ясно, что $F_{i, \alpha} = 0$. Теперь рассмотрим $\{ N_i^{(\beta)} \}_{\beta}$~--- все мономы (от переменных $u_i$) степени $s - k_i$.  Тогда условие $(*)$ означает, что все $M^{(\alpha)}$~--- всевозможные линейные комбинации $g_i(u, y_0) N_i^{(\beta)}$. Это, в свою очередь, равносильно тому, что 
		 \[
		 	S = \Span\{ g_{i}(u, y_0)N_i^{(\beta)} \},
		 \]
		 где $S$~--- пространство однородных многочленов степени $\alpha$. Тогда ясно, что  
		 \[
		 	I_{s} \not\subset (g_{1}(u, y_0), \ldots, g_{t}(u, y_0)) \Leftrightarrow S \neq \Span\{ g_{i}(u, y_0)N_i^{(\beta)} \}.
		 \]

		 Это уже просто условие на ранг матрицу, а это уже условие на определитель. Это полиномиальное условие по $y$, то есть мы получим некоторое замнутое множество относительно переменных $y$. 
	\end{proof}

	\begin{corollary}
		Регулярная функция на неприводимом проективном многообразии постоянна. 
	\end{corollary}

	\begin{proof}
		Действительно, пусть $f$~--- регулярная функция на неприводимом многообразии $X$. Тогда рассмотрим 
		\[
			X \xrightarrow{f} \AA^1 \to \PP^1
		\]
		

		Образ $X$ неприводим и замкнут в $\AA^1$, тогда он либо точка (и отображение постоянно), либо это всё $\AA^1$. Но, второго случая быть не может, так как мы можем рассмотреть композицию (как и написано выше), а образ $\AA^1$ в $\PP^1$ замкнутым не будет. 
	\end{proof}

	\subsection{Рациональные отображения многообразий }

	\begin{definition} 
		Пусть $X, Y$~--- многообразия, $X$ неприводимо. Рассмотрим множество пар $(U, f)$, где $U \subset X$~--- открытое, а $f\colon U \to Y$~--- регулярная функция. На этом множестве мы можем ввести такое отношение эквивалентности: 
		\[
			(U_1, f_1) \sim (U_2, f_2) \Leftrightarrow \exists W \subset U_1 \cap U_2 \colon f_1 = f_2\vert_{W}.
		\]

		Класс эквивалентности по этому отношению называется \emph{рациональным} отображением и обозначается, как $f\colon X \dashrightarrow Y$.
	\end{definition}

	\begin{definition} 
		Рациональное отображение $f\colon X \dashrightarrow Y$ называется \emph{доминатным}, если $f(U)$ плотно в $Y$.
	\end{definition}

	\begin{remark}
		Композицию рациональных отображений определить не всегда возможно (по понятным) причинам, а вот с доминатными рациональными отображениями дело обстоит лучше. 
	\end{remark}

	Пусть у нас есть доминантные рациональные отображения $X \dashrightarrow Y \dashrightarrow Z$ и они представляются отображениями $f\colon U \to Y$ и $g\colon V \to Z$. Тогда $f^{-1}(V)$ лежит в $U$ и мы можем рассмотреть отображение 
	\[
	 	f^{-1}(V) \to V \xrightarrow{g} Z
	 \] 
	 и получить рациональное отображение $X \dashrightarrow Z$. Доминантность тут используется в том месте, где мы полагаем $f^{-1}(V)$ непустым. 

	 Отметим также, что композиция доминантных отображений является доминантным. Действительно, предположим противное, а именно, что образ $f^{-1}(V)$ под действием композиции попадает в некоторое замкнутое $T \subset Z$. Но тогда $f(f^{-1}(V)) \subset g^{-1}(T)$, которое замкнуто, но тогда $g^{-1}(T) = V$, что противоречит доминантности. Тогда $g^{-1}(T) \subsetneq V$, то это тоже противоречит  доминатности (по каким-то общетопологическим причинам). 

	 Значит, квазипроективные многообразия с доминатными рациональными морфизмами образуют категорию. Кроме того, если у нас есть доминантное рациональное отображение $\varphi\colon X \dashrightarrow Y$, то у нас есть и отображение полей функций $\Bbbk(Y) \to \Bbbk(X)$. Действительно, рассмоттрим $(V, f)$~--- элемент поля функций ($V$~--- открытое подмножество $Y$, а $f$~--- регулярная функция на $V$). Тогда $\varphi^{-1}(V)$~--- непустое открытое подмножество $X$, и на нём мы можем рассмотреть функцию 
	 \[
	 	\varphi^{-1}(V) \xrightarrow{\varphi} V \xrightarrow{f} \Bbbk.
	 \]
	 Иными словами, у нас есть отображение 
	 \[
	 	\Bbbk(Y) \to \Bbbk(X), \quad f \mapsto \varphi^{*}(f).
	 \]

	 Как мы помним, если $U \in \Aff$, то $\Bbbk(U) = \mathrm{Frac}(A(U))$. 

	 \subsection{Рациональные многообразия}

	 \begin{definition} 
	 	Многообразие $X$ называется \emph{рациональным}, если его поле рациональных функций $\Bbbk(X)$ изоморфно полю рациональных функций от конечного числа переменных (т.е. $\Bbbk(t_1, \ldots, t_n)$).
	 \end{definition}

	 \begin{example}
	 	Например, окружность $x^2 + y^2 = 1$ является рациональным многообразием. Действительно, так как поле алгебраически замкнуто, 
	 	\[
	 		x^2 + y^2 = (x + iy)(x - iy) = st
	 	\]
	 	и окружность задаётся как $st = 1$. Тогда видно, что $\Bbbk(X) \cong \Bbbk(t)$.
	 \end{example}

	 А знаем ли мы многообразия, которые не являются рациональными? Рассмотрим \emph{эллиптическую кривую}
	 \[
	 	y^2 = x^3 + ax + b,
	 \]
	 с условием, что $x^3 + ax + b$ не имеет кратных корней и $\Char{\Bbbk} \neq 2, 3$. Условие про кратные корни гарантированно, например, тем, что $4a^3 + 27b^2 \neq 0$ и $a, b \neq 0$.

	 Делая линейные замены переменных, мы можем свести ситуацию к 
	 \[
	 	y^2 = x(x - 1)(x - \alpha), \quad \alpha \neq 0.
	 \]

	 Покажем, что эта кривая не является рациональной. Посмотрим на проективное замыкание этой кривой, для этого нужно взять гомогенизацию этого многочлена: 
	 \[
	 	y^2z = x^3 + axz^2 + bz^3.
	 \]

	 Несложно видеть, что преоктивное замыкание от самой кривой отличается лишь на бесконечно удалённую точку: если $z \neq 0$, то мы получаем все аффинные точки, а если $z = 0$, то мы как раз получаем бесконечно удалённую точку $(0 : 1 : 0)$. 

	 Так как аффинная кривая содержится в проективной, как открытое подмножество, поля функций у них совпадают. Из этого в частности следует, что если мы докажем, что у аффинная кривая не рациональна, то мы получим, что её проективизация не изоморфна $\PP^1$.

	 Теперь докажем, что аффинная кривая не рациональна. Предположим, что $\Bbbk(X) \cong \Bbbk(t)$, и
	 \[
	 	y \mapsto \frac{p_1(t)}{p_2(t)}, \quad x \mapsto \frac{q_1(t)}{q_2(t)}.
	 \]

	 Не умаляя общности, $(p_1, p_2) = (q_1, q_2) = 1$. Тогда должно быть выполнено соотношение 
	 \[
	 	\frac{p_1^2}{p_2^2} = \frac{q_1}{q_2}\lr*{\frac{q_1}{q_2} - 1}\lr*{\frac{q_1}{q_2} - \alpha} \rightsquigarrow p_1^2 \cdot q_2^3 = p_2^2 q_1(q_1 - q_2)(q_1 - \alpha q_2). 
	 \]
	 Отсюда $p_2^2 \divby q_2^3$  и $q_2^3 \divby p_2^2$, откуда они пропорциональны, то есть $q_2^3 / p_2^2 = c \in \Bbbk$, то мы получим 
	 \[
	 	c p_1^2 = q_1(q_1 - q_2)(q_1 - \alpha q_2)
	 \]
	 и не умаляя общности, $c$ здесь мы можем просто не писать. Отсюда $q_2$~--- квадрат, тогда $  q_1, q_2, \ q_1 - q_2, \ q_1 - \alpha q_2$~--- квадраты. 

	 \begin{statement} 
	 	Пусть $Q_1, Q_2$~--- два взаимнопростых многочлена. Тогда в пространстве $\Span_{\Bbbk}(Q_1, Q_2)$ нет четврёх полных квадратов, каждые два из которых непропорциональны. 
	 \end{statement}
	 \begin{proof}
	 	Пусть $R_1, R_2 \in \Span_{\Bbbk}(Q_1, Q_2)$ непропорциональные квадраты. Так как они пропорциональны, $\Span_{\Bbbk}(R_1, R_2) = \Span_{\Bbbk}(Q_1, Q_2)$. Тогда оставшиеся два квадрата можно записать в виде $\alpha_1 R_1 + \alpha_2 R_2$ и $\beta_1 R_1 + \beta_2 R_2$. Пусть $R_1 = S_1^2, \ R_2 = S_2^2$ и  
	 	\[
	 		\alpha_1 R_1 + \alpha_2 R_2 = (\sqrt{\alpha_1} S_1 + \sqrt{\alpha_2}S_2)(\sqrt{\alpha_1} S_1 - \sqrt{-\alpha_2}S_2), \quad \beta_1 R_1 + \beta_2 R_2 = (\sqrt{\beta_1} S_1 + \sqrt{\beta_2}S_2)(\sqrt{\beta_1} S_1 - \sqrt{-\beta_2}S_2)
	 	\]

	 	Так как $(S_1, S_2) = 1$, значит взаимнопросты и правые части равенств, а отсюда $sqrt{\alpha_1} S_1 + \sqrt{\alpha_2}S_2, \sqrt{\alpha_1} S_1 - \sqrt{-\alpha_2}S_2$, $\sqrt{\beta_1} S_1 + \sqrt{\beta_2}S_2, \sqrt{\beta_1} S_1 - \sqrt{-\beta_2}S_2$~--- квадраты. 

	 	Выберем изначально $Q_1, Q_2$ с наименьшим максимумом степеней. Тогда мы только что смогли спуститься. 
	 \end{proof}

	 Если у нас есть рациональное отображение $f\colon X \dashrightarrow Y$, то мы всегда можем рассматривать максимальное открытое множество, на котором $f$ регулярно (объединение всех открытых, на которых $f$ регулярна. 

	 \textcolor{red}{Тут в конце еще что-то было, видимо, но я ничего не понял и не записал :(}

	 



