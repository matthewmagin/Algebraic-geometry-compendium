	
	\begin{statement} 
		Пусть $X$~--- неособая неприводимая кривая, пусть у нас есть рациональное отображение $f\colon X  \dashrightarrow \PP^n$. Тогда оно регулярно во всех точках. 	
	\end{statement}
	\begin{proof}
		Действительно, рассмотрим открытое $U \subset X$, на котором $f$ регулярно, то есть морфизм. Если мы рассмотрим $f$, как отображение $U \to \AA_{0}^n \subset \PP^n$, то мы знаем, что $f$ мы можем записать, как 
		\[
			u \mapsto (1 : f_1(u) : f_2(u) : \ldots : f_n(u)) = (f_0(u) : f_1(u) : f_2(u) : \ldots : f_n(u)), \quad f_i \text{~--- регулярны на } U. 
		\]
		Тогда $f_i \in \Bbbk(X)$. Рассмотрим $P \in X$, тогда $\cO_{P} \subset \Bbbk(X)$. Так как кривая $X$ неособая, $\cO_{P}$~--- дискретно нормированное кольцо. Обозначим соответствующее дискретное нормирование за $\v_{P}$, пусть $k_i = \v_{P}(f_i)$, а $k = \min_{i}\{k_i\}$. В силу симметрии, мы можем считать, что минимум достигается при $i = 0$. Тогда мы можем записать каждую функцию, как 
		\[
			f_i = t^{k_i} g_{i},
		\]
		где $t$~--- локальный параметр, то есть образующая $\fm_{P}$ (или же, такой элемент, что $\v_{P}(t) = 1$). Разделим каждую координату на $t^{k_0}$. Так мы получим другое рациональное отображение $X \dashrightarrow Y$, имеющее вид 
		\[
			u \mapsto (g_{0}(u) : t^{k_1 - k_0} f_1(u) : \ldots : t^{k_n - k_0} g_n(u)).
		\]
		Так как локальный параметр не обращается в 0 на некотором открытом множестве, это отображение определено на открытом множестве. Нетрудно видеть, что приведённое выше рациональное отображение регулярно в точке $P$. Действительно, $\v_{P}(g_0) = 0$, то есть $g_0 \in \cO_{P}^{*}$, то есть $g_0 \notin \fm_{P}$. Это означает, что $g_0$ определена в точке $P$ и не обращается в 0 (и даже в некоторой окрестности этой точки). Для остальных $i$ мы получаем, что $\v_{P}(g_i) \ge 0 \implies g_i \in \cO_{P}$, то есть они регулярны в некоторой окрестности $P$. Тогда мы получаем, что отображение определено в некоторой окрестности точки $P$ и точка $P$ является точкой регулярности (так как в её окрестности не все координаты равны нулю). По произвольности точки $P$ мы имеем нужное. 
		
	\end{proof}

	При помощи этого утверждения также можно доказать, что эллиптическая кривая не бирационально изоморфна $\PP^1$.

	Предположим противное, пусть у нас есть рациональное отображение $C \dashrightarrow \PP^1$. Тогда по предыдущему предложению мы можем полагать, что это отображение морфизм. Тогда у нас есть композиция 
	\[
		C \xrightarrow{f} \PP^1 \xrightarrow{g} C,
	\]
	где $f, g$~--- бирациональные изоморфизмы. Тогда на некотором открытом множестве $g \circ f = \mathrm{id}$. 
	\begin{lemma} 
	 	Если $h_1, h_2\colon X \to Y$~--- морфизмы, $X$ неприводимо и $h_1$ совпадает с $h_2$ на некотором открытом $U \subset X$. Тогда $h_1 = h_2$.
	 \end{lemma} 

	 Тогда мы получаем, что $C \cong \PP^1$. Не умаляя общности, можно считать, что $\infty \mapsto \infty$ и тогда достаточно доказывать, что аффинная эллиптическая кривая не может быть изоморфна $\AA^1$. Чтоб доказать это, нужно смотреть на кольца регулярных функций. Оказывается, что кольцо регулярных функций на эллиптической кривой не факториально. 

	   

	   \subsection{Локальное кольцо в точке}

	   \begin{definition} 
	   	Пусть $X$~--- квазипроективное многообразие, $P \ni X$, а $\cO(X)$~--- его кольцо регулярных функций. Для точки $P$ определим её \emph{локальное кольцо} $\cO_{P}$, как 
	   	\[
	   		\cO_{P} = \varinjlim_{U \ni P} \cO(U).
	   	\]
	   	Говоря более изысканно, это кольцо ростков регулярных функций на $Y$ в окрестности $P$. Или, иными словами, элемент $\cO_P$~--- это пара $(U, f)$, где $U$~--- открытая окрестность $P$ в $Y$, а $f$~--- регулярная функция на $U$, причём пары $(U, f)$ и $(V, g$ отождествляются, если $f = g$ на $U \cap V$. 
	   \end{definition}

	  Отметим, что кольцо $\cO_P$ на самом деле является локальным кольцом: его единственный максимальный идеал $\fm_P$ состоит из всех ростков регулярных функций, обращающихся в нуль в точке $P$. В самом деле, если $f(P) \neq 0$, то $1/f$ регулярна в некоторой окрестности $P$ и в максимальном идеале лежать не может. Значит, всё вне $\fm_P$ обратимо, то есть $\fm_P$ максимальный. Также несложно видеть, что поле вычетов $\cO_P/\fm_{P} \cong \bk$. 



	   Пусть $A$~--- локальное кольцо, $\fm$~--- его максимальный идеал. Тогда $\fm/\fm^2$~--- векторное пространство над $A/\fm$. 

	   \begin{statement}\label{Nakayama_cor} 
	   		Пусть у нас есть набор $x_i \in \fm$ таких, что $(\overline{x_1}, \ldots, \overline{x_k}) = \fm/\fm^2$. Тогда $(x_1, \ldots, x_k) = \fm$.
	   \end{statement}
	   \begin{proof}
	   		Рассмотрим модуль $M = \fm/(x_1, \ldots, x_k)$. Тогда по лемме Накаямы:
	   \[
	   		M = 0 \iff \fm M = M \iff \fm^2 + (x_1, \ldots, x_k) = \fm,
	   \]
	   что как равносильно тому, что $(\overline{x_1}, \ldots, \overline{x_k})$~--- система образующих $\fm/\fm^2$.
	   \end{proof}


	   \begin{lemma} 
	   	Пусть $A$~--- коммутативное кольцо, $I_1, \ldots, I_n$~--- набор идеалов ($n \ge 2$), причем среди них есть не более двух \bf{не} простых. И, пусть
	   	\[
	   		J \lei A, \quad J \lei I_1 \cup \ldots \cup I_n.
	   	\]
	   	Тогда $J \subset I_k$ для некоторого $k$. 
	   \end{lemma}

	   Доказательство этой леммы предоставляется читателю как упражнение. 

	   	\begin{exercise}
	   		Локальное кольцо точки $P = (0, 0)$ для кривой $X = Z(y^2 - x^3)$ не является дискретно нормированным.  
	   	\end{exercise}

	   	\begin{example}
   		Нетрудно заметить,что в случае упражнения выше у нас всё устроено так: 
   		\[
   			\dim{X} = 1, \quad \fm = (x, y), \quad \dim_{\bk} \fm_P/\fm_P^2 = 2. 	
   		\]
		Возводя в степень максимальный идеал это кольца можно заметить интересную вещь: 
	   	\[
	   		\fm^3 = (x^3, x^2 y, xy^2, y^3) = (y^2, x^2 y) \subset (y) \implies \fm^3 \subset (y) \subset \fm
	   	\]
	   	\end{example}
	   	
	   	Это явление имеет такое коммутативно алгебраическое происхождение:    
        \begin{theorem}\label{another_def_of_dim_of_ring} 
	   		Пусть $A$~--- нётерово локальное кольцо. Тогда $\dim{A} < \infty$ и она равна минимальному такому $n$, для которого  $\exists k, \ x_1, x_2 \ldots, x_n \in \fm$ такие, что 
	   		\[
	   			\fm^k \subset (x_1, \ldots, x_n) \subset \fm.
	   		\]
	   	\end{theorem}
	   	\begin{proof}
	   		Мы будем доказывать эту теорему для \emph{колец геометрического происхождения}, то есть колец вида $\cO_P$. Зафиксируем $d$ и предположим, что для некоторого $k \in \N$ мы имеем 
	   		\[
	   			\fm_P^k \subset (x_1, \ldots, x_d).
	   		\]
	   		\bf{Шаг 1.} Покажем сначала, что в таком случае $\dim{\cO_P} \le d$. 

	   		Так как если мы рассмотрим вместо всего многообразия неприводимую компоненту максимальной размерности, содержащую точку $P$, то $\dim{\cO_P}$ не изменится, и всё еще будет $\fm_P^k \subset (x_1, \ldots, x_d)$, не умаляя общности можно полагать кольцо целостным. 

	   		 $\Specm{A}$~--- аффинная окрестность, содержащая точку $P$, а $A$~--- её координатное кольцо. 

	 		Так как $\cO_{P} = \varinjlim A_{a}$, мы с самого начала можем полагать, что мы работаем в некоторой локализации $A_{s}$. Теперь докажем вот такую лемму: 

	 		\begin{lemma} 
	 			Пусть $I, J$~--- идеалы в целостном нётеровом кольце $A$, $\fm \in \Specm{A}$ и $I_{\fm} \subset J_{\fm}$. Тогда существует $a \in A \setminus \fm$ такой, что $I_{a} \subset J_{a}$.
	 		\end{lemma}
	 		\begin{proof}[Доказательство леммы]
	 			Рассмотрим короткую точную последовательность 
	 			\[
	 				0 \to J \to I + J \to I + J/J \to 0.
	 			\]
	 			Так как локализация~--- это точный функтор, точной будет и последовательность 
	 			\[
	 				0 \to J_{\fm} \xrightarrow{\sim} (I + J)_{\fm} \to (I + J/J)_{\fm} \to 0.
	 			\] 

	 			Тогда, так как $I_{\fm} \subset J_{\fm}$, вторая слева стрелка~--- изоморфизм, откуда $(I + J/J)_{\fm} = 0$. Рассмотрим конечнопорожденный $A$-модуль $M = I + J/J$. Так как $M_{\fm} = 0$, существует $a \in A \setminus \fm$ такой, что $M_a = 0$, то есть $(I + J/J)_{a} = 0$. Но, последовательность 
	 			\[
	 				0 \to J_{a} \xrightarrow{\sim} (I + J)_{a} \to (I + J/J)_{a} \to 0 \iff 0 \to J_{a} \xrightarrow{\sim} (I + J)_{a} \to 0 
	 			\]
	 			также точна, откуда $I_a \subset J_a$.

	 			\end{proof}

	 			 Применяя эту лемму к $\cO_P = A_{\fm_{P}}$ мы получаем, что $\exists b \colon $ в кольце $B = A_{b}$ имеется включение идеалов 
	 			\[
	 				\fm_{P}^k \subset (x_1, \ldots, x_d).
	 			\]
	 			Тогда мы имеем включения 
	 			\[
	 				\fm_P \subset \sqrt{(x_1, \ldots, x_d)} \subset \fm_{P},
	 			\]
	 			откуда (так как $\fm_P = I(P)$), мы получаем, что $P = Z(x_1, \ldots, x_d)$. Тогда, если $\dim{\cO_P} > d$, то мы пришли к противоречию, так как $Z(x_1, \ldots, x_d)$~--- это пересечение $d$ гиперповерхностей, а оно  либо пустое, либо степени $\dim{\cO_P} - d > 0$.  
	 			
	 			\noindent\bf{Шаг 2.} Покажем, что для $d = \dim{\cO_{P}}$ существует натуральное $k$ такое, что 
	 			\[
	 				\fm_P^k \subset (x_1, \ldots, x_{d}).
	 			\]
	 			Будем доказывать это индукцией по $d$. 

	 			\noindent\bf{База.} Случай $d = 0$ очевиден, так как кольцо $\cO_P$ локальное, откуда $\Rad(\cO_P) = \fm_P$, но в Артиновом кольце радикал Джекобсона нильпотентен, то есть для некоторо $N$ $\fm_P^N = 0$ и условие будет выполнено. 

	 			\noindent\bf{Переход.} Рассмотрим элемент $x$, не лежащий в объединении минимальных простых идеалов. Тогда, как мы видели в одном из параграфов ранее, 
	 			\[
	 				\dim{\cO_P/(x)} \le \dim{\cO_P} - 1,
	 			\]
	 			так как если у нас есть максимальная цепочка вложенных простых $\overline{\fp_0} \subset \ldots \subset \overline{\fp_n}$, то цепочка $\fp_0 \subset \ldots \subset \fp_n$ уже не будет максимальной, так как идеал $\fp_0$ не минимальный. Тогда по индукционному предположению существует $k$ такое, что 
	 			\[
	 				\fm_P^k/(x) \subset (\overline{x_1}, \ldots, \overline{x_{d - 1}}) \implies  \fm_P^k \subset (x, x_1, \ldots, x_d).
	 			\]
	   	\end{proof}

	   	\begin{remark}
	   		Комментарий про $< \infty$ тут по существу, так как произвольное нётерово кольцо, вообще говоря, не обязано быть конечномерным. 
	   	\end{remark}

	   	\begin{corollary}[Из теоремы~\ref{another_def_of_dim_of_ring}]
	   		Пусть $A$~--- нётерово локальное кольцо с максимальным идеалом $\fm$. Тогда 
	   		\[
	   			\dim{A} \le \dim_{A/\fm}{\fm/\fm^2}.
	   		\]
	   	\end{corollary}
	   	\begin{proof}
	   		Пусть $\overline{x_1}, \ldots, \overline{x_n}$~--- базис $\fm/\fm^2$. Тогда по предложению~\ref{Nakayama_cor} мы имеем $(x_1, \ldots, x_n) = \fm$, а тогда $\fm \subset (x_1, \ldots, x_n) \subset \fm$. Но тогда по теореме~\ref{another_def_of_dim_of_ring} мы имеем $\dim{A} \le n$. 
	   	\end{proof}

	   	Применяя это к кольцу $\cO_{P}$, мы получаем, что 
	   	\[
	   		\dim{\cO_{P}} \le \dim_{\bk}{\fm_P/\fm_P^2}.
	   	\]

	   	Это наводит на мысль, что полезно рассматривать следующие объекты:

	   	\begin{definition}\label{regular_rings} 
	   		Пусть $A$~--- локальное кольцо с максимальным идеалом $\fm$. Оно называется \emph{регулярным}, если $\dim{A} = \dim_{\Bbbk}{\fm/\fm^2}$.
	   	\end{definition}

	   	Нам понадобятся следующие факты о регулярных локальных кольцах: 

	   	\begin{enumerate}
	   		\item Регулярное локальное кольцо целостное. 

	   		\item Регулярное локальное кольцо факториально. 

	   		\item Любая локализация регулярного локального кольца относительно простого идеала~--- тоже регулярное локальное кольцо. 
	   	\end{enumerate}

	   	Некоторые из них мы позже даже докажем. Перед этим, введём объект, который в принципе мотивирует изучение регулярных колец (т.е. поймём, что условие регулярности означает геометрически). 

        

        
	   	
    

	 

	