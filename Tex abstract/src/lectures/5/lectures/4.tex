	
	\subsection{Размерность редуцированного кольца, в котором каждый необратимый элемент является делителем нуля}


 	\begin{lemma}\label{4_lemma_1} 
 		Пусть $R$~--- нётерово кольцо. Тогда 
 		\begin{enumerate}
 			\item Любой минимальный простой идеал состоит из делителей нуля. 
 			\item Множество всех делителей нуля нётерова кольца $R$ представляет из себя объединение конечного числа ассоциированных простых идеалов $\fp_{i}$, а $\fp_i = \Ann(a_i)$ для некоторого $a_i \in R$.
 			\item Пусть $R$~--- нётерово кольцо, причем любой его элемент либо обратим, либо делитель нуля и вдобавок $R$ редуцировано (т.е. $\NRad(R) = 0$). Тогда $\dim{R} = 0$ (т.е. $R$~--- артиново кольцо). 
 		\end{enumerate}
 	\end{lemma}
 	\begin{proof}

 		Докажем сначала \bf{первый пункт}.\hypertarget{bilet_9}{} Пусть $\fp$~--- минимальный простой идеал. Тогда, так как в кольце $R_{\fp}$ идеал $\fp R_{\fp}$ максимальный, 
 		\[
 			\NRad(R_{\fp}) = \fp R_{\fp}.
 		\]
 		Так $R_{\fp}$ нётерово (локализация нётерова кольца нётерова), значит идеал конечнопорождён: $\fp R_{\fp} = (e_1, \ldots, e_n)$. Тогда, если $\forall j \quad e_{j}^n = 0$, то
 		\[
 			\NRad(R_{\fp})^{n m} = 0.
 		\]
 		Значит, $\forall a \in \fp \  a^{nm} = 0$ в колцье $R_{\fp}$. Тогда $\exists s \in \fp \colon a^N s = 0$ в $R$, значит $a$~--- делитель нуля. 
 		

 		\bf{Второй}\hypertarget{bilet_10}{} пункт был в курсе коммутативной алгебры. 
 

 		Теперь докажем \bf{третий}\hypertarget{bilet_11}{} пункт. Возьмём $\fm \in \Specm{R}$, тогда он полностью состоит из делителей нуля. Тогда по пункту $2$:
 		\[
 			\fm \subset \bigcup_{i = 1}^{m} \fp_i \implies \fm \subset \fp_i \implies \fm = \fp_i.
 		\]
 		Тогда $\fm = \Ann(a)$ для некоторого $a \in R$. Предположим, что $\fp \subsetneq \fm$. Рассмотрим два случая: 
 		\begin{enumerate}
 			\item $a \in \fm$. Тогда $a \in \Ann(a)$, откуда $a^2 = 0$, но это противоречит тому, что $\NRad(R) = 0$. 

 			\item $a \notin \fm$. Возьмём тогда $b \in \fm \setminus \fp$. Тогда $a b = 0$. Заметим, что $a \notin \fp, b \notin \fp$. Но тогда мы получили противоречие с тем, что идеал $\fp$ простой. 
 		\end{enumerate}
 	\end{proof}

 	\hypertarget{bilet_12}{}

 	\begin{theorem} 
 		Пусть $R$~--- нётерово кольцо, а $S = R[x]$. Пусть $\dim{R} = d - 1, \ d \ge 1$, а $I \lei S$. Тогда $\exists f_1, \ldots, f_{d} \in I\colon$
 		\[
 			\sqrt{I} = \sqrt{(f_1, \ldots, f_d)}.
 		\]
 	\end{theorem}
 	\begin{proof}
 		Пусть $d = 1$, тогда $\dim{R} = 0$ и $R/\NRad(R)$~--- редуцированное артиново кольцо, то есть прямая сумма конечного числа полей 
 		\[
 			R/\NRad(R) = K_1 \oplus K_2 \oplus \ldots K_n.
 		\]

 		С другой стороны, легко проверить, что $\NRad(R[x]) = \NRad(R)[x]$. Но тогда мы получаем, что 
 		\[
 			S/\NRad(S) \cong K_1[x] \oplus \ldots K_n[x],
 		\]
 		а справа написана область главных идеалов. Тогда идеал $I + \NRad(S)/\NRad(S)$ главный, пусть он пораждается $f \in I$. Тогда 
 		\[
 			I + \NRad(S) = (f) + \NRad(S) \implies \sqrt{I} = \sqrt{I + \NRad(S)} = \sqrt{(f) + \NRad(S)} = \sqrt{(f)}. 
  		\]

  		Поясним равенство $\sqrt{I} = \sqrt{I + \NRad(S)}$. Включение $(\subset)$ очевидно, докажем включение $(\supset)$. Пусть $x \in \sqrt{I + \NRad(S)}$, тогда $x^m = a + b$, где $a \in I$, $b \in \NRad(S)$. Тогда, так как $b^N = 0$ для некоторого $N$, $(x^m)^N \in I$, откуда $x \in \sqrt{I}$. 
  		

  		Сделаем теперь переход $d - 1 \mapsto d$. Так как при факторизации по $\NRad(R)$ размерность не меняется, не умаляя общности мы можем полагать, что с самого начала кольцо редуцированное. 

  		Рассмотрим $U$~--- множество всех не делителей нуля в $R$. Рассмотрим локализацию $U^{-1}R = R[U^{-1}]$. Заметим, что в $R[U^{-1}]$ любой элемент либо обратим, либо является делителем нуля.  Тогда по лемме~\ref{4_lemma_1} это кольцо будет редуцированным нётеровым кольцом размерности 0, то есть произведением конечного числа полей. Тогда 
  		\[
  			S[U^{-1}] = \prod K_i[x],
  		\]
  		то есть это в частности кольцо главных идеалов. Тогда  $IS[U^{-1}] = (f_1) S[U^{-1}]$, где $f_1 \in I$.

  		 Тогда, так как $I$ конечно порожден, $\exists r \in U\colon r I \subset (f_1) \subset S$\footnote{Можно считать, что $r \in U$, домножая на НОК знаменателей образующих. }. Так как $r$~--- не делитель нуля, он не лежит в объединении всех минимальных простых идеалов кольца $R$ (тут мы вновь пользуемся леммой~\ref{4_lemma_1}). Перейдём к фактору и покажем, что

  		\[
  			\dim{R/(r)} \le d - 2 
  		\]

  		В самом деле, если $\dim{R/(r)} = d - 1$, то у нас есть цепочка 
  		\[
  			\fp_0/(r) \subsetneq \fp_1/(r) \subsetneq  \ldots \ \subsetneq \fp_{d - 1}/(r). 
  		\]

  		Тогда, поднимаясь к исходному кольцу, мы получаем такую цепочку: 
  		\[
  			(r) \subset \fp_0 \subsetneq \fp_1 \subsetneq \ldots \subsetneq \fp_{d - 1}.
  		\]
  		Но тогда идеал $\fp_0$ не может быть минимальным (так как $r$ не лежит ни в каком минимальном), а значит, мы можем увеличить цепочку и получить противоречие. 

  		Теперь мы можем применить к кольцу $R/(r)$ индукционное предположение: $\exists \overline{f_2}, \ldots, \overline{f_d} \in I + (r)/(r)\colon$
  		\[
  			\sqrt{I + (r)/(r)} = \sqrt{(\overline{f_1}, \ldots, \overline{f_d})}.
  		\]
 		Теперь остается проверить только, что 
 		\[
 			\sqrt{I} = \sqrt{(f_1, \ldots, f_d)}.
 		\]
 		Действительно, если $x \in \sqrt{I}$, то $x^k \in I$, а кроме того, $\overline{x}^m \in \sqrt{(\overline{f_2}, \ldots, \overline{f_d})} \cdot R/(r)$, то есть $x^m \in (r, f_2, \ldots, f_d)$.

 		Тогда $x^{k + m} \in x^m I \in (r, f_2, \ldots, f_d)I$, но так как $rI \subset (f)1$, то есть 
 		\[
 			x^{k + m} \in x^m I \in (r, f_2, \ldots, f_d)I \subset (f_1, \ldots, f_d). 
 		\]

 	\end{proof}

 	Пусть $X \subset \AA^n$~--- аффинное многообразие, тогда $I(X) \subset S = \Bbbk[x_1, \ldots, x_{n - 1}][x_n] = R[x]$, $\dim{R} = n - 1$. Тогда по предыдущей теореме мы можем найти $f_1, \ldots, f_d$ такие, что 
 	\[
 		\sqrt{I(X)} = I(X) = \sqrt{(f_1, \ldots, f_d)}. 
 	\]
 	Тогда ясно, что $X = Z(I) = Z(\sqrt{f_1, \ldots, f_n}) = Z(f_1, \ldots, f_n)$, чего мы и хотели. 

 	\begin{corollary}
 		В $\AA^n$ любое аффинное многообразие задаётся не более чем $n$ уравнениями. 
 	\end{corollary}

    \section{Проективные многообразия}

 	\subsection{Проективные многообразия}

 	Пусть $\bk$~--- наше базовое алгебраически замкнутое поле, рассмотрим проективное пространство $\PP^n = \PP^n_{\bk}$. 

	В этом контексте кольцо многочленов $S = \bk[x_0, x_1, \ldots, x_n]$ мы будем рассматривать, как градуированное кольцо. Для этого вкратце напомним терминологию: 

	\begin{definition} 
		Кольцо $S$ называется \emph{градуированным}\footnote{Если точнее, $\N_{\ge 0}$-градуированным. }, если оно обладает разложением в прямую сумму 
		\[
 			S	 = \bigoplus_{d \ge 0} S_{d}
 		\] 
 		абелевых групп $S_d$ таких, что $S_{d} \cdot S_{e} \subset S_{d e}$. Элементы из $S_{d}$ мы будем называть \emph{однородными степени } $d$. 

 		Идеал $\fa \subset S$ мы будем называть \emph{однородным}, если он представляется в виде 
 		\[
 			\fa = \bigoplus_{d \ge 0} (\fa \cap S_{d}).
 		\]
 	\end{definition} 	

 	Приведём несколько полезных фактов про однородные идеалы: 

 	\begin{itemize}
 		\item Идеал однородный тогда и только тогда, когда он может быть порожден однородными элементами. 

 		\item Сумма, произведение, пересечение однородных идеалов, а также радикал однородного идеала однородны.

 		\item Однородный идеал $\fa$ простой тогда и только тогда, когда для любых двух \emph{однородных} $f, g$ из условия $fg$ следует, что либо $f \in \fa$, либо $g \in \fa$. 
 	\end{itemize}

 	Кольцо $S = \bk[x_0, x_1, \ldots, x_n]$ мы превратим в градуированное так: обозначим за $S_d$ множество всех линейных комбинаций одночленов полной степени $d$ от переменных $x_0, \ldots, x_n$. 

 	Кроме того, многочлены мы уже не можем рассматривать как функции на $\PP^n$ ввиду неоднозначности координатных представлений точек $\PP^n$. Но, заметим, что если $f$~--- однородный многочлен, то очевидно, что свойство $f$ обращаться в $0$ зависит только от класса эквивалентности $(a_0, \ldots, a_n)$. Тем, самым, для однородных многочленов имеет смысл говорить о множестве 
 	\[
 		Z(f) \eqdef \{ P \in \PP^n \ \vert \ f(P) = 0 \}. 
 	\]
 	Соотвественно, для любого множества $T$ однородных элементов мы определяем 
 	\[
 		Z(T) \eqdef \{ P \in \PP^n \ \vert \ f(P) = 0 \quad \forall f \in T \}.
 	\]

 	Если $\fa$~--- однородный идеал в $S$, то определим $Z(\fa)$, как $Z(\fa) = Z(T)$, где $T$~--- множество всех однородных элементов из $\fa$. В силу нётеровости кольца $S$ любое множество однородных элементов $T$ содержит такое конечное подмножество $f_1, \ldots, f_r$, что $(T) = (f_1, \ldots, f_r)$.


 	\begin{definition} 
 		Подмножество проективного пространства $Y \subset \PP^n$ называется \emph{проективным алгебраическим многообразием}, если существует такое множество $T \subset S$ однородных элементов, что $Y = Z(T)$.
 	\end{definition}

 	Таким образом, мы можем задать на $\PP^n$ тополгию Зарисского, объявив замкнутыми алгебраические многообразия. 

 	\begin{definition} 
 		Также, для любого $Y \subset \PP^n$ определим его \emph{однородный идеал} $I(Y) \subset S$, как идеал, порожденный множеством однородных элементов $f \in S$ таких, что $f(P) = 0$ для всех $P \in Y$. \emph{Однородное координатное кольцо $S(Y)$} проективного многообразия $Y$ определим как факторкольцо $S(Y) = S/I(Y)$. 	
 	\end{definition}
 	

 	\begin{exercise}
 		Пусть $S = \Bbbk[x_1, \ldots, x_n]$, обозначим за $S^h$ множество однородных многочленов. Тогда 
 		\begin{enumerate}
		 	\item Если $T_1 \subset T_2 \subset S^h$, то $Z(T_2) \subset Z(T_1)$.

		 	\item Если $Y_1 \subset Y_2 \subset \PP^n$, то $I(Y_2) \subset I(Y_1)$.

		 	\item $I(Y_1 \subset Y_2) = I(Y_1) \cap I(Y_2)$.

		 	\item Пусть $I \subset S$~--- однородный идеал, $Z(I) \neq \varnothing$. Тогда $I(Z(I)) = \sqrt{I}$. 

		 	\item Пусть $I \subset S$~--- однородный идеал. Тогда следующие условия равносильны: 
		 	\begin{enumerate}
		 		\item $Z(I) = \varnothing$

		 		\item $\sqrt{I} = (1)$ или $\sqrt{I} = S+ = \sum_{d > 0} S_d$

		 		\item $S_d \subset I$ для некоторого $d$. 
		 	\end{enumerate}
		 \end{enumerate}		 
	\end{exercise}



 	\begin{theorem}[Однородный Nullstelensatz] 
 		Пусть $I \subset \Bbbk[x_0, \ldots, x_n]$~--- однородный идеал, а $f \in \Bbbk[x_0, \ldots, x_n]$~--- однородный элемент положительной степени. Пусть $f(P) = 0$. Тогда 
 		\[
 			\forall P \in Z(I) \subset \PP^n \implies \exists m \colon f^m \in I.
 		\]
 	\end{theorem}

 	Это теорема легко сводится к аффинному случаю. 

 	\begin{definition} 
 		\emph{Квазипроективным многообразием} мы будем называть открытое подмножество проективного многообразия. 
 	\end{definition}

 	Наша дальнейшая цель состоит в том, чтоб показать, что $n$-мерное проективное пространство обладает открытым покрытием, состоящим из $n$-мерных аффинных пространств. И, как весьма полезное следствие, что \emph{всякое проективное/квазипроективное многообразие) обладает открытым покрытием состоящие из аффинных/квазиаффинных многообразий} (это очень удобно, если мы доказываем что-то локально).

 	Пусть $H_i = Z(x_i)$~--- координатные гиперплоскости. Рассмотрим множества 
 	\[
 		U_i = \PP^n \setminus H_i = \{ (x_0 : \ldots : x_n ) \in \PP^n \ \vert \ x_i \neq 0 \}. 
 	\]

 	Совершенно ясно, что $\PP^n$ покрывается множествами $U_i$ (так как у любой точки хотя бы одна из однородных координат отлична от нуля). Рассмотрим отображение 
 	\[
 		 \varphi_i\colon U_i \to \AA^n, \quad (a_0 : \ldots : a_n) \mapsto \lr*{\frac{a_0}{a_i} : \ldots : \frac{a_n}{a_i}}.
 	\]
 	Отметим, что это отображение опредедлено корректно, так как частное $a_j/a_i$ не зависит от выбора однородных координат. 

 	\begin{statement}\label{U_icongA^n} 
 		Отображение $\varphi_i$ осуществляет гомеоморфизм $U_i \xrightarrow{\sim} \AA^n$.
 	\end{statement}
 	\begin{proof}
 		Очевидно, что оно биективно. Достаточно показать, что замкнутые множества в $U_i$ соотвествуют замкнутым множествам в $\AA^n$. Не умаляя общности, $i = 0$, $U_0 = U, \ \varphi_0 = \varphi$.

 		Пусть $A = \bk[y_1, \ldots, y_n]$. Рассмотрим отображения 
 		\[
 			\alpha\colon S^h \to A, \quad \alpha(f) = f(1, y_1, \ldots, y_n), \ \beta \colon A \to S^h, \quad \beta(g) = x_0^{\deg{g}} \cdot g\lr*{\frac{x_1}{x_0}, \ldots, \frac{x_n}{x_0}}.
 		\]

 		Пустть $Y \subset U$~--- замкнутое подмножество.Тогда $\overline{Y}$ (тут замыкание берётся в $\PP^n$)~--- проективное многообразие, то есть $\overline{Y} = Z(T)$ для некотрого $T \subset S^h$. Положим $\alpha(T) = T'$. Тогда непосредственно проверяется, что $\varphi(Y) = Z(T')$. И обратно, если $W$~--- замкнутое подмножество в $\AA^n$, то $W = Z(T')$ для некторого $T' \subset A$ и легко проверить, что 
 		\[
 			\varphi^{-1}(W) = Z(\beta(T')) \cap U..
 		\]
 		Значит и $\varphi$ и $\varphi^{-1}$ замкнутые, что и требовалось. 
 	\end{proof}

 	\begin{corollary}
 		Пусть $Y$~--- проективное (квазипроективное) многообразие. Тогда $Y$ покрывается открытыми множествами $Y \cap U_i$, гомеоморфными аффинным (квазиаффинным) многообразиям, причем гомеоморфизм осуществляется определённым выше отображением $\varphi_i$.
 	\end{corollary}



	

 	








