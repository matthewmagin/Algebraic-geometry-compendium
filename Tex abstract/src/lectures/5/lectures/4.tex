	
	Если же $X, Y \in \qAff$, то их произведение также буде квазиаффинным. 

	Проверим теперь, что для неприводимых многообразий $\dim{X \times Y} = \dim{X} + \dim{Y}$. Так как размерность открытого подмножества аффинного совпадает с рахмерностью всего многообразия, достаточно показывать это для аффинных. 

	Начнем с того, что для неприводимого аффинного многообразия $X$ определение поля рациональных функций $\Bbbk(X)$ можно дать несколько иначе от определения~\ref{rat_func}. 

	Предположем, что $X$ неприводимо, тогда мы можем рассмотреть поле частных координатного кольца $A(X)$. Кроме того, рассматривая очевидное отображение 
	\[
		A(X) \to \Bbbk(X).
	\]
	мы получаем и вложение поле $\mathrm{Frac}(A(X)) \hookrightarrow \Bbbk(X)$. С другой стороны, нетрудно видеть, что это изоморфизм. 

	Пусть $\dim{X} = r, \ \dim{Y} = s$. Поле функций $\Bbbk(X)$ порождается ровно  $r$ алгебраически независимыми координатными функциями $u_1, \ldots, u_r$. Аналогично, $\Bbbk(Y)$ порождается координатныими функциями $v_1, \ldots, v_s$ и они алгебраически независимы. Тогда совершенно ясно, что 
	\[
		\dim{(X \times Y)} = \mathrm{trdeg}\lr*{\Bbbk(X \times Y)} \le r + s.
	\]

	Остаётся показать, что система $(u_1, \ldots, u_r, v_1, \ldots, v_s)$ будет алгебраически независимой в $\Bbbk(X \times Y)$. 

	Предположим, что 
	\[
		\sum f_{i_1 i_2 \ldots i_r}(v_1, \ldots, v_s) u_1^{i_1} \cdot \ldots \cdot u_{r}^{i_r} = 0.
	\]

	Подставляя $a_i \in \Bbbk$, мы получаем полиномиальное соотношение на $u_i:$

	\[
		\sum f_{i_1 i_2 \ldots i_r}(a_1, \ldots, a_s) u_1^{i_1} \cdot \ldots \cdot u_{r}^{i_r} = 0,
	\]
	а так как $u_i$ алгебраически независимы, отсюда следует, что $f_{i_1 \ldots i_r}(a_1, \ldots, a_s) = 0$. По произвольности набора  $a_1, \ldots, a_s$, мы получаем, что $f_{i_1 i_2 \ldots i_r}(v_1, \ldots v_s) = 0$, но так как $v_i$ алгебраически независимы, отсюда следует, что $f_{i_1 \ldots i_r} = 0$, что и требовалось. 

	Обсудим, что происходит в случае приводимых многообразий. Если 
	\[
		X = \bigcup_{i} X_{i}, Y = \bigcup_{j} Y_{j} \quad X_{i}, Y_{j} \text{~--- неприводимые},
 	\]
 	то $\dim{X} = \max{\dim{X_i}}$, а $\dim{Y} = \max{\dim{Y_j}}$. Тогда у нас есть разложение 
 	\[
 		X \times Y = \bigcup X_i \times Y_j 
 	\]
 	в неприводимые и ясно, что $\dim{(X \times Y)} = \dim{X} + \dim{Y}$. 

 	Далее нам понадобится несколько лемм из курса коммутативной алгебры. 

 	\begin{lemma}\label{4_lemma_1} 
 		Пусть $R$~--- нётерово кольцо. Тогда 
 		\begin{enumerate}
 			\item Любой минимальный простой идеал состоит из делителей нуля. 
 			\item Множество всех делителей нуля нётерова кольца $R$ представляет из себя объединение конечного числа ассоциированных простых идеалов $\fp_{i}$, а $\fp_i = \Ann(a_i)$ для некоторого $a_i \in R$.
 			\item Пусть $R$~--- нётерово кольцо, причем любой его элемент либо обратим, либо делитель нуля и вдобавок $R$ редуцировано (т.е. $\NRad(R) = 0$). Тогда $\dim{R} = 0$ (т.е. $R$~--- артиново кольцо). 
 		\end{enumerate}
 	\end{lemma}
 	\begin{proof}
 		Пусть $\fp$~--- минимальный простой идеал. Тогда мы помним, что $\NRad(R_{\fp}) = \fp R_{\fp}$ (так как и то и другое будет пересечением всех простых идеалов кольца $R_{\fp})$. Но, $\NRad(R_{\fp})$  нильпотентен. Тогда $a \in \fp \implies a^{N} = 0 \in R_{\fp}$, а тогда $a^{N}s = 0$, где $s \notin \fp$. Отсюда ясно, что $a$ является делителем нуля.  

 		Второй пункт сразу получается из примарного разложения. 

 		Теперь докажем первый пункт. Возьмём $\fm \in \Specm{R}$, по условию, он полностью состоит из делителей нуля. Тогда по пункту $2$:
 		\[
 			\fm \subset \bigcup_{i = 1}^{m} \fp_i \implies \fm \subset \fp_1 \implies \fm = \fp_1.
 		\]
 		Тогда $\fm = \Ann(a)$ для некоторого $a \in R$. Предположим, что $\fp \subsetneq \fm$. Рассмотрим два случая: 
 		\begin{enumerate}
 			\item $a \in \fm$. Тогда $a \in \Ann(a)$, откуда $a^2 = 0$, но это противоречит тому, что $\NRad(R) = 0$. 

 			\item $a \notin \fm$. Возьмём тогда $b \in \fm \setminus \fp$. Тогда $a b = 0$. Заметим, что $a \notin \fp, b \notin \fp$. Но тогда мы получили противоречие с тем, что идеал $\fp$ простой. 
 		\end{enumerate}
 	\end{proof}

 	\begin{theorem} 
 		Пусть $R$~--- нётерово кольцо, а $S = R[x]$. Пусть $\dim{R} = d - 1, \ d \ge 1$, а $I \lei S$. Тогда $\exists f_1, \ldots, f_{d} \in I\colon$
 		\[
 			\sqrt{I} = \sqrt{(f_1, \ldots, f_d)}.
 		\]
 	\end{theorem}
 	\begin{proof}
 		Пусть $d = 1$, тогда $\dim{R} = 0$ и $R/\NRad(R)$~--- редуцированное артиново кольцо, то есть прямая сумма конечного числа полей 
 		\[
 			R/N(R) = K_1 \oplus K_2 \oplus \ldots K_n.
 		\]

 		С другой стороны, легко проверить, что $\NRad(R[x]) = \NRad(R)[x]$. Но тогда мы получаем, что 
 		\[
 			S/\NRad(S) \cong K_1[x] \oplus \ldots K_n[x],
 		\]
 		а справа написана область главных идеалов. Тогда идеал $I + \NRad(S)/\NRad(S)$ главный, пусть он пораждается $f \in I$. Тогда 
 		\[
 			I + \NRad(s) = (f) + \NRad(S) \implies \sqrt{I} = \sqrt{I + \NRad(S)} = \sqrt{(f) + \NRad(S)} = \sqrt{(f)}. 
  		\]

  		Сделаем теперь переход $d - 1 \mapsto d$. Так как при факторизации по $\NRad(R)$ рахмерность не меняется, не умаляя общности мы можем полагать, что с самого начала кольцо редуцированное. 

  		Рассмотрим $U$~--- множество всех не делителей нуля в $R$. Рассмотрим локализацию $R[U^{-1}]$. Заметим, что в $R[U^{-1}]$ любой элемент либо обратим, либо является делителем нуля.  Тогда по лемме~\ref{4_lemma_1} это кольцо будет редуцированным нётеровым кольцом размерности 0, то есть произведением конечного числа полей. Тогда 
  		\[
  			S[U^{-1}] = \prod K_i[x],
  		\]
  		то есть кольцо главных идеалов. Тогда  $IS[U^{-1}] = (f_1) S[U^{-1}]$, где $f_1 \in I$. Тогда, так как $I$ конечно порожден, $\exists r \in U\colon r I \subset (f_1) \subset S$. Так как $r$~--- не делитель нуля, он не лежит в объединении всех минимальных простых идеалов кольца $R$ (тут мы вновь пользуемся леммой~\ref{4_lemma_1}). Тогда мы можем перейти к фактору $R/(r)$:

  		\[
  			\dim{R/(r)} \le d - 2 
  		\]

  		В самом деле, если $\dim{R/(r)} = d - 1$, то у нас есть цепочка 
  		\[
  			0 \subsetneq \fp_1/(r) \subsetneq  \ldots \ \subsetneq \fp_{d - 1}/(r). 
  		\]

  		Тогда, поднимаясь к исходному кольцу, мы получаем такую цепочку: 
  		\[
  			(r) \subset \fp_0 \subsetneq \fp_1 \subsetneq \ldots \subsetneq \fp_{d - 1}.
  		\]
  		Но тогда идеал $\fp_0$ не может быть минимальным (так как $r$ не лежит ни в каком минимальном), а значит, мы можем увеличить цепочку и получить противоречие. 

  		Теперь мы можем применить к кольцу $R/(r)$ индукционное предположение: $\exists \overline{f_2}, \ldots, \overline{f_d} \in I + (r)/(r)\colon$
  		\[
  			\sqrt{I + (r)/(r)} = \sqrt{(\overline{f_1}, \ldots, \overline{f_d})}.
  		\]
 		Теперь остается проверить только, что 
 		\[
 			\sqrt{I} = \sqrt{(f_1, \ldots, f_d)}.
 		\]
 		Действительно, если $x \in \sqrt{I}$, то $x^k \in I$, а кроме того, $\overline{x}^m \in \sqrt{(\overline{f_1}, \ldots, \overline{f_d})}$, то есть $x^m \in (r, f_2, \ldots, f_d)$.

 		Тогда $x^{k + m} \in x^m I \in (r, f_2, \ldots, f_d)I$, но так как $rI \subset (f)1$, то есть 
 		\[
 			x^{k + m} \in x^m I \in (r_1, f_2, \ldots, f_d)I \subset (f_1, \ldots, f_d). 
 		\]

 	\end{proof}

 	Пусть $X \subset \AA^n$~--- аффинное многообразие, тогда $I(X) \subset S = \Bbbk[x_1, \ldots, x_{n - 1}][x_n] = R[x]$, $\dim{R} = n - 1$. Тогда по предыдущей теореме мы можем найти $f_1, \ldots, f_d$ такие, что 
 	\[
 		\sqrt{I(X)} = I(X) = \sqrt{(f_1, \ldots, f_d)}. 
 	\]
 	Тогда ясно, что $X = Z(I) = Z(\sqrt{f_1, \ldots, f_n}) = Z(f_1, \ldots, f_n)$, чего мы и хотели. 

 	\subsection{Проективные многообразия}

 	\begin{definition} 
 		Пусть $I$~--- однородный идеал в кольце $\Bbbk[x_0, \ldots, x_{n}]$. Тогда определим $Z(I)$~--- множество нулей всех однородных многочленов из $I$. 
 		
 		Замнкутыми подмножествами $\PP^n$ мы объявим множества вида $Z(I)$.
 	\end{definition}

 	Таким образом на $\PP^n$ мы заводим топологию Зарисского. Сформулируем однородуню теорему о нулях: 

 	\begin{theorem}[Однородный Nullstelensatz] 
 		Пусть $I \subset \Bbbk[x_0, \ldots, x_n]$~--- однородный идеал, а $f \in \Bbbk[x_0, \ldots, x_n]$~--- однородный элемент положительной степени. Пусть $f(P) = 0$. Тогда 
 		\[
 			\forall P \in Z(I) \subset \PP^n \implies \exists m \colon f^m \in I.
 		\]
 	\end{theorem}

 	Это теорема легко сводится к аффинному случаю. 

	\begin{exercise}
 		Пусть $S = \Bbbk[x_1, \ldots, x_n]$, обозначим за $S^h$ множество однородных многочленов. Тогда 
 		\begin{enumerate}
 				 	\item Если $T_1 \subset T_2 \subset S^h$, то $Z(T_2) \subset Z(T_1)$.

 				 	\item Если $Y_1 \subset Y_2 \subset \PP^n$, то $I(Y_2) \subset I(Y_1)$.

 				 	\item $I(Y_1 \subset Y_2) = I(Y_1) \cap I(Y_2)$.

 				 	\item Пусть $I \subset S$~--- однородный идеал, $Z(I) \neq \varnothing$. Тогда $I(Z(I)) = \sqrt{I}$. 
 				 \end{enumerate}		 
	\end{exercise}
 	








