	
	Если же $X, Y \in \qAff$, то их произведение также буде квазиаффинным. 

	Проверим теперь, что для неприводимых многообразий $\dim{X \times Y} = \dim{X} + \dim{Y}$. Так как размерность открытого подмножества аффинного совпадает с рахмерностью всего многообразия, достаточно показывать это для аффинных. 

	Начнем с того, что для неприводимого аффинного многообразия $X$ определение поля рациональных функций $\Bbbk(X)$ можно дать несколько иначе от определения~\ref{rat_func}. 

	Предположем, что $X$ неприводимо, тогда мы можем рассмотреть поле частных координатного кольца $A(X)$. Кроме того, рассматривая очевидное отображение 
	\[
		A(X) \to \Bbbk(X).
	\]
	мы получаем и вложение поле $\mathrm{Frac}(A(X)) \hookrightarrow \Bbbk(X)$. С другой стороны, нетрудно видеть, что это изоморфизм. 

	Пусть $\dim{X} = r, \ \dim{Y} = s$. Поле функций $\Bbbk(X)$ порождается ровно  $r$ алгебраически независимыми координатными функциями $u_1, \ldots, u_r$. Аналогично, $\Bbbk(Y)$ порождается координатныими функциями $v_1, \ldots, v_s$ и они алгебраически независимы. Тогда совершенно ясно, что 
	\[
		\dim{(X \times Y)} = \mathrm{trdeg}\lr*{\Bbbk(X \times Y)} \le r + s.
	\]

	Остаётся показать, что система $(u_1, \ldots, u_r, v_1, \ldots, v_s)$ будет алгебраически независимой в $\Bbbk(X \times Y)$. 

	Предположим, что 
	\[
		\sum f_{i_1 i_2 \ldots i_r}(v_1, \ldots, v_s) u_1^{i_1} \cdot \ldots \cdot u_{r}^{i_r} = 0.
	\]

	Подставляя $a_i \in \Bbbk$, мы получаем полиномиальное соотношение на $u_i:$

	\[
		\sum f_{i_1 i_2 \ldots i_r}(a_1, \ldots, a_s) u_1^{i_1} \cdot \ldots \cdot u_{r}^{i_r} = 0,
	\]
	а так как $u_i$ алгебраически независимы, отсюда следует, что $f_{i_1 \ldots i_r}(a_1, \ldots, a_s) = 0$. По произвольности набора  $a_1, \ldots, a_s$, мы получаем, что $f_{i_1 i_2 \ldots i_r}(v_1, \ldots v_s) = 0$, но так как $v_i$ алгебраически независимы, отсюда следует, что $f_{i_1 \ldots i_r} = 0$, что и требовалось. 

	Обсудим, что происходит в случае приводимых многообразий. Если 
	\[
		X = \bigcup_{i} X_{i}, Y = \bigcup_{j} Y_{j} \quad X_{i}, Y_{j} \text{~--- неприводимые},
 	\]
 	то $\dim{X} = \max{\dim{X_i}}$, а $\dim{Y} = \max{\dim{Y_j}}$. Тогда у нас есть разложение 
 	\[
 		X \times Y = \bigcup X_i \times Y_j 
 	\]
 	в неприводимые и ясно, что $\dim{(X \times Y)} = \dim{X} + \dim{Y}$. 

