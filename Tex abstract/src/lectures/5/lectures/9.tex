	
	\subsection{Касательное пространство}

	\begin{definition} 
		Пусть $X \subset \AA^n$~--- аффинное многообразие, $I(X) = (f_1, \ldots, f_m), \ P \in X$. Тогда пространство решений системы линейных уравнений 
		\begin{equation}
	 		\begin{cases} \cfrac{\partial f_1}{\partial x_1}(P) t_1 + \ldots + \cfrac{\partial f_1}{\partial x_n}(P) t_n = 0 \\ 
	 		\vdots \\
	 		\cfrac{\partial f_m}{\partial x_1}(P) t_1 + \ldots + \cfrac{\partial f_m}{\partial x_n}(P) t_n = 0 
	 		 \end{cases}	 \label{tangent_space}
	 	\end{equation}
	 	называется \emph{касательным пространством к многообразию $X$ в точке $P$}.
	\end{definition}

	\begin{remark}
		Определение корректно, то есть оно не зависит от выбора образующих в идеале $I(X)$.
	\end{remark}
	\begin{proof}
		Возьмём $f \in I(X)$ и разложим его по образующим 
		\[
			f = f_1 g_1 + \ldots + f_m g_m.
		\]
		Теперь продифференцируем: 
		\[
			\frac{\partial f}{\partial x_1}(P) = \sum_{j = 1}^{m} \lr*{ \frac{\partial f_j}{\partial x_1}(P) \cdot g_j(P) + \underbrace{f_j(P)}_{ = 0, \text{ т.к. } f_j \in I(X)} \cdot \frac{\partial g_j}{\partial x_1}(P) } = \sum_{j = 1}^{m} \frac{\partial f_j}{\partial x_1}(P) \cdot g_j(P).
		\]
		Тогда отсюда мы заключаем, что 
		\[
			\sum_{i} \frac{\partial f}{\partial x_i} t_i = \sum_{i, j} \frac{\partial f_j}{\partial x_i}(P) g_j(P) t_i = \sum_{j} g_j(P) \cdot \lr*{ \sum_{i} \frac{\partial f_j}{\partial x_1}(P) \cdot t_i }
		\] 
		Теперь заметим, что если $(t_1, \ldots, t_n)$ удовлетворяют системе~\ref{tangent_space}, то каждое сллагаемое будет равно нулю. Значит, можно определять касательное пространство более инвариантно: записать бесконечную систему таких уравнений по все $f \in I(X)$, а это~--- то, что нам нужно. 
	\end{proof}

	Теперь рассмотрим билинейное спаривание 
	\[
		\fm_{P}/\fm_{P}^2 \times T_pX \to \bk, \quad (\overline{g}, (t_1, \ldots, t_n)) \mapsto \sum_{i} \frac{\partial g}{\partial x_i}(P)t_i \in \bk.
	\]
	Покажем, что оно невырождено. Для начала, поясним, что это отображение определено корректно. 

	\begin{itemize}
		\item Так как $g \in \cO_P$, в окрестности точки $P$ мы можем представить $g$ в виде $g = r/s$. Тогда по определению логично думать, что 
		\[
			g' = \frac{r' s - r s'}{s^2}.
		\]
		\item Если $\overline{g_1} = \overline{g_2}$, то $g_1 - g_2 = h \in \fm_P^2$. Тогда 
		\[
			h = \sum_{j} \ell_j \ell_j', \text{ где } \ell_i, \ell_i' \in \fm_P \implies \frac{\partial h}{\partial x_i}(P) = \sum_j \lr*{ \ell_j(P) \frac{\partial \ell_j'}{\partial x_i}(P) + \frac{\partial \ell_j}{\partial x_i}(P) \cdot \ell_j'(P) }. 
		\]
		
		\item Теперь, ддля $g \in \cO_{P}$, $s \in I(X)$, тогда 
		\[
			(\overline{g}, (t_1, \ldots, t_n)) = (\overline{g + s}, (t_1, \ldots, t_n)),
		\]
		так как $s \in I(X)$. 
	\end{itemize}

	Теперь наконец покажем, что оно невырождено.

	\begin{itemize}
		\item  Зафиксируем $(t_1, \ldots, t_n) \in T_P X$. Предположим, что 
	\[
		\forall g \in \fm_P \quad \sum_i \frac{\partial g}{\partial x_i}(P) t_i = 0.
	\] 
	Пусть $P = (p_1, \ldots, p_n)$. Тогда, если мы возьмём $g = x_i - p_i \in \fm_{P}$, то из равенства выше будет следовать, что $t_i = 0 \ \forall i$. 

	\item Теперь зафиксируем $g \in \fm_{P} \subset \cO_{P} \subset \bk(x_1, \ldots, x_n)$. Предположим, что 
	\[
	 	\sum_{i} \frac{\partial g}{\partial x_i}(P) \cdot t_i = 0 \quad \forall (t_1, \ldots, t_n) \in T_p X.
	 \] 
	 Это уравнение является следствием уравнений для касательного пространства, откуда 

	 \[
	 	\sum_{} \frac{\partial g}{\partial x_i}(P) t_i =  \alpha_1 \ell_1 + \ldots + \alpha_m \ell_m, \text{ где } \ell_i = \sum \frac{\partial f_i}{\partial x_j}(P) t_i.
	 \]
	 Приравнивая коэффициенты слева и справа мы получаем, что 
	 \[
	 	\frac{\partial g}{\partial x_i}(P) = \alpha_1 \frac{\partial f_1}{\partial x_i}(P) + \ldots + \alpha_m \frac{\partial f_m}{\partial x_i}(P) \ \forall i \implies \frac{\partial \lr*{g - \sum \alpha_j f_j} }{\partial x_i}(P) = 0.
	 \]
	 Положим $\widetilde{g} = g - \sum \alpha_j f_j$ и разложим $\widetilde{g}$ по формуле Тейлора в точке $P$: 
	 \[
	 	\widetilde{g}(x_1, \ldots, x_n) = \underbrace{\widetilde{g}(p_1, \ldots, p_n)}_{ = 0} + \sum_i \underbrace{\frac{\partial \widetilde{g}}{\partial x_i}(P)}_{ = 0} \cdot (x_i - p_i) + \varepsilon, \quad \varepsilon \in \fm_P^2,
	 \]
	 откуда $\overline{\widetilde{g}} = 0$ в $\fm_P/\fm_P^2$. С другой стороны, $\sum \alpha_j f_j \equiv 0$ на $X$, откуда $g \in \fm_P^2$, то есть $\overline{g} = 0$ в $\fm_P/\fm_P^2$, чего мы и хотели.  
	\end{itemize}

	Значит, мы только что доказали, что 
	\[
		T_{p}X \cong \lr*{\fm_P/\fm_P^2}^*.
	\]

	Это замечательное наблюдение позволяет нам распространить понятие касательного пространства с аффинного многообразия на произвольное квазипроективное: 
	
	\begin{definition} 
		Пусть $X \in \qProj$ а $P \in X$. Тогда \emph{касательным пространством к $X$ в точке $P$} мы будем называть векторное пространство 
		\[
			T_{p}X \eqdef \lr*{\fm_P/\fm_P^2}.
		\]
	\end{definition}

	\begin{remark}
	 	Так как идеал $\fm_P$ конечнопорожден, это пространство всегда конечномерное. 
	 \end{remark}

	 
	\begin{theorem} 
		Регулярное локальное кольцо геометрического происхождения является областью целостности. 
	\end{theorem}
	\begin{proof}
		Будем вести индукцию по $\dim{R}$.

		\noindent\bf{База.} Пусть $\dim{R} = 0$. Тогда, так как $R$ регулярно, 
		\[
			\dim{R} = \dim_{\bk}\fm_{P}/\fm_{P}^2 = 0 \iff \fm_{P} = \fm_{P}^2.
		\]
		Так как $R$~--- нётерово кольцо размерности 0, оно Артиново, а отсюда (и из того факта, что оно локальное) $\fm_{p} = \Rad(R) = \NRad(R)$, откуда существует такое $n$, что $\fm_{P}^n = 0$, но тогда $\fm_P = 0$, то есть $R$~--- поле. 

		\noindent\bf{Переход.} Пусть $\dim{R} \ge 1$. Рассмотрим элемент $x \in \fm$, не лежащий в объединении минимальных простых и $\fm^2$. Такой существует, так как 
		\[
			\fm = \bigcup_{\fp \text{~--- минимальный простой }} \fp \cup \fm^2 \implies \fm \subset \fp \text{ или } \fm \subset \fm^2.
		\]
		В первом случае $\dim{R} = 0$. Во втором случае по лемме Накаямы $\fm = 0$, откуда $R$~--- поле. 

		Теперь рассмотрим кольцо $S = R/(x)$. Ясно, что $\dim{S} \le \dim{R} - 1$.  Пусть $\dim{S} = d$. Заметим, что 
		\[
			\overline{\fm^2} \subset (\overline{x_1}, \ldots, \overline{x_d}) \subset \overline{\fm} \implies \fm^2 \subset (x, x_1, \ldots, x_d) \subset \fm, 
		\]
		откуда $\dim{S} \ge \dim{R} - 1$. Теперь заметим, что $S$~--- регулярное локальное кольцо. То, что оно локальное с единственным идеалом $\overline{\fm}$ ясно. Тогда мы знаем, что 
		\[
			\dim{S} \le \dim_{\bk}{\overline{\fm}/\overline{\fm}^2}. 
		\]
		С другой стороны, мы имеем
		\[
			\dim_{\bk}\overline{\fm}/\overline{\fm}^2 \le \dim_{\bk} \fm/\fm^2 - 1 = \dim{R} - 1 = \dim{S},
		\]
		так как отображение $\fm/\fm^2 \hookrightarrow \fm/\fm^2 + (x)$ имеет нетривиальное ядро. Значит, кольцо $S$ регулярно. Тогда по индукционному предположению $S$~--- область целостности, откуда $(x) \subset R$~--- простой идеал. Так как $x$ не лежит ни в одном минимальном простом идеале, существует $\fp \in \Spec{R}$ такой, что $\fp \subseteq (x)$. Рассмотрим $y \in \fp$, тогда $y = ax$ для некоторого $a \in R$. Но, так как $x \notin \fp$, отсюда $a \in \fp$. То есть $\fp = (x) \fp$, откуда по лемме Накаямы $\fp = 0$, то есть $R$~--- область целотсности. 
	\end{proof}

	\begin{definition} 
		Пусть $X$~--- многообразие, $P \in X$. Точка $P$ называется \emph{неособой}, если $\cO_P$ регулярно. Многообразие $X$ называтся \emph{неособым}, если каждая его точка неособая. 

		Точка, в которой локальное кольцо нерегулярно называется \emph{особой точкой}. 
	\end{definition}

	\subsection{Разложение в ряд Тейлора}

	Пусть $X$~--- многообразие, $P \in X$. Возьмём $f \in \cO_P$, тогда ясно, что $f - f(P) \in \fm_{P}$. Так как идеал $\fm_{P}$ конечнопорожден, мы можем выбрать какую-то систему образующих $\fm_P = (u_1, \ldots, u_n)$. Тогда 
	\[
		f - f(P) = g_1 u_1 + \ldots + g_n u_n, \quad g_i \in \cO_P.
	\]

	Аналогично, $g_i \in g_i(P) + \fm_P$, тогда 
	\[
		f - f(P) - \sum_{i = 1}^{n} g_i(P) u_i \in \fm_P^2 \implies f - f(P) - \sum_{i = 1}^{n} g_i(P) u_i = \sum_{i, j = 1}^{n} h_{i j} u_i u_j, \quad h_{i, j} \in \cO_P.
	\]
	Продолжая в том же духе мы получаем, что 
	\[
		\forall g \in \cO_P \ \exists F_0 + F_1 + \ldots \in \bk[[x_1, \ldots, x_n]]\colon f - \sum_{i = 0}^{s} F_i(u_1, \ldots, u_n) \in \fm_P^{s + 1} \ \forall s \in \N,
	\]
	где $F_i$~--- однородный многочлен степени $i$ от переменных $x_1, \ldots, x_n$. 

	\begin{definition} 
		Полученное выше представление и называется \emph{рядом Тейлора} для функции $f$ относительно системы локальных параметров $u_1, \ldots, u_n$. 
	\end{definition}

	Ответим сразу на естественный вопрос о единственности такого представления. 

	\begin{theorem} 
		Пусть $X$~--- многообразие, $P \in X$~--- неособая точка, а $\dim{\cO_{P}} = n$. Выберем систему образующих $\fm = (u_1, \ldots, u_n)$ и рассмотрим функцию $f \in \cO_P$. Тогда существует единственный ряд Тейлора для функции $f$ относительно системы $(u_1, \ldots, u_n)$.
	\end{theorem}

	\begin{proof}
		Докажем сначала вот такую лемму: 
		\begin{lemma}\label{taylor_series_lemma} 
			Пусть $F$~--- $s$-форма от $x_1, \ldots, x_n$ с коэффициентами из поля $\bk$. Предположим, что $F(u_1, \ldots, u_n) \in \fm_P^{s + 1}$. Тогда $F \equiv 0$.
		\end{lemma}
		\begin{proof}[Доказательство леммы]
			\bf{1)} Предположим сначала, что $F(0, \ldots, 1) = [u_n^s]F \neq 0$. Так как $a_n u_n^s + \ldots \in \fm_P^{s + 1}$, $a_n u_n^s + \ldots = G(u_1, \ldots, u_n)$, где $G$~--- форма степени  $s$ с коэффициентами из $\fm_P$. 
			\[
				G(u_1, \ldots, u_n) = H_0 u_n^s + H_1 u_n^{s - 1} + \ldots + H_s, \text{ где }
			\]
			$H_i$~--- форма от $x_1, \ldots, x_{n - 1}$ степени $i$. Тогда 
			\[
				(\underbrace{a_n}_{\in \bk} - \underbrace{H_0(u_1, \ldots, u_{n - 1})}_{\in \fm_{P}})u_{n}^s \in (u_1, \ldots, u_{n - 1}) \implies u_n^s \in (u_1, \ldots, u_{n - 1}),
			\]
			так как $(\underbrace{a_n}_{\in \bk} - \underbrace{H_0(u_1, \ldots, u_{n - 1}}_{\in \fm_P}) \in \cO_P^*$. Значит, мы получаем 
			\[
				\fm_P^s \subset (u_1, \ldots, u_{n - 1}) \subset \fm_P,
			\]
			но тогда по лемме~\ref{another_def_of_dim_of_ring} мы имеем $\dim{\cO_P} = n - 1$, что приводит нас к противоречию. 

			\noindent\bf{2)} В общем случае сделаем замену переменных: возьмём
			\[
				G(u_1, \ldots, u_n) = F(\alpha_11 u_1 + \ldots + \alpha_{1n}u_n, \ldots, \alpha_{n 1}u_1 + \ldots + \alpha_{n n} u_n)
			\]
			где $(\alpha_{i j}) \in \mathrm{GL}_n(\bk)$ и 
			\[
				G(0, \ldots, 0, 1) = F(\alpha_{1 n}, \ldots, \alpha_{n n}) \neq 0
			\]
			и таким образом сведём ситуацию к \bf{1)}.
		\end{proof}

		Теперь докажем теорему. Так как нас интересует лишь вопрос единственности, достаточно показать, что у нулевой функции ряд Тейлора также будет нулевым. Пусть $F_0 + F_1 + \ldots $~--- ряд Тейлора для нуля. Тогда, так как функция 0 (очевидно) обнуляется в точке $P$, из определения и леммы мы имеем 
		\[
			F_0(u_1, \ldots, u_n)\in \fm_{P}, \implies F_0 \equiv 0.
		\]

		Пусть теперь $s = 1$. Тогда, так как $F_0 = 0$, отсюда 
		\[
			F_1(u_1, \ldots, u_n) \in \fm_P^2 \implies F_1 \equiv 0
		\]
		по лемме. Продолжая пользоваться леммой мы получаем, что $F_j \equiv 0$. 
	\end{proof}

	Итак, выбор системы образующих $\fm_P = (u_1, \ldots, u_n)$ определяет гомоморфизм 
	\[
		\tau\colon \cO_P  \to  \bk[[x_1, \ldots, x_n]].
	\]

	Естественно задуматься о том, каково его ядро. Нетрудно видеть, что 
	\[
		\tau(f) = 0 \iff f \in \bigcap_{s \in \N} \fm^s,
	\]
	что мотивирует изучить, как устроен идеал справа. Оказывается, в нашем случае он устроен не слишком уж сложно. 

	\begin{theorem} 
		Пусть $A$~--- локальное нётерово кольцо с максимальным идеалом $\fm$. Тогда 
		\[
			\bigcap_{s \in \N} \fm^s = 0.
		\]
	\end{theorem}
	\begin{proof}
		Рассмотрим $\alpha \in \bigcap_{s \in \N} \fm^s = M$.  Тогда $\alpha = F_k(u_1, \ldots, u_n)$, где $F_k$~--- однородный многочлен из $A[x_1, \ldots, x_n]$ степени $k$.

		Так как $A[x_1, \ldots, x_n]$ нётерово, не умаляя общности, мы можем считать, что 
		\[
			(F_1, F_2, \ldots, F_k, \ldots) = (F_1, \ldots, F_s).
		\]
		Но тогда мы получаем, что 
		\[
			F_{s + 1} = G_1 F_1 + \ldots + G_s F_s, \quad G_i \in A[x_1, \ldots, x_n], \deg{G_i} = s + 1 - i
		\]
		и $G_i$ однородные. Тогда, подставляя $u_1, \ldots, u_n$:
		\[
			\alpha = F_{s + 1}(u_1, \ldots, u_n) = \alpha(G_1(u) + \ldots + G_s(u)) \implies \alpha = \alpha a, \ a \in \fm.
		\]
		Но тогда $\fm M = M$ и, применяя лемму Накаямы, мы получаем $M = 0$.
	\end{proof}

	Таким образом, как мы видим, ядро построенного выше гомомрфизма тривиально и $\cO_P \hookrightarrow \bk[[x_1, \ldots, x_n]]$ и отсюда сразу следует целостность кольца $\cO_P$. 

	Отметим также, что из леммы~\ref{taylor_series_lemma}  можно извлечь достаточно полезное следствие. А именно, рассмотрим градуированное кольцо 
	\[
		\bigoplus_{k = 0}^{\infty} \fm_P^k/\fm_P^{k + 1}.
	\]
	Тогда, когда $P$~--- неособая точка, мы можем рассмотреть гомоморфизм 
	\[
		\bk[x_1, \ldots, x_n] \to \sum_{k = 0}^{\infty} \fm_P^k/\fm_P^{k + 1}, \quad x_i \mapsto u_i.
	\]
	Из леммы следует, что это мономорфизм. Сюръективность очевидна. Значит, мы доказали такое следствие. 

	\begin{corollary}
		Пусть $P \in X$~--- неособая точка. Тогда имеет место следующий изоморфизм $\bk$-алгебр: 
		\[
			\bigoplus_{k = 0}^{\infty} \fm_P^k/\fm_P^{k + 1} \cong \bk[x_1, \ldots, x_n]
		\]
	\end{corollary}

	В частности, отсюда следует, что на неособом многообразии градуированная алгебра 
	\[
	 	\bigoplus_{k = 0}^{\infty} \fm_P^k/\fm_P^{k + 1}
	 \] 

	 не зависит от выбора точки $P$ (что вообще говоря неочевидно). 

  \subsection{Локальное кольцо точки на неособой кривой. Индексы ветвления и степень инерции. }

  Рассмотри неособоую проективную кривую $X$ и точку $P \in X$. Мы знаем, что в силу того, что кривая неособая, $\dim{\cO_{P}} = \dim_{\bk}\fm_{P}/\fm_{P}^2$, а из этого условия по лемме Накаямы следует, что идеал $\fm_{P}$ главный. 

  Тогда ясно (на самом деле это было ясно и из общих соображений\footnote{Действительно, если мы рассматриваем к примеру аффинный случай, то $A(X)$ в этом случае Дедекиндово, а любая простая локализация Дедекиндова кольца~--- это кольцо дискретного нормирования. }), что кольцо $\cO_P$ в этом случае~--- дискретно нормированное кольцо и с каждой точкой кривой у нас ассоциировано нормирование $\v_{P}$.

  Пусть теперь $\varphi\colon Y \to X$~--- морфизм неособых кривых и $\varphi^{-1}(P) = \{ Q_1, \ldots, Q_n \}$. Попробуем понять, как же связаны нормирования $\v_{P}$ и $\v_{Q_i}$. Ясно, что в этом случае у нас есть расширение  колец $\cO_{P} \to \cO_{Q_i}$ и $\fm_P \subset \fm_{Q_i}$, $\fm_{P} = \fm_{Q_i} \cap \cO_{Q_i}$. И, видно, что в этой ситуации нормирования $\v_{Q_i}$ \emph{продолжают} нормирование $\v_{P}$. Действительно, мы можем рассмотреть функцию 
  \[ 
        \psi(f) = \v_{Q_i}(\varphi^*(f)) \colon \cO_{P} \to \Z,
  \]
  она уже возможно не будет дискретным нормированием, но $\Im{\psi} \le \Z$~--- подгруппа, обозначим её $e\Z$. Нетрудно проверить, что $e$ не может быть равен нулю.  Отсюда мы получаем, что 
  \[ 
        \v_{Q_i} = e \cdot \v_{P}.
  \]

  Число $e$ в этом контексте называют \emph{индексом ветвления} и обозначают $e = e(\Q_{Q_i}/\cO_{P})$.  Сразу видно, что индекс ветвления можно вычислить вот так: 
  \[ 
        \v_{Q_i}(z_P) = e(\cO_{Q_i}/\cO_{P}),
  \]
  где $(z_P) = \fm_P$~--- локальный параметр для кольца $\cO_{P}$. Или, иными словами, такой элемент, что $\v_P(\pi_P) = 1$. 
  
  
  \emph{Степенью инерции} называют степень расширения полей вычетов $f_i = [\bk_{Q_i} : \bk_{P}]$, где поле вычетов~--- это $\bk_{P} = \cO_P/\fm_P$. 
  
  \begin{remark}
    Более подробно обо всём этом можно прочесть в конспекте 
    \begin{center}
     \url{https://www.overleaf.com/read/khfyxghsbmnn#50fedd}
    \end{center}
  \end{remark}

    Посмотрим теперь, как в общей ситуации для колец нормирования связаны между собой индексы ветвления, степени инерции и степень расширения. Пусть $L/K$~--- конечное сепарабельное расширение, $\v$~--- нормирование на $K$, $\cO_{\v}$~--- кольцо нормирования. 
  
  В коммутативной алгебре у нас была такая теорема:

	\begin{theorem} 
		Пусть $A$~--- Дедекиндово кольцо, $K$~--- его поле частных, $L/K$~--- конечное расширение (полей), а $B = \Int_{L}{A}$. Тогда $B$~--- дедекиндово. 
	\end{theorem}

	В нашей ситуации $A = \cO_{\v}$, а $B = \Int_{L}(A)$. Пусть  $\fm_{\v} = \fp$ и рассмотрим $\fp B$, он раскладывается в произведение простых: 

	\[
		\fp B = \prod_{i = 1}^{n} \fp_{i}^{e_i} 
	\]

	Каждая локализация $B_{\fp_i}$ является дискретно-нормированным кольцом (и, более того, кольцом нормирования $w_i$). Кроме того, все нормирования $w_i$, связанные с идеалами $\fp_i$, продолжают нормирование $\v$ (т.е. $w_i \mid \v$) и, это в точности все нормирования, продолжающие нормирование $\v$.

	В самом деле, если $w$ продолжает $\v$, то $(w \vert v \implies \fm_v \subset \fm_{w} \implies \fm_{v}B \divby \fm_{w})$  $\fm_{w} \cap B$~--- простой идеал, висящий над идеалом $\fp$, то есть, один из $\fp_i$. Пусть $\fm_{w} \cap B = \fp_i$, тогда $\fm_{w} \cap B_{\fp_i} = \fp_i B_{\fp_i}$. Тогда $\fm_{w} \cap \cO_{w_i} = \fm_{w} \cap B_{\fp_i} = \fm_{w_i}$, откуда $w = w_i$.

	Как мы уже отмечали, $B_{\fp_i} /\fp_i B_{\fp_i} = \cO_{w_i}/\fm_{w_i}$. Положим 
	\[
		f_i = [\cO_{w_i}/\fm_{w_i} : A/\fp] = [\Bbbk_{w_i} : \Bbbk_{\v}] 
	\]
	и будем (как и в первой части курса) называть $f_i$ \bf{степенью инерции}.

	Отметим так же, что, как и в случае колец целых, $B$~--- свободный $\cO_{\v}$-модуль ранга $n$ (как и кольцо целых $\cO_{K}$, которое было свободным $\Z$-модулем ранга $n$).


	\begin{theorem} 
		Для индексов ветвления и степеней инерции справедлива следующая формула: 
		\[
			\sum_{i = 1}^{k} e_i f_i = n.
		\]
	\end{theorem}
	\begin{proof}
		Пусть $\fp = \fm_{\v}$. Так как $B$~--- свободный $\cO_{\v}$-модуль ранга $n$, ясно, что что $B/\fp B$~--- векторное пространство над $A/\fp$ размерности $n$. 
		\[
			B/\fp B \cong B/\prod \fp_i^{e_i} = B/\fp_{i}^{e_1} \times  B/\fp_{i}^{e_2} \times \ldots \times  B/\fp_{k}^{e_k}.
		\]
		Вычисляя размерности обеих частей равенства и приравнивая, мы получаем 
		\[
			n = \sum_{i = 1}^{k} \dim_{A/\fm} B/\fp_i^{e_i}.
		\]
		Рассмотрим на кольце $B$ фильтрацию степенями идеалов $\fp_i$:
		\[
			\fp_{i}^{e_i} \subset \fp_i^{e_i - 1} \subset \ldots \subset \fp_i \subset B.
		\]
		Посмотрим на факторы этой фильтрации, то есть, на  $\fp_{i}^{m}/\fp_{i}^{m + 1}$, они являются векторными пространствами над $A/\fp$. Покажем, что $\forall m \ge 1$ $B/\fp_i \cong \fp_i^m / \fp_i^{m + 1}$. Выберем $x \in \fp_i^{m}\setminus \fp_{i}^{m + 1}$ и отображение $m_{x}\colon B/\fp_i \to \fp_i^m/\fp_i^{m + 1}, \ y \mapsto xy$. Вполне ясно, что это корректно определённый гомоморфизм, вычислим его ядро.  Рассмотрим $y \in B$ такой, что $xy \in \fp_i^{m + 1}$ и покажем, что тогда $y \in \fp_i$. Рассмотрим главный идеал $(xy)$. Так как $xy \in \fp_i^{m + 1}$, его разложение на простые имеет вид 
		\[
			(xy) = \fp_i^{m + 1} \cdot I.
		\]
		С другой стороны, $(x) = \fp_i^{m} \cdot J, \ J \notdivby \fp_i$ и тогда  $(y) = \fp_i \cdot \widetilde{J}$, откуда $y \in \fp_i$. Значит, мы показали, что $\ker{m_{x}} = \{ 0\}$. Сюръективность отображения $m_{x}$ следует из того, что $(x) + \fp_{i}^{m + 1} = \fp_{i}^m$. Так как $x \in \fp_i^m$, очевидно, что левая часть лежит в правой. Тогда $(x) + \fp_{i}^{m + 1} =  \fp_i^m \cdot I$, покажем, что $I = (1)$. Предположим противное, тогда 
		\[
			I = \fp_i^{s} \cdot \fq_{1}^{r_1} \cdot \ldots \fq_{\ell}^{r_{\ell}}, \quad \fq_j \neq \fp_i.
		\]

		Тогда $I = \fp_{i}^{s}$, откуда $(x) + \fp_{i}^{m + 1} = \fp_{i}^{m + s}$. Предположим, что $s$ положительно. Тогда $(x) \subset \fp_i^{m + 1}$, что противоречит тому, что мы брали $x \in \fp_i^{m}\setminus \fp_{i}^{m + 1}$. Тогда мы показали, что 
		\[
			B/\fp_i \cong \fp_i^m/\fp_i^{m + 1},
		\]
		и отсюда уже следует теорема: 
		\begin{multline*}
			B/\fp_{i}^{e_i} \cong \frac{B}{\fp_{i}} \cdot \frac{\fp_i}{\fp_{i}^2}  \cdot \ldots \frac{\fp_i^{e_i - 1}}{\fp_{i}^{e_i}} \implies \\ \implies \dim_{A/\fp} B/\fp_i^{e_i} = \dim_{A/\fp}{B/\fp_i} + \dim_{A/\fp}{\fp_i/\fp_i^2} + \ldots + \dim_{A/\fp}{\fp_i^{e_i - 1}/\fp_i^{e_i}} = e_i \cdot \dim_{A/\fp}{B/\fp_i} = e_i \cdot f_i.  
		\end{multline*}
	\end{proof}
  






	
	
	






























	

