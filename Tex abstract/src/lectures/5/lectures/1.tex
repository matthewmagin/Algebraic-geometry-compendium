	\subsection{Введение. Аффинные алгебраические многообразия. }
	
	\bf{Литература:}

	\begin{enumerate}
		\item Хатсхорн, Алгебраическая геометрия. 

		\item Шафаревич, Основы алгебраической геометрии, том \RNum{1}. 
	\end{enumerate}

	Прежде всего отметим, что в данном курсе будет изучаться аффинная алгебраическая геометрия и речь пойдет об \emph{аффинных многообразиях}. 

	Пусть $\Bbbk$~--- фиксированное алгебраически замкнутое поле, через $\AA_{\Bbbk}^n$ мы будем обозначать $\Bbbk^n$, рассматриваемое как множество, и будем называть это множество \emph{$n$-мерным аффинным пространством} над $\Bbbk$. Определим на нём топологию. 

	Рассмотрим $T \subset \Bbbk[x_1, \ldots, x_n]$ и определим 
	\[
		Z(T) \eqdef \left\{ (a_1, \ldots, a_n) \in \AA_{\Bbbk}^n \vert f(a_1, \ldots, a_n) = 0 \quad \forall f \in T \right\}. 
	\]

	\begin{remark}
		В случае $T = \varnothing$ естественно полагать $Z(T) = \AA_{k}^n$. Кроме того, сразу заметим, что $Z((1)) = \varnothing$. 
	\end{remark}

	Пусть $I = (T)$~--- идеал, порождённый множеством $T$. В этом случае $Z(T) = Z(I)$, так как если $(a_1, \ldots, a_n)$ является нулём для всех элементов идеала, то для $T \subset I$ уж тем более, и, кроме того, если $(a_1, \ldots, a_n)$ является общим нулём многочленов из $T$, то $f \in I$ мы можем представить 
	\[
		f = \sum g_i h_i, \quad h_i \in T, \ g_i \in \Bbbk[x_1, \ldots, x_n]
	\]
	и тогда ясно, что $f(a_1, \ldots, a_n) = 0$. 

	Отсюда ясно, что можно рассматривать не произвольные подмножества $k[x_1, \ldots, x_n]$, а идеалы этого кольца. 

	\begin{definition} 
		Введём на $\AA_{\Bbbk}^n$ \emph{топологию Зарисского} следующим образом: объявим замкнутыми все множества вида $Z(T)$ для некоторого $T$. 
	\end{definition}

	Покажем, что это действительно топология. 

	\begin{enumerate}
		\item Ясно, что $Z(T_1) \cup Z(T_2) = Z(T_1 T_2)$, где $T_1 T_2 = \{ f_1 f_2 \ \vert f_1 \in T_2, \ f_2 \in T_2 \}$. 

		Совсем очевидно, что левая часть лежит в правой. Пусть $f_1f_2(a_1, \ldots, a_n) = 0 \ \forall f_1 \in T_1, \ f_2 \in T_2$. Если $f_1 f_2(a_1, \ldots, a_n) = 0$, то хотя бы один многочлен из произведения зануляется, откуда ясно обратное включение. 

		\item Кроме того, ясно, что  $\bigcap_{i \in I} Z(T_i) = Z\lr*{\bigcup_{i \in I} T_i }$.

	\end{enumerate}

	\begin{example}
		Рассмотрим $\AA_{\Bbbk}^1$. Тогда $Z(I) = Z(f)$, то есть любое замкнутое множество~--- просто некоторое конечное множество. С другой стороны, если у нас есть конечное множество $\{ a_1, \ldots, a_n \} \subset \AA_{\Bbbk}^1$, то оно является множеством нулей многочлена 
		\[
			f(x) = \prod_{i = 1}^{n} (x - a_i).
		\]
		Таким образом, замкнутые подмножества аффинной прямой~--- в точности все конечные подмножества $\AA_{\Bbbk}^1$.

		Также отсюда ясно, что любые два открытых подмножества пересекаются и топология Зарисского не хаусдорфова. 
	\end{example}

	\begin{definition} 
		Пусть $X$~--- тополгическое пространство, $Y \subset X, \ Y \neq \varnothing$. $Y$ называется \emph{неприводимым}, если из равенства $Y = Y_1 \cup Y_2$, где $Y_i$ замкнуты в $Y$, следует, что $Y_1 = Y$ или $Y_2 = Y$.

		То есть, неприводимые подмножества~--- в точности те, которые нельзя представить в виде объединения двух замкнутых множеств.
	\end{definition}

	\begin{example}
		Аффинная прямая $\AA_{\Bbbk}^1$ неприводима (просто из соображений мощности). 
	\end{example}

	Теперь докажем такой общетопологический факт:

	\begin{statement} 
		Открытое\footnote{непустое\ldots} подмножество неприводимого~--- неприводимо и плотно. Замыкание неприводимого множества неприводимо. 
	\end{statement}

	\begin{proof}
		Пусть $U \subset Y$~--- открытое подмножество. Покажем, что оно неприводимо. Пусть 
		\[
			U = (U \cap F_1) \cup (U \cap F_2), \quad F_1, F_2 \text{ замкнуты в } X.
		\]

		Тогда $Y = (Y \cap F_1) \cup (Y \cap F_2) \cup (Y \setminus U)$, а так как $Y$~--- неприводимо, $Y$ совпадает с каким-то из этих множеств. Но, если $Y \subset F_1 \implies U \cap F_1 = U$, что мы и хотели. 

		Теперь докажем, что $\overline{U} = Y$. Действтельно, $Y = (Y \setminus U) \cup \overline{U}$ и из неприводимости $Y$ следует, что $\overline{U} = Y$.
	\end{proof}

	\begin{definition} 
		Замкнутые непустые подмножества в $\AA_{\Bbbk}^n$ мы будем называть \emph{аффинными алгебраическими многообразиями}.\footnote{Отметим, что в книге Хартсхорна все аффинные алгебраические многообразия предполагаются неприводимыми. }
	\end{definition}

	\begin{example}
		Рассмотрим $\AA^1 \setminus \{ 0 \}$. Это множество не является замкнытм в нашей топологии. С другой стороны, есть взаимно-однозначное соотвествие между этим множеством и множеством $ \{ (x, y) \in \AA^2 \ \vert \ xy - 1 = 0 \}$, которое является аффинным многообразем. 

		В одну сторону нам надо сделать проекцию графика гиперболы на горизонтальную ось: $(x, y) \mapsto x \in \AA^1 \setminus \{ 0 \}$, а в обратную сторону, мы можем отобразить $\AA^1 \setminus \{ 0 \} \ni x \mapsto (x, x^{-1})$.
	\end{example}

	Соотвественно, понятие аффиного многообразия будет расширяться, когда мы определим морфизмы алгебраических многообразий. 

	Рассмотрим $Y \subset \AA_{\Bbbk}^n$, положим 
	\[
		I(Y) \eqdef \left\{ f \in \Bbbk[x_1, \ldots, x_n] \ \vert \ f(y) = 0 \quad \forall y \in Y \right\}.
	\]

	Совершенно ясно, что $I(Y)$~--- идеал. Кроме того, отметим, что для $Y = \AA_{\Bbbk}^n \ I(Y) = 0,$ а для $Y = \varnothing, \ I(Y) = (1)$. Таким образом, у нас есть отображения 
	\[
		\Bbbk[x_1, \ldots, x_n] \supset T \mapsto Z(T), \quad \AA_{\Bbbk}^n \supset Y \mapsto I(Y).
	\]

	\begin{statement}\label{ag_prop_1} 
		Определённые выше отображения имеют следующие свойства: 
		\begin{enumerate}
		\item $T_1 \subset T_2$, где $T_1, T_2 \subset \Bbbk[x_1, \ldots, x_n], \ Z(T_1) \supset Z(T_2)$.

		\item Если $Y_1 \subset Y_2 \subset \AA_{\Bbbk}^n$, то $I(Y_1) \supset I(Y_2)$.

		\item $I(Y_1 \cup Y_2) = I(Y_1) \cap I(Y_2)$.

		\item Пусть $I \lei \Bbbk[x_1, \ldots, x_n]$, тогда $I(Z(I)) = \sqrt{I}$.

		\item $Y \subset \AA_{\Bbbk}^n \implies Z(I(Y)) = \overline{Y}$.

	\end{enumerate}		
	\end{statement}	
	\begin{proof}
		Перые три пункта очевидны. Четвертый и пятый пункт следуют из (сильной) теоремы Гильберта о нулях: 

		\begin{theorem}[Теорема Гильберта о нулях]
			Пусть $\Bbbk = \Bbbk^{alg}$, $I \lei F[t_{1}, \ldots, t_{n}]$, а $f \in \Bbbk[x_{1}, \ldots, x_{n}]$. Тогда 
			$f(Z(I)) = 0 \Leftrightarrow f \in \sqrt{I}$. Иными словами, $I(Z(I)) = \sqrt{I}$.
		\end{theorem}

		Докажем теперь пятый пункт. Влючение $\overline{Y} \subset Z(I(Y))$ очевидно, так как 
		$Y \subset Z(I(Y))$, а правое множество замкнуто. 

		Пусть $Y \subset Z(\fa)$, тогда $I(Z(\fa)) \subset I(Y) \implies \fa \subset I(Y)$ (так как $I(Z(\fa)) = \sqrt{\fa}$). Тогда мы имеем $Z(I(Y)) \subset Z(\fa)$. 
	\end{proof}

	\begin{statement} 
		Отображения $Y \mapsto I(Y)$ устанавливает взаимно однозначное соотвествие между аффинными многообразиями в $\AA_{\Bbbk}^n$ и радикальными идеалами кольца многочленов $\Bbbk[x_1, \ldots, x_n]$. 

		При этом, неприводимым аффинным многообразиям соотвествуют простые идеалы и наоборот. 
	\end{statement}
	\begin{proof}
		На самом деле, то, что отображения $Z$ и $I$ взаимно обратные, мы уже фактически видели в предложении~\ref{ag_prop_1}.

		Во-первых, очевидно, что $\sqrt{I(Y)} = I(Y)$.  Покажем инъективность: 
		\[
			I(Y_1) = I(Y_2) \implies \overline{Y_1} = Z(I(Y_1)) = Z(I(Y_2)) = \overline{Y_2} = Y_2. 
		\]

		Возьмём теперь некоторый радикальный идеал $I = \sqrt{I}$. Пусть $Y = Z(I)$, тогда $I(Y) = \sqrt{I} = I$. 

		Пусть теперь $Y$ неприводимо. Покажем, что $I(Y)$~--- простой идеал. Рассмотрим $f, g \in \Bbbk[x_1, \ldots, x_n]$, пусть $fg \in I(Y)$. Заметим, что 
		\[
			Y = (Y \cap Z(f)) \cup (Y \cap Z(g)),
		\]
		тогда одно из этих множеств совпадает с $Y$. Например, $Y = Y \cap Z(f) \implies Y \subset Z(f) \implies f \in I(Y)$.

		Наоборот, предположим, что $I(Y)$~--- простой идеал. Пусть $Y = Y_1 \cup Y_2$, тогда 
		\[
			I(Y) = I(Y_1) \cap I(Y_2) \supset I(Y_1) I(Y_2).
		\]
		Так как идеал простой, не умаляя общности, $I(Y_1) \subset I(Y) \implies Y \subset Y_1 \implies Y = Y_1$, что мы и хотели 

	\end{proof}

	Посмотрим, куда при этом соотвествии переходят точки. Пусть $P = (a_1, \ldots, a_n)$.  Ясно, что 
	\[
		I(P) = \{ f \ \vert \ f(a_1, \ldots, a_n) = 0 \} = (x_1 - a_1, \ldots, x_n - a_n ). 
	\]

	Слабая теорема Гильберта о нулях говорит нам, что все максимальные идеалы $\Bbbk[x_1, \ldots, x_n]$ имеют вид $(x_1 - a_1, \ldots, x_n - a_n)$, а значит, мы только что поняли, что есть соответствие 
	\[
		\text{точки } \AA_{\Bbbk}^n \longleftrightarrow \Specm(\Bbbk[x_1, \ldots, x_n]).
	\]

	\begin{definition} 
		Топологическое пространство $X$ называется \emph{Нётеровым}, если оно удовлетворяет $\mathrm{DCC}$ для замкнутых множеств. Иными словами, если у нас есть цепочка 
		\[
			\forall \ Z_0 \supset Z_1 \supset Z_2 \supset \ldots  \ \exists n \colon Z_n = Z_{n + 1}. 
		\]

		Или же, еще более иными словами, в любом семействе замкнутых множеств содержится минимальный (по включению) элемент. 
	\end{definition}

	\begin{example}
		$\AA_{\Bbbk}^n$ является Нётеровым, так как по теореме Гильберта о базисе $\Bbbk[x_1, \ldots, x_n]$~--- Нётерово кольцо. 

		В самом деле, если $Z_0 \supset Z_1 \supset \ldots$, то $I(Z_0) \subset I(Z_1) \subset \ldots$. Так как $\Bbbk[x_1, \ldots, x_n]$~--- Нётерово, $\exists m\colon I(Z_m) = I(Z_{m + 1}) = \ldots$, и, применяя $Z$, мы имеем $Z_m = Z_{m + 1} = \ldots$. 
	\end{example}

	\begin{homework}
		Задачи:
		\begin{enumerate}
			\item Любое подпространство Нётерова пространства Нётерово. 

			\item Нётерово пространство квазикомпактно\footnote{В алгебраической геометрии это означает просто обычную компактность. }. 
			\item 
		\end{enumerate}
	\end{homework}

	\begin{theorem}[] 
		Пусть $X$~--- нётерово пространство, $Y \subset X$~--- замкнутое. Тогда существует единственное разложение $Y = Y_1 \cup Y_2 \cup \ldots \cup Y_m$, где $Y_i$~--- замкнутые нприводимые множества и $\forall i, j \ Y_i \not\subset Y_j$. 
	\end{theorem}
	\begin{proof}
		\emph{Существование.} Пусть сущестуют замкнутые множества $Y$, не разлагающиеся в объединение неприводимых. В силу Нётеровости пространства, мы можем выбрать минимальное множество с таким свойством и обозначить его за $Y$.

		Совершенно ясно, что оно не может быть неприводимым. Тогда его можно представить в виде $Y = T_1 \cup T_2$, где $T_1, T_2$~--- замкнутые и не совпадают с $Y$. Так как $T_1, T_2 \subset Y$, а $Y$~--- минимальное, $T_i$ мы уже можем представить в виде объединения неприводимых, а значит, и $Y$, что даёт нам противоречие. 

		\emph{Единственность.} Пусть $Y = Y_1 \cup Y_2 \cup \ldots \cup Y_m = Y_1' \cup \ldots \cup Y_s'$. Тогда 
		\[
			Y_1 = \bigcup_{i} (Y_1 \cap Y_i') \implies Y_1 \subset Y_i'.
		\]
		Проводя аналогичное рассуждение для $Y_i'$, мы получаем, что $Y_i' \subset Y_j$ для некоторого $j$. Но, так как между компонентами не может быть включений, отсюда $Y_1 = Y_j$.
	\end{proof}


	\begin{definition} 
		Замкныте неприводимые множества $Y_i$, определённые в теореме выше, называют \emph{неприводимыми компонентами $Y$}.
	\end{definition}

	Итак, если $Y = Y_1 \cup \ldots \cup Y_m$, тогда $I(Y) \subset I(Y_i)$, а $I(Y_i)$~--- простые идеалы. Оказывается, что идеалы $I(Y_i)$~--- наименьшие простые идеалы, содержащие $I(Y)$. Действительно, доказать это совсем легко: 

	Пусть $T \subset Y$~--- неприводимое подмножество, тогда
	\[
		T = \bigcup_{i = 1}^{m}(T \cap Y_i) \implies T \subset Y_i.
	\]
	Пусть $Y = Z(I)$, а $Y_i = Z(\fp_i)$, $I \subset \fp_i \in \Spec(\Bbbk[x_1, \ldots, x_n])$. Предположим, что $I \subset \fp \subsetneq \fp_i$. Тогда так как $Z(\fp)$ неприводимо, оно целиком содержится в некоторой компоненте $Z(\fp_j)$. Но тогда $\fp \supset \fp_j$. То есть, мы получили, что $\fp_j \subset \fp \subsetneq \fp_i$, но тогда $Y_i \subset Y_j$, что даёт нам противоречие. 

	Теперь, возьмём произвольный минимальный простой идеал $\fp \supset I$ и покажем, что он даст нам неприводимую компоненту (т.е., что он будет совпадать с одним из $\fp_i$). 

	В самом деле, для некоторого $i$ мы имеем $Z(\fp) \subset Z(\fp_i) \implies \fp_i \subset \fp$, откуда по минимальности $\fp$ мы имеем $\fp_i = \fp$.

	\begin{definition} 
		Пусть $Y$~--- аффинное многообразие. Его \emph{аффинным координатным кольцом} мы будем называть $A(Y) = \Bbbk[x_1, \ldots, x_n]/I$.
	\end{definition}




		
	



	