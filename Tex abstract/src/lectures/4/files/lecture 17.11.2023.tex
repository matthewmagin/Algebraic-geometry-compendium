
\subsection{Геометрия чисел}

	Рассмотрим евклидово пространство $\R^n$, выберем в нём набор из $k$ линейно независимых векторов $e_1, \ldots, e_k$ и рассмотрим порожденную ими свободную абелеву группу:
	\[ 
		L = \bigoplus_{i = 1}^{k} \Z e_i 
	\]
	Тогда $L$ мы будем называть \emph{решёткой}, натянутой на вектора $e_1, \ldots, e_k$. В случае $k = n$ решётка $L$ называется \emph{полной}. 

	\begin{example}
		Картинка для $L \subset \R^2$.
	\end{example}

	\begin{statement} 
		В любом ограниченном подмножестве $\R^n$ лежит конечное число точек решётки. 
	\end{statement}

	\begin{proof}
		В самом деле, можно сделать линейное преобразование, которое переводит произвольную решетку в прямоугольную. Оно будет переводить ограниченное множество в ограниченное, а для прямоугольной решетки необходимое свойство очевидно. 
	\end{proof}

	Оказывается, верно и обратное утверждение. 

	\begin{statement} 
		Пусть $A \le \R^n$~--- подгруппа (как абелевой группы), причем такая, что в любом ограниченном подмножестве $\R^n$ лежит конечное число элементов из $A$. Тогда $A$~--- решётка. 
	\end{statement}

	\begin{proof}
		Рассмотрим подпространство $\Span(A)$ в $\R^n$. Оно прождено некоторым линейно независимым набором векторов: 
		\[
			\Span(A) = \langle e_1, \ldots, e_m \rangle, \ e_i \in A \text{~--- линейно независимы}.
		\]
		Рассмотрим свободную абелеву группу, порожденную этими векторами: 
		\[
			A_0  = \Z e_1 + \ldots + \Z e_m.
		\]
		Покажем, что $A/A_0$ конечна. Расммотрим \emph{фундаментальную область}
		\[
			\Delta = \left\{  \sum_{i = 1}^{m} x_i e_i \ \bigg\vert \ 0 \le x_i < 1 \right\}
		\]
		Ясно, что $\forall a \in A \ \exists a_0 \in A_0\colon a - a_0 \in \Delta$. Но так как $\Delta$ ограничено, в нём может лежать только конечное число элементов решётки, значит количество значений, которые может принимать $a - a_0$ конечно и $A/A_0$ конечна. 

		Значит, $\exists s \in \Z \setminus \{ 0 \}\colon sA \subset A_0$.  Тогда
		\[
			 A \subset \frac{1}{s}A_0 = \Z \frac{e_1}{s} \oplus \Z \frac{e_2}{s} \oplus \ldots \oplus \Z \frac{e_m}{s} = A_1.
		 \]

		  Значит, $A_0 \subset A \subset A_1$, а $A_0$ и $A_1$~--- свободные абелевы группы одного и того же ранга. Значит и $A$~--- свободная абелева группа, 
		\[
		 	A = \Z u_1 \oplus \Z u_2 \oplus \ldots \oplus \Z u_m.
		 \] 

		 Остаётся проверить, что $u_1, \ldots, u_m$ будут линейно независимы над $\R$. Но, это так, потому что 
		 \[
		 	m = \dim\Span(A_0) \le \dim\Span(A_1) = \dim\langle u_1, \ldots, u_m \rangle. 
		 \]
	\end{proof}

	Это предложение даёт хороший критерий для проверки, является ли какое-то подпространство решёткой. 

	\begin{definition} 
		Если $L \le \R^n$~--- решётка с порождающим набором $e_1, \ldots, e_m$, то множество 
		\[
			\Delta = \left\{  \sum_{i = 1}^{m} x_i e_i \ \bigg\vert \ 0 \le x_i < 1 \right\}
		\]
		называют \emph{основным параллелипипедом} решётки или же \emph{фундаментальной областью} решётки. 
	\end{definition}

	Если $e_1, \ldots, e_m$~--- порождающий набор решётки $L$, то $\R^m = \Span\{ e_1, \ldots, e_m \}$ и тогда 
	мы можем вычислить объем фундаментальной области, как
	\[
		\Vol(\Delta) = \det\left\lvert (a_{i j})\right\rvert,  
	\]
	где $e_i = (a_{i 1}, a_{i 2}, \ldots, a_{i n})$.

	\begin{lemma}\label{lattice_prop_volume}
		Пусть $T$~--- ограниченное измеримое множество в $\R^m$, $L$~--- решётка ранга $m$ в $\R^m$, $\Delta$~--- её фундаментальная область. Предположим, что $\forall \ell_1, \ell_2 \in L $ множества $T + \ell_1$ и $T + \ell_2$ не пересекаются.  Тогда 
		\[
			\Vol(T) \le \Vol(\Delta).
		\]
	\end{lemma}
	\begin{proof}
		В самом деле, так как множества  $T + \ell$ дизъюнктны, 
		\[
			\Vol(\Delta) \ge \sum_{\ell \in L} \Vol(\Delta \cap T_{\ell}) = \sum_{\ell \in L} \Vol(\Delta_{-\ell} \cap T) = \Vol\lr*{\ \bigcup_{\ell \in L} \Delta_{\ell} \cap T} = \Vol(\R^m \cap T) = \Vol(T).
		\]
	\end{proof}

	\begin{lemma}[Г. Минковский, О выпуклом теле]\label{Mink_theorem}
		Пусть $T$~--- ограниченное выпуклое центрально-симметричное (относительно нуля) измеримое подмножество $\R^n$, $L$~--- решётка ранга $n$ в $\R^n$, $\Delta$~--- её фундаментальная область. Предположим, что выполнена следующая оценка на объемы:
		\[
			\Vol(T) > 2^n \Vol(\Delta).
		\]
		Тогда $\exists 0 \neq \ell \in L\colon \ell \in T$. Кроме того, если $T$ компактно, то это будет верно и в случае нестрогого неравенства $\Vol(T) \ge 2^n \Vol(\Delta)$.
	\end{lemma}
	\begin{proof}
		Рассмотрим тело $\frac{1}{2}T$, тогда $\Vol\lr*{\frac{1}{2}T} = \frac{1}{2^n} \Vol(T) > \Vol(\Delta)$. Тогда по предыдущей лемме~\ref{lattice_prop_volume}  $\exists \ell_1, \ell_2 \in L\colon \frac{1}{2}T_{\ell_1} \cap \frac{1}{2}T_{\ell_{2}} \neq \varnothing$. Это означает, что
		 \[ \exists t_1, t_2 \in T\colon \frac{x_1}{2} + \ell_1 = \frac{x_2}{2} + \ell_2 \implies 0 \neq \frac{x_1 - x_2}{2} = \ell_1 - \ell_2 \in L \cap T.\]

		 В последнем равенстве $\frac{x_1 - x_2}{2} \in T$ так как $T$~--- выпукло и центрально симметрично. 

		 Теперь докажем вторую часть теоремы. Рассмотрим $T_{\varepsilon} = (1 + \varepsilon)T$, для него неравенство уже будет строгим и по первой части теоремы мы получим $0\neq \ell \in L \cap T_{\varepsilon}$. Понятно, что если $\ell \in T$, всё доказано. Пусть теперь $\ell \in T_{\varepsilon} \setminus T$. Вообще говоря, в $T_{\varepsilon} \setminus T$  лежит лишь конечное число точек из $L$. Так как $T_{\varepsilon}$ замкнуто, мы можем уменьшить $\varepsilon$ так, чтоб все точки из $\ell$, лежащие в $T_{\varepsilon}\setminus T$ уже не лежали там. 
	\end{proof}

	\begin{definition}\hypertarget{real_and_complex_inclusions}{}
		Рассмотрим конечное расширение $K/\Q$, $[K : \Q] = n$. Тогда у нас есть $n$ вложений $\sigma_i \colon K \to \Q^{alg}$. Среди них есть \emph{вещественные} вложения, то есть такие, что $\Im(\sigma_i) \subset \R$. Остальные вложения  называют \emph{комплексными} (или, \emph{невещественными}).
	\end{definition}

	 С каждым невещественным вложением $\sigma_i$ связано вложение $\overline{\sigma_i} \neq \sigma_i$. Пронумеруем наши вложения следующим образом:
	  \[
	   \sigma_1, \ldots, \sigma_s \text{~--- вещественные вложения, } \sigma_{s + 1}, \overline{\sigma_{s + 1}}, \sigma_{s + 2}, \overline{\sigma_{s + 2}}, \ldots, \sigma_{s + t}, \overline{\sigma_{s + t}} \text{~--- комплексные вложения.} 
	  \] 
	   Так как количество вложений равно степени расширения, $s + 2t = n$.

	Рассмотрим отображение $\varphi\colon K \to \R^n$, которое действует так:
	\[
		\alpha \in K, \ \alpha \mapsto (\sigma_1(\alpha), \sigma_2(\alpha), \ldots, \sigma_s(\alpha), \Re(\sigma_{s + 1}(\alpha)), \Im(\sigma_{s + 1}(\alpha)),  \ldots, \Re(\sigma_{s + t}(\alpha)), \Im(\sigma_{s + t}(\alpha))) \in \R^n.
	\]
	\hypertarget{phi_from_this}{}

	Пусть $I$~--- ненулевой идеал в $\cO_{K}$. Возьмём его базис как абелевой группы~--- $\alpha_1, \ldots, \alpha_n$. Тогда $\varphi(I)$~--- решётка в $\R^n$ с базисом $\varphi(\alpha_1), \ldots, \varphi(\alpha_n)$. Нужно проверить равзе что линейную независимость. 

	Пусть $\sigma_{s + k}(\alpha) = a + bi$, будем делать слудеющие преобразования:
	\begin{multline*}
	 	(\Re\sigma_{s+ k}\alpha, \Im\sigma_{s + k}\alpha) = (a, b) \mapsto (a, bi) \mapsto (a + bi, bi) \mapsto (a + bi, 2bi) \mapsto (a + bi, -a + bi) \mapsto \\ \mapsto (a + bi, a - bi) = (\sigma_{s + k}\alpha, \overline{\sigma_{s+k}\alpha}).
	 \end{multline*}
	 Посмотрим, что будет происходить с определителем при проделывании этих операций. Нетрудно проследить, что по итогу определитель умножится на $-2i$. В итоге мы получим, что 
	\[
		\det\lr*{\begin{pmatrix} \varphi(\alpha_1) \\ \varphi(\alpha_2) \\ \vdots \\ \varphi(\alpha_n) \end{pmatrix}} = \frac{1}{(2i)^t} \det{\begin{pmatrix} \sigma_{1}(\alpha_1) & \ldots& \sigma_{s + 1}(\alpha_1) & \overline{\sigma_{s + 1}(\alpha_1)} & \ldots &  \overline{\sigma_{s + t}(\alpha_1)} \\ \vdots & \vdots & \vdots & \vdots & \vdots & \vdots & \vdots \\
		\sigma_{1}(\alpha_n) & \ldots& \sigma_{s + 1}(\alpha_n) & \overline{\sigma_{s + 1}(\alpha_n)} & \ldots &  \overline{\sigma_{s + t}(\alpha_n)} \end{pmatrix}} = \pm \frac{1}{(2i)^t} \sqrt{|\disc(\alpha_1, \ldots, \alpha_n)|}.
	\]

	А теперь заметим, что левая часть~--- объем фундаментальной области, а правую мы можем переписать в терминах $\disc(K)$, пользуясь предложением~\ref{dic_and_ind}
	\[
		\Vol(\Delta) = \left\lvert \det\lr*{\begin{pmatrix} \varphi(\alpha_1) \\ \varphi(\alpha_2) \\ \vdots \\ \varphi(\alpha_n) \end{pmatrix}} \right\rvert
 = \frac{1}{2^t} \sqrt{|\disc(\alpha_1, \ldots, \alpha_n)|} = \frac{1}{2^t} \sqrt{|\disc(K)| \cdot [\cO_{K} : I]^2} = \frac{1}{2^t} \Nm(I) \sqrt{|\disc(K)|}.	
 	\]

 	Рассмотрим теперь для некоторого фиксированного $a > 0$.
 	\[
 		T = \left\{ (x_1, x_2, \ldots, x_{s}, y_{1}, z_{1}, \ldots, y_{t}, z_{t})\vert \ |x_1| + \ldots + |x_s| + 2\sqrt{y_1^2 + z_1^2} + \ldots + 2\sqrt{y_t^2 + z_t^2} \le a \right\}. 
 	\]

 	$T$~--- выпуклое, центрально-симметричное и $\Vol(T) = 2^s \lr*{\frac{\pi}{2}}^t \frac{a^n}{n!}$.
 	Подберём $a$ так, что для $0 \neq I \subset \cO_{K}$ будет выполнено неравенство 
 	\begin{equation}
 		2^s \lr*{\frac{\pi}{2}}^t \frac{a^n}{n!} > 2^n \frac{\sqrt{|\disc(K)|}}{2^t} \Nm(I). \label{calculating_a}
 	\end{equation}

 	Тогда по лемме Минковского~\ref{Mink_theorem} $\exists 0 \neq \alpha \in I\colon$
 	\[
 	 	|\sigma_1(\alpha)| + \ldots + |\sigma_{s}(\alpha)| + 2 |\sigma_{s + 1}(\alpha)| + \ldots +  2|\sigma_{s + t}(\alpha)| \le a.
 	 \] 
 	 Это неравенство мы можем переписать в виде: 
 	 \[
 	 	|\sigma_1(\alpha)| + \ldots + |\sigma_{s}(\alpha)| + |\sigma_{s + 1}(\alpha)| + |\overline{\sigma_{s + 1}}(\alpha)| + \ldots +  |\sigma_{s + t}(\alpha)| + |\overline{\sigma_{s + t}}(\alpha)| \le a.
 	 \]

 	  Тогда по неравенству о средних мы имеем 
 	  \[
 	  	\left\lvert \sigma_{1}(\alpha) \cdot \ldots \cdot \sigma_{s}(\alpha) \sigma_{s + 1}(\alpha) \overline{\sigma_{s + 1}(\alpha)} \cdot \ldots \cdot \sigma_{s + t}(\alpha) \overline{\sigma_{s + t}}(\alpha) \right\rvert \le \lr*{\frac{a}{n}}^n \Leftrightarrow \Nm(\alpha) \le \lr*{\frac{a}{n}}^n
 	  \]

 	  Заметим, что в неравенстве~\eqref{calculating_a} равенство будет достигаться при 
 	  \[
 	  	a^n = \frac{2^n \sqrt{|\disc(K)|} \Nm(I) n!}{2^t 2^s} \cdot \lr*{\frac{2}{\pi}}^t = \lr*{\frac{4}{\pi}}^{t} n! \Nm(\alpha) \sqrt{|\disc{(K)}|}.
 	  \]
 	  В этом случае будет выполняться неравенство
 	  \[
 	  		\Nm(\alpha) \le \lr*{\frac{a}{n}}^n = \lr*{\frac{4}{\pi}}^t \frac{n!}{n^n} \Nm(I) \sqrt{\disc{K}}.
 	  \]

 	  \begin{remark}
 	  	Подставим в этом неравенство, например, единичный идеал. Тогда мы получим неравенство 
 	  \[
 	  	1 \le \lr*{\frac{4}{\pi}}^t \frac{n!}{n^n} \sqrt{\disc{K}}.
 	  \]

 	  Например, из этого неравенства следует, что $\disc(K) \neq 1$ (в случае нетривиального расширения).  

 	  \begin{theorem} 
 	  	Пусть $K/\Q$~--- конечное расширение, $[K : \Q] = n > 1$. Тогда $\disc(K) \neq 1$. Кроме того, 
 	  	\[
 	  	 	\lim\limits_{n \to \infty} \disc(K) = \infty.
 	  	 \] 
 	  \end{theorem}

 	  \end{remark}
 	  
 	  Получим теперь при помощи новых методов некоторые результаты, связанные с группой классов идеалов числового поля. 

 	  Мы доказали, что для ненулевого идеала $I$ в $\cO_{K}$ существует $\alpha \in I\colon$
 	  \[
 	  		\Nm(\alpha) \le \lr*{\frac{4}{\pi}}^t \frac{n!}{n^n} \Nm(I) \sqrt{\disc{K}}.
 	  \]

 	  Возьмём класс $[J] \in \Class(K)$, для него существует целый идеал $I$ такой, что  $[J] = [I^{-1}] \in \Class(K)$.Пусть $\alpha \in I$. С одной стороны, $\alpha I^{-1}$ представляет тот же класс в группе классов, а с другой стороны, 
 	  \[
 	  	|\Nm(\alpha)| \le \lr*{\frac{4}{\pi}}^{t} \cdot \frac{n!}{n^n} \Nm(I) \sqrt{|\disc(K)|}. 
 	  \]
 	  \[
 	  	\Nm(\underbrace{\alpha I^{-1}}_{\text{целый}}) = \Nm(\alpha) \Nm\lr*{I^{-1}} \le \lr*{\frac{4}{\pi}}^{t} \frac{n!}{n^n} \sqrt{|\disc(K)|}.
 	  \]

 	  Но, как мы уже замечали, существует лишь конечное число целых идеалов с ограниченной нормой. Значит, мы получили еще одно (геометрическое) доказательство теоремы~\ref{Cl(K)_is_finite}.

 	  \begin{example}\label{Cl(Q(sqrt[3]{6}))}
 	  	Рассмотрим $K = \Q\lr*{\sqrt[3]{6}}$. Посмотрим сначала на количество вложений. Нетрудно убедиться в том, что $n = 3, t = 1$. Сосчитаем теперь дискриминант $K$. Для этого покажем, что $\cO_{K} = \Z[\sqrt[3]{6}]$. Обозначим $\theta = \sqrt[3]{6}$ и посчитаем $[\cO_{K} : \Z[\theta]]$.
 	  Заметим, что минимальный многочлен $\theta$~--- это $x^3 - 6$, а минимальный многочлен $\theta^2$~--- это $x^2 - 36$, а тогда 

 	  \[
 	  	\disc(K) \cdot \lr*{\ind(\theta)}^2 = \disc(1, \theta, \theta^2) = \det\begin{pmatrix} \Tr(1) & \Tr(\theta) & \Tr(\theta^2) \\ \Tr(\theta) & \Tr(\theta^2) & \Tr(\theta^3) \\ \Tr(\theta^2) & \Tr(\theta^3) & \Tr(\theta^4) \end{pmatrix} = \det\begin{pmatrix} 3 & 0 & 0 \\ 0 & 0 & 18 \\ 0 & 18 & 0 \end{pmatrix} = - 2^{2} 3^{5}.
 	  \]

 	  Заметим, что многочлен $x^3 - 6$ является многочленом Эйзенштейна относительно и двойки и тройки, а значит, $\ind(\theta)$ не может делиться на $2$ и $3$, откуда $\ind(\theta) = 1$. Значит, $\disc(K) = -2^5 \cdot 3^5$. Подставим это в полученное выше неравенство: 
 	  \[
 	  	\Nm(\underbrace{\alpha I^{-1}}_{\text{целый}}) = \Nm(\alpha) \Nm\lr*{I^{-1}} \le \lr*{\frac{4}{\pi}}^{t} \frac{n!}{n^n} \sqrt{|\disc(K)|} = \frac{4}{\pi} \cdot \frac{6}{27} \cdot \sqrt{2^5 \cdot 3^5} < \frac{16}{\sqrt{3}} < 10. 
 	  \]

 	  То есть, в любом классе есть целый представитель, норма которого меньше 10.  Значит, чтоб показать, что группа классов тривиальна, нам достаточно показать, что любой идеал, висящий над 2, 3, 5, 7~--- главный. 

 	  Из теоремы Куммера~\ref{Kummer_theorem} легко понять, что 
 	  \[
 	    	2\cO_{K} = (2, \theta)^3 = (\theta - 2)^3, 
 	    \]  
 	    так как $\Nm(\theta - 2) = -2 \implies (\theta - 2)$~--- простой и он тоже висит над двойкой, откуда  $(2, \theta) = (\theta - 2)$. 

 	  Аналогичное явление будет и с тройкой: 
 	  \[
 	    	3\cO_{K} = (3, \theta)^3 = \lr*{\frac{\theta}{\theta - 2}}^3 = \lr*{\frac{6 + 2\theta^2 + 4\theta}{-2}}^3 = (3 + \theta^2 + 2\theta)^3.
 	    \]  
 	    Действительно, $\Nm(\theta) = 6, \ \Nm(\theta - 2) = -2$, откуда $\Nm\lr*{\frac{\theta}{\theta - 2}} = -3$, то есть $\frac{\theta}{\theta - 2}$~--- простой, висящий над тройкой, $(3, \theta) = \lr*{\frac{\theta}{\theta - 2}}$. 

 	    Теперь, опять же из теоремы Куммера~\ref{Kummer_theorem}, мы получаем, что 
 	    \[
 	     	5\cO_{K} = (5, \theta - 1)(5, \theta^2 + \theta + 1)^2
 	     \] 
 	     и надо показать, что эти идеалы главные. В самом деле, $\Nm(\theta - 1) = 5$ и при этом 
 	     \[
 	     	\frac{5}{\theta - 1} = \theta^2 + \theta + 1, 
 	     \]
 	     откуда $(5, \theta^2 + \theta + 1) = (\theta^2 + \theta + 1)$, \ $(5, \theta - 1) = (\theta - 1)$.  Теперь, делаем то же самое над семёркой: 
 	     \[
 	     	x^3 - 6 = x^3 + 1 = (x + 1)(x^2 - x + 1) = (x + 1)(x - 3)(x - 5) \text{ в } \F_{7}[x].
 	     \]
 	     Тогда семёрка будет раскладываться, как 
 	     \[
 	     	7\cO_{K} = (7, \theta + 1) (7, \theta - 3)(7, \theta - 5).
 	     \]

 	     Во-первых, заметим, что 
 	     \[
			\frac{7}{\theta + 1} = \frac{7(\theta^2 - \theta + 1)}{\theta^3 + 1} \implies (7, \theta + 1) = (\theta + 1). 	     	
 	     \]
 	     \[
 	     	\Nm(\theta - 3) = 21, \Nm\lr*{\frac{\theta}{\theta - 2}} = -2 \implies \Nm\lr*{\frac{(\theta - 3)(\theta - 2)}{\theta}} = -7
 	     \]
			Этот элемент даёт нам максимальный главный идеал, висящий над семёркой. Нетрудно заметить, что 
			\[ 
				(7, \theta - 3) \subset \lr*{\frac{(\theta - 3)(\theta - 2)}{\theta}} 
			\]

			а так как оба идеала максимальные, они совпадают. Значит и третий идеал главный (так как произведение трех главных идеалов равно главному идеалу). 

 	     Таким образом, мы показали, что любой идеал кольца $\cO_{K}$ является главным, то есть, что $\Cl(\Q(\sqrt[3]{6})) = 0$.
 	  \end{example}

 	  \begin{homework}\label{hw:10}
 	  	Задачи: 
 	  	\begin{enumerate}
 	  		\item Докажите, что для любого простого $p$ уравнение  
 	  		\[
 	  			3x^2 + 4y^3 + 5z^3 = 0
 	  		\]
 	  		имеет ненулевое решение над $\F_{p}$.
 	  		\item Докажите, что $1 - 6\theta + 3\theta^2 \in \cO_{K}^{*}$, где $K = \Q(\sqrt[3]{6})$.

 	  		\item Докажите, что $\Cl(\Q(\sqrt{-23})) = \Z/3\Z$.
 	  	\end{enumerate}
 	  \end{homework}












