	
	\subsection{Мультипликативная группа кольца целых числового поля}

	Пусть числа $s$ и $t$, связанные с количеством вложений числового поля $K \to \Q^{alg}$ определены как в \hyperlink{real_and_complex_inclusions}{}. В этом параграфе мы докажем, что мультипликативная группа кольца целых числового поля имеет вид 
	\[
		\cO_{K}^{*} \cong \mu \oplus \Z^{s + t - 1},
	\]
	где $\mu$~--- группа корней из единицы. Этот факт будет иметь множество приложений. 
	Этот факт обычно называют \emph{сильной формой теоремы Дирихле о единицах}.
	

	\begin{example}
		Рассмотрим квадратичное расширение $K = \Q(\sqrt{d})$. Если $\Q(\zeta_n) \subset \in \Q(\sqrt{d})$, то $[\Q(\zeta_n) : \Q] \le 2$, но с другой стороны $[\Q(\zeta_n) : \Q] = \varphi(n)$, откуда $n = 1, 2, 3, 4, 6$.  Если $n = 3$, то $d = -3$, если $n = 4$, то $d = -1$, если $n = 6$, то $d = -3$, но $\Q(\zeta_6) = \Q(\zeta_3)$, так как $\zeta_6 = -\zeta_3$, а в остальных случаях нетривиальных корней из единицы в этом поле нет. 

		Пусть $s$ и $t$ опредедлены как \hyperlink{real_and_complex_inclusions}{тут}. Соотвественно, если $d > 0$, то $s = 2, t = 0 \implies s + t - 1 = 1 \implies \cO_{K}^{*} = \{ \pm \theta \ \vert \ m \in \Z\}$.

		Если $d > 0$ и $d \not\equiv 1 \pmod{4} \implies \cO_{K} = \Z[\sqrt{d}]$. Как мы помним, $u \in \cO_{K}^{*} \Leftrightarrow \Nm_{K/\Q}(u) = \pm 1$. В нашем случае 
		\begin{equation}
			\Nm(x + y\sqrt{d}) = x^2 - dy^2 = \pm 1 \label{Pell_equation}
		\end{equation}
		и из другого описания $\cO_{K}^{*}$ мы получаем, что все решения уравнения~\eqref{Pell_equation} имеют вид $\{ \pm \theta^{m} \ \vert \ m \in \Z \}$. Из этого, например, следует, что решений уравнения~\eqref{Pell_equation} бесконечно много. 

		Вообще говоря, этот самый элемент $\theta = \theta_{d}$ может иметь очень неприятный вид. Например, $\theta_2 = 1 + \sqrt{2}, \ \theta_3 = 2 + \sqrt{3}, \ \theta_{94} = 2143295 + 221064\sqrt{94}$.

		Если же $d < 0$, то вполне ясно, что $s = 0, \ t = 1 \implies s + t - 1 = 0$, откуда следует, что $\cO_{K}^{*} = \mu$, откуда, в частности, следует, что уравнение~\eqref{Pell_equation} имеет конечно число решение.
	\end{example}

	\begin{theorem}[Дирихле, о единицах, \emph{слабая форма}]\label{Weak_dirichlet_theorem} 
		Мультипликативная группа кольца целых $\cO_{K}$ числового поля $K$ имеет вид 
		\[
			\cO_{K}^* \cong \mu \oplus \Z^m, \text{ где } m \le s + t - 1.
		\]
	\end{theorem}
	\begin{proof}
		Рассмотрим отображение $\ell\colon K^* \to \R^{s + t}$, действующее следующим  образом 
		\[
			\ell(\alpha) = \lr*{\log|\sigma_1\alpha|, \log|\sigma_2\alpha|, \ldots, \log|\sigma_s\alpha|, \log|\sigma_{s + 1}\alpha|^2, \ldots, \log|\sigma_{s + t}\alpha|^2}.
		\]
		Заметим, что это гомомомрфизм групп, $\ell(\alpha\beta) = \ell(\alpha) + \ell(\beta)$.
		Рассмотрим сужение $\ell\colon \cO_{K}^* \to \R^{s + t}$ (чтоб не перегружать обозначения, с этого момента мы называем сужение той же буквой $\ell$). Посчитаем ядро этого отображения: 
		\[
			\alpha \in \Ker{\ell} \Leftrightarrow \forall i = 1, \ldots, s + t \quad |\sigma_i \alpha| = 1 \underbrace{\implies}_{\text{\bf{Л.}~\ref{lemma:14}}} \Ker{\ell} = \mu.
		\]
		Теперь посчитаем $\Im{\ell}, \ \ell \colon \cO_{K}^{*} \to \R$. Пусть $\alpha \in \cO_{K}^{*}$, тогда мы знаем, что $\Nm(\alpha) = \pm 1$, откуда
		\[
			\log{|\Nm(\alpha)|} = \log|\sigma_1\alpha| + \ldots + \log{|\sigma_{s + 1}\alpha|} + \log{|\overline{\sigma}_{s + 1}\alpha|} + \ldots  = 0,
		\]
		что даёт нам, что образы всех обратимых элементов лежат в гиперплоскости 
		\[
			x_1 + x_2 + \ldots + x_{s + t} = 0.
		\]

		\begin{lemma} 
			$\Im{\ell}$~--- решётка в $\R^{s + t}$.
		\end{lemma}
		\begin{proof}
			Надо проверить, что $\Im{\ell}$~--- это дискретная подгруппа в $\R^{s +t}$ (то, что это подгруппа~--- очевидно). Иными словами, нам надо показать, что в любом ограниченном множестве содержится конечное число точек из $\Im{\ell}$. Ясно, что это достаточно проверять для шаров, рассмотрим шар $\overline{B}_{r}(0)$.  

			Ясно, что неравенства
			\[
				\log|\sigma_{j}\alpha| \le r \ \forall j \Leftrightarrow |\sigma_{j}\alpha| \le e^{r} \ \forall j = 1, \ldots, s, \quad |\sigma_{j}\alpha| < e^{\frac{r}{2}} \ \forall j = s + 1,\ldots, s + t.
			\]
			Нетрудно заметить, что из этих неравенств следуют неравенства на мнимую и вещественную часть координат $s + 1, \ldots, s + t$, то есть мы имеем
			\[
				|\sigma_{j}\alpha| \le e^{r} \ \forall j = 1, \ldots, s, \quad |\Im\sigma_{j}\alpha| \le  e^{\frac{r}{2}}, \ |\Re\sigma_{j}\alpha| \le  e^{\frac{r}{2}} \ \forall j = s + 1,\ldots, s + t.
			\]
			Но, в предыдущем параграфе мы уже рассматривали отображение $\varphi$
			\[	  	
				\varphi(\alpha) = (\sigma_1(\alpha), \sigma_2(\alpha), \ldots, \sigma_s(\alpha), \Re(\sigma_{s + 1}(\alpha)), \Im(\sigma_{s + 1}(\alpha)),  \ldots, \Re(\sigma_{s + t}(\alpha)), \Im(\sigma_{s + t}(\alpha))) \in \R^n,
			\]
			и доказывали, что $\Im{\varphi}$~--- решётка. Тогда по предложению~\ref{finite_points_of_lattice} в шаре $\overline{B}_{e^{r}}$  лежит лишь конечное число точек из образа $\varphi$, но тогда, по отмеченному выше, там будет лежать лишь конечное число точек из $\Im{\ell}$, а тогда по предложению~\ref{discrete_subgroup_is_lattice} $\Im{\ell}$~--- решётка. 
		\end{proof}

		Значит, $\Im{\ell}$ порождён $m$ линейно-независимыми в $\R^{s + t}$ векторами и  $\Im{\ell} \cong \Z^{m}$. С другой стороны, так как $\Im{\ell}$ лежит в гиперплоскости, $m \le s + t - 1$.  Тогда у нас есть короткая точная последовательность 
		\[
		 	0 \to \Ker{\ell} \hookrightarrow \cO_{K}^{*} \xrightarrow{\ell} \Im{\ell} \to 0.
		 \] 
		 Как мы уже выяснили выше, $\Ker{\ell} \cong \mu,$ а $\Im{\ell} \cong \Z^{m}, m \le s + t - 1$, а значит 
		 \[
		 	0 \to \mu \hookrightarrow \cO_{K}^{*}  \xrightarrow{\ell} \Z^{m} \to 0, \quad m \le s + t - 1. 
		 \]
		 откуда $\cO_{K}^{*} \cong \mu \oplus \Z^{m}, \quad m \le s + t - 1$. 
	\end{proof}

	Даже слабая форма теоремы Дирихле о единицах позволяет успешно вычислять мультипликативные группы колец целых числовых полей. 

	\begin{example}
		Из~\ref{hw:10} мы знаем, что в $K = \Q(\theta)$, где $\theta^3 = 6$ мы имеем
		\[ \frac{1}{1 - 6 \theta + 3\theta^2} = 109 + 60\theta + 33\theta^2 \]

		В данном случае $s = 1, t = 1 \implies s + t - 1 = 1$, откуда $m = 1$, так как ясно, что $m \le 1$, так как если $m = 0$, то никаких обратимых элементов, кроме $\mu$ в $\cO_{K}$ нет, а из корней из единицы в этом кольце есть только $\pm 1$, так как  если $\Q(\zeta_n) \subset \Q(\sqrt[3]{6})$, то $\varphi(n) = 2$. Таким образом, мы имеем 
		\[
			\cO_{K}^* \cong \mu_{2} \oplus \Z.
		\]
	\end{example}

	\begin{statement} 
		Элемент $\varepsilon = 1 - 6\theta + 3\theta^2$~--- основная единица в $\cO_{K}^*$.
	\end{statement}
		
	\begin{proof}
		Во-первых, как мы уже отметили выше, этот элемент обратим и его обратный~--- это $109 + 60\theta + 33\theta^2$. 

		Предположим, что основная единица~--- это $\alpha$, тогда, так как $\mu_2 = \{ \pm 1 \}$ 
		\[
			1 - 6\theta + 3 \theta^2 = \pm \alpha^d, \quad d \ge 2.
		\]
		Пусть $\omega^3 = 1$, тогда все вложения $K \to \Q^{alg}$~--- это 
		\[
			\theta \mapsto \theta, \quad \theta \mapsto \omega \cdot \theta, \quad \omega^2 \cdot \theta.
		\]
		Рассмотрим $\varepsilon'$ и $\varepsilon''$~--- образы $\varepsilon$ при комплексных вложениях, 
		\[
			\varepsilon' = 1 - 6\theta \omega + 3 \theta^2 \omega^2, \quad \varepsilon'' = 1 - 6\theta \omega^2 + 3\theta^2 \omega, \ \varepsilon' = \overline{\varepsilon''}. 
		\]

		Заметим, что тогда, по определению нормы:
		\[
			\pm 1 = \Nm(\varepsilon) = \varepsilon \cdot \varepsilon' \cdot \varepsilon''.
		\]
		\[
			\begin{cases} |\varepsilon'| = |\varepsilon''| \\ |\varepsilon \varepsilon' \varepsilon''| = 1  \end{cases} \implies |\varepsilon'| = |\varepsilon''| = \sqrt{|\varepsilon|^{-1}} = \sqrt{|109 + 60\theta + 33\theta^2|} < \sqrt{109 + 60 \cdot 2 + 33 \cdot 4} = \sqrt{361} = \sqrt{19}.
		\]
		Запишем $\alpha$ в виде $\alpha = x + y\theta + z\theta^2, \ x, y, z \in \Z$ и рассмотрим его образы при комплексных вложениях: 
		\[
			\alpha' = x + y\theta\omega + z\theta^2\omega^2, \quad \alpha'' = x + y\theta \omega^2 + z \theta^2 \omega. 
		\]
		Так как $1 + \omega + \omega^2 = 0$, мы имеем 
		\[
			y\theta = \frac{\alpha + \alpha''\omega + \alpha'\omega^2}{3}, \quad z \theta^2 = \frac{\alpha + \alpha' \omega + \alpha'' \omega^2}{3}.
		\]
		Теперь заметим, что $|\varepsilon| = |\alpha|^d$, откуда $|\alpha| = |\varepsilon|^{\frac{1}{d}}$. Тогда 
		\[
			|\alpha'| \le |\varepsilon'|^{\frac{1}{d}} \le \sqrt{|\varepsilon'|} < \sqrt{19}, \quad |\alpha''| \le |\varepsilon''|^{\frac{1}{d}} \le \sqrt{|\varepsilon''|} < \sqrt{19},
		\]
		так как $d \ge 2$. Оценим теперь $|y|$. Заметим, что $|\alpha| = |\varepsilon|^{\frac{1}{d}} < 1$\footnote{в том, что $|\varepsilon| < 1$ легко убедиться непосредственно.}, откуда 
		\[
			|y| = \frac{|\alpha + \alpha'' \omega + \alpha' \omega^2|}{3|\theta|} \le \frac{|\alpha + \sqrt{19}|}{3|\theta|} < \frac{1 + \sqrt{19}}{3|\theta|} < 2.
		\]

		Абсолютно аналогичным образом мы получаем, что 
		\[
			|z| < \frac{9,8}{3|\theta|^2} < 1.
		\]

		Так как $z \in \Z$, отсюда следует, что $z = 0$. Значит, $\alpha = x + \theta y$, $|y| < 2$. Как мы помним, 
		\[
			\pm 1 = \Nm(\alpha) = x^3 + 6y^3 \implies x^3 \equiv \pm 1 \pmod{3} \implies x \equiv \pm 1 \pod{3} \equiv x^3 \equiv \pm 1 \pmod{9},
		\]
		а тогда $y \divby 9$ и отсюда $y = 0$, но тогда $\alpha = x \in \Z$, что даёт нам противоречие. 
	\end{proof}

	\begin{homework}
	\begin{itemize}
		\item Рассмотрим $K = \Q(\zeta_{5})$. Докажите, что $\cO_{K}$~--- евклидово. 

		\item Приведите пример неизоморфных расширений $K_1, K_2$ над $\Q$ одинаковой степени и таких, что $\disc(K_1) = \disc(K_2)$.

		Рассмотрим $K_1 = \Q(\theta), \ \theta^3 - 18\theta - 6 = 0$, $K_2 = K(\xi), \ \xi^3 - 36\xi - 78 = 0$, $K_3 = \theta^3 - 54\theta - 150 = 0$.

		\item \textcolor{red}{Тут была еще задача, её надо с фотки переписать.} 
	\end{itemize}
	\end{homework}




