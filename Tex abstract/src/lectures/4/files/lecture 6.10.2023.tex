	
	\noindent\bf{Напоминание про нормальную форму Смитта:}\\
	Пусть $B \subset A$~--- свободные абелевы группы ранга $n$, причем $A = \bigoplus \Z x_{i}$, $B = \langle \sum_{j = 1}^{n} a_{i j} x_{j}, \ 1 \le i \le n \rangle$.  Тогда мы можем явно вычислить задание факторгруппы $A/B$ образующими и соотношениями. 

	Рассмотрим матрицу 
	\[
		\begin{pmatrix} a_{11} & a_{12} & \ldots & a_{1n} \\ \vdots & \ldots & \ldots & \vdots \\ a_{n 1} & a_{n 2} & \ldots & a_{n n} \end{pmatrix}
	\]
	Рассмотрим автоморфизм группы $A$, который переводит $x_{1}$ в $x_{1} + c x_{2}, \ c \in \Z$, а остальные образующие переводит в себя. 
	Что произойдет с матрцией в результате этого изоморфизма?  Ко второму столбцу прибавится первый, умноженный на $c$.  Аналогично мы можем делать для любых столбцов. Кроме того, мы можем менять их местами посредством изоморфизмов вида $x_{1} \mapsto x_{2}, x_{2} \mapsto x_{1}$. При таких преобразованиях факторгруппа $B/A$ будет оставаться такой же, так как: $A/B \cong A / f(B)$. Соотвественно, с помощью таких операций матрицу мы можем диагонализовать. В итоге мы получим диагональную матрциу 
	\[
		\begin{pmatrix} \alpha_{1} & 0 & 0 & \ldots & 0 \\ 0 & \alpha_{2} & 0 & \ldots & 0 \\ 0 & 0 & \alpha_{3} & \ldots &  0  \\ \vdots & \vdots & \vdots & \ddots & \vdots  \\ 0 & 0 & 0 & \ldots & \alpha_{nn}\end{pmatrix}
	\]

	\subsection{Норма идеала}

	\begin{definition} 
		Пусть $K/\Q$~--- конечное расширение, $0 \neq I \subset \cO_{K}$~--- идеал. Тогда, как мы знаем, $|\cO_{K}/I| < \infty$. \emph{Нормой идеала $I$} мы будем называть целое число
		\[
			\mathrm{N}_{K/\Q}\lr*{I} \eqdef \left\lvert \cO_{K} / I \right\rvert
		\]
	\end{definition}

	\begin{remark}
		Вообще говоря, норма идеала определяется для любого дедекиндова кольца, соответствующего некоторому расширению и обычно является идеалом. В нашем случае мы рассматриваем кольцо целых, где для любого идеала можно выбрать наименьшую по модулю неотрицательную порождающую, поэтому у нас норма~--- число. 
	\end{remark}

	Хотелось бы, чтоб норма главного идеала была равна норме порождающего его элемента (в смысле нормы для расширения полей). 
	\begin{statement} 
		Пусть $\omega_{1}, \ldots, \omega_{n}$~--- целый базис $\cO_{K}$. Тогда $\mathrm{N}((a)) = \left\lvert\mathrm{N}_{K/\Q}(a)\right\rvert$.
	\end{statement}
	\begin{proof}
		Пусть $a \omega_{i} = \sum\limits_{j = 1}^{n} b_{i j} \omega_{j}, \ b_{i j} \in \Z$. Тогда $\Nm(a) = \det{(b_{i j})}$. С другой стороны, мы доказали, что $|\det{(b_{i j})}| = \left\lvert \cO_{K}/a\cO_{K} \right\rvert$.
	\end{proof}

	Заметим, что тогда мы получаем и мультипликативность для главных идеалов: 
	\[
		\Nm((a))\Nm((b)) = |\Nm_{K/\Q}(a)||\Nm_{K/\Q}(b)| = |\Nm_{K/\Q}(ab)| = \Nm((ab)).
	\]

	Хотелось бы теперь обобщить это на произвольные идеалы. Для этого нам понадобятся задачи из ДЗ~\ref{hw_4}. 

	\begin{lemma}[Задача 5 из ДЗ~\ref{hw_4}] 
		 Пусть $\fp_{1}, \ldots, \fp_{k}$~--- максимальные идеалы кольца $\cO_{K}$, $n_{1}, \ldots, n_{k} \in \Z$. Докажите, что существует $\alpha \in K^{*}\colon \upsilon_{\fp_{i}}(\alpha) = n_{i} \ \forall 1 \le i \le n_{k}$.
	\end{lemma}
	\begin{proof}
		Заметим, что идеалы $\fp_{i}^{n_{i}}$ попарно взаимнопросты. Выберем $x_{i} \in \fp_{i}^{n_i} \setminus \fp_{i}^{n_i + 1}$. Тогда по КТО существует $x \equiv x_{i} \pmod{\fp_{i}^{n_{i} + 1}}$. Тогда 
		\[
			\upsilon_{\fp_{i}}(x) = \upsilon_{\fp_{i}}((x - x_{i}) + x_{i}) = n_{i}.
		\]
	\end{proof}

	\begin{lemma}[Задача 6 из ДЗ~\ref{hw_4}]\label{hw_4_task_6}
		Пусть $I \subset \cO_{K}$~--- идеал, $J$~--- дробный идеал. Докажите, что $\exists x \in K^{*}\colon x J + I = \cO_{K}$.
	\end{lemma}
	\begin{proof}
		Во-первых,  $J$ сразу можно полагать целым, так как мы можем сначала домножить его на элемент, превращающий его в целый, а потом уже что-то с ним делать. Разложим $I$ в произведение простых: 
		\[
			I = \fp_{1}^{k_1} \cdot \ldots \cdot \fp_{m}^{k_m}.
		\]
		Соотвественно, легко найти $y \in K^{*}\colon \upsilon_{\fp_{i}}(yJ) = 0$. Проблема в том, что $yJ$ может оказаться не целым идеалом. Предположим, что это так.
		\[
			yJ = \prod_{i = 1}^{\ell} \fq_{i}^{-r_{i}} \cdot \prod_{j = 1}^{r} \fr_{j}^{\ell_{i}}, \text{ где } \ell_i \ge 0, \ r_i \ge 0.
		\]

		Выберем $\widetilde{y} \in \cO_{K}\colon \upsilon_{\fq_{i}}(\widetilde{y}) = r_i, \ \upsilon_{\fp_{i}}(\widetilde{y}) = 0$, тогда ясно, что $y \widetilde{y} J$~--- целый идеал, который не делится на $\fp_{i}$, следовательно он взаимнопрост с $I$, что и требовалось.
	\end{proof}

	\begin{theorem} [Задача 7 из ДЗ~\ref{hw_4}]
		Любой дробный идеал $I$ порождается двумя элементами. 
	\end{theorem}
	\begin{proof}
		Возьмем $x \in \cO_{K}$ такой, что $x I^{-1} \subset \cO_{K}$~--- целый идеал. Тогда по лемме~\ref{hw_4_task_6} (тут у нас $x I^{-1}$~--- целый идеал, $I^{-1}$~--- дробный) найдётся $y \in K^{*}$ такой, что 
		\[
			xI^{-1} + yI^{-1} = \cO_{K} \implies x I^{-1}I + y I^{-1}I = I \implies I = (x) + (y) = (x, y).
		\]
	\end{proof}

	\begin{homework}[Осторожно, открытая задача]
		Существует ли кольцо, в котором каждый идеал порождается тремя элементами, причём, есть идеал, который не порождается двумя элементами. 
	\end{homework}

	\begin{theorem}[Мультипликативность нормы идеала]
		Если $I, J$~--- два ненулевых идеала в $\cO_{K}$, то для их норм верно равенство $\Nm(I J) = \Nm(I) \Nm(J)$.
	\end{theorem}
	\begin{proof}
		Сравним индексы: $\left\lvert \cO_{K}/IJ \right\rvert = \left\lvert \cO_{K}/I \right\rvert \cdot \left\lvert I/IJ \right\rvert$. Значит, остаётся показать, что $\left\lvert \cO_{K}/J \right\rvert = \left\lvert I/IJ \right\rvert$. По лемме~\ref{hw_4_task_6} найдём $x \in K^{*}\colon xI + J = \cO_{K}$. Тогда воспользуемся теоремой о гомоморфизме и взаимной простотой:
		\[
			\left\lvert  \cO_{K} / J \right\rvert = \left\lvert \lr*{xI + J} / J \right\rvert = \left\lvert xI / xI \cap J \right\rvert = \left\lvert xI / x I J \right\rvert = \left\lvert I / IJ \right\rvert.
		\]
	\end{proof}

	\subsection{Индекс ветвления и степень инерции}

	Возьмем простое число $p \in \Z$ и рассмотрим главный идеал $(p) = p\Z \subset \Z$. Этот же идеал мы можем рассматривать, как главный идеал в коьлце $\cO_{K}$. Там он уже не обязательно будет простым, но будет раскладываться в произведение простых: 
	\[
		p\cO_{K} = \fp_{1}^{e_1} \fp_{2}^{e_2} \cdot \ldots \cdot \fp_{k}^{e_k},
	\]
	причем набор идеалов $\fp_{i}$ будет своим для каждого простого числа $p$ (т.е. для различных простых чисел эти наборы не будут пересекаться). Кроме того, если $\fp \subset \cO_{K}$~--- простой идеал, то $\fp \cap \Z$ будет идеалом в $\Z$, причем простым, значит для некоторого простого $p$ мы получим $(p) \subset \fp$. Тогда $(p) \fp^{-1} \subset \cO_{K}$, следовательно мы можем разложить его на простые: 
	\[
		(p) = \fp \cdot \ldots.
	\]
	Таким образом, простые идеалы в $\cO_{K}$ находятся в соотвествии с простыми числами. 

	Иными словами, над каждым простым числом $p, q, \ldots \in \Z$ находится сколько то идеалов $\{ \fp_{1}, \ldots, \fp_{2}, \ldots, \}$,  $\{ \fq_{1}, \fq_{2}, \ldots\}$. Эти наборы не будут пересекаться и, кроме того, будут покрывать все простые идеалы в $\cO_{K}$.
	\begin{definition} 
		Степень $e_i$ называется \emph{индексом ветвления} идеала $\fp_{i}$.
	\end{definition}

	\begin{definition} 
		Как известно, для $\fp \in \Spec{\cO_{K}}$ факторкольцо $\cO_{K}/\fp$ будет полем. Это поле~--- конечное расширение $\F_{p}$ так как у нас есть естественное вложение $\F_{p}  = \Z/p\Z \to \cO_{K}/\fp$. Значит, $\left\lvert \cO_{K}/\fp \right\rvert = p^{f}$. Число $f$ называется \emph{степенью инерции} идеала $\fp$. Иными словами, \emph{степень инерции}~--- это $[\cO_{K}/\fp : \F_{p}]$.
	\end{definition}

	Заметим, что $|\cO_{K}/\fp| = \Nm(\fp)$. Возьмем простое число $p$ и рассмотрим главный идеал $p\cO_{K}$. Тогда  если $n = [K : \Q]$, то
	\[
		p^n = \Nm_{K/\Q}(p) = \Nm\lr*{p\cO_{K}} = \prod_{i = 1}^{k} \Nm\lr*{\fp_{i}}^{e_{i}} = \prod_{i = 1}^{k} \lr*{p^{f_{i}}}^{e_i}.
	\]

	Тогда, приравнивая степени, мы получаем формулу, устанавливающую соотношение между \emph{индексом ветвления}, \emph{степенью инерции} и степенью расширения: 
	\begin{equation}
		\sum\limits_{i = 1}^{k} e_i f_i = n. \label{deg_ind_eq}
	\end{equation}

	Нетрудно заметить, что случае квадратичного расширения индекс ветвления, как и степень инерции, будут равны единице.  Таакэе ясно, что $1 \le e_{i} f_{i} \le n$, то есть, эти числа не могут быть произвольными. \\

	\noindent\bf{Ветвление при расширении Галуа:}

	Пусть $K/\Q$~--- конечное расширение. Тогда группа Галуа $\Gal(K/\Q)$ действует и на идеалах кольца $\cO_{K}$. Кроме того, она оставляет на месте $\Spec{\cO_{K}}$, так как $ \forall \sigma \in \Gal(K/\Q) \ \cO_{K}/\fp \cong \sigma\cO_{K}/\sigma \fp \cong \cO_{K}/\sigma \fp$. 

	\begin{theorem} 
		Действие $\Gal(K/\Q)$ на множестве простых идеалов, висящих над простым числом $p$.
	\end{theorem}
	\begin{proof}
		Предположим, что есть два простых идеала $\fp, \widetilde{\fp}\colon \fp \cap \Z = (p) = \widetilde{\fp} \cap \Z$, для которых утверждение теоремы не верно.  Тогда 
		\[
			\{ \sigma \fp \ \vert \ \sigma \in \Gal(K/\Q) \} \cap \{ \sigma \widetilde{\fp} \ \vert \ \sigma \in \Gal(K / \Q) \} = \varnothing.
		\]

		По КТО мы можем выбрать такой элемент $x \in \cO_{K}$, что 
		\[
			x \equiv 0 \pmod{\sigma \fp} \quad x \equiv 1 \pmod{\sigma \widetilde{\fp}} \quad \forall \sigma \in \Gal(K/\Q).			
		\]
		Применим теперь норму: 
		\[
			\Nm_{K/\Q}(x) = \prod_{\tau \in \Gal(K/Q)}\tau x \in \fp \cap \Z = p\Z = \widetilde{p} \cap \Z \implies \Nm_{K/\Q}(x) \in \widetilde{\fp}. 
		\]
		Значит, так как $\widetilde{\fp} \in \Spec{\cO_{K}}$, $\exists \tau \in \Gal(K/\Q)\colon \tau x \in \widetilde{\fp} \Leftrightarrow x \in \tau^{-1}\widetilde{\fp}$. Но, с дугой стороны, ранее мы отметили, что $\forall \tau \in \Gal(K/\Q)$ $\tau x \equiv 1 \pmod{\widetilde{\fp}}$.
	\end{proof}


	Так как действие транзитивно, $\exists \sigma \in \Gal(K/\Q) \colon \sigma\fp_{1} = \fp_{2}$
	\[
		\fp \cO_{K} = \fp_{1}^{e_1} \fp_{2}^{e_2} \cdot \ldots \fp_{k}^{e_k} = \sigma(\fp \cO_{K}) = \sigma(\fp_{1})^{e_1} \cdot \ldots \cdot \sigma(\fp_{k})^{e_k} = \fp_{2}^{e_{1}} \cdot \ldots. 
	\]
	Значит, в силу единственности разложения, мы получаем $e_{1} = e_{2}$. В силу транзитивности, мы можем сделать так для любой пары индексов, из чего следует нужное нам. Тогда в случае расширения галуа все индексы ветвления равны. Аналогично мы можем сделать и для степеней инерции. Тогда равенство~\eqref{deg_ind_eq} примет весьма простой вид: $e f k = n$.\\


	\noindent\bf{Ветвление при квадратичном расширении:}

	Пусть $p \neq 2$~--- простое число, рассмотрим расширение $\Q(\sqrt{d})/\Q$, где $d$~--- целое и свободно от квадратов. Тогда в силу формулы $\sum e_i f_i = 2$ мы получаем, что возможны такие варианты разложения:
	\[
		(p) = \fp_1 \fp_2, \ \fp_1 \neq \fp_2, \quad (p) = \fp, \quad (p) = \fp^2.
	\]

	Пусть $p \mid d$, тогда $(p) = (p, \sqrt{d})^2$. Действительно, нам надо проверить
	\[
		(p) = (p^2, p\sqrt{d}, d) \Leftrightarrow (1) = \lr*{p, \sqrt{d}, \frac{d}{p}},
	\]
	а это так, потому что $(p, \frac{d}{p}) = 1$. Кроме того, заметим, что отсюда в частности следует, что идеал $(p, \sqrt{d})^2$~--- простой. 

	Теперь рассмотрим случай, когда $p \not\mid d$. Начнём со случая, когда $\lr*{\frac{d}{p}} = 1$. Тогда $x^2 - d = pm$. Тогда 
	\[
		(p) = \fp_{1} \fp_{2}, \text{  где } \fp_{1} = (p, x + \sqrt{d}), \ \fp_{2} = (p, x - \sqrt{d}).
	\]
	Действительно, перемножим эти идеалы: 
	\[
		\fp_{1}\fp_{2} = (p^2, p(x - \sqrt{d}), p(x + \sqrt{d}), pm) = (p) \Leftrightarrow (p, x - \sqrt{d}, x + \sqrt{d}, m) = (1). 
	\]
	Идеал слева не будет единичным тогда и только тогда все образующие делятся на $p$. На $m$ здесь вообще можно не смотреть.  Но, $p \not \ \mid x \implies (p, 2x) = (1)$.

	Остаётся случай, когда $\lr*{\frac{d}{p}} = -1$. Предположим, что $d \not\equiv 1 \pmod{4}$. Тогда 
	\[
		\cO_{K} = \Z[\sqrt{d}] = \Z[x]/(x^2 - d) \implies \cO_{K}/(p) \cong \Z[x] /(x^2 - d, p) = \F_{p}[x]/(x^{2} - d) \text{~--- поле}
	\]
	откуда следует, что идеал $(p)$ максимален.  Тепеь, если $d \equiv 1 \pmod{4}$, 
	\[
		\cO_{K} = \Z\left[ \frac{1 + \sqrt{d}}{2} \right] \implies \cO_{K}\bigg/(p) \cong \Z[x]\bigg/\lr*{x^2 - x + \frac{1 - d}{4}, p} \cong \F_{p}/\lr*{x^2 - x + \frac{1 - d}{4}} \text{~--- поле,}
	\]

	так как дискриминант многочлена $x^2 - x + \frac{1 - d}{4}$ равен $d$, а $d$~--- невычет по модулю $p$.

	\begin{homework}\label{hw_5}
	Задачи:
		\begin{enumerate}
			\item Разобрать случай $p = 2$ в выкладках выше. 
			\item Пусть $K/\Q$~--- расширение степени $n$, $K = \Q(\theta)$, где $\theta^n + a_{n - 1}\theta^{n - 1} + \ldots + a_0 = 0$ и пусть $p$~--- такое простое число, что $\vp(a_0) = 1$ и $\vp(a_i) \ge 1$. Докажите, что тогда $p \not \ \mid \ind(\theta)$.
		\end{enumerate}
		
	\end{homework}


	








	

	