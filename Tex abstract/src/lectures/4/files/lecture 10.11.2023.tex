	
	\subsection{Первый случай Last Fermat's theorem}

	Мы можем полагать, что показатель $n$~--- простое число, а также рассматривать уравнение в виде 
	\begin{equation}
		x^p + y^p + z^p = 0, \quad (x, y = z) = 1. \label{1_st_Last_Fermat's_Theorem}
	\end{equation}
	

	\begin{theorem}[Первый случай Большой теоремы Ферма]
		Уравнение~\ref{1_st_Last_Fermat's_Theorem} не имеет целых решений, если $p \not \mid xyz$.
	\end{theorem}

	\begin{theorem}[Софи Жермен] 
		Если простое число $p$ таково, что $2p + 1 = q$~--- простое число, то имеет место первый случай Большой теоремы Ферма. 
	\end{theorem}
	\begin{proof}
		Перепишем уравнение в виде 
		\[ 
			y^p + z^p = (-x)^p \Leftrightarrow (y + z)(y^{p - 1} + y^{p - 2}z + \ldots + z^{p - 1}) = (-x)^{p}.
		\]

		Покажем, что $(y + z, y^{p - 1} + y^{p - 2}z + \ldots + z^{p - 1}) = 1$. Пусть $r$~--- простое ($r \neq p$) и такое, что $r \mid y + z, \ r \mid y^{p - 1} + y^{p - 2}z + \ldots + z^{p - 1}$. Тогда 
		\[
			y \equiv -z \pmod{r}, \quad py^{p - 1} \equiv 0 \pmod{r} \implies py^{p - 1}\equiv 0 \pmod{r},
		\]
		что противоречит взаимной простоте. Тогда мы имеем 
		\[
			\begin{cases} 
				y + z = A^p \\
				y^{p - 1} + y^{p - 2}z + \ldots + z^{p - 1} = T^p
			\end{cases}
		\]

		Так как наше условие симметрично относительно переменных, $x + y = B^p, \ x + z = C^p$, откуда 
		\begin{equation}
				x^{\frac{q - 1}{2}} + y^{\frac{q - 1}{2}} + z^{\frac{q - 1}{2}} = 0. \label{ferm_2}
		\end{equation}

		Заметим, что если $q \not \mid x$,  то по малой теореме Ферма:
		\[
				x^{q - 1} \equiv 1 \pmod{q} \implies x^{\frac{q - 1}{2}} \equiv \pm 1 \pmod{q}.
			\]	
		Ясно, что отсюдо следует протвиоерчие~\eqref{ferm_2}. Значит, $q \mid xyz$.  Не умаляя общности, $q \ \mid x$. Тогда 
		\[
			2x = B^p + C^p - A^p = B^{\frac{q - 1}{2}} + C^{\frac{q - 1}{2}} - A^{\frac{q - 1}{2}} \ \vdots \ q.
		\]

		Заметим, что $q \not \ \mid B, \ q \not\  \mid C$, значит $q \not \ \mid BC \implies q \mid A$. Тогда $q \mid y + z$.  

		Заметим, что  $T^p \equiv py^{p - 1} \pmod{q}$. Так как $(A, T) = 1$, $q \not\  \mid T$. Тогда $T^{\frac{q - 1}{2}} \equiv py^{p - 1} \pmod{q}$, тогда по малой теореме Ферма $\pm 1 = p y^{p - 1} \pmod{q}$.  Так как $q \mid x$,  $B^p = x + y \equiv y \pmod{q}$. Значит, 
		\[
			y \equiv B^{\frac{q - 1}{2}} \equiv \pm 1 \pmod {1}, \text{ так как } q \not \ \mid B.
		\]

		Знчит, $\pm 1 \equiv \pm p \pmod{q}$, а этого быть не может, так как $q = 2p + 1$.

	\end{proof}

	\begin{homework}
		Получите элементарное доказательство случая $p = 5$ в первом случае большой теоремы Ферма. 
	\end{homework}

	Как мы знаем, если мы рассматриваем расширение $K = \Q(\zeta_m)$ над $\Q$ и простое число $q \not \mid m$, число $q$ не равзетвлено. 

	В частности для $m = p$~--- простого, $q \neq p \implies q$ неразветвлено в $\Q(\zeta_p)$.

	\begin{statement}\label{stm:11} 
		Простое число $p$ полностью разветвлено в $\Q(\zeta_{p})$.
	\end{statement}
	\begin{proof}
		\[
	 	x^p - 1 = (x - 1)(x - \zeta)(x - \zeta^2)\ldots(x - \zeta^{p - 1})
	 \] 
	 \[
	 	x^{p - 1} + \ldots + x + 1 = (x - \zeta)(x - \zeta^2)\ldots(x - \zeta^{p - 1})
 	 \]

 	 Подставим $x = 1$ и от числового равенства перейдём к равенству идеалов: 
 	 \[
 	 	p = ((1 - \zeta))((1 - \zeta^2))\ldots((1 - \zeta^{p - 1})),
 	 \]
 	 отсюда, так как $\zeta = \lr*{\zeta^i}^j$, все идеалы в правой части совпадают, а значит, $(p) = ((1 - \zeta))^{p - 1}$, а это означает, что $p$ полностью разветвлено. Действительно, пусть 
 	 \[
 	 	1 - \zeta = \fp_{1}\fp_{2} \ldots \fp_{k} \implies (1 - \zeta)^{p - 1} = \fp_{1}^{p - 1}\fp_{2}^{p - 1} \ldots \fp_{k}^{p - 1},
 	 \]
 	 из чего следует, что степень ветвления каждого $\fp_i$ хотя бы $p - 1$. С другой стороны, $\sum e_j f_j = p - 1$, то есть $(p) = \fp^{p - 1}$.
	\end{proof}

	\begin{lemma} 
		Пусть $p$~--- простое число, не равное двум. Множество корней из единицы\footnote{не обязательно степени $p$} в поле $\Q(\zeta_{p})$ равно $\{ \pm \zeta_{p}^{i} \}.$
	\end{lemma}
	\begin{proof}
		Возьмем $\zeta_n \in \Q(\zeta_{p})$. Предположим, что $n \equiv 0 \pmod{4}$. Тогда $i = \zeta_{4} \in \Q(\zeta_{p})$. Заметим, что $2i = (1 + i)^2$ а значит, $(2) = ((1 + i))^2$. Значит, двойка разветвлена в $\Q(\zeta_{p})$ (а это противоречит предыдущему утверждению). Значит $n \not\equiv 0 \pmod{4}$. 

		Теперь рассмотрим случай  $n = 2n_{0}$, $n \equiv 1 \pmod{2}$. Тогда $\{\zeta^{i}_{n}\} = \{ \pm \zeta_{n_0}^{i}\}$ и нам достаточно рассматривать $n_0$. Предположим, что существует простое $p' \neq p \colon p \ \mid \ n_0$. Тогда $\zeta_{p'} \in \Q(\zeta_{p})$, но $\zeta_{p'} \in \Q(\zeta_{p'})$, а так как $p' \mid n_0$, $\Q(\zeta_{p'}) \subset \Q(\zeta_{n_0}) \subset \Q(\zeta_{p})$. Простое число $p'$ будет полностью разветвлено в $\Q(\zeta_{p'})$, а значит, оно будет полностью разветвлено и $\Q(\zeta_{p})$, но это противоречит утверждению~\ref{stm:11}.
		

		Значит, $n_0 = p^a$, а тогда $\zeta_{p^{a}} \in \Q(\zeta_{p})$. Но это приводит нас к тому, что $n_0 = p$, ведь 
		\[
			[\Q(\zeta_{p^{a}}) : \Q] = p^{a} - p^{a - 1} \le [\Q(\zeta_{p}) : \Q] = p - 1 \implies a - 1 \implies n_0 = p 
		\]
	\end{proof}

	\begin{lemma}\label{lemma:14}
		Пусть $K/\Q$~--- конечное расширение, $\sigma_i \colon K \to \Q^{alg}$~--- все вложения ($1 \le i \le n$, $n = [K : \Q]$). Предположим, что $\alpha \in \cO_{K}$ и $\forall i \ |\sigma_{i}\alpha| = 1$. Тогда $\alpha$ является корнем из единицы какой-то степени. \footnote{Обратное утверждение очевидно.}
 	\end{lemma}
 	\begin{proof}
 		Выпишем многочлен с целыми коэффициентами, корнем которого является $\alpha$:
 		\[
 			\prod_{i} (x - \sigma_{i} \alpha) \in \Z[x].
 		\]
 		В силу предположения теоремы, его коэффиценты ограничены, так как они являются симметрическими функциями от $\sigma_{i}\alpha$. Заметим теперь, что из условия следует, что $|\sigma_{i}(\alpha^k)| \le 1$, а значит, для $\alpha^k$ мы также получим многочлен с ограниченными коэффициентами. Заметим, что $k$~--- произвольное натуральное, а значит, мы получаем бесконечное число $\alpha^k$, которые являются корнями коненого набора многочленов над $\Z$ (так как коэффициенты  каждого мы можем ограничить одной и той же константой).  Значит, $\exists m, n \colon \alpha^m = \alpha^n$, что и даёт нам, что $\alpha$~--- корень из 1. 
 	\end{proof}

 	\begin{lemma} 
 		Пусть $u \in \cO_{K}^{*} = \Z[\zeta_{p}]$ для $K = \Q(\zeta_{p})$. Тогда $\exists s \colon u \zeta^{s} \in \R$.

 		\begin{proof}
 			Рассомтрим $v = u / \overline{u}$ и возьмем $\rho \in \Gal\lr*{\Q(\zeta_{p})/\Q} \cong (\Z/p\Z)^{*}$. Тогда по лемме~\ref{lemma:14}:
 			\[
 				\rho(v) = \frac{\rho(u)}{\rho(\overline{u})}  = \frac{\rho(u)}{\overline{\rho(u)}}	 \implies |\rho(v)| = 1 \implies v = \pm \zeta_{p}^{n}.
		 	\] 	

		 	Рассмотрим $\lambda = 1 - \zeta$, тогда 
		 	\[
		 		\rho(\zeta) \equiv \zeta^k \equiv \zeta \pmod{\lambda} \implies \rho(\zeta^i) \equiv \zeta^i \pmod{\lambda},
		 	\]
		 	а так как $\cO_{K} = \Z[\zeta_{p}]$, мы имеем такое сравнение для всеё элементов $\cO_{K}$. В частности, из этого следует, что $\rho(u) \equiv u \pmod{\lambda}$. В частности, мы можем положить $\rho(\cdot) = \overline{\cdot}$ и отсюда получить, что $\overline{u} \equiv u \pmod{\lambda}$. Из этого следует, что 
		 	\begin{multline*}
		 		\pm \zeta^{n}_{p} \overline{u} \equiv u \equiv \overline{u} \pmod{\lambda}, \text{пусть } -\zeta^{n}_{p}u \equiv u \pmod{\lambda}, \\ -\zeta^{n}_{p}\overline{u} \equiv \overline{u} \pmod{\lambda} \implies -u \equiv u \pmod{\lambda} \implies 2u \equiv 0 \pmod{\lambda},
		 	\end{multline*}
		 	но если $2u \ \vdots \  (1 - \zeta)$, то так как $u$ обратим, $2 \ \vdots \ (1 - \zeta)$.  Но, если $2 \in (1 - \zeta)$, то $2^{p - 1} \in ((1 - \zeta))^{p - 1} = (p)$. Значит, знак минус невозможен и реализуется случай 
		 	\[
		 		\zeta^n_{p} \overline{u} \equiv \overline{u} \equiv u \pmod{\lambda}. 
		 	\]
		 	Так как $u - v \overline{u}$, $u = \zeta^{n}_{p} \overline{u}$, тогда $u \zeta_{p}^{s} = \zeta_{p}^{n + s}\overline{u}$. Найдём такое $s$, что $\overline{\zeta_{p}^{n + s} \overline{u}} = u$. То есть, такое $s$, что $\overline{\zeta^{n + s}} = \zeta^{s} \implies 2s + n \equiv 0 \pmod{p}$, что вполне возможно реализовать. 
 		\end{proof}
 	\end{lemma}

 	\begin{lemma}\label{lemma:15}
 		Пусть $x^p + y^p = z^p, \ p \not \ \mid xyz, \ (x, y, z) = 1$, разложим левую часть в линейные множители:
 	\[
 		(x + y)(x + \zeta y)\ldots(x + \zeta^{p - 1}y)
 	\]
 	Тогда  эти сомножители взаимнопросты. 	
 	\end{lemma}
 	

 	\begin{proof}
 		Предположим противное, тогда $x + \zeta^i y, \ x + \zeta^{j} y \in \fq$ для некоторых $i, j$. Тогда $\zeta^i(1 - \zeta^{j - i}) y \in \fq$, но $((1 - \zeta^{j - i}))^{p - 1} = (p)$ и если $\fq \neq (1 - \zeta)$, мы получаем, что $y \in \fq$, но тогда и $x \in \fq$, а это противоречит взаимной простоте $x$ и $y$. Если же $\fq = (1 - \zeta)$, мы получаем, что $x + y = x + \zeta^{i}y + (y - \zeta^{i})y \in \fq$, откуда следует, что $z \in \fq$, а значит, $z^{p - 1} \ \vdots p$, а этого не может быть по предположению. 
 	\end{proof}


 	\begin{theorem} 
 		Пусть $p \not \ \mid |\Cl(\Q_{\zeta_{p}})|$\footnote{такие простые числа называются регуярными}. Тогда имеет место первый случай Великой теоремы Ферма.  
 	\end{theorem}
 	\begin{proof}
 		Пусть $x^p + y^p = z^p$, разложим левую часть на множители: 
 		\[
 			\prod_{i = 0}^{p - 1} (x + \zeta^{i}y) = z^p,
 		\]
 		а так как все сомножители в левой части равенства взаимнопросты по лемме~\ref{lemma:15}. Значит, все они являются $p$-ми степенями, в частности для $i = 1$. То есть 
 		\[
 			(x + \zeta y) = I^{p},
 		\]
 		значит $I$ находится в $p$-кручении $\Cl(\Q_{\zeta_{p}})$, но так как $p$ не делит порядок группы классов, оно тривиально, значит  $I$--- главный, то есть  $I = (\alpha)$.  Значит,

 		\[
 			(x + \zeta y) = (\alpha^p) \implies x + \zeta y = \varepsilon \alpha^p, \ \varepsilon \in \Z[\zeta]^{*}, \ \varepsilon = \zeta^s u \implies x + \zeta y = \zeta^s u \alpha^p.
 		\]

 		С другой стороны, $\alpha = a_0 + a_1 \zeta_{p} + \ldots + a_{p - 2}\zeta^{p - 2}$. Тогда 
 		\[
 		 	\alpha^p \equiv \underbrace{a_0^p + a_1^p + \ldots + a_{p - 2}^{p}}_{\eqdef n} \pmod{p}.
 		 \] 

 		 Тогда $x + \zeta y \equiv \zeta^s u n \pmod{p}$. Переходя в этом сравнении к сопряженным, мы получаем 
 		 \[
 		 	x + \zeta y \equiv \zeta^s u n \pmod{p}, \quad x + \zeta^{-1}y \equiv \zeta^{-s} \overline{u} n = \zeta^{-s} u n \pmod{p}, \text{ так как } u \in \R.
 		 \]

 		 Отсюда мы получаем, что $\zeta^{-s}(x + \zeta y) \equiv \zeta^{s} (x + \zeta^{-1}y)\pmod{p}$. Значит, $x + \zeta y - \zeta^{2s} x - \zeta^{2s - 1}y \in p\Z[\zeta]$. Теперь рассмотрим несколько случаев: 

 		 \begin{enumerate}
 		 	\item Элементы $S = \{ 1, \zeta, \zeta^{2s}, \zeta^{2s - 1}\}$~--- попарно различны. 
 		 	\begin{enumerate}
 		 		\item Если $\zeta^{p - 1} \notin S$, то $p \mid x, p \mid y$. 

 		 		\item Если $\zeta^{p - 1} \in S$, то $p = 2s - 1$ и тогда $s \ \vdots p $, а значит $(\zeta - \zeta^{-1})y \in p\Z[\zeta]$ откуда следует, что $p \mid 1 - \zeta^2$, что даёт нам противоречие. 

 		 		\item Если $\zeta^{p - 1} = -(1 + \zeta + \ldots + \zeta^{p - 2})= \zeta^{2s}$, то $x + \zeta y - \zeta^{-1}x - \zeta^{-2}y \in \Z[\zeta]$, а тогда 
 		 		\[
 		 			x + \zeta y + ((1 + \zeta + \ldots + \zeta^{p - 2}))x - \zeta^{p - 2}y \ \vdots \ p \implies x + y \ \vdots \  p \implies z \ \vdots \ p,
 		 		\]
 		 		что даёт нам противоречие. 
 		 	\end{enumerate}

 		 	\item Некоторые из этих степеней совпадают. 
 		 		\begin{enumerate}
 		 			\item $\zeta^{2s} = 1$. В этом случае $(\zeta - \zeta^{-1})y \ \vdots \ p$,  а мы уже видели, что это невозможно. 

 		 			\item $\zeta^{2s - 1} = 1$.  В этом случае $x - y + \zeta y - \zeta x \ \vdots \ p \implies (x - y)(1 - \zeta) \ \vdots \ p \implies x - y \ \vdots \ p$, то есть $x \equiv y \pmod{p}$. Исходное уравнение мы можем записать в виде 
 		 			\[
 		 				z^p + (-y)^p = x^p,
 		 			\]
 		 			и рассуждая аналогично, мы можем получить, что $z \equiv - y \pmod{p}$. Тогда 
 		 			\[
 		 				x^p + x^p \equiv (-x)^{p} \pmod{p} \implies 3x^{-} \equiv 0 \pmod{p}.
 		 			\]
 		 			Значит, либо $p = 3$ (в этом случае мы полностью разобрали вообще всю теорему Ферма), либо $p \mid x$, что даёт нам противоречие. 
 		 			

 		 			\item $\zeta = \zeta^{2s - 2}$. Тогда $x - \zeta^2 x \ \vdots \ p$, откуда следует, что $1 - \zeta^2 \ \vdots \ p$, а это, как мы уже видели, противоречие. 
 		 		\end{enumerate}
 		 \end{enumerate}

 		 Дадим теперь хороший критерий для проверки условия теоремы. Этот критерий мы дадим без доказательства, так как он доказывается методами аналитической теории чисел. 

 		 Как мы помним, числа Бернулли можно начинать через экспоненциальную производящую функцию: 
 		 \[
 		 	\frac{t}{e^t - 1} = \sum_{m = 0}^{\infty} B_{m} \frac{t^m}{m!}.
 		 \]
 		 Подставим $-t$:
 		 \[
 		 	\frac{-t}{\frac{1}{e^{t}} - 1} = \frac{t e^{t}}{e^{t} - 1} = \frac{t(e^t - 1) + t}{e^t - 1} = t + \frac{t}{e^t - 1}.
 		 \]

 		 Отсюда мы можем понять, чему равны коэффициенты: 
 		 \[
 		 	m > 1, \ m \equiv_{2} 1 \implies B_m = -B_m \implies B_m = 0.
 		 \]
 		 \[
 		 	-(n + 1)B_n = \binom{n + 1}{n - 1} B_{n - 1} + \ldots + \binom{n + 1}{k}B_{k} + \ldots + \binom{n + 1}{1} B_{1} + 1. 
 		 \]
 		 Так как все нечётные коэффициенты равны нулю, авторы часто используют обозначение $B_n$ для $2n$-го числа Бернулли. Например, известно, что 
 		 \[
 		 	B_{n} = -n \zeta(1 - n), n > 1.
 		 \]

 		 \begin{theorem} 
 		 	Простое число $p$~--- регулярно тогда и только тогда, когда числители всех чисел Бернулли $B_{2}, B_{3}, \ldots, B_{p - 3}$ не делятся на $p$.
 		 \end{theorem}

 		 Из этого рекуррентного соотношения следует, что знаменатели $B_n$ не делятся на $p$ при $n \le p - 3$ , так как 
 		 \[
 		 	\upsilon_{p}((n + 1)B_n) \ge 0 \implies \upsilon_{p}(B_n) \ge 0.
 		 \]

 		 \begin{example}
 		 	Например, таким образом нетрудно показать, что число 7 является регулярным. 
 		 \end{example}

 	\end{proof}

 	Теперь, немного отвлечёмся и приведём алгоритм построение целого базиса.

 	\subsection{Алгоритм построения целого базиса}

 	Итак, для $d = \disc(1, \alpha, \alpha^2, \ldots, \alpha^{n - 1})$ мы наем, что 
 	$d \cO_{k} \subset \Z[\alpha] \subset \cO_{k}$, откуда 
 	\[
 		\Z[\alpha] \subset \cO_{K} \subset d^{-1}\Z[\alpha].
 	\]

 	Тогда, как мы помним, $\bigcup (\omega_i + \Z[\alpha]) = d^{-1}\Z[\alpha]$, причем, каждый класс, либо вообще не содержит целых алгебраических чисел, либо целиком из них состоит. То есть, отсюда мы имеем 
 	\[
 		\cO_{K} = \bigcup_{i \in I} (\omega_{i} + \Z[\alpha]),
 	\]
 	откуда следует, что $\cO_{K}$ порождена $\omega_{i}$ и $\alpha^s$ при $0 \le s \le n - 1$. Значит, $d \cO_{K}$ будет порождена  $d\omega_{i}$ и $d \alpha^s$. С другой стороны, $d\cO_{K}$~--- подгруппа свободной абелевой группы $\Z[\alpha]$, значит мы попадаем в контекст \hyperlink{smith_normal_form}{нормальной формы Смита}.

 	\begin{homework}

 	\begin{enumerate}
 		\item \begin{theorem}[Штикельберг] 
 			Дискриминант конечного расширения $K/\Q$ сравним с нулём или единицей по модулю 4. 
 		\end{theorem}
 		\emph{Hint:} Можно действовать так: $\disc(K) = \lr*{\det(\sigma_i)}^2$, и раскрывая определитель, мы получаем $\det(\sigma_{j}\omega_{j}) = P - N$, где $P$~--- сумма произведений со знаком +, а $N$~--- сумма произведений со знаком минус. Тогда $\disc(K) = (P - N)^2 = (P + N)^2 - 4 PN$. Значит, достаточно показать, что числа $P + N$ и $PN$~--- целые.  \emph{Hint:} Целое число~--- это рациональное число, которое еще и целое алгебраическое. 
 		\item
 	\end{enumerate}
 		
 	\end{homework}





	