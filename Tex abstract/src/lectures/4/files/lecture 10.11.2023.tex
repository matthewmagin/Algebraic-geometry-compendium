	
	\subsection{Первый случай Last Fermat's theorem}

	Мы можем полагать, что показатель $n$~--- простое число, а также рассматривать уравнение в виде 
	\begin{equation}
		x^p + y^p + z^p = 0, \quad (x, y, z) = 1. \label{1_st_Last_Fermat's_Theorem}
	\end{equation}
	
	\emph{Первым случаем} Большой теоремы Ферма называют доказательство большой теоремы ферма в предположениии $p \not \ \mid xyz$.  

	\begin{theorem}[Софи Жермен] 
		Если простое число $p$ таково, что $2p + 1 = q$~--- простое число, то имеет место первый случай Большой теоремы Ферма. 
	\end{theorem}
	\begin{proof}
		Перепишем уравнение в виде 
		\[ 
			y^p + z^p = (-x)^p \Leftrightarrow (y + z)(y^{p - 1} - y^{p - 2}z + \ldots - y z^{p - 1} + z^{p - 1}) = (-x)^{p}.
		\]

		Покажем, что $(y + z, y^{p - 1} - y^{p - 2}z + \ldots + z^{p - 1}) = 1$. Пусть $r$~--- простое ($r \neq p$) и такое, что $r \mid y + z, \ r \mid y^{p - 1} + y^{p - 2}z + \ldots + z^{p - 1}$. Тогда 
		\[
			y \equiv -z \pmod{r} \implies y^{p - 1} - y^{p - 2}z + \ldots - y z^{p - 1} + z^{p - 1} \equiv p y^{p - 1} \pmod{r} \implies y \divby r \implies z \divby r,
		\]
		что противоречит  тому, что $(y, z) = 1$. 
		
		\[
			\begin{cases} 
				y + z = A^p \\
				y^{p - 1} - y^{p - 2}z + \ldots - y z^{p - 1} + z^{p - 1} = T^p
			\end{cases}
		\]

		Так как наше условие симметрично относительно переменных, $x + y = B^p, \ x + z = C^p$. Теперь заметим, что по условию 
		\begin{equation}
				x^p + y^p + z^p = 0 \Leftrightarrow x^{\frac{q - 1}{2}} + y^{\frac{q - 1}{2}} + z^{\frac{q - 1}{2}} = 0. \label{ferm_2}, \quad p = \frac{q - 1}{2}.
		\end{equation}

		Заметим, что если $q \not \ \mid x$,  то по малой теореме Ферма:
		\[
				x^{q - 1} \equiv 1 \pmod{q} \implies x^{\frac{q - 1}{2}} \equiv \pm 1 \pmod{q}.
			\]	
		Отсюда ясно, что не может быть такого, что $q \not \ \mid x, \ q \not \ \mid y, \ q \not \ \mid z$, так как иначе~\ref{ferm_2} выполняться не может.  Значит, $q \mid xyz$.  Не умаляя общности, пусть $q \ \mid x$. Тогда 
		\[
			2x = B^p + C^p - A^p = B^{\frac{q - 1}{2}} + C^{\frac{q - 1}{2}} - A^{\frac{q - 1}{2}} \equiv 0 \pmod{p}.
		\]

		Отсюда ясно, что по аналогичным соображениям не может быть такого, что $ABC \notdivby q$. С другой стороны, если $B \divby q$, то $B^p \divby q$, а тогда 
		\[
			\begin{cases} x + y = B^p \divby q \\ x \divby q \end{cases}  \implies y \divby q,
		\]
		что противоречит $(x, z) = 1$. По аналогичным причинам $q \not \ \mid C$. Тогда $A \divby q$, откуда $y + z = A^p \divby q$, откуда $T^p \equiv py^{p - 1} \pmod{q}$. С другой стороны, так как $(A, T) = 1$, как мы показали выше, $T \notdivby q$, а тогда по малой теореме Ферма 
		\begin{equation}
			T^{\frac{q - 1}{2}} \equiv \pm 1 \pmod{q} \implies p y^{p - 1} \equiv \pm 1 \pmod{q}.   \label{ferm_3}
		\end{equation}
		
		А так как $x \divby q$, $B^{\frac{q - 1}{2}} = B^p = x + y \equiv y \pmod{q}$, но тогда так как $B \notdivby q$, по малой теореме Ферма
		\[
			y \equiv B^{\frac{q - 1}{2}} \equiv \pm 1 \pmod{q}.  
		\]
		Подставляя это в~\ref{ferm_3}, получаем, что
		\[
			p y^{p - 1} \equiv p (\pm 1)^{p - 1} \equiv p \equiv \pm 1 \pmod{q},
		\]
		что даёт нам противоречие, так как $q = 2p + 1$. 

		 Так как $(A, T) = 1$, $q \not\  \mid T$. Тогда $T^{\frac{q - 1}{2}} \equiv py^{p - 1} \pmod{q}$, тогда по малой теореме Ферма $\pm 1 = p y^{p - 1} \pmod{q}$.  Так как $q \mid x$,  $B^p = x + y \equiv y \pmod{q}$. Значит, 
		\[
			y \equiv B^{\frac{q - 1}{2}} \equiv \pm 1 \pmod {1}, \text{ так как } q \not \ \mid B.
		\]

		Знчит, $\pm 1 \equiv \pm p \pmod{q}$, а этого быть не может, так как $q = 2p + 1$.

	\end{proof}

	\begin{homework}
		Получите элементарное доказательство случая $p = 5$ в первом случае большой теоремы Ферма. 
	\end{homework}

	Рассмотрим $K = \Q(\zeta_m)$ над $\Q$. Мы доказывали, что если $q$ простое и $q \not \ \mid m$, то оно неравзетвлено, то есть 
	\[
		q\cO_{K} = \fp_1 \cdot \fp_2 \cdot \ldots \cdot \fp_k.
	\]

	В частности, если $m = p$~--- простое, то $q \not \ \mid p$ и это будет выполнено. А вот $p$ будет полностью разветвлено в $\Q(\zeta_p)$. Убедимся в этом:

	\begin{statement}\label{stm:11} 
		Простое число $p$ полностью разветвлено в $\Q(\zeta_{p})$.
	\end{statement}
	\begin{proof}
		\[ 
	 	x^p - 1 = (x - 1)(x - \zeta)(x - \zeta^2)\ldots(x - \zeta^{p - 1}) = (x - 1)(x^{p - 1} + \ldots + x + 1). 
	 \] 
	 \[
	 	x^{p - 1} + \ldots + x + 1 = (x - \zeta)(x - \zeta^2)\ldots(x - \zeta^{p - 1})
 	 \]

 	 Подставим $x = 1$ и от числового равенства перейдём к равенству идеалов: 
 	 \[
 	 	p\cO_{K} = ( 1 - \zeta )( 1 - \zeta^2 )\ldots( 1 - \zeta^{p - 1} ).
 	 \]
 	 Ясно, что $1 - \zeta^j \divby 1 - \zeta$, а так как $\exists i \colon (\zeta^j)^i = \zeta$, $1 - \zeta \divby 1 - \zeta^j$, то есть все идеалы в правой части совпадают и мы имеем 
 	 \[
 	 	p\cO_{K} = ((1 - \zeta))^{p - 1}.
 	 \]

 	  Покажем теперь, что $(1 - \zeta)$~--- простой идеал. Действительно, пусть 
 	 \[
 	 	(1 - \zeta) = \fp_{1}\fp_{2} \ldots \fp_{k} \implies p\cO_{K} = ((1 - \zeta))^{p - 1} = \fp_{1}^{p - 1}\fp_{2}^{p - 1} \ldots \fp_{k}^{p - 1},
 	 \]
 	 откуда индекс ветвления $e_i \ge p - 1$, но с другой стороны, $e f k = [K : \Q] = p - 1$ (так как у нас расширение Галуа), откуда ясно, что 
 	 \[
 	 	p\cO_{K} = ((1 - \zeta))^{p - 1} = \fp^{p - 1}, \quad \fp \in \Specm(\cO_{K}). 
 	 \]
	\end{proof}

	\begin{lemma}\label{lemma:13} 
		Пусть $p$~--- простое число, не равное двум. Множество корней из единицы\footnote{не обязательно степени $p$} в поле $\Q(\zeta_{p})$ равно $\{ \pm \zeta_{p}^{i} \}.$
	\end{lemma}
	\begin{proof}
		Возьмем $\zeta_n \in \Q(\zeta_{p})$. 

		\bf{1)}. Предположим, что $n \equiv 0 \pmod{4}$. Тогда $i = \zeta_{4} \in \Q(\zeta_{p})$. Заметим, что $2i = (1 + i)^2$ а значит, $(2) = ((1 + i))^2$, то есть двойка разветвлена в $\Q(\zeta_{p})$ (а это противоречит предыдущему утверждению). Значит $n \not\equiv 0 \pmod{4}$. 

		\bf{2)}. Теперь рассмотрим случай  $n = 2n_{0}$, $n$~--- нечётное. Тогда $\zeta^{i}_{n} = \pm \zeta_{n_0}^{i}$ и нам достаточно рассматривать $n_0$. Пусть у $n_0$ есть какие-то простые делители, кроме $p$, например, $p'$. Тогда, так как $n_0 \divby p'$, 
		\[
			\Q(\zeta_{p'}) \le \Q(\zeta_{n_0}) \le \Q(\zeta_{p}).
		\]
		Но тогда по предложению~\ref{stm:11} $p'$ будет полность разветвлено в $\Q(\zeta_{p'})$ и при этом, так как $p' \not \ \mid p$, неразветлвено в $\Q(\zeta_{p})$, что даёт нам противоречие. 
		
		\bf{3)}. Значит, $n_0 = p^a$, а тогда $\zeta_{p^{a}} \in \Q(\zeta_{p})$, то есть $\Q(\zeta_{p^{a}}) \le \Q(\zeta_{p})$. С другой стороны, тогда 
		\[
			[\Q(\zeta_{p^{a}}) : \Q] = p^{a} - p^{a - 1} \le [\Q(\zeta_{p}) : \Q] = p - 1 \implies a = 1 \implies n_0 = p.
		\]
	\end{proof}

	\begin{lemma}\label{lemma:14}
		Пусть $K/\Q$~--- конечное расширение, $\sigma_i \colon K \to \Q^{alg}$~--- все вложения ($1 \le i \le n$, $n = [K : \Q]$). Предположим, что $\alpha \in \cO_{K}$ и $\forall i \ |\sigma_{i}\alpha| \le 1$. Тогда $\alpha$ является корнем из единицы какой-то степени. \footnote{Обратное утверждение очевидно.}
 	\end{lemma}
 	\begin{proof}
 		Выпишем многочлен с целыми коэффициентами, корнем которого является $\alpha$:
 		\[
 			\prod_{i} (x - \sigma_{i} \alpha) \in \Z[x].
 		\]
 		В силу предположения теоремы, его коэффиценты ограничены, так как они являются симметрическими функциями от $\sigma_{i}\alpha$. Заметим теперь, что из условия следует, что $|\sigma_{i}(\alpha^k)| \le 1$, а значит, для $\alpha^k$ мы также получим многочлен с ограниченными коэффициентами. Заметим, что $k$~--- произвольное натуральное, а значит, мы получаем бесконечное число $\alpha^k$, которые являются корнями коненого набора многочленов над $\Z$ (так как коэффициенты  каждого мы можем ограничить одной и той же константой).  Значит, $\exists m, n \colon \alpha^m = \alpha^n$, что и даёт нам, что $\alpha$~--- корень из 1. 
 	\end{proof}

 	\begin{lemma}\label{lemma:26} 
 		Пусть $u \in \cO_{K}^{*} = \Z[\zeta_{p}]$ для $K = \Q(\zeta_{p})$. Тогда $\exists s \colon u \zeta_{p}^{s} \in \R$.

 		\begin{proof}
 			Положим $\zeta = \zeta_{p}$. Рассомтрим $v = u / \overline{u}$ и возьмем $\rho \in \Gal\lr*{\Q(\zeta_{p})/\Q} \cong (\Z/p\Z)^{*}$. Тогда по лемме~\ref{lemma:14}:
 			\[
 				\rho(v) = \frac{\rho(u)}{\rho(\overline{u})}  =\footnote{Так как сопряжение~--- тоже автоморфизм, а группа Галуа абелева, $\rho(\overline{x}) = \overline{\rho(x)}$} \frac{\rho(u)}{\overline{\rho(u)}}	 \implies |\rho(v)| = 1.
		 	\]
		 	Значит, по лемме~\ref{lemma:14} $v$~--- является корнем из единицы какой-то степени, а тогда по лемме~\ref{lemma:13} $v = \pm \zeta^{n}$.


		 	Положми $\lambda = 1 - \zeta$, тогда 
		 	\[
		 		\rho(\zeta) \equiv \zeta^k \equiv \zeta \pmod{\lambda} \implies \rho(\zeta^i) \equiv \zeta^i \pmod{\lambda},
		 	\]
		 	а так как $\cO_{K} = \Z[\zeta_{p}]$, мы имеем такое сравнение для всех элементов $\cO_{K}$. В частности, из этого следует, что $\rho(u) \equiv u \pmod{\lambda}$. В частности, мы можем положить $\rho(x) = \overline{x}$ и отсюда получить, что $\overline{u} \equiv u \pmod{\lambda}$. Так как $v = u/\overline{u}$, а $v = \pm \zeta^n$, то есть $u = \pm \zeta^n \overline{u}$, мы имеем
		 	\[
		 		\pm \zeta^n = \overline{u} \equiv u \pmod{\lambda}.
		 	\]
		 	Предположим, что реализуется знак минус. Тогда, так как $\zeta^n \equiv 1 \pmod{\lambda}$, отсюда мы получаем
		 	\[
		 		- \overline{u} \equiv \overline{u} \pmod{\lambda} \implies 2 \overline{u} \equiv 0 \pmod{\lambda},
		 	\]
		 	а так как $\overline{u}$ обратим, отсюда $2 \divby \lambda = 1 - \zeta$. Но тогда $2^{p - 1} \divby (1 - \zeta)^{p - 1}$, а как мы уже видели в предложении~\ref{stm:11}, $((1 - \zeta)^{p - 1}) = p\cO_{K}$, то есть $2^{p - 1} \divby p$,  что даёт нам противоречие.  

			Значит, знак минус невозможен и реализуется случай 
		 	\[
		 		\zeta^n \overline{u}  \equiv \overline{u} \equiv u \pmod{\lambda}. 
		 	\]
		 	Тогда $\zeta^n \overline{u} = u$, значит $u \zeta^s = \zeta^{n + s}\overline{u}$. Попробуем подобрать такое $s$, что 
		 	\[
		 		u \zeta^{s} = \overline{\zeta^{n + s}\overline{u}} \implies \overline{\zeta^{n + s}} = \zeta^{s} \implies 2s + n \equiv 0 \pmod{p},
		 	\]
		 	и достаточно взять $s \equiv -n/2\pmod{p}$.
		 	
 		\end{proof}
 	\end{lemma}

 	\begin{lemma}\label{lemma:15}
 		Пусть $x^p + y^p = z^p, \ p \not \ \mid xyz, \ (x, y, z) = 1$, разложим левую часть в линейные множители:
 	\[
 		x^p + y^p = (x + y)(x + \zeta y)\ldots(x + \zeta^{p - 1}y). 
 	\]
 	Тогда сомножители в правой части равенства попарно взаимнопросты. 
 	\end{lemma}
 	\begin{proof}
 		Предположим противное, тогда $x + \zeta^i y, \ x + \zeta^{j} y \in \fq$ для некоторых $i, j$. Тогда 
 		\[
 			(x + \zeta^i y) - (x + \zeta^{j} y) = \zeta^i(1 - \zeta^{j - i})y \in \fq.
 		\]
 		\begin{enumerate}
 			\item Если $y \in \fq$, то, так как $1 + \zeta^i y \in \fq$, мы имеем $x \in \fq$, но по условию $(x, y) = (1)$.

 			\item Если $(1 - \zeta^{j - i}) \in \fq $, то $1 - \zeta \in \fq$, а так как $(1 - \zeta) \in \Specm(\cO_{K})$ (что мы доказывали в~\ref{stm:11}), $\fq = (1 - \zeta)$. Но тогда $x + y \in \fq \implies z^p \in \fq \implies z \in \fq$. Тогда 
 			\[
 				(z)^{p - 1} \divby \fq^{p - 1} = ((1 - \zeta))^{p - 1} = (p) \implies z \divby p,
 			\]
 			что даёт нам противоречие. 
 		\end{enumerate}
 	\end{proof}

 	Сейчас, пользуясь всей подготовкой выше, мы докажем первый случай большой теоремы Ферма для регулярных простых. 
 	\begin{theorem} 
 		Пусть $p \not \ \mid |\Class(\Q\lr*{\zeta_{p}})$\footnote{такие простые числа называются \emph{регулярными}}. Тогда имеет место первый случай Великой теоремы Ферма.  
 	\end{theorem}
 	\begin{proof}
 		Пусть $x^p + y^p = z^p$, разложим левую часть на множители: 
 		\[
 			\prod_{i = 0}^{p - 1} (x + \zeta^{i}y) = z^p.
 		\]
 		Так как все сомножители в левой части равенства взаимнопросты по лемме~\ref{lemma:15}, все они являются $p$-ми степенями, в частности для $i = 1$. То есть 
 		\[
 			(x + \zeta y) = I^{p},
 		\]
 		значит $I$ находится в $p$-кручении $\Class(\Q_{\zeta_{p}})$. Но так как $p$ не делит порядок группы классов, оно тривиально, значит  $I$--- главный, то есть  $I = (\alpha)$ для некоторого $\alpha$.  Значит,

 		\[
 			(x + \zeta y) = (\alpha^p) \implies x + \zeta y = \varepsilon \alpha^p, \ \text{ где }\varepsilon \in \Z[\zeta]^{*}, 
 		\]
 		Тогда по лемме~\ref{lemma:26} мы имеем 
 		\[
 			  x + \zeta y = \zeta^s u \alpha^p, \quad u \in \R.
 		\]

 		С другой стороны, $\alpha = a_0 + a_1 \zeta + \ldots + a_{p - 2}\zeta^{p - 2}, \ a_i \in \Z$. Тогда 
 		\[
 		 	\alpha^p \equiv \underbrace{a_0^p + a_1^p + \ldots + a_{p - 2}^{p}}_{\eqdef n} \pmod{p}.
 		 \] 

 		 Тогда $x + \zeta y \equiv \zeta^s u n \pmod{p}$. Переходя в этом сравнении к сопряженным, мы получаем 
 		 \[
 		 	\overline{x + \zeta y} = x + \zeta^{-1}y \equiv \zeta^{-s}\overline{u}n = \zeta^{-s}un \pmod{p} \implies \zeta^{s}(x + \zeta^{-1}y) \equiv un \pmod{p}.
 		 \]
 		 \[
 		 	\begin{cases} \zeta^{-s}(x + \zeta y) \equiv u n \pmod{p} \\ \zeta^{s}(x + \zeta^{-1}y) \equiv un \pmod{p} \end{cases} \implies \zeta^{-s}(x + \zeta y) \equiv \zeta^{s}(x + \zeta^{-1}y) \pmod{p} 
 		 \]
 		 \[
 		 	x + \zeta y \equiv \zeta^{2s}(x + \zeta^{-1}y) \pmod{p} \implies x + \zeta y - \zeta^{2s} x - \zeta^{2s - 1}y \in p\Z[\zeta]
 		 \]
 		 
 		 Теперь рассмотрим несколько случаев: 

 		 \begin{enumerate}
 		 	\item Элементы $S = \{ 1, \zeta, \zeta^{2s}, \zeta^{2s - 1}\}$ попарно различны. 
 		 	\begin{enumerate}
 		 		\item Если $\zeta^{p - 1} \notin S$, то $p \mid x, p \mid y$. 

 		 		\item Если $\zeta^{p - 1} = \zeta^{2s - 1}$, то $s \divby p$, а значит $(\zeta - \zeta^{-1})y \in p\Z[\zeta]$ откуда следует, что $p \mid 1 - \zeta^2$, что даёт нам противоречие. 

 		 		\item Если $\zeta^{p - 1} = -(1 + \zeta + \ldots + \zeta^{p - 2})= \zeta^{2s}$, а тогда 
 		 		\[
 		 			x + \zeta y + ((1 + \zeta + \ldots + \zeta^{p - 2}))x - \zeta^{p - 2}y = 2x + \zeta(x + y) + x\zeta^2 + \ldots + x\zeta^{p - 3}  + (x - y)\zeta^{p - 2} \implies 2x \divby p,
 		 		\]
 		 		что даёт нам противоречие. 
 		 	\end{enumerate}

 		 	\item Некоторые из этих степеней совпадают. 
 		 		\begin{enumerate}
 		 			\item $\zeta^{2s} = 1$. В этом случае $s \divby p$, а как мы уже видели, это влечёт $(\zeta - \zeta^{-1})y \ \divby \ p$, что невозможно. 

 		 			\item $\zeta^{2s - 1} = 1$.  В этом случае $x - y + \zeta y - \zeta x \divby p \implies (x - y)(1 - \zeta) \divby  p \implies x - y \divby  p$, то есть $x \equiv y \pmod{p}$. Исходное уравнение мы можем записать в виде 
 		 			\[
 		 				z^p + (-y)^p = x^p,
 		 			\]
 		 			и рассуждая аналогично, мы можем получить, что $(y + z) \divby p, \ (x + z) \divby p$. Тогда 
 		 			\[
 		 				0 \equiv x^p + y^p - z^p \equiv x + y - z \equiv 3x \pmod{p},
 		 			\]
 		 			откуда либо $p = 3$ (а в этом случае мы Большую теорему Ферма доказали), либо $p \mid x$, что даёт нам противоречие. 
 		

 		 			\item $\zeta = \zeta^{2s - 1}$. Тогда $x - \zeta^2 x \divby p$, откуда следует, что $1 - \zeta^2 \divby p$, а это, как мы уже видели, противоречие. 
 		 		\end{enumerate}
 		 \end{enumerate}
 		 \end{proof}

 		 Дадим теперь хороший критерий для проверки условия теоремы. Этот критерий мы дадим без доказательства, так как он доказывается методами аналитической теории чисел. 

 		 \begin{definition} 
 		 	Рассмотрим экспоненцальную прозводящую функцию 
 		 	\[
 		 		\frac{t}{e^t - 1} = \sum_{m = 0}^{\infty} B_{m} \frac{t^m}{m!},
 		 	\]
 		 	тогда 
 		 \end{definition}
 		 
 		 Подставим $-t$:
 		 \[
 		 	\frac{-t}{\frac{1}{e^{t}} - 1} = \frac{t e^{t}}{e^{t} - 1} = \frac{t(e^t - 1) + t}{e^t - 1} = t + \frac{t}{e^t - 1}.
 		 \]

 		 Отсюда мы можем понять, чему равны коэффициенты: 
 		 \[
 		 	m > 1, \ m \equiv_{2} 1 \implies B_m = -B_m \implies B_m = 0.
 		 \]
 		 \begin{remark}
 		 	Так как все нечётные коэффициенты равны нулю, авторы часто используют обозначение $B_n$ для $2n$-го числа Бернулли. Например, известно, что 
 		 \[
 		 	B_{n} = -n \zeta(1 - n), n > 1.
 		 \]
 		 \end{remark}
 		 Также отсюда мы имеем такую формулу: 
 		 \begin{statement} 
 		 	\[
 		 	-(n + 1)B_n = \binom{n + 1}{n - 1} B_{n - 1} + \ldots + \binom{n + 1}{k}B_{k} + \ldots + \binom{n + 1}{1} B_{1} + 1. 
 		 \]	
 		 \end{statement}
 		 \begin{corollary}
 		 	Знаменатели $B_{2}, B_{4}, \ldots, B_{p - 3}$ не делятся на $p$.
 		 \end{corollary}
 		 \begin{proof}
 		 	Докажем это утверждение по индукци, база очевидна, докажем переход. 
 		 	\[
 		 		\v_{p}\lr*{\binom{n + 1}{n - 1} B_{n - 1} + \ldots + \binom{n + 1}{k}B_{k} + \ldots + \binom{n + 1}{1} B_{1} + 1} \ge 0 \implies \v_{p}((n + 1)B_{n}) \ge 0 \implies \v_{p}(B_n) \ge 0.
 		 	\]
 		 \end{proof}

 		 Так вот, нам числа Бернулли полезны, так как справедлива такая теорема:
 		 \begin{theorem} 
 		 	Простое число $p$~--- регулярно тогда и только тогда, когда числители всех чисел Бернулли $B_{2}, B_{3}, \ldots, B_{p - 3}$ не делятся на $p$.
 		 \end{theorem}
 		 

 		 \begin{example}
 		 	Например, таким образом нетрудно показать, что число 7 является регулярным. Действительно, 
 		 	\[
 		 		\overline{B_0} = 1, \ \overline{B_1} = \overline{3}, \ -3\overline{B_2} = 3 \overline{B_1} + \overline{1} = \overline{10} \implies \overline{B_3} = -\frac{\overline{10}}{\overline{3}} = -\overline{1}, \ \overline{B_3} = 0, \ \overline{B_4} = 10\overline{B_2} + 5 \cdot \overline{3} + 1 = -\overline{1} \implies \overline{B_4} = \frac{\overline{1}}{\overline{5}} = \overline{3}. 
 		 	\]
 		 \end{example}

 	

 	Теперь, немного отвлечёмся и приведём алгоритм построение целого базиса.

 	\subsection{Алгоритм построения целого базиса}

 	Итак, для $d = \disc(1, \alpha, \alpha^2, \ldots, \alpha^{n - 1})$ мы наем, что 
 	$d \cO_{k} \subset \Z[\alpha] \subset \cO_{k}$, откуда 
 	\[
 		\Z[\alpha] \subset \cO_{K} \subset d^{-1}\Z[\alpha].
 	\]

 	Тогда, как мы помним, $\bigcup (\omega_i + \Z[\alpha]) = d^{-1}\Z[\alpha]$, причем, каждый класс, либо вообще не содержит целых алгебраических чисел, либо целиком из них состоит. То есть, отсюда мы имеем 
 	\[
 		\cO_{K} = \bigcup_{i \in I} (\omega_{i} + \Z[\alpha]),
 	\]
 	откуда следует, что $\cO_{K}$ порождена $\omega_{i}$ и $\alpha^s$ при $0 \le s \le n - 1$. Значит, $d \cO_{K}$ будет порождена  $d\omega_{i}$ и $d \alpha^s$. С другой стороны, $d\cO_{K}$~--- подгруппа свободной абелевой группы $\Z[\alpha]$, значит мы попадаем в контекст \hyperlink{smith_normal_form}{нормальной формы Смита}.

 	\begin{homework}

 	\begin{enumerate}
 		\item \begin{theorem}[Штикельберг] 
 			Дискриминант конечного расширения $K/\Q$ сравним с нулём или единицей по модулю 4. 
 		\end{theorem}
 		\emph{Hint:} Можно действовать так: $\disc(K) = \lr*{\det(\sigma_i)}^2$, и раскрывая определитель, мы получаем $\det(\sigma_{j}\omega_{j}) = P - N$, где $P$~--- сумма произведений со знаком +, а $N$~--- сумма произведений со знаком минус. Тогда $\disc(K) = (P - N)^2 = (P + N)^2 - 4 PN$. Значит, достаточно показать, что числа $P + N$ и $PN$~--- целые.  \emph{Hint:} Целое число~--- это рациональное число, которое еще и целое алгебраическое. 
 		\item
 	\end{enumerate}
 		
 	\end{homework}





	