\subsection{Алгебраические числа и целые алгебраические числа}

	\begin{definition} 
		Число $\alpha \in \C$ называется \emph{алгебраическим}, если существует $p \in \Z[x]$, аннулирующий $\alpha$. 
	\end{definition}

	\begin{remark}
		Это частный случай общей терминологии, тут речь о том, что $\alpha$ алгебраичен над $\Q$.
	\end{remark}

	\begin{statement} 
		Пусть $\alpha \in \C$. Тогда следующие утверждения эквивалентны: 
		\begin{enumerate}
			\item $\alpha$~--- алгебраическое. 
			\item $\Q[\alpha]$~--- конечномерное векторное пространство над $\Q$.
		\end{enumerate}
	\end{statement}

	\begin{proof}
		$\mathbf{(1) \implies (2)}$: очевидно, так как если $\alpha$~--- алгебраичен над $\Q$,  базисом $\Q[\alpha]$ над $\Q$ будет множество $\{ 1, \alpha, \ldots, \alpha^{n - 1} \}$.

		$\mathbf{(2) \implies (1)}$: действительно, если $\dim_{\Q} \Q[\alpha] = n$, то $1, \alpha, \ldots, \alpha^n$ линейно зависимы, то есть $\exists a_0, \ldots, a_n \in \Q$:
		\[
			a_n \alpha^n + \ldots + a_1 \alpha + a_0 = 0.
		\]
		Домножая на знаменатель, мы имеем нужный многочлен. 
	\end{proof}

	\begin{statement} 
		Множество алгебраических чисел является полем. 
	\end{statement}
	\begin{proof}
		Пусть $\alpha$~--- алгебраическое число. Тогда 
		\[
			a_n \alpha^n + a_{n - 1}\alpha^{n - 1} + \ldots + a_1 \alpha + a_0 = 0, \ a_i \in \Z.
		\]
		Но тогда $a_n + \ldots + a_1 \lr*{\alpha^{-1}}^{n - 1} + a_0 \lr*{\alpha^{-1}}^{n} = 0$, то есть $\alpha^{-1}$ алгебраическое. Теперь, пусть $\alpha$ и $\beta$ алгебраические. Тогда 
		$\dim_{\Q}\Q[\alpha, \beta] < \infty \implies \dim_{\Q}\Q[\alpha \beta], \ \dim_{\Q}\Q[\alpha + \beta] < \infty$.    

	\end{proof}

	\begin{remark}
		Искушенный читатель сразу заметит, что это поле~--- это в точности $\Q^{alg}$.
	\end{remark}

	\begin{definition} 
		$\alpha \in \C$ мы будем называть \emph{целым алгебраическим числом}, если существует унитарный многочлен $p \in \Z[x]$, аннулирующий $\alpha$.
	\end{definition}

	\begin{example}
		$\sqrt{2}$~--- целое алгебраическое число, а вот $\sqrt{2}/2$~--- нет!
	\end{example}

	\begin{statement} 
		Следующие утверждения эквивалентны: 
		\begin{enumerate}
			\item $\alpha$~--- целое алгебраическое число. 
			\item $\Z[\alpha]$~--- конечно-порожденный $\Z$-модуль. 
		\end{enumerate}
	\end{statement}

	\begin{proof}
		Опять же, $\mathbf{(1) \implies (2)}$ следует просто из того, что если $\alpha$~--- целое алгебраическое, то $\{ 1, \ldots, \alpha^{n - 1} \}$~--- базис $\Z[\alpha]$ над $\Z$. 

		Теперь докажем $\mathbf{(2) \implies (1)}$. Ясно, что все образующие $\Z[\alpha]$ над $\Z$~--- многочлены от $\alpha$, пусть они $p_1(\alpha), \ldots, p_m(\alpha)$. Пусть $N = \max{\deg(p_i)}$, тогда 
		\[
			\alpha^{N + 1} = \sum_{i = 1}^{m} a_i p_i(\alpha), \quad \alpha^{N + 1} - \sum_{i = 1}^{m} a_i p_i(\alpha) = 0.
		\]
	\end{proof}

	\begin{theorem} 
		Множество целых алгебраических чисел является кольцом. 
	\end{theorem}
	\begin{proof}
		Возьмём $\alpha, \beta$~--- целые алгебраические. Тогда по предыдущему предложению $\Z[\alpha, \beta]$~--- конечнопорожденный $\Z$-модуль, а так как $\Z$~--- нётерово, тогда подумодули $\Z[\alpha + \beta]$ и $\Z[\alpha \beta]$ конечнопорождены, откуда $\alpha\beta$ и $\alpha + \beta$ целые алгебраические (также по предыдущему предложению). 
	\end{proof}

	Обозначим кольцо целых алгебраических чисел, как $\cO$. В основном в этом курсе мы будем изучать подкольца в $\cO$, а именно

	\begin{definition} 
		Пусть $K/\Q$~--- конечное расширение. Тогда 
		\[
			\cO_{K} \eqdef \cO \cap K
		\]
		мы будем называть \emph{кольцом целых} числового поля $K$. Иными словами, $\cO_{K}$~--- множество элементов $K$, для которых существует унитарный целочисленный многочлен, аннулирующий их. 
	\end{definition}

	\subsection{След элемента и целый базис кольца $\cO_{K}$}

	Заведём теперь некоторый полезный аппарат. 

	\begin{definition} 
		Пусть $L/K$~--- конечное расширение, $[L : K] = n$. Возьмём $\alpha \in L$, его можно рассматривать, как эндоморфизм понятным образом
		\[
			T_{\alpha}\colon L \to L, \quad x \mapsto \alpha x. 
		\]

		Соотвественно, след этого оператора называют следом элемента $\alpha$ относительно расширения $L/K$ и обозначают $\Tr_{L/K}(\alpha)$.

		У этого оператора есть характеристический многочлен $\chi_{\alpha}$. Выбрав базис $L/K$, мы можем записать матрицу оператора $T_{\alpha}$  и тогда 
		\[
			\chi_{\alpha}(t) = \det\lr*{Et - T_{\alpha}} = t^n - \Tr_{L/K}(\alpha)t^{n - 1} + \ldots 
		\]
		Если $L/K$~--- расширение Галуа, то можно определять след, как 
		\[
			\Tr_{L/K}(\alpha) = \sum_{\sigma \in \Gal(L/K)} \sigma(\alpha).
		\]
	\end{definition}

	Соотвественно, след~--- это $K$-линейный функционал $L \to K$, то есть 
	\[
		\forall \alpha, \beta \in K \quad \Tr_{L/K}(\alpha a + \beta b) = \alpha \Tr_{L/K}(a) + \beta \Tr_{L/K}(b).
	\]
	Кроме того, для $\alpha \in K$ $\Tr_{L/K}(\alpha) = [L : K] \cdot \alpha$. Кроме того, след хорошо ведёт себя относительно башни расширений. Если $M$~--- расширение $K$, а $K$~--- расширение $L$, то 
	\[
		\Tr_{M/K} = \Tr_{M/L} \circ \Tr_{L/K}.
	\]

	Кроме того, след можно рассматривать и как невырожденную билинейную симметричную форму 
	\[
		K \times K \to \Q, \ (x, y) \mapsto \Tr_{K/\Q}(xy).
	\]

	\begin{remark}
		Если $\alpha \in \cO_{K}$, то $\Tr_{K/\Q}(\alpha) \in \Z$. Действительно, во-первых, $\sigma(\alpha) \in \cO_k \ \forall \sigma \in \Gal(K/\Q)$, так как если $\alpha \in \cO_{K}$, то 
		\[
			\alpha^n + a_{n - 1} \alpha^{n - 1} + \ldots + a_0 = 0 \implies (\sigma\alpha)^n + a_{n - 1} (\sigma \alpha)^{n - 1} + a_0 = 0 \implies \sigma\alpha \in \cO_{K}
		\]

		, откуда $\Tr_{K/\Q}(\alpha) \in \cO_{K}$.

		С другой стороны, по первому определению $\Tr_{K/\Q} \in \Q$, а $\cO_{K} \cap \Q = \Z$. 
	\end{remark}

	\begin{statement}
		Любой элемент поля $K$ представим в виде $\frac{\beta}{d}$, где $\beta \in \cO_{K}$, $m \in \Z$. Иными словами, $K$~--- поле частных кольца $\cO_{K}$. 
	\end{statement}
	\begin{proof}
		Во-первых, $\alpha$ является корнем некоторого унитарного многочлена с коэффициентами из $\Q$:
		\[
			\alpha^n + c_{n - 1}\alpha^{n - 1} + \ldots + c_{1}\alpha + c_0 = 0, \quad c_i \in \Q.
		\]
		Запишем $c_i = \frac{b_i}{d}$, $b_i, d \in \Z$. Тогда, домножив равенство выше на $d^n$, мы получаем  
		\[
			(\alpha d)^n + b_{n - 1}(\alpha d)^{n - 1} + b_{n - 2} d (\alpha d)^{n - 2} + \ldots + b_0 d^{n - 1} = 0.
		\]
		Соотвественно, полагая $\beta = d\alpha$ мы видим, что $\beta \in \cO_{K}$ и $\alpha = \beta / d$. 
	\end{proof}

	Так вот, возьмём базис $K/\Q$. Из предыдущего предложения ясно, что можно полагать, что этот базис состоит из элементов $\cO_{K}$. Обозначим их за $\omega_1, \ldots, \omega_n$. Выберем для этого базиса взаимный базис $\omega_1^*, \ldots, \omega_n^*$ относительно формы $\Tr_{K/\Q}$, т.е. такой базис, что 
	\[
		\Tr(\omega_i \omega_j^*) = \begin{cases} 0, i \neq j \\ 1, i = j \end{cases} = \delta_{i j}.
	\]
	Покажем, что выполнено 
	\[
		\bigoplus_{i} \Z \omega_i \subset \cO_{K} \subset \bigoplus_{i} \Z \omega_i^{*}.
	\]

	Первое включение очевидно, докажем второе. Возмём $\alpha \in \cO_{K}$,
	\[
	 	\alpha = \sum_{i = 1}^{n} x_i \omega_i^{*}, \quad x_i \in \Q.
	 \] 
	 Покажем, что на самом деле $x_i \in \Z$. 
	 \[
	 	\alpha \omega_j = \sum_{i = 1}^{n} x_i \omega_j \omega_i^{*} \implies \Tr_{K/\Q}(\alpha \omega_j) = x_j. 
	 \]
	 С другой стороны, так как $\alpha \omega_j \in \cO_{K}$, $\Tr_{K/\Q}(\alpha \omega_j) \in \Z$ (как мы отвечали выше). Таким образом, мы имеем  
	 \[
		\bigoplus_{i} \Z \omega_i \subset \cO_{K} \subset \bigoplus_{i} \Z \omega_i^{*}.
	\]
	Так как слева и справа конечнопорождённые абелевы группы ранга $n$, мы только что доказали такую теорему: 

	\begin{theorem}\label{integral_basis_O_K} 
		Пусть $\cO_{K}$~--- кольцо целых числового поля $K/\Q$, где $K/\Q$~--- расширение степени $n$. Тогда, как абелева группа оно изоморфно конечнопорожденной свободной абелевой группе ранга $n$:
		\[
			\cO_{K} \cong \bigoplus_{i = 1}^{n} \Z u_i,
		\]
		где $\{ u_i \}$~--- базис $K/\Q$, состоящий из элементов кольца $\cO_{K}$. 

		В данном контексте $\{ u_i \}_{i = 1}^{n}$ называют \emph{целым базисом}. 
	\end{theorem}

	Из этой теоремы сразу следует вот такой факт:

	\begin{theorem} 
		Кольцо $\cO_{K}$~--- нётерово. 
	\end{theorem}
	\begin{proof}
		В самом деле, по теореме~\ref{integral_basis_O_K} кольцо $\cO_{K}$ конечно порождена, как абелева группа, а значит, любой его идеал $I \subset \cO_K$ тоже конечно порожден, как абелева группа, откуда следует, что он конечно порожден и как идеал. 
	\end{proof}

	\begin{example}\label{ring_of_integers_Q(sqrt(-3))}
		Рассмотрим поле $K = \Q(\sqrt{-3})$. Чему равно его кольцо целых $\cO_{K}$?

		Ясно, что $\Z[\sqrt{-3}] \subset \cO_{K}$, но вот равенства нет, так как можно рассмотреть 
		\[
			\alpha = \frac{1 + \sqrt{-3}}{2}, \quad 2\alpha - 1 = \sqrt{-3} \implies 4\alpha^2 - 4\alpha + 4 = 0 \implies \alpha^2 - \alpha + 1 = 0,		
		\]
		то есть $\alpha \in \cO_{K}$ и $\alpha \notin \Z[\sqrt{-3}]$. 
	

	Воспользуемся понятием \emph{нормы}:

	\begin{definition} 
		Пусть $L/K$~--- конечное расширение, $[L : K] = n$. Возьмём $\alpha \in L$, его можно рассматривать, как эндоморфизм понятным образом
		\[
			T_{\alpha}\colon L \to L, \quad x \mapsto \alpha x. 
		\]
		\emph{Нормой элемента $\alpha$ относительно расширения $L/K$} мы будем называть $\Nm_{L/K}(\alpha) = \Nm(\alpha) = \det(T_{\alpha})$. В случае, когда расширение сепарабельно, норму можно определять, как 
		\[
			\Nm(\alpha) = \prod_{\sigma \in \Gal(L/K)} \sigma(\alpha).
		\]
	\end{definition}

	\begin{remark}
		Как и в случае со следом, для $\alpha \in \cO_{K}$ $\Nm_{K/\Q}(\alpha) \in \cO_{K}$ (доказывается это так же, как для следа), а из определения через определитель ясно, что $\Nm_{K/\Q}(\alpha) \in \Q$, то есть $\Nm_{K/\Q}(\alpha) \in \Z$.
	\end{remark}

	Ясно, что в случае квадратичного расширения $\Q(\sqrt{d})$ норма $a + b\sqrt{d}$ норма элемента~--- произведение его на его сопряженный:
	\[
		\Nm(a + b\sqrt{d}) = (a + b\sqrt{d}))(a - b\sqrt{d}) = a^2 - d b^2
	\]

	Соотвественно, рассмотрим $a + b\sqrt{-3} \in \Q(\sqrt{-3})$ (т.е. $a, b \in \Q$) и пусть $\alpha = a + b\sqrt{-3} \in \cO_{K}$. Тогда 
	\[
		\Nm_{K/\Q}(a + b\sqrt{-3}) = (a + b\sqrt{-3})(a - b\sqrt{-3}) = a + 3b^3 \in \Z, \quad \Tr_{K/\Q}(a + b\sqrt{-3}) = (a + b \sqrt{-3}) + (a - b\sqrt{-3}) = 2a \in \Z. 
	\]

	Соответственно, $2a \in \Z$, то есть или $a = n/2$, где $n \in \Z$ и нечётное, или $a \in \Z$. 

	\begin{enumerate}
		\item Пусть $a \in \Z$, тогда так как $a^2 + 3b^2 \in \Z$, $3b^2 \in \Z \implies b \in \Z$. 

		\item Пусть $2a = 2n + 1$, тогда $4(a^2 + 3b^2) = 4a^2 + 12b^2 \in 4\Z \implies 12b^2 \in 4\Z$, откуда $2b \in \Z$. 
	\end{enumerate}

	Значит, либо $a$ и $b$ одновременно целые, либо $a$ и $b$ одновременно полуцелые. То есть 
	\[
	 	\alpha = \frac{2n + 1}{2} + \frac{2m + 1}{2}\sqrt{-3} = (n + m\sqrt{-3}) + \frac{1 + \sqrt{-3}}{2} = (n - m) + (2m + 1)\frac{1 +\sqrt{-3}}{2},
	 \] 
	 откуда $\cO_{K} \cong \Z \oplus \Z \cdot \frac{1 + \sqrt{-3}}{2}$.

	\end{example}

	\begin{example}
		Если $K = \Q(i)$, то $\cO_{K} = \Z[i]$.
	\end{example}

	\begin{homework}\label{hw_1}
		Задачи:
		\begin{enumerate}
			\item Опишите $\cO_{K}$ для $K = \Q(\sqrt{d})$, где $d \in \Z$ и $d$ свободно от квадратов. 

			\item Докажите, что любое конечное целостное кольцо является полем. 

			\item Рассмотрим поле $K = \Q(\sqrt{-5})$, $\cO_{K} = \Z[\sqrt{-5}]$ и идеал $I = (2, 1 + \sqrt{-5})$. Покажите, что он не является главным. 

			\item Докажите, что кольцо $\Z[\sqrt{-5}]$  не факториальное. А именно, рассмотрите
			\[
				21 = 3 \cdot 7 = (1 + 2\sqrt{-5})(1 - 2\sqrt{-5})
			\]
			и покажите: что это два существенно различных разложения в произвдение простых. 
		\end{enumerate}
	\end{homework}

	\subsection{Размерность кольца целых $\cO_{K}$}

	Докажем теперь, что для любого числового поля $K$ кольцо целых $\cO_{K}$ одномерно. 

	\begin{lemma}\label{integer_in_ideal_of_O_K} 
		Пусть $K/\Q$~--- конечное расширение и  $0 \neq I$~--- идеал кольца $\cO_{K}$. Тогда $I \cap \Z \neq 0$, то есть $I$ содержит целое число. 
	\end{lemma}

	\begin{proof}
		Возьмём $\alpha \in I$, $\alpha \neq 0$. Тогда 
		\[
			\alpha^n + a_{n - 1} \alpha^{n - 1} + \ldots + a_{1}\alpha + a_0 = 0, \quad a_i \in \Z.
		\]
		Не умаляя общности, $a_0 \neq 0$. Но тогда 
		\[
			a_0 = - (\alpha^n + a_{n - 1} \alpha^{n - 1} + \ldots + a_{1}\alpha) \implies a_0 \in \Z \cap I.
		\]
	\end{proof}

	\begin{corollary}\label{O_K/I}
		Пусть $I$~--- ненулевой идеал в $\cO_{K}$. Тогда $\cO_{K}/I$ конечно. 
	\end{corollary}
	\begin{proof}
		По лемме~\ref{integer_in_ideal_of_O_K} выберем $n \in I \cap \Z$, $n \neq 0$. Тогда 
		$(n) = n\cO_{K} \lei I$, значит достаточно доказать, что $\cO_{K}/n\cO_{K}$ конечно. А это сразу же следует из того, что $\cO_{K}$~--- это конечнопорожденная свободная абелева группа ранга $n$. 
	\end{proof}
	
	\begin{theorem} 
		Кольцо $\cO_{K}$ одномерно, т.е. $\dim(\cO_{K}) = 1$. 
	\end{theorem}

	\begin{proof}
		Пусть $\fp \in \Spec(\cO_{K})$. Тогда $\cO_{K}/\fp$~--- область целостности и конечно по  следствию~\ref{O_K/I}. Но тогда по задаче 2 в~\ref{hw_1} $\cO_{K}/\fp$~--- поле, что равносильно тому, что $\fp$~--- максимальынй. 
	\end{proof}

	\begin{remark}
		Эквивалентная формулировка этой теоремы состоит в том, что любой ненулевой простой идеал  кольца $\cO_{K}$ является максимальным. 
	\end{remark}

	Поговорим теперь еще про строение идеалов в кольце $\cO_{K}$. Как мы уже убеждались в задаче 3 Д/З~\ref{hw_1}, кольцо $\cO_{K}$ далеко не всегда является областью главных идеалов. 

	\subsection{Примеры евклидовых колец целых алгебраических чисел}

	\begin{statement} 
		Рассмотрим $K = \Q(\sqrt{-3})$. Тогда $\cO_{K}$~--- евклидово. 
	\end{statement}

	\begin{proof}
		Как мы убедились в примере~\ref{ring_of_integers_Q(sqrt(-3))}, 
		\[
			\cO_{K} \cong \Z \oplus \Z \omega, \quad \omega = \frac{1 + \sqrt{-3}}{2}. 
		\]
		Рассмотрим $a + b \omega$, тогда  положим
		\[
			\Nm(\alpha) = (a + b\omega)(a + b\overline{\omega}) = a^2 + ab + b^2.
		\]

		Пусть $a, b, c, d \in \Z$, тогда
		\[
			\frac{a + b\omega}{c + d\omega} = \alpha + \beta \omega, \quad \alpha, \beta \in \Q.
		\]

		Тогда существуют $u, v \in \Z$ такие, что $|u - \alpha| \le \frac{1}{2}$, $|v - \beta| \le \frac{1}{2}$. Положим $\alpha - u = \alpha'$, $\beta - v = \beta'$.

		\[
			a + b\omega = (c + d\omega)(\alpha + \beta \omega) = (c + d\omega)(u + v\omega) + (c + d\omega)(\alpha' + \beta' \omega) = (c + d \omega)(u + v \omega)  + r.
		\]
		\[
			\Nm(r) = \Nm((c + d\omega)(\alpha' + \beta' \omega)) = \Nm(c + d \omega)\Nm(\alpha' + \beta' \omega) = \Nm(c + d\omega)(\alpha'^2 + \alpha'\beta' + \beta'^2) < \Nm(c + d \omega),
		\]
		так как $\alpha'^2 + \alpha'\beta' + \beta'^2 \le \frac{1}{4} + \frac{1}{4} + \frac{1}{4} = \frac{3}{4}$, что и требовалось. 
			
	\end{proof}

	Сейчас мы посмотрим 
	\begin{statement} 
		Уравнение $y^2 = x^3 - 2$ над $\Z$ в качестве решений имеет лишь $(3, \pm 5)$.
	\end{statement}
	\begin{proof}
		Во-первых заметим, что можно сразу полагать $y$ нечётным. 

		Попробуем решить уравнение в $\Z[\sqrt{-2}] = \cO_{K}$ для $K = \Q(\sqrt{-2})$. 
		\[
			(y + \sqrt{-2})(y - \sqrt{-2}) = x^3.
		\]
		Пусть $d = (y + \sqrt{0-2}, y - \sqrt{-2})$. Покажем, что $d = 1$. Действительно, 
		\[
			\begin{cases} y + \sqrt{-2} \divby d \\ y - \sqrt{-2} \divby d \end{cases}	\implies \begin{cases} 2y \divby d \\ 2\sqrt{-2} \divby d \end{cases}.
		\]	
		Заметим, что $2\sqrt{-2} \divby d \implies \Nm(2 \sqrt{-2}) \divby \Nm(d)$. Пусть $d = a + b \sqrt{-2}, \ a, b \in \Z$. Тогда  мы имеем $8 \divby (a^2 + 2b^2)$. 
		
		Т.е. $a^2 + 2b^2 = 1, 2, 4$ или 8. Разберём соотвествующие случаи: 
		\begin{enumerate}
			\item $a^2 + 2b^2 = 1  \rightsquigarrow a = \pm 1,\  b = 0$. 

			\item $a^2 + 2b^2 = 2 \rightsquigarrow a = 0, \ b = \pm 1$. 

			\item $a^2 + 2b^2 = 4 \rightsquigarrow a = \pm 2, \ b = \pm 0$. 

			\item $a^2 + 2b^2 = 8 \rightsquigarrow a = \pm 0, \ b = \pm 2$. 
		\end{enumerate}

		Заметим, что так как $y$~--- нечётное целое, 
		\[
			\Nm(y + \sqrt{-2}) = y^2 + 2 \notdivby \Nm(\sqrt{-2}) = 2,
		\]
		откуда следует, что случаи $(2)$ и $(4)$ нам не годятся. Случай $(3)$ нам не подходит просто в силу того, что $y$ нечётное. 

		Значит, мы доказали, что $y + \sqrt{-2}$ и $y - \sqrt{-2}$~--- взаимнопросты, а так как кольцо $\Z[\sqrt{-2}]$ факториально, отсюда мы имеем, что 
		\[
			y + \sqrt{-2} = z^3, \quad y - \sqrt{-2} = t^3. 
		\]
		Пусть опять же $z = a + b \sqrt{-2}$. Честно возведём в куб: 
		\[
			y + \sqrt{-2} = (a + b\sqrt{-2})^3 = a^3 + 3a^2 b \sqrt{-2} - 6 a b^2 - 2 b^3 \sqrt{-2},
		\]
		октуда мы имеем 
		\[
			\begin{cases} a^3 - 6ab^2 = y \\ 3a^2 b - 2b^3 = 1 \end{cases}
		\]
		Соотвественно, из второго уравнения ясно, что $b = \pm1$, откуда либо $3a^2 = 3 \implies a = \pm 1$, либо $3a^2 = -1$, чего быть не может ($a \in \Z$). Соотвественно, мы получили, что 
		\[
			y = a^3 - 6ab^2 = \pm 5 \implies y = \pm 5.
		\]
	\end{proof}






