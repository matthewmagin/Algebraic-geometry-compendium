\subsection{Алгебраические числа и целые алгебраические числа}

	\begin{definition} 
		Число $\alpha \in \C$ называется \emph{алгебраическим}, если существует $p \in \Z[x]$, аннулирующий $\alpha$. 
	\end{definition}

	\begin{remark}
		Это частный случай общей терминологии, тут речь о том, что $\alpha$ алгебраичен над $\Q$.
	\end{remark}

	\begin{statement} 
		Пусть $\alpha \in \C$. Тогда следующие утверждения эквивалентны: 
		\begin{enumerate}
			\item $\alpha$~--- алгебраическое. 
			\item $\Q[\alpha]$~--- конечномерное векторное пространство над $\Q$.
		\end{enumerate}
	\end{statement}

	\begin{proof}
		$\mathbf{(1) \implies (2)}$: очевидно, так как если $\alpha$~--- алгебраичен над $\Q$,  базисом $\Q[\alpha]$ над $\Q$ будет множество $\{ 1, \alpha, \ldots, \alpha^{n - 1} \}$.

		$\mathbf{(2) \implies (1)}$: действительно, если $\dim_{\Q} \Q[\alpha] = n$, то $1, \alpha, \ldots, \alpha^n$ линейно зависимы, то есть $\exists a_0, \ldots, a_n \in \Q$:
		\[
			a_n \alpha^n + \ldots + a_1 \alpha + a_0 = 0.
		\]
		Домножая на знаменатель, мы имеем нужный многочлен. 
	\end{proof}

	\begin{statement} 
		Множество алгебраических чисел является полем. 
	\end{statement}
	\begin{proof}
		Пусть $\alpha$~--- алгебраическое число. Тогда 
		\[
			a_n \alpha^n + a_{n - 1}\alpha^{n - 1} + \ldots + a_1 \alpha + a_0 = 0, \ a_i \in \Z.
		\]
		Но тогда $a_n + \ldots + a_1 \lr*{\alpha^{-1}}^{n - 1} + a_0 \lr*{\alpha^{-1}}^{n} = 0$, то есть $\alpha^{-1}$ алгебраическое. Теперь, пусть $\alpha$ и $\beta$ алгебраические. Тогда 
		$\dim_{\Q}\Q[\alpha, \beta] < \infty \implies \dim_{\Q}\Q[\alpha \beta], \ \dim_{\Q}\Q[\alpha + \beta] < \infty$.    

	\end{proof}

	\begin{remark}
		Искушенный читатель сразу заметит, что это поле~--- это в точности $\Q^{alg}$.
	\end{remark}

	\begin{definition} 
		$\alpha \in \C$ мы будем называть \emph{целым алгебраическим числом}, если существует унитарный многочлен $p \in \Z[x]$, аннулирующий $\alpha$.
	\end{definition}

	\begin{example}
		$\sqrt{2}$~--- целое алгебраическое число, а вот $\sqrt{2}/2$~--- нет!
	\end{example}

	\begin{statement} 
		Следующие утверждения эквивалентны: 
		\begin{enumerate}
			\item $\alpha$~--- целое алгебраическое число. 
			\item $\Z[\alpha]$~--- конечно-порожденный $\Z$-модуль. 
		\end{enumerate}
	\end{statement}

	\begin{proof}
		Опять же, $\mathbf{(1) \implies (2)}$ следует просто из того, что если $\alpha$~--- целое алгебраическое, то $\{ 1, \ldots, \alpha^{n - 1} \}$~--- базис $\Z[\alpha]$ над $\Z$. 

		Теперь докажем $\mathbf{(2) \implies (1)}$. Ясно, что все образующие $\Z[\alpha]$ над $\Z$~--- многочлены от $\alpha$, пусть они $p_1(\alpha), \ldots, p_m(\alpha)$. Пусть $N = \max{\deg(p_i)}$, тогда 
		\[
			\alpha^{N + 1} = \sum_{i = 1}^{m} a_i p_i(\alpha), \quad \alpha^{N + 1} - \sum_{i = 1}^{m} a_i p_i(\alpha) = 0.
		\]
	\end{proof}

	\begin{theorem} 
		Множество целых алгебраических чисел является кольцом. 
	\end{theorem}
	\begin{proof}
		Возьмём $\alpha, \beta$~--- целые алгебраические. Тогда по предыдущему предложению $\Z[\alpha, \beta]$~--- конечнопорожденный $\Z$-модуль, а так как $\Z$~--- нётерово, тогда подумодули $\Z[\alpha + \beta]$ и $\Z[\alpha \beta]$ конечнопорождены, откуда $\alpha\beta$ и $\alpha + \beta$ целые алгебраические (также по предыдущему предложению). 
	\end{proof}

	Обозначим кольцо целых алгебраических чисел, как $\cO$. В основном в этом курсе мы будем изучать подкольца в $\cO$, а именно

	\begin{definition} 
		Пусть $K/\Q$~--- конечное расширение. Тогда 
		\[
			\cO_{K} \eqdef \cO \cap K
		\]
		мы будем называть \emph{кольцом целых} числового поля $K$. Иными словами, $\cO_{K}$~--- множество элементов $K$, для которых существует унитарный целочисленный многочлен, аннулирующий их. 
	\end{definition}

	\subsection{След элемента и целый базис кольца $\cO_{K}$}

	Заведём теперь некоторый полезный аппарат. 

	\begin{definition} 
		Пусть $L/K$~--- конечное расширение, $[L : K] = n$. Возьмём $\alpha \in L$, его можно рассматривать, как эндоморфизм понятным образом
		\[
			T_{\alpha}\colon L \to L, \quad x \mapsto \alpha x. 
		\]

		Соотвественно, след этого оператора называют следом элемента $\alpha$ относительно расширения $L/K$ и обозначают $\Tr_{L/K}(\alpha)$.

		У этого оператора есть характеристический многочлен $\chi_{\alpha}$. Выбрав базис $L/K$, мы можем записать матрицу оператора $T_{\alpha}$  и тогда 
		\[
			\chi_{\alpha}(t) = \det\lr*{Et - T_{\alpha}} = t^n - \Tr_{L/K}(\alpha)t^{n - 1} + \ldots 
		\]
		Если $L/K$~--- расширение Галуа, то можно определять след, как 
		\[
			\Tr_{L/K}(\alpha) = \sum_{\sigma \in \Gal(L/K)} \sigma(\alpha).
		\]
	\end{definition}

	Соотвественно, след~--- это $K$-линейный функционал $L \to K$, то есть 
	\[
		\forall \alpha, \beta \in K \quad \Tr_{L/K}(\alpha a + \beta b) = \alpha \Tr_{L/K}(a) + \beta \Tr_{L/K}(b).
	\]
	Кроме того, для $\alpha \in K$ $\Tr_{L/K}(\alpha) = [L : K] \cdot \alpha$. Кроме того, след хорошо ведёт себя относительно башни расширений. Если $M$~--- расширение $K$, а $K$~--- расширение $L$, то 
	\[
		\Tr_{M/K} = \Tr_{M/L} \circ \Tr_{L/K}.
	\]

	Кроме того, след можно рассматривать и как невырожденную билинейную симметричную форму 
	\[
		K \times K \to \Q, \ (x, y) \mapsto \Tr_{K/\Q}(xy).
	\]

	\begin{remark}
		Если $\alpha \in \cO_{K}$, то $\Tr_{K/\Q}(\alpha) \in \Z$, так как во-первых, $\sigma(\alpha) \in \cO_k \ \forall \sigma \in \Gal(K/\Q)$, откуда $\Tr_{K/\Q}(\alpha) \in \cO_{K}$.

		С другой стороны, по первому определению $\Tr_{K/\Q} \in \Q$, а $\cO_{K} \cap \Q = \Z$. 
	\end{remark}

	\begin{statement}
		Любой элемент поля $K$ представим в виде $\frac{\beta}{d}$, где $\beta \in \cO_{K}$, $m \in \Z$. Иными словами, $K$~--- поле частных кольца $\cO_{K}$. 
	\end{statement}
	\begin{proof}
		Во-первых, $\alpha$ является корнем некоторого унитарного многочлена с коэффициентами из $\Q$:
		\[
			\alpha^n + c_{n - 1}\alpha^{n - 1} + \ldots + c_{1}\alpha + c_0 = 0, \quad c_i \in \Q.
		\]
		Запишем $c_i = \frac{b_i}{d}$, $b_i, d \in \Z$. Тогда, домножив равенство выше на $d^n$, мы получаем  
		\[
			(\alpha d)^n + b_{n - 1}(\alpha d)^{n - 1} + b_{n - 2} d (\alpha d)^{n - 2} + \ldots + b_0 d^{n - 1} = 0.
		\]
		Соотвественно, полагая $\beta = d\alpha$ мы видим, что $\beta \in \cO_{K}$ и $\alpha = \beta / d$. 
	\end{proof}

	Так вот, возьмём базис $K/\Q$. Из предыдущего предложения ясно, что можно полагать, что этот базис состоит из элементов $\cO_{K}$. Обозначим их за $\omega_1, \ldots, \omega_n$. Выберем для этого базиса взаимный базис $\omega_1^*, \ldots, \omega_n^*$ относительно формы $\Tr_{K/\Q}$, т.е. такой базис, что 
	\[
		\Tr(\omega_i \omega_j^*) = \begin{cases} 0, i \neq j \\ 1, i = j \end{cases} = \delta_{i j}.
	\]
	Покажем, что выполнено 
	\[
		\bigoplus_{i} \Z \omega_i \subset \cO_{K} \subset \bigoplus_{i} \Z \omega_i^{*}.
	\]

	Первое включение очевидно, докажем второе. Возмём $\alpha \in \cO_{K}$,
	\[
	 	\alpha = \sum_{i = 1}^{n} x_i \omega_i^{*}, \quad x_i \in \Q.
	 \] 
	 Покажем, что на самом деле $x_i \in \Z$. 
	 \[
	 	\alpha \omega_j = \sum_{i = 1}^{n} x_i \omega_j \omega_i^{*} \implies \Tr_{K/\Q}(\alpha \omega_j) = x_j. 
	 \]
	 С другой стороны, так как $\alpha \omega_j \in \cO_{K}$, $\Tr_{K/\Q}(\alpha \omega_j) \in \Z$ (как мы отвечали выше). Таким образом, мы имеем  
	 \[
		\bigoplus_{i} \Z \omega_i \subset \cO_{K} \subset \bigoplus_{i} \Z \omega_i^{*}.
	\]
	Так как слева и справа конечнопорождённые абелевы группы ранга $n$, мы только что доказали такую теорему: 

	\begin{theorem} 
		Пусть $\cO_{K}$~--- кольцо целых числового поля $K/\Q$. Тогда, как абелева группа оно изоморфно конечнопорожденной абелевой группе ранга $n$:
		\[
			\cO_{K} \cong \bigoplus \Z u_i.
		\]

		В данном контексте $\{ u_i \}$ называют \emph{целым базисом}. 
	\end{theorem}




