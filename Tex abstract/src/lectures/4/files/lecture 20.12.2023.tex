
	\subsection{Теорема Минковского-Хассе}

	\begin{theorem}[Принцип Минковского-Хассе]\label{MinkHasse} 
		Пусть $f$~--- рациональная квадратичная форма. Тогда $f$ изотропна над $\Q_{p}$ для всех  $p \in \mathbb{P} \cup \{ \infty \}$ тогда и только тогда, когда она изотропна над $\Q$.
	\end{theorem}

	\begin{remark}
		Ясно, что с самого начала $f$ можно полагать невырожденной и диагональной, то есть $f \simeq \langle a_1, \ldots, a_n \rangle$, $a_1 \cdot \ldots \cdot a_n \neq 0$.
	\end{remark}

	Пусть $n = \dim{f}$, случай $n = 1$ тривиален. 

	\begin{proof}[Доказательство теоремы~\ref{MinkHasse} в случае $n = 2$:]
		Ясно, что с самого начала можно полагать $f = \langle 1, -a \rangle$, где $a \in \Z_{>0}$ (так как $f$ изотропна над $\R$). Кроме того, можно считать $a$ свободным от квадратов (так как все квадраты простых, входящих в $a$, можно занести внутрь переменной, делая линейную замену).

		 Пусть $a = p_1 \ldots p_k$. Тогда, так как $\forall i \ \v_{p_i}(a) = 1$, $a$ не является квадратом ни над каким $\Q_{p_i}$, значит $a = 1$ и $f \simeq \langle 1, - 1\rangle$.
	\end{proof}
	\begin{proof}[Доказательство теоремы~\ref{MinkHasse} в случае $n = 3$:]
		По соображениям аналогичным размерности $2$, с самого начала можно полагать $f \simeq \langle 1, -a, -b \rangle$, где  $a, b \in \Z$, $ab \neq 0$ и $a$ и $b$ свободны от квадратов. 

		Будем вести индукцию по $|a| + |b|$.

		\bf{База:} Пусть $|a| + |b| = 1$. Тогда $a = \pm 1, \ b = \pm 1$, а так как $f$ изотропна над $\R$, они не могут быть одновременно отрицательными. Но тогда $f$ имеет гиперболическую плоскость $\langle 1, -1 \rangle$ в качестве подформы, а она представляет $0$ над $\Q$. 

		\bf{Переход:} Не умаляя общности, пусть $|a| \le |b|$. Пусть
		\[
			b = \pm p_1 p_2 \ldots p_k, \quad p_i \neq p_j \text{ при } i \neq j.
		\]
		Тогда, так как $f$ изотропна над каждым $\Q_{p_i}$, $a$ является квадратичным вычетом по модулю каждого $p_i$, откуда $a$ является квадратом в $\Z/b\Z$. Тогда 
		\[
			\exists t, b'\in \Z\colon t^2 - a = b b'.
		\]
		и при этом, ясно, что можно полагать $|t| < b/2$, а отсюда $|b'| \le |b|/4 + 1$. Рассмотрим форму 
		\[
			f' \cong \langle 1, -a, -b' \rangle. 
		\]
		\[
			|a| + |b'| \le |a| + \left\lvert \frac{|b|}{4} + 1\right\rvert < |a| + |b|.
		\]
		Чтоб применить индукционное предположение, осталось понять, почему $f'$ изотропна над $\Q_{p}$ для всех простых $p$. Действительно, по свойству, отмеченному в \hyperlink{(1 - a, a)_p}{
		этом замечении}, 

		\[
			1 = (a, t^2 - a)_{p} = (a, b'b)_{p} = \underbrace{(a, b)_{p}}_{= 1} \cdot (a, b')_{p} \implies (a, b')_{p} = 1.
		\]
		Значит, $\langle 1, -a, -b' \rangle$ изотропна над $\Q_{p}$ для всех $p \in \mathbb{P} \cup \{ \infty \}$. Тогда, по индукционному предположению она изотропна над $\Q$. По предложению~\ref{guilbert_simbol_norm} это равносильно тому, что
		\[
			b' = \Nm_{\Q(\sqrt{a})/\Q}(\alpha).
		\]
		С другой стороны, так как $b b' = t^2 - a$, $b b' = \Nm_{\Q(\sqrt{a})/\Q}(\beta)$. Но тогда 
		\[
			b b'^2 = \Nm_{\Q(\sqrt{a})/\Q}(\alpha \beta) \implies b = \Nm_{\Q(\sqrt{a})/\Q}\lr*{\frac{\alpha \beta}{b'}}, 
		\]
		откуда по предложению~\ref{guilbert_simbol_norm} форма $\langle 1, -a, -b \rangle$ изотропна над $\Q$.
	\end{proof}
	\begin{proof}[Доказательство теоремы~\ref{MinkHasse} в случае $n = 4$:]
		Пусть $f \simeq \langle a_1, a_2, a_3, a_4 \rangle, \ a_i \in \Z$. Так как $f$ изотропна над $\R$, мы можем полагать $a_1 > 0, a_4 < 0$. Рассмотрим формы 
		\[
			g \simeq \langle a_1, a_2 \rangle, \quad h \simeq \langle -a_3, -a_4 \rangle.
		\]
		Заметим, что для доказательства нам достаточно найти такое ненулевое рациональное $c$, которое представляют обе эти формы, то есть
		\[
			a_1 \xi_1^2 + a_2 \xi_2^2 = c = -a_3 \xi_3^2 - a_4 \xi_4^2,
		\]
		Рассмотрим множество $S$, определённое как 
		\[
			S \eqdef \{ 2, \text{ все простые делители } a_i \}.
		\]
		По условию теоремы для каждого простого $p$ сущесвтуют $\xi_{i p} \in \Q_{p}$ (не все равные нулю) и  $b_p \in \Z_{p}$ такие, что 
		\[
			a_1 \xi_{1p}^2 + a_2 \xi_{2p}^{2} = -a_{3}\xi_{3p}^{2} - a_{4}\xi_{4p}^{2} = b_p \neq 0. 
		\]
		По китайской теореме об остатках мы можем выбрать такое $a \in \Z_{>0}$, что 
		\[
			\forall p_i \in S, \ p_i \neq 2  \quad a \equiv b_{p_i} \pmod{p_i^2}, \ a \equiv b_{2} \pmod{16}. 
		\]
		Так как мы можем полагать, что $0 \le \v_{p_i}(b_{p_i}) \le 1$, мы имеем
		\[
			\begin{cases} \v_{p_i}(a) = \v_{p_i}(b_{p_i}) \\ \v_{2}(a) = \v_{2}(b_2) \end{cases} \implies  b_{p_i} - a = p_i^2 d \rightsquigarrow \frac{b_{p_i}}{a} = 1 + \frac{p_i^2}{a}d \in \Z_{p}.
		\]
		То есть мы имеем
		\[
			\frac{b_i}{a} \in \Z_{p_i} \text{ и } \frac{b_i}{a} \equiv 1 \pmod{p_i}.
		\]
		Тогда, так как $\lr*{\frac{1}{p_i}} = 1$, по предложению~\ref{squares_of_Z_p} $a^{-1}b_{p_i}$ является квадратом в $\Z_{p_i}$. Аналогично и $a^{-1}b_{2}$ является квадратом в $\Z_{2}$.

		Теперь заметим, что по определению $b_{p_i}$ формы $\langle a_1, a_2, - b_{p_i} \rangle$  и $\langle -a_3, -a_4, - b_{p_i} \rangle$ изотропны над $\Q_{p_i}$, а так как $a^{-1} b_{p_i}$~--- квадрат в $\Z_{p_i}$, формы $\langle a_1, a_2, -a \rangle$ и $\langle -a_3, -a_4, -a \rangle$ изотропны над $\Q_{p_i}$ для всех $p_i \in S$. В самом деле, пусть $a^{-1}b_{p_i} = \theta^2$, тогда $b_{p_i} = a \cdot \theta^2$ и 
		\[
			a_1 \eta_1^2 + a_2 \eta_2^2 - \cdot b_{p_i} = 0 \rightsquigarrow a_1 \eta_1^2 + a_2 \eta_2^2 - a \cdot \theta^2 = 0, 
		\]
		и аналогично для второй формы. 


	\end{proof}
