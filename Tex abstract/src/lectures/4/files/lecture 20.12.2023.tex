	Теперь докажем некоторую техническую теорему об изотропнсоти квадратичных форм с $p$-адическими коэффициентами, которая поможет нам в доказательстве теоремы Минковского-Хассе.  	
	Рассмотрим неособую квадратичную форму $f \simeq \langle b_1, \ldots, b_n \rangle, \ b_i \in \Q_{p}^{*}$. Линейной заменой перменных её всегда можно привести к виду 
	\[
		f = f_0 + p f_1 = a_1 x_1^2 + \ldots + a_r x_r^2 + p(a_{r + 1}x_{r + 1}^2 + \ldots + a_{n} x_{n}^2), \quad a_i \in \Z_{p}^{*}.
	\]
	Если мы говорим об изотропности формы, можно всегда полагать $r \ge n - r$, так как $pf$ и $f$ изотропны одновременно, а $pf \simeq f_1 + pf_0$. 

	\begin{theorem}\label{f_0 + pf_1} 
	 	Пусть $p \neq $ и $0 < r < n$.  Тогда форма $f$ изотропна над $\Q_{p}$ тогда и только тогда, когда  хоть одна из форм $f_0$ или $f_1$ изотропна над $\Q_{p}$. 
	 \end{theorem} 
	 \begin{proof}
	 	Пусть форма $f$ представляет нуль:
	 	\begin{equation}
	 		a_1 \xi_1^2 + \ldots + a_r \xi_r^2 + p(a_{r + 1}\xi_{r + 1}^2 + \ldots + a_{n} \xi_{n}^2) = 0. \label{form_n_ge_5}
	 	\end{equation}
	 	Мы можем полагать, что $\xi_i \in \Z_{p}$ и хотя бы одно из них не делится на $p$. 

	 	\begin{itemize}
	 		\item Если не все $\xi_1, \ldots, \xi_r$ делятся на $p$, то переходя в равенстве~\eqref{form_n_ge_5} к сравнению по модулю $p$ мы имеем 
	 		\[
	 			f_0(\xi_1, \ldots, \xi_r) \equiv 0 \pmod{p}, \quad \frac{\partial f_0}{\partial x_1}(\xi_1, \ldots, \xi_r) = 2a_1 \xi_1 \not\equiv 0 \pmod{p}.
	 		\]
	 		Тогда по лемме Гензеля~\ref{Henzel_lemma} форма $f_0$ изотропна. 
	 		\item  Пусть $\xi_j \divby p \ \forall j = 1, \ldots, r$. Тогда 
	 		\[
	 			a_1 \xi_1^2 + \ldots + \xi_r^2 \equiv  0 \pmod{p^2}.
	 		\]
	 		Тогда перейдём в равенстве~\eqref{form_n_ge_5} к сравнению по модулю $p^2$ и сократим его на $p$, получится 
	 		\[
	 			f_1(\xi_{r + 1}, \ldots, \xi_n) \equiv 0 \pmod{p},
	 		\]
	 		причём хоть одно из $\xi_{r + 1}, \ldots, \xi_n$ не делится на $p$. Применяя лемму Гензеля аналогично первому случаю, имеем нужное. 
	 	\end{itemize}
	 \end{proof}

	 \begin{corollary}\label{r_ge_3}
	 	Если $a_1, \ldots, a_n \in \Z_{p}^{*}$, то при $p \neq 2$ форма $f = \langle a_1, \ldots, a_{r} \rangle$ всегда изотропна над $\Q_{p}$ при $r \ge 3$.
	 \end{corollary}
	 \begin{proof}
	 	По теореме Шевалле квадратичная форма от хотя бы  трёх переменных всегда имеет нуль в $\Z/p\Z$, значит, сравнение 
	 	\[
	 		f(x_1, \ldots, x_n) \equiv 0 \pmod{p}
	 	\]
	 	будет иметь нетривиальное решение. Но тогда остаётся лишь применить лемму Гензеля. 
	 \end{proof}

	Теперь перейдём к случаю $p = 2$. В этом случае, как и обычно, нужно давать некоторые корректировки. 

	\begin{theorem} 
		В поле $\Q_{2}$ квадратичная форма $f = f_0 + 2f_1$ изотропна тогда и только тогда, когда разрешимо сравнение $f \equiv 0 \pmod{16}$ разрешимо с нечетным значением хоть одной переменной. 
	\end{theorem}
	\begin{proof}
		Пусть $f(\xi_1, \ldots, \xi_n) \equiv 0 \pmod{16}$ и хотя бы одно $\xi_i$ не делится на 2. 
		\begin{itemize}
			\item  Предположим, что некоторый $\xi_i$, где $1 \le i \le r$, не делится на 2. Не умаляя общности, пусть это $\xi_1$. 
			\[
				f(\xi_1, \ldots, \xi_n) \equiv 0 \pmod{8}, \quad \frac{\partial f}{\partial x_1}(\xi_1, \ldots, \xi_n) = 2a_1\xi_1 \not\equiv 0 \pmod{4}.
			\]
			Тогда по лемме Гензеля~\ref{Henzel_lemma} форма $f$ изотропна. 
			\item Пусть $\xi_1, \ldots, \xi_r \divby 2$, то есть $\xi_i = 2\eta_i$.  Но тогда 
			\[
				4 (a_1 \eta_1^2 + \ldots + a_r \eta_r^2) + 2 (a_{r + 1} \xi_r^2 + \ldots + a_n \xi_n^2) \equiv 0 \pmod{16}.
			\]
			Сократим это сравнение на $2$, получим: 
			\[
				2 (a_1 \eta_1^2 + \ldots + a_r \eta_r^2) +  (a_{r + 1} \xi_r^2 + \ldots + a_n \xi_n^2) \equiv 0 \pmod{8}
			\]
			и хотя бы одно $\xi_{i}$, где $r + 1 \le i \le n$ не делится на $2$. Тогда мы можем в точности, как в первом случае, применить лемму Гензеля и получить, что $f_1 + 2f_0$ изотропна, но $f_1 + 2 f_0 \simeq 2f$, а $f$ и $2f$ изотропны одновременно. 
		\end{itemize}
	\end{proof}

	Из доказательства сразу можно извлечь такое следствие:
	\begin{corollary}\label{fmod8}
		Если для $f = f_0 + 2f_1$ сравнение $f \equiv 0 \pmod{8}$ имеет решение с нечетным значением хоть одной из неизвестных $x_1, \ldots, x_r$, то эта форма изотропна над $\Q_{2}$. 
	\end{corollary}
	
	Теперь мы наконец можем доказать следующую теорему: 

	\begin{theorem} 
		В поле $p$-адических чисел $\Q_{p}$ всякая неособая квадратичная форма от пяти и более переменных изотропна. 
	\end{theorem}
	\begin{proof}
		Не умаляя общности, можем полагать, что наша форма имеет вид 
		\[
		f = f_0 + p f_1 = a_1 x_1^2 + \ldots + a_r x_r^2 + p(a_{r + 1}x_{r + 1}^2 + \ldots + a_{n} x_{n}^2), \quad a_i \in \Z_{p}^{*}.
	\]
	и $r \ge n - r$. Пусть $p \neq 2$. Тогда, так как $n \ge 5, \ n - r \ge 3$ и по следствию~\ref{r_ge_3} форма $f_0$ изотропна, а значит, и $f$ изотропна вместе с ней. 

	Теперь, пусть $p = 2$. Если $n - r > 0$, рассмотрим форму 
	\[
		g = a_1 x_1^2 + a_2 x_2^2 + a_3 x_3^2 + 2a_n x_n^2.
	\]
	Покажем, что такая форма всегда изотропна над $\Q_2$. Так как $a_1 + a_2 = 2\alpha, \ \alpha \in \Z_{2}$,
	\[
		a_1 + a_2 + 2a_n \alpha^2 \equiv 2\alpha + 2\alpha^2 \equiv 2\alpha(\alpha + 1) \equiv 0 \pmod{4}. 
	\]
	Значит, $a_1 + a_2 + 2a_n \alpha^2 = 4\beta$, $\beta \in \Z_{2}$. Тогда мы можем положить $x_1 = x_2 = 1$, $x_3 = 2\beta, \ x_n = \alpha$ и 
	\[
		a_1 + a_2 + a_3(2\beta)^2 + 2a_n\alpha^2 \equiv 4\beta + 4\beta^2 \equiv 0 \pmod{8}.
	\]
	Тогда по следствию~\ref{fmod8} форма $g$ изотропна над $\Q_2$, а значит, и $f$ изотропна над $\Q_2$. 

	Если же $n = r$, то мы рассмотрим 
	\[
		g = a_1 x_1^2 + a_2 x_2^2 + a_3 x_3^2 + a_4 x_4^2 + a_5 x_5^2
	\]
	Если $a_1 + a_2 \equiv a_3 + a_4 \equiv 2 \pmod{4}$, то можем положить $x_1 = x_2 = x_3 = x_4 = 1$, а если $a_1 + a_2 \equiv 0 \pmod{4}$, то $x_1 = x_2 = 1, \ x_3 = x_4 = 0$. И в том, и в другом случае мы получим 
	\[
		a_1 x_1^2 + a_2 x_2^2 + a_3 x_3^2 + a_4 x_4^2 = 4\gamma, \quad \gamma \in \Z_{2}.
	\]
	Тогда положим $x_5 = 2\gamma$ и получим 
	\[
		g \equiv 4 \gamma + 4\gamma^2 \equiv 4\gamma(\gamma + 1) \equiv 0 \pmod{8}
	\]
	Тогда по следствию~\ref{fmod8} мы получаем, что $g$ изотропна, а значит и $f$ изотропна. 

	\end{proof}



	\subsection{Теорема Минковского-Хассе}

	\begin{theorem}[Принцип Минковского-Хассе]\label{MinkHasse} 
		Пусть $f$~--- рациональная квадратичная форма. Тогда $f$ изотропна над $\Q_{p}$ для всех  $p \in \mathbb{P} \cup \{ \infty \}$ тогда и только тогда, когда она изотропна над $\Q$.
	\end{theorem}

	\begin{remark}
		Ясно, что с самого начала $f$ можно полагать невырожденной и диагональной, то есть $f \simeq \langle a_1, \ldots, a_n \rangle$, $a_1 \cdot \ldots \cdot a_n \neq 0$.
	\end{remark}

	Пусть $n = \dim{f}$, случай $n = 1$ тривиален. 

	\begin{proof}[Доказательство теоремы~\ref{MinkHasse} в случае $n = 2$:]
		Ясно, что с самого начала можно полагать $f = \langle 1, -a \rangle$, где $a \in \Z_{>0}$ (так как $f$ изотропна над $\R$). Кроме того, можно считать $a$ свободным от квадратов (так как все квадраты простых, входящих в $a$, можно занести внутрь переменной, делая линейную замену).

		 Пусть $a = p_1 \ldots p_k$. Тогда, так как $\forall i \ \v_{p_i}(a) = 1$, $a$ не является квадратом ни над каким $\Q_{p_i}$, значит $a = 1$ и $f \simeq \langle 1, - 1\rangle$.
	\end{proof}
	\begin{proof}[Доказательство теоремы~\ref{MinkHasse} в случае $n = 3$:]
		По соображениям аналогичным размерности $2$, с самого начала можно полагать $f \simeq \langle 1, -a, -b \rangle$, где  $a, b \in \Z$, $ab \neq 0$ и $a$ и $b$ свободны от квадратов. 

		Будем вести индукцию по $|a| + |b|$.

		\bf{База:} Пусть $|a| + |b| = 1$. Тогда $a = \pm 1, \ b = \pm 1$, а так как $f$ изотропна над $\R$, они не могут быть одновременно отрицательными. Но тогда $f$ имеет гиперболическую плоскость $\langle 1, -1 \rangle$ в качестве подформы, а она представляет $0$ над $\Q$. 

		\bf{Переход:} Не умаляя общности, пусть $|a| \le |b|$. Пусть
		\[
			b = \pm p_1 p_2 \ldots p_k, \quad p_i \neq p_j \text{ при } i \neq j.
		\]
		Тогда, так как $f$ изотропна над каждым $\Q_{p_i}$, $a$ является квадратичным вычетом по модулю каждого $p_i$, откуда $a$ является квадратом в $\Z/b\Z$. Тогда 
		\[
			\exists t, b'\in \Z\colon t^2 - a = b b'.
		\]
		и при этом, ясно, что можно полагать $|t| < b/2$, а отсюда $|b'| \le |b|/4 + 1$. Рассмотрим форму 
		\[
			f' \cong \langle 1, -a, -b' \rangle. 
		\]
		\[
			|a| + |b'| \le |a| + \left\lvert \frac{|b|}{4} + 1\right\rvert < |a| + |b|.
		\]
		Чтоб применить индукционное предположение, осталось понять, почему $f'$ изотропна над $\Q_{p}$ для всех простых $p$. Действительно, по свойству, отмеченному в \hyperlink{(1 - a, a)_p}{
		этом замечении}, 

		\[
			1 = (a, t^2 - a)_{p} = (a, b'b)_{p} = \underbrace{(a, b)_{p}}_{= 1} \cdot (a, b')_{p} \implies (a, b')_{p} = 1.
		\]
		Значит, $\langle 1, -a, -b' \rangle$ изотропна над $\Q_{p}$ для всех $p \in \mathbb{P} \cup \{ \infty \}$. Тогда, по индукционному предположению она изотропна над $\Q$. По предложению~\ref{hilbert_simbol_norm} это равносильно тому, что
		\[
			b' = \Nm_{\Q(\sqrt{a})/\Q}(\alpha).
		\]
		С другой стороны, так как $b b' = t^2 - a$, $b b' = \Nm_{\Q(\sqrt{a})/\Q}(\beta)$. Но тогда 
		\[
			b b'^2 = \Nm_{\Q(\sqrt{a})/\Q}(\alpha \beta) \implies b = \Nm_{\Q(\sqrt{a})/\Q}\lr*{\frac{\alpha \beta}{b'}}, 
		\]
		откуда по предложению~\ref{hilbert_simbol_norm} форма $\langle 1, -a, -b \rangle$ изотропна над $\Q$.
	\end{proof}
	\begin{proof}[Доказательство теоремы~\ref{MinkHasse} в случае $n = 4$:]
		Пусть $f \simeq \langle a_1, a_2, a_3, a_4 \rangle, \ a_i \in \Z$. Так как $f$ изотропна над $\R$, мы можем полагать $a_1 > 0, a_4 < 0$. Рассмотрим формы 
		\[
			g \simeq \langle a_1, a_2 \rangle, \quad h \simeq \langle -a_3, -a_4 \rangle.
		\]
		Заметим, что для доказательства нам достаточно найти такое ненулевое рациональное $c$, которое представляют обе эти формы, то есть
		\[
			a_1 \xi_1^2 + a_2 \xi_2^2 = c = -a_3 \xi_3^2 - a_4 \xi_4^2,
		\]
		Рассмотрим множество $S$, определённое как 
		\[
			S \eqdef \{ 2, \text{ все простые делители } a_i \}.
		\]
		По условию теоремы для каждого простого $p$ сущесвтуют $\xi_{i p} \in \Q_{p}$ (не все равные нулю) и  $b_p \in \Z_{p}$ такие, что 
		\[
			a_1 \xi_{1p}^2 + a_2 \xi_{2p}^{2} = -a_{3}\xi_{3p}^{2} - a_{4}\xi_{4p}^{2} = b_p \neq 0. 
		\]
		По китайской теореме об остатках мы можем выбрать такое $a \in \Z_{>0}$, что 
		\[
			\forall p_i \in S, \ p_i \neq 2  \quad a \equiv b_{p_i} \pmod{p_i^2}, \ a \equiv b_{2} \pmod{16}. 
		\]
		Так как мы можем полагать, что $0 \le \v_{p_i}(b_{p_i}) \le 1$, мы имеем
		\[
			\begin{cases} \v_{p_i}(a) = \v_{p_i}(b_{p_i}) \\ \v_{2}(a) = \v_{2}(b_2) \end{cases} \implies  b_{p_i} - a = p_i^2 d \rightsquigarrow \frac{b_{p_i}}{a} = 1 + \frac{p_i^2}{a}d \in \Z_{p}.
		\]
		То есть мы имеем
		\[
			\frac{b_i}{a} \in \Z_{p_i} \text{ и } \frac{b_i}{a} \equiv 1 \pmod{p_i}.
		\]
		Тогда, так как $\lr*{\frac{1}{p_i}} = 1$, по предложению~\ref{squares_of_Z_p} $a^{-1}b_{p_i}$ является квадратом в $\Z_{p_i}$. Аналогично и $a^{-1}b_{2}$ является квадратом в $\Z_{2}$.

		Теперь заметим, что по определению $b_{p_i}$ формы $\langle a_1, a_2, - b_{p_i} \rangle$  и $\langle -a_3, -a_4, - b_{p_i} \rangle$ изотропны над $\Q_{p_i}$, а так как $a^{-1} b_{p_i}$~--- квадрат в $\Z_{p_i}$, формы $\langle a_1, a_2, -a \rangle$ и $\langle -a_3, -a_4, -a \rangle$ изотропны над $\Q_{p_i}$ для всех $p_i \in S$. В самом деле, пусть $a^{-1}b_{p_i} = \theta^2$, тогда $b_{p_i} = a \cdot \theta^2$ и 
		\[
			a_1 \eta_1^2 + a_2 \eta_2^2 - \cdot b_{p_i} = 0 \rightsquigarrow a_1 \eta_1^2 + a_2 \eta_2^2 - a \cdot \theta^2 = 0, 
		\]
		и аналогично для второй формы. 

		Если же $p \notin S$ и $p \not \ \mid a$, то так как $a_i \notdivby p$, формы $\langle a_1, a_2, -a \rangle$ и $\langle -a_3, -a_4, -a\rangle$  также будут изотропны над $\Q_{p}$.

		Значит, нам остаётся рассмотреть лишь простые делители $a$, не лежаще в $s$. От $a$ мы требовали, чтоб 
		\[
			\forall p_i \in S, \ p_i \neq 2  \quad a \equiv b_{p_i} \pmod{p_i^2}, \ a \equiv b_{2} \pmod{16}, \quad a \in \Z_{> 0}.
		\]

		Так как мы свободно можем изменять $a$ на $16p_1^2\ldots p_k^2 \cdot n$, всевозможные подходящие $a$ лежат в арифметической прогрессии $\{ c_n \}$, где $c_0$ , а 
		\[
			c_n = c_0 + 16p_1^2\ldots p_k^2 \cdot n = d \cdot\lr*{\frac{c_0}{d} + \frac{16p_1^2\ldots p_k^2}{d}n}, \quad d = (c_0, 16p_1^2\ldots p_k^2 \cdot).
		\]
		Тогда по теореме Дирихле о простых в арифметической прогрессии существует $n$ такой, что  $c_n = d p_0$, где $p_0$~--- простое. Возьмём в качестве искоимого $a$ этот самый $c_n$. Так как любой простой делитель $d$ будет лежать в $S$, всего один простой делитель $a$ не будет лежать там~--- это $p_0$. 

		Заметим, что тогда форма $\langle a_1, a_2, -a \rangle$ изотропна над $\Q_{p}$ для всех $p$, кроме $p_0$. Но тогда, в силу закона взаимности Гильберта~\ref{Hilbert_reciprocity}, они изотропна и над $\Q_{p_0}$. Аналогично и для формы $\langle -a_3, -a_4, -a \rangle$. 

		Тогда для каждой из них мы можем применить теорему Минковского-Хассе для $n = 3$ и получить, что $\langle a_1, \ldots, a_4 \rangle$ изотропна над $\Q$.
	\end{proof}

	\begin{proof}[Доказательство теоремы~\ref{MinkHasse} в случае $n \ge 5$:]

	Пусть $f \simeq \langle a_1, \ldots, a_n \rangle$. Рассмотрим формы 
	\[
		g \simeq \langle a_1, a_2 \rangle, \quad h \simeq \langle -a_3, -a_4, \ldots, -a_n \rangle.
	\]
	Определим множество $S$ аналогичным образом и также, как и в случае $n = 4$ найдём $a$ такое, что формы 
	\[
		\langle a_1, a_2, - a \rangle \text{ и } \langle -a_3, \ldots, -a_n, -a \rangle
	\]
	изотропны над всеми $\Q_{p}$, кроме, может быть, $\Q_{p_0}$. Тогда по закону взаимности Гильберта~\ref{Hilbert_reciprocity} форма $\langle a_1, a_2, - a \rangle$ будет изотропна над всеми $\Q_{p}, \ p \in \mathbb{P} \cup \{\infty\}$. 

	Пусть теперь $n \ge 5$. По следствию~\ref{r_ge_3} (оно применимо, так как $a_3, a_4, a_5 \notdivby p_0 \implies a_3, a_4, a_5 \in \Z_{p_0}^{*}$)  форма $h$ будет изотропна над $\Q_{p_0}^{*}$. Значит, она представляет любой элемент $\Q_{p_0}$\footnote{Это общий факт, справедливый над любым полем: изотропная форма представляет любой элемент.}, в частности $a$.  Но, тогда форма $\langle -a_3, -a_4, -a_5, -a \rangle$ изотропна над $\Q_{p_0}$. Значит, по теореме Минковского-Хассе для $n = 4$ она изотропна над $\Q$. Но тогда формы $g$ и $h$ представляют рациональное число $a$, откуда $f$ изотропна над $\Q$.
	
	Теперь пусть $n > 5$.  Тогда достаточно заметить, что нашу форму  $f$ можно будет представить в  виде $f_0 + f_1$, где $f_0$~--- неопределённая форма от пяти переменных. По доказанному выше она будет изотропна над $\Q$, а значит и $f$ будет изотропна над $\Q$. 
	\end{proof}

	\begin{corollary}
		Пусть $f$~--- рациональная квадратичная форма и $f$ гиперболична над всеми $\Q_{p}$. Тогда $f$ гиперболична. 
	\end{corollary}
	\begin{proof}
		Так как $f$ гиперболична над всеми $\Q_{p}$, она, в частности, изотропна над всеми $\Q_{p}$, откуда по теореме Минковского-Хассе она изотропна над $\Q$. 
		Тогда мы можем представить её в виде 
		\[
			f = \HH \oplus f' \implies f_{\Q_{p}} = \HH \oplus f'_{\Q_{p}}. 
		\]
		С другой стороны, так как $f$ гиперболична над всеми $\Q_{p}$, мы имеем 
		\[
			\underbrace{\HH \oplus \ldots \HH}_{m \text{ раз }} = f_{\Q_{p}} = \HH \oplus f'_{\Q_{p}}.
		\]
		Применяя теорему Витта о сокращении, мы получаем, что $f'$ гиперболична, а значит, $f$ гиперболична. 
	\end{proof}

	\begin{corollary}
		Пусть $f, g$~--- рациональные квадратичные формы одинаковой размерности, $\dim{f} = \dim{g} = n$. Тогда они эквивалентны над полем рациональных чисел тогда и только тогда, когда они эквивалентны над  полями $\Q_{p}$ для всех $p \in \mathbb{P} \cup \{ \infty \}$.
	\end{corollary}
	\begin{proof}
		Так как $f \simeq g$ над  $\Q_{p}$ для всех $p \in \mathbb{P} \cup \{ \infty \}$, форма $f \oplus (-g)$  будет гиперболична над $\Q_{p}$ для всех $p \in \mathbb{P} \cup \{ \infty \}$. Тогда по предыдущему следствию $f \oplus (-g)$ будет гиперболична над $\Q$. С другой стороны, форма $g \oplus (- g)$ тоже гиперболична и имеет такую же размерность. Тогда остаётся применить теорему Витта о сокращении: 
		\[
			f \oplus (-g) \simeq g \oplus (- g) \implies f \simeq g
		\]
		над полем $\Q$.
	\end{proof}



