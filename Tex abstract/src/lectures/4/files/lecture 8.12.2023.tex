 	\subsection{Поле $p$-адических чисел и лемма Гензеля}

	 Напомним вкратце  определение поля $\Q_{p}$. Как мы помним из курса алгебры,  \emph{кольцо целых $p$-адических чисел} определяется как 
	 \[
	 	\Z_{p} = \varprojlim \Z/p^{\ell}\Z
	 \]

	 Соотвественно, его элементы имеют вид $\sum_{k = 1}^{\infty} a_k p^{k}$, а операции определяются покоординатно по модулю $p$. Кроме того ясно, что элемент $\Z_p$ обратим тогда и только тогда, когда $a_0 \neq 0 \pmod{p}$. Отсюда в частности следует, что кольцо $\Z_{p}$ локальное с единственным максимальным идеалом  $(p)$. 

	 Кольцо $\Z_{p}$ целостное и его поле частных мы называем полем $p$-адических чисел $\Q_{p}$. Кроме того, в данном случае оно совпадает с локализацией $\Z_{p}$ в идеале $(p)$. 

	 Отметим также, что кольцо $\Z_{p}$ является кольцом $\mathrm{DVR}$ (со всеми вытекающими из этого хорошими свойствами), нормирование на него естественно продолжается с $\Z$, как 
	 \[
	 	x = p^n u, \quad u \in \Z_{p}^{*} \rightsquigarrow \v_{p}(x) = n. 
	 \]

	 и с него оно также естественно продолжается на $\Q_{p}$. 

	 Полагая $\cU = \Z_{p}^{*}$ мы имеем такую  точную последовательность 
	 \[
	 	1 \to \cU \to \Q_{p}^{*} \xrightarrow{\v} \Z \to 0.
	 \]

	 Кроме того, на $\Q_{p}$ при помощи этой метрики можно определить \emph{неархимедову $p$-адическую норму}
	 \[
	 	|x|_{p} = \begin{cases} p^{-\v_{p}(x)}, \quad x \neq 0 \\ 0, \quad x = 0 \end{cases}
	 \]

	 Она удовлетворяет всем аксиомам нормы, но вместо неравенства треугольника имеет место более сильное \emph{ультраметрическое неравенство}: 
	 \[
	 	\v_{p}(x + y) \ge \min(\v_{p}(x), \v_{p}(y)) \rightsquigarrow |x + y|_{p} \le \max(|x|_{p}, |y|_{p}).
	 \]

	 Соотвественно, нетрудно убедиться в том, что $\Q_{p}$~--- пополнение $\Q$ по $p$-адической норме (и это даёт другую конструкцию этого поля). Одним из самых частых применений $p$-адических чисел является следующая известаня многим со школьных лет лемма:    

	 \begin{lemma}[Лемма Гензеля] 
	 	Пусть $f \in \Z_{p}[x]$, причём  для некоторого $x_0 \in \Z_{p}$
	 	\[
	 		f(x_0) \equiv 0 \pmod{p^{2a + 1}}, \quad f'(x_0) \equiv 0 \pmod{p^{a}}, \quad f'(x_0) \not\equiv 0 \pmod{p^{a + 1}}.
	 	\]

	 	Тогда $\exists x \in \Z_{p}$ такое, что \tabularnewline
	 	\[
	 		x \equiv x_0 \pmod{p^{a + 1}}, \quad f(x) = 0.
	 	\]
	 \end{lemma}

	 \begin{proof}
	 	\emph{Метод касательных Ньютона:}\\

	 	Пусть $x_0 = x_0$, построим далее индуктивно последовательность $\{x_n\}$, предел которой даст нам нужный корень. Положим  
	 	\[
	 		x_{n + 1} = x_n - \frac{f(x_n)}{f'(x_n)}
	 	\]

	 	и докажем, что $x_n \in \Z_{p}$ и они удовлетворяют следующим свойствам: 
	 	\[
	 		\begin{cases} f(\alpha_n) \equiv 0 \pmod{2a + 1 + n}, n \ge 0 \\ x_n \equiv x_{n - 1} \pmod{p^{a + n}}, n \ge 1 \end{cases}.
	 	\]

	 	Докажем это по индукции, \textcolor{red}{мне лень писать сейчас перепишу потом из боревича чесслово. это и так база и все всё это знают зачем вообще я это техаю???? да и вообще как-то многовато вопросов к жизни в последнее время. :( }
	 \end{proof}

	 Итак, мы знаем, что уравнение 
	 \[
	 	3x^3 + 4x^3 + 5z^3 = 0
	 \] 
	 имеет корни над любым $\Z/p\Z$.  Рассмотрим случаи $p \neq 2, 3, 5$. Существуют числа $x_0, y_0, z_0$ такие, что 
	 \[
	 	3x_0^3 + 3y_0^3 + 4 z_0^3 \equiv 0\pmod{p}.
	 \]

	 Тогда по лемме Гензеля с $a = 0$ для многочлена 
	 \[
	 	f(x) = 3x^3 + 4y^3 + 5z^3, \quad f'(x) = 9x^2.
	 \]
	 мы получаем, что существует корень в $\Z_{p}$.  Случаи $p = 2, 3, 5$ разбираютяс отдельно. Для $p = 2$ достаточно рассмотреть набор $(1, 0, 1)$ и применить лемму Гензеля с $a = 0$. Для $p = 3$ с $(0, 2, -1)$ достаточно применить лемму Гензеля с $a = 1$, а для $p = 5$ достаточно рассмотреть набор $(2, -1, 0)$ с $a = 0$. 

	 \subsection{Локально-глобальный принцип для квадратичных форм}

	 \begin{statement} 
	 	При $p \neq 2$ $\Q_{p}^{*}/\Q_{p}^{*2} \cong \Z/2\Z \times \Z/2\Z = \{ 1, \varepsilon, p, \varepsilon p\}$.
	 \end{statement}

	 \begin{proof}
	 	\textcolor{red}{тоже лучщше написать из боревича, а предварительно написать про квадратичные формы оттуда же!!!}
	 \end{proof}

	 \begin{statement} 
	 	В случае $p = 2$ предыдущий результат имеет такой вид 
	 	\[
	 		\Q_{2}^{*}/\Q_{2}^{*2} \cong \{1, 3, 5, 7, 2, 6, 10, 14 \}
	 	\]
	 \end{statement}
	 \begin{proof}
	 	
	 	\begin{lemma} 
	 		Если $x \in \Z_{2}$ и $x \equiv 1 \pmod{8}$, то $x$ является квадратом. 
	 	\end{lemma}
	 	\begin{proof}
	 		Рассмотрим многочлен $t^2 - x$ и применим лемму Гензеля для $a = 1$. Так как $x \in \Z_{2}$.
	 	\end{proof}

	 	Запишем $x \in \Z_2$ в виде 
	 	\[
	 		x = a_0 + 2 a_2 + 2^2 a_2 + 8y.
	 	\]

	 	Если $a_0 = 1$, то  мы можем рассмотреть 
	 	\[
	 		\frac{x}{a_0 + 2 a_1 + 2^2 a_2} \equiv 1 \pmod{8},
	 	\]

	 	а тогда по лемме это число является квадратом, откуда мы получаем конечное число вариантов на $a_i$ (тут реализуются $1, 3, 5, 7)$ . Аналогично разбирается случай $a_0 = 0$.  

	 	Кроме того, ясно, что все эти числа не равны по модулю квадратов. 
	 \end{proof}

	 Рассмотрим поле $\Q_{p}$ и $a \in \Q_{p}^{*} \setminus \Q_{p}^{*2}$, такое $a$ задаёт квадратичное расширение $\Q_{p}(\sqrt{a})$ с соотвествующей нормой. Тогда возникает \emph{группа норм}

	 \[
	 	\Nm \eqdef \{ 0 \neq x^2 - ay^2 \in \Q_{p} \ \vert x, y \in \Q_{p} \}.
	 \]

	 Ясно, что $\Q_{p}^{*2} \subset \N \subset \Q_{p}^{*}$. Индекс подгруппы 
	 \[
	  	[\Q_{p}^{*} : N] = 2,
	  \] 
	  независимо от четности $p$.

	  \begin{proof}
	  Для этого достаточно проверить, что $\Nm \neq \Q_{p}^{*}$ и $\Nm \neq \Q_{p}^{*2}$. Если $-a \not \Q_{p}^{*2}$, то ясно, что $\Nm \neq \Q_{p}^{*2}$. Если же $-a \in \Q_{p}^{*2}$, то $\Nm = \{ 0 \neq x^2 + y^2 \}$, а форма $x^2 + y^2$ представляет все элементы конечного поля $\F_{p}$. Тогда, взяв квадратичный невычет, мы можем поднять его до нужного нам элемента $\Q_{p}$. 

	  Также нетрудно показать, что она не совпадает со всей группой $\Q_{p}^{*}$.

	  В случае $p = 2$ ситуация заметно сложнее. 

	  	Докажем сначала для нечётного $p$. 
	  \end{proof}

	  \begin{definition} 
	  	Определим \emph{символ Гильберта}, для простого $p$ и $a, b \in \Q_{p}^{*}$, как 
	  	\[
	  		(a, b)_{p} = \begin{cases} x^2 - ay^2 - bz^2 \text{ представляет нуль }\end{cases}
	  	\]
	  \end{definition}


	 




