	Докажем теперь сильную теорему Дирихле о единицах: 

	\begin{theorem} 
		Пусть $K/\Q$~--- конечное расщирение, $[K : \Q]$, а числа $s, t$ \hyperlink{real_and_complex_inclusions}{связаны с количествами вещественных и комплексных вложений}. Тогда 
		\[
			\cO_{K}^{*} \cong \mu \oplus \Z^{s + t - 1},
		\]
		где $\mu$~--- группа всех корней из единицы в $\cO_{K}$. 
	\end{theorem}
	\begin{proof}
	 Ясно, что для этого нам достаточно доказать оценку $m \ge s + t - 1$, а это равносильно тому, что $\Im{\ell}$~--- полная решётка в гиперплоскости 

	\[
			x_1 + x_2 + \ldots + x_{s + t} = 0.
	\]

	Для этого необходимо найти систему из $(s + t - 1)$-го линейно независимого над $\R$ элемента $\ell\lr*{\cO_{K}^{*}}$. 

	Соотвественно, надо найти $s + t$ элементов $u_{1}, \ldots, u_{s + t} \in \cO_{K}^{*}$, которые дадут нам $s + t - 1$ линейно независимый над $\R$ вектор в образе. Мы постараемся найти такие $u$, что их образы имеют вид 
	\[
		u_1 \mapsto (+, -, - , \ldots, -), u_2 \mapsto (-, +, -, \ldots, -), \ldots u_n \mapsto (-, -, \ldots, +).
	\]

	Обозначение выше означает, что на соовтествующей координате стоит число соотвествующего знака. Покажем сначала, что такие векторы нам подойдут. Возьмём первые $s + t - 1$ координату первых $s + t - 1$ столбцов матрицы, где  $\ell(u_i)$ записаны по строкам и обозначим за $A$. Для ясности, выпишем еще раз эту матрицу: 
	\[
		A = \begin{pmatrix} + & - & - & \ldots & - \\ - & + & - & \ldots & - \\ \vdots & \vdots & \ldots & \ldots & \vdots \\ - & - & - & \ldots & +  \end{pmatrix}, \quad A \in \mathrm{M}_{s + t - 1}(\R).
	\]

	Ясно, что достаточно доказать, что эта матрица имеет полный ранг.  Заметим, что так как образ лежит в гиперплоскости $x_1 + \ldots + x_{s + t} = 0$ изначально сумма по каждой строке равна нулю, то есть 
	\[
		a_{i,1} + a_{i,2} + \ldots + a_{i, s + t} = 0,
	\]
	а так как $\forall i = 1, \ldots, s + t - 1 \quad a_{i, s + t} < 0$, в усеченной матрице (которую мы обозначили за $A$), сумма по каждой строке будет равна 
	\[
	 	a_{i,1} + a_{i,2} + \ldots + a_{i, s + t - 1} > 0.
	 \] 

	 Докажем теперь такую лемму: 

	\begin{lemma} 
		Пусть $A \in \mathrm{M}_{m}(\R)$ такая, что $\forall i  \ a_{ii} > 0$, $\forall i \neq j \ a_{i j } < 0, \ \forall i \ \sum_{j = 1}^{m} a_{i j} > 0$. Тогда $\rank{A} = m$. 
	\end{lemma}

	\begin{proof}
		 Предположим, что $\Ker{A} \neq \{ 0 \}$, то есть система
		 \[
		 	\begin{cases} a_{11} x_1 + \ldots + a_{1 m } x_m = 0 \\ \vdots \\ a_{m 1} x_1 + \ldots + a_{m m} x_m = 0 \end{cases}.
		 \]
		 имеет нетривиальное решение. 

		 Не умаляя общности, $x_1$~--- максимальная по модулю координата. Тогда 
		 \[
		 	 0 = |a_{11} x_1 + \ldots + a_{1m} x_m| \ge |a_{11} x_{1}| - |a_{1 2} x_2| - \ldots - |a_{1 m}||x_m| \ge |x_1|\underbrace{(a_{1 1} - |a_{1 2} | - \ldots - |a_{1 m }|)}_{> 0} \ge 0,
		 \]
		 откуда $|x_1| = 0 \implies |x_i| = 0 \ \forall i = 1, \ldots, m$.
	\end{proof}

	Остаётся найти систему $u_1, \ldots, u_{s + t}$, которые в образе дадут нужные знаки координат. Пусть $n = s + 2t$, рассмотрим множество
	\[
	 	Y = \left\{ (x_1, \ldots, x_{s}, y_1, z_1, \ldots y_t, z_t), \quad |x_i| < C_i \forall 1 \le i \le s, y_i^2 + z_i^2 < C_{s + i} \right\}.
	 \] 

	 Нетрудно проверить, что $Y$~--- ограниченное, выпуклое и центрально-симметричное. Кроме того, 
	 \[
	 	\Vol\lr*{Y} = 2^s \prod_{i = 1}^{s} C_i \cdot \pi^t \cdot \prod_{i = 1}^t C_{s + i} = 2^s \pi^t \cdot \prod_{i = 1}^{s + t} C_i. 
	 \]

	 Пусть $\Gamma$~--- полная решётка, $\Delta$~--- объем фундаментальной области. Тогда, если 
	 \[
	 	2^{s}\pi^{t} \prod_{i = 1}^{s + t} C_i > 2^n \Delta,
	 \]
	 то $Y$ будет содержать точку из решетки $\Gamma$ (по лемме Минковкого о выпуклом теле~\ref{Mink_theorem}).  В качестве $\Gamma$ мы возьмём $\Im{\varphi}$, где 
	 \[	  	
				\varphi(\alpha) = (\sigma_1(\alpha), \sigma_2(\alpha), \ldots, \sigma_s(\alpha), \Re(\sigma_{s + 1}(\alpha)), \Im(\sigma_{s + 1}(\alpha)),  \ldots, \Re(\sigma_{s + t}(\alpha)), \Im(\sigma_{s + t}(\alpha))) \in \R^n.
			\]

	 Заметим, что неравенство выше равносильно тому, что 
	 \[
	 	\prod_{i = 1}^{s + t}C_i >  \lr*{\frac{4}{\pi}}^t \cdot \Delta.
	 \]


	 Возьмём $C > \lr*{\frac{4}{\pi}}^t \Delta$ и рассмотри  все главные идеалы $a_i \cO_{K} \subset \cO_{K}\colon \Nm(a_i \cO_{K}) < C$. 
	 Пусть $\varepsilon = \min\lr*{|\sigma_i a_{j}|, |\sigma_{s + i} a_{j}|^2} > 0$.

	  Зафиксируем теперь некоторый $\sigma_j \in \{ \sigma_{1}, \ldots, \sigma_{s}, \sigma_{s + 1}, \ldots, \sigma_{s + t}\}$ и определим 
	  \[
	  	C_{i} = \begin{cases} \varepsilon, i \neq j \\  C \cdot \varepsilon^{-(s + t - 1)}, i = j\end{cases}.
	  \]
	  Нетрудно заметить, что всё подрбрано таким образом, что 
	  \[
	  		\prod_{i = 1}^{s + t} C_i > \lr*{\frac{4}{\pi}}^{t} \Delta.
	  \]
	  Тогда по лемме~\ref{Mink_theorem} $\exists 0 \neq x \in \cO_{K}\colon $
	  \[
	  		|\sigma_{1}x| < C_1, \ldots, |\sigma_{s}x| < C_{s}, \quad |\sigma_{s + 1}x|^2 < C_{s + 1}, \ldots, |\sigma_{s + t}x|^2 < C_{s + t}.
	  \]
	  Вычислим норму этого $x \in \cO_{K}$
	  \[
	  	\Nm(x\cO_{K}) = |\Nm(x)|  = |\sigma_{1}x| \ldots |\sigma_{s}x| |\sigma_{s + 1}x|^2 \ldots |\sigma_{s + t}x|^2 < \prod_{i = 1}^{s + t} C_{i} = C. 
	  \]
	  Значит, для некоторого $i$ мы имеем $x \cO_{K} = a_i \cO_{K}$. Положим $u = \frac{x}{a_i}$. Тогда $\Nm(u) = 1 \implies u \in \cO_{K}^{*}$. 

	  Так для каждого $\sigma _j$ мы находим свой $u$ (назовём его $u_j$). Проверим, что $\{ u_j \}$ подойдут.  Пусть $\tau = \sigma_i$,
	  \[
	   	|\tau u_j| = \frac{|\tau x|}{|\tau a_k},
	   \] 
	   докажем, что для всех $\tau \neq \sigma_j$ будет выполнено $|\tau u_j| < 1$ (это означает, что в соотвествующей координате будет знак минус). Ясно, что этого будет достаточно, так как сумма координат равна нулю. Рассмотрим два случая: 
	   \begin{itemize}
	   	\item Пусть $\tau \in \{ \sigma_1, \ldots, \sigma_{s} \}, \ \tau = \sigma_i$, тогда 
	   	\[
	   		|\tau u_j| = \frac{|\tau x|}{|\tau a_k|} < \frac{C_i}{\varepsilon} = 1, \text{ так как } i \neq j.
	   	\]
	   	\item Пусть $\tau \in \{ \sigma_{s + 1}, \ldots, \sigma_{s + t} \}$, тогда 
	   	\[
	   		|\tau u_j| = \frac{|\tau x|}{|\tau a_j|} < \frac{\sqrt{C_i}}{\sqrt{\varepsilon}} = 1.
	   	\]
	   \end{itemize}

	   Таким образом, мы показали, что $\Im{\ell}$~--- полная решетка в гипеплоскости, то есть $\Im{\ell} \cong \Z^{s + t - 1}$, откуда, как мы уже замечали в доказательстве слабой теоремы Дирихле о единицах~\ref{Weak_dirichlet_theorem}
	   \[
	   		\cO_{K}^{*} \cong \mu \oplus \Z^{s + t - 1}.
	   \]
	  
	   \end{proof}
	  

	  \subsection{Контр-пример к принципу Минковского-Хассе}

	  Начнём с вот такого утверждения. 

	  \begin{statement}[ДЗ 11, задача 3]\label{prop-alpha} 
	  	Пусть $K = \Q(\alpha), \ \alpha^3 + a \alpha + b = 0$ где $a, b \in \Z$. Пусть $p\cO_{K} = \fp_1 \fp_2 \fp_3$ и $\alpha \in \fp_1 \fp_2$. Тогда $\alpha \in \fp_3$.
	  \end{statement}
	  \begin{proof}
	  	Перепишем данное равенство, как $\alpha(\alpha^2 + a) = - b$ и возьмём норму от обеих частей: 
	  	\[
	  		\Nm(\alpha)\Nm(\alpha^2 + a) = \Nm(-b) = - b^3.
	  	\]
	  	Так как $\Nm(\alpha) = (-1)^3 b = -b$, отсюда $\Nm(\alpha^2 + a) = b^2$.  Сразу заметим, что $b \divby p$, так как 
	  	\[
	  		-b = \Nm(\alpha) , \quad (\alpha) = \fp_1^{k_1} \fp_{2}^{k_2} \cdot \ldots \implies \Nm(\alpha) \divby p,
	  	\]
	  	так как $\Nm(\fp_1), \Nm(\fp_2) \divby p$, так как они висят над $p$.

	  	\begin{enumerate}
	  		\item Пусть $a \in p\Z$. Тогда, так как $\alpha^3 = a\alpha - b$,  а $b \divby p$, в этом случае $\alpha^3 \divby p$, откуда $\alpha^3 \in \fp_{3}$, а так как $\fp_3 \in \Spec{\cO_{K}}$, $\alpha \in \fp_3$, что мы и хотели. 

	  		\item Пусть $a \notin p\Z$. Заметим, что тогда $a \notin \fp_1 \fp_2$, так как если $a$ лежит хоть в одном из них, $a \divby p$.  Но тогда $\alpha^2 + a \notin \fp_1 \fp_2$. Теперь заметим, что 
	  		\[
	  			\Nm(\alpha^2 + a) = b^2 \divby p \implies \alpha^2 + a \in p\cO_{K}, 
	  		\]
	  		а так как $\alpha^2 + a \notin \fp_1\fp_2$, $\alpha^2  + a \in \fp_3$. Пусть $(\alpha^2 + a) = \fp_3^s \fq$, тогда 
	  		\[
	  			b^2 = \Nm(\alpha^2 + a) = \Nm(\fp_3)^s \underbrace{\Nm(\fq)}_{\notdivby p}
	  		\]
	  		Из условия все индексы втевления $e_i$ равны единицы. Но тогда, так как $1 \cdot f_1 + 1 \cdot f_2 + 1 \cdot f_3 = 3$, все степени инерции равны единице, а тогда $\Nm(\fp_3) = p$. Тогда 
	  		\[
	  			p^{2n} \cdot \underbrace{\ldots}_{\notdivby p} = b^2 = p^s \cdot \underbrace{\ldots}_{\notdivby p} \implies s = 2n.
	  		\]
	  		То есть $\v_{\fp_3}(\alpha^2 + a) = 2n$. С другой стороны, $\v_{\fp_3}(\alpha) = 0$, откуда  $\v_{\fp_3}(\alpha^2 + a) = 2n$. Но тогда, так как $\v_{p}(b) = n$
	  		\[
	  			\alpha(\alpha^2 + a) = -b = p^n \cdot d, \quad (p, d) = 1, \quad (\alpha(\alpha^2 + a)) = \fp_1^n \fp_2^n \fp_3^n \cdot \underbrace{\ldots}_{\notdivby \fp_3},
	  		\]
	  		то есть $\v_p(\alpha(\alpha^2 + a)) = n$, что даёт нам противоречие. 
	  	\end{enumerate}
	  	\end{proof}
	  	

	 	\begin{theorem} 
	 		Уравнение $3x^3 + 4y^3 + 5z^3 = 0$ не имеет целых решений. 
	 	\end{theorem}
	 	\begin{proof}
	 		Предположим противное, пусть 
	 		\[
	 			3x_1^3 + 4y_1^3 + 5z_1^3 = 0 \implies 6x_1^3 + 8 y_1^3 + 10 z_1^3 = 0.
	 		\]
	 		Сделаем замены переменных $x = 2y_1, \ y = x_1, z = -z_1,$ получим уравнение 
	 		\begin{equation}
	 			x^3 + 6y^3 = 10z^3. \label{selm_1}
	 		\end{equation}
	 		Выберем среди таких решений решение с минимальным ненулевым $|z|$. Рассмотрим расширение $K = \Q(\theta)$, где $\theta^3 = 6$. Тогда уравнение~\ref{selm_1} выражает тот факт, что 
	 		\begin{equation}
	 			\Nm(x + \theta y) = 10z^3. 
	 		\end{equation}

	 		Положим $\alpha = x + \theta y$. Предположим, что в разложение идеала $(\alpha)$ на простые входит идеал $\fp_1$, не лежащий над двойкой и пятёркой (т.е. $\fp_1 \not \ \mid 2\cO_K, \ \fp_1 \not \ \mid 5\cO_K$) и предположим, что этот $\fp_1$ висит над некоторым простым числом $p$. То есть, пусть $(\alpha) = \fp_1^m \cdot \fq$.  Рассмотрим два случая: 

	 		\begin{enumerate}
	 		 	\item Пусть $\fq \not \mid p\cO_K$. Тогда применим норму: 
	 		 	\[
	 		 		\Nm(\fp_1)^m \cdot \Nm(\fq) = \Nm(\alpha) = \Nm(x + \theta y) = 10z^3.
	 		 	\]
	 		 	Так как степень инерции не больше степени расширения, $\Nm(\fp_1) = p^s$, где $s \in \{1, 2, 3\}$. Так как $\v_{p}(10z^3) \divby 3$ и $\fq \not \mid p\cO_K$, мы имеем $sm \divby 3$, значит либо $m \divby 3$, либо $s = 3$. 
 		 		
 		 		Если $s = 3$, то $\Nm(\fp_1) = p^3$, а тогда $e_{1}(p) = 1$ и так как степень расширения равна трём, $\fp_1 = (p)$. В таком случае $\alpha \divby p \implies x \divby p, \ y \divby p \implies z \divby p$, а тогда мы можем сделать спуск. 

 		 		Отсюда мы заключаем, что $m \divby 3$. 

 		 		\item Пусть $(\fq, (p)) \neq (1)$. Тогда $(\alpha) = \fp_1^m \fp_2 \fq'$ (где $\fp_2$~--- еще один простой идеал, лежащий над $p$). 

 		 		\begin{itemize}
 		 			\item Если $p\cO_K = \fp_1 \fp_2 \fp_3$, то попробуем применить предложение\footnote{С $\alpha = \alpha \theta$, как бы абсурдно это не звучало.}~\ref{prop-alpha} для $\alpha \theta = (x + \theta y)\theta$.  Проверим, что коэффициент при $t^2$ минимального многочлена $\alpha \theta$ равен нулю. В самом деле, так как это многочлен, этот коэффициент с точностью до знака равен следу, а след равен 
 		 			\[
 		 			 	\Tr(\alpha \theta) = \Tr(x\theta) + \Tr(y\theta^2) = 0 + 0 = 0.
 		 			 \] 
 		 			Тогда по предложению~\ref{prop-alpha} мы имеем $\alpha \theta \in \fp_3$, то есть $\alpha \divby p$, то есть $x \theta + y \theta^2 \divby p$,  откуда $x \divby p, \ y \divby p$ и мы снова можем сделать спуск.  

 		 				\item Если $p\cO_K = \fp_1^2 \fp_2$  или $p\cO_K = \fp_1 \fp_2^2$. В любом из этих случаев мы получаем $\alpha^2 \divby p$, но
 		 				\[
 		 				 	\alpha^2 = x^2 + 2xy \theta + y^2 \theta^2 \divby p \implies x \divby p, \ y \divby p
 		 				 \] 
 		 				 и мы можем сделать спуск. 
 		 		\end{itemize}

	 		 \end{enumerate} 

	 		 Итого мы получили, что $(\alpha) = I^3 \cdot \fm$, где $\fm$~--- произведение максимальных идеалов, висящих над 2 и 5. Поймём при помощи теоремы Куммера~\ref{Kummer_theorem}, какие идеалы висят над 2 и 5. Это мы уже делали при вычислении группы классов идеалов $\Q(\sqrt[3]{6})$ (см. пример~\ref{Cl(Q(sqrt[3]{6}))}). 

	 		\[
	 			x^3 - 6 \equiv x^2 \pmod{2} \rightsquigarrow 2\cO_K = (2, \theta)^3  = (\theta - 2)^3, \quad x^3 - 6 = (x - 1)(x^2 + x + 1) \pmod{5} \rightsquigarrow 5 \cO_K = (5, \theta - 1)(5, \theta^2 + \theta + 1) = (\theta - 1)(\theta^2 + \theta + 1).
	 		\]

	 		Заметим, что $(\theta - 1)$ и $(\theta^2 + \theta + 1)$ не могут входить в $\fm$ одновременно, так как тогда $\alpha \divby 5$, откуда $x \divby 5, \ y \divby 5$ и мы можем спустится. 

	 		Так как $\Nm(\alpha) \divby 2, \divby 5$, в разложение $\alpha$ обязательно входит как идеал, всящий над двойкой, так и идеал, висящий над пятеркой. 

	 		Посмотрим сначала на идеалы, висящие над двойкой. Заметим, что с самого начала мы можем полагать $z$ нечётным, так как иначе можно сделать спуск. Но тогда $\v_{2}(10 z^3) = \v_{p}\lr*{\Nm(\alpha)} = 1$. Тогда, так как $\Nm(\theta - 2) = 2$, идеал $(\theta - 2)$ не может входить в разложение $(\alpha)$ в больше чем первой степени. 

	 		Теперь посмотрим на идеалы, висящие над пятеркой. $\v_5(\Nm(\alpha)) = \v_5(10z^3) \equiv 1 \pmod{3}$. Но тогда в $(\alpha)$ может входить либо $(\theta - 1)$ в первой степени, так как $\v_5\lr*{\Nm(\theta - 1) = 1}$, либо $(\theta^2 + \theta + 1)^2$, так как $\v_5((\theta^2 + \theta + 1)^2) = 4 \equiv 1 \pmod{5}$ и других случаев не бывает. 

	 		Соотвественно, $\alpha$ имеет вид 
	 		\[
	 			\alpha = \alpha_0 \cdot t^3, \quad \alpha_0 \in \{ (\theta - 2)(\theta - 1), \ (\theta - 2)(\theta^2 + \theta + 1)^2 \} \cdot \{ 1, \varepsilon, \varepsilon^2, \}. 
	 		\]
	 		где $\varepsilon = 1 - 6\theta + 3\theta^2$~--- основная единица в $\cO_K$. 
	 		Пусть $t = u + v\theta + w \theta^2$. Рассмотрим, например, случай, когда 
	 		\[
  			\alpha = (\theta - 2)(\theta - 1)\lr*{u + v\theta + w\theta^2}^3 = (\theta^2 - 3\theta + 2)(u + v\theta + w\theta^2)^3 = x + y\theta. 
  			\]
  			Раскроем скобки и приравняем коэффициенты при соотвествующих степенях $\theta$, а после, перейдём от равенства к сравнению по модулю 3. Так как $\theta^3 = 6$, 
  			\[
  				\lr*{u + v\theta + w\theta^2}^3 \equiv u^3 \pmod{3}.
  			\]
  			Значит, в левой части равенства коэффициент при $\theta^2$ будет сравним с $u^3$ по модулю 3. С другой стороны, коэффициент при $\theta^2$ в правой части равен нулю, откуда $u \divby 3$. Но тогда  
  			\[
  				\lr*{u + v\theta + w\theta^2}^3 \equiv 0 \pmod{3} \implies x + y \theta \divby 3 \implies x, y \divby 3
  			\]
  			и мы можем сделать спуск. Если 
  			\[
  			\alpha = (\theta - 2)(\theta^2 + \theta + 1)^2\lr*{u + v\theta + w\theta^2}^3 = (\theta^2 - 3\theta + 2)(u + v\theta + w\theta^2)^3 = x + y\theta,
  			\]
  			то будет работать абсолютно такой же аргумент, так как 
  			\[
  				(\theta - 2)(\theta^2 + \theta + 1)^2 \equiv (\theta - 2)(2\theta + 1) \equiv 2\theta^2 - 2 \pmod{3}.
  			\]
  			Остаётся сказать, что если мы вместо единицы возьмём какой-то другой элемент $\cO_{K}^*$, ничего не изменится, так как основная единица $\varepsilon \equiv 1 \pmod{3}$.

  			Таким образом, во всех случаях мы смогли сделать спуск и теорема доказана. 



	 	\end{proof}

	 	



	  	

	  


 



