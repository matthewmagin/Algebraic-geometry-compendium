	Докажем теперь сильную теорему Дирихле о единицах: 
	\begin{theorem} 
		Пусть $K/\Q$~--- конечное расщирение, $[K : \Q]$, а числа $s, t$ \hyperlink{real_and_complex_inclusions}{связаны с количествами вещественных и комплексных вложений}. Тогда 
		\[
			\cO_{K}^{*} \cong \mu \oplus \Z^{s + t - 1},
		\]
		где $\mu$~--- группа всех корней из единицы в $\cO_{K}$. 
	\end{theorem}
	\begin{proof}
	 Ясно, что для этого нам достаточно доказать оценку $m \ge s + t - 1$, а это равносильно тому, что $\Im{\ell}$~--- полная решётка в гиперплоскости 

	\[
			x_1 + x_2 + \ldots + x_{s + t} = 0.
	\]

	Для этого необходимо найти систему из $(s + t - 1)$-го линейно независимого над $\R$ элемента $\ell\lr*{\cO_{K}^{*}}$. 

	Соотвественно, надо найти $s + t$ элементов $u_{1}, \ldots, u_{s + t} \in \cO_{K}^{*}$, которые дадут нам $s + t - 1$ линейно независимый над $\R$ вектор в образе. Мы постараемся найти такие $u$, что их образы имеют вид 
	\[
		u_1 \mapsto (+, -, - , \ldots, -), u_2 \mapsto (-, +, -, \ldots, -), \ldots u_n \mapsto (-, -, \ldots, +).
	\]

	Обозначение выше означает, что на соовтествующей координате стоит число соотвествующего знака. Покажем сначала, что такие векторы нам подойдут. Возьмём первые $s + t - 1$ координату первых $s + t - 1$ столбцов матрицы, где  $\ell(u_i)$ записаны по строкам и обозначим за $A$. Для ясности, выпишем еще раз эту матрицу: 
	\[
		A = \begin{pmatrix} + & - & - & \ldots & - \\ - & + & - & \ldots & - \\ \vdots & \vdots & \ldots & \ldots & \vdots \\ - & - & - & \ldots & +  \end{pmatrix}, \quad A \in \mathrm{M}_{s + t - 1}(\R).
	\]

	Ясно, что достаточно доказать, что эта матрица имеет полный ранг.  Заметим, что так как образ лежит в гиперплоскости $x_1 + \ldots + x_{s + t} = 0$ изначально сумма по каждой строке равна нулю, то есть 
	\[
		a_{i,1} + a_{i,2} + \ldots + a_{i, s + t} = 0,
	\]
	а так как $\forall i = 1, \ldots, s + t - 1 \quad a_{i, s + t} < 0$, в усеченной матрице (которую мы обозначили за $A$), сумма по каждой строке будет равна 
	\[
	 	a_{i,1} + a_{i,2} + \ldots + a_{i, s + t - 1} > 0.
	 \] 

	 Докажем теперь такую лемму: 

	\begin{lemma} 
		Пусть $A \in \mathrm{M}_{m}(\R)$ такая, что $\forall i  \ a_{ii} > 0$, $\forall i \neq j \ a_{i j } < 0, \ \forall i \ \sum_{j = 1}^{m} a_{i j} > 0$. Тогда $\rank{A} = m$. 
	\end{lemma}

	\begin{proof}
		 Предположим, что $\Ker{A} \neq \{ 0 \}$, то есть система
		 \[
		 	\begin{cases} a_{11} x_1 + \ldots + a_{1 m } x_m = 0 \\ \vdots \\ a_{m 1} x_1 + \ldots + a_{m m} x_m = 0 \end{cases}.
		 \]
		 имеет нетривиальное решение. 

		 Не умаляя общности, $x_1$~--- максимальная по модулю координата. Тогда 
		 \[
		 	 0 = |a_{11} x_1 + \ldots + a_{1m} x_m| \ge |a_{11} x_{1}| - |a_{1 2} x_2| - \ldots - |a_{1 m}||x_m| \ge |x_1|\underbrace{(a_{1 1} - |a_{1 2} | - \ldots - |a_{1 m }|)}_{> 0} \ge 0,
		 \]
		 откуда $|x_1| = 0 \implies |x_i| = 0 \ \forall i = 1, \ldots, m$.
	\end{proof}

	Остаётся найти систему $u_1, \ldots, u_{s + t}$, которые в образе дадут нужные знаки координат. Пусть $n = s + 2t$, рассмотрим множество
	\[
	 	Y = \left\{ (x_1, \ldots, x_{s}, y_1, z_1, \ldots y_t, z_t), \quad |x_i| < C_i \forall 1 \le i \le s, y_i^2 + z_i^2 < C_{s + i} \right\}.
	 \] 

	 Нетрудно проверить, что $Y$~--- ограниченное, выпуклое и центрально-симметричное. Кроме того, 
	 \[
	 	\Vol\lr*{Y} = 2^s \prod_{i = 1}^{s} C_i \cdot \pi^t \cdot \prod_{i = 1}^t C_{s + i} = 2^s \pi^t \cdot \prod_{i = 1}^{s + t} C_i. 
	 \]

	 Пусть $\Gamma$~--- полная решётка, $\Delta$~--- объем фундаментальной области. Тогда, если 
	 \[
	 	2^{s}\pi^{t} \prod_{i = 1}^{s + t} C_i > 2^n \Delta,
	 \]
	 то $Y$ будет содержать точку из решетки $\Gamma$ (по лемме Минковкого о выпуклом теле~\ref{Mink_theorem}).  В качестве $\Gamma$ мы возьмём $\Im{\varphi}$, где 
	 \[	  	
				\varphi(\alpha) = (\sigma_1(\alpha), \sigma_2(\alpha), \ldots, \sigma_s(\alpha), \Re(\sigma_{s + 1}(\alpha)), \Im(\sigma_{s + 1}(\alpha)),  \ldots, \Re(\sigma_{s + t}(\alpha)), \Im(\sigma_{s + t}(\alpha))) \in \R^n.
			\]

	 Заметим, что неравенство выше равносильно тому, что 
	 \[
	 	\prod_{i = 1}^{s + t}C_i >  \lr*{\frac{4}{\pi}}^t \cdot \Delta.
	 \]


	 Возьмём $C > \lr*{\frac{4}{\pi}}^t \Delta$ и рассмотри  все главные идеалы $a_i \cO_{K} \subset \cO_{K}\colon \Nm(a_i \cO_{K}) < C$. 
	 Пусть $\varepsilon = \min\lr*{|\sigma_i a_{j}|, |\sigma_{s + i} a_{j}|^2} > 0$.

	  Зафиксируем теперь некоторый $\sigma_j \in \{ \sigma_{1}, \ldots, \sigma_{s}, \sigma_{s + 1}, \ldots, \sigma_{s + t}\}$ и определим 
	  \[
	  	C_{i} = \begin{cases} \varepsilon, i \neq j \\  C \cdot \varepsilon^{-(s + t - 1)}, i = j\end{cases}.
	  \]
	  Нетрудно заметить, что всё подрбрано таким образом, что 
	  \[
	  		\prod_{i = 1}^{s + t} C_i > \lr*{\frac{4}{\pi}}^{t} \Delta.
	  \]
	  Тогда по лемме~\ref{Mink_theorem} $\exists 0 \neq x \in \cO_{K}\colon $
	  \[
	  		|\sigma_{1}x| < C_1, \ldots, |\sigma_{s}x| < C_{s}, \quad |\sigma_{s + 1}x|^2 < C_{s + 1}, \ldots, |\sigma_{s + t}x|^2 < C_{s + t}.
	  \]
	  Вычислим норму этого $x \in \cO_{K}$
	  \[
	  	\Nm(x\cO_{K}) = |\Nm(x)|  = |\sigma_{1}x| \ldots |\sigma_{s}x| |\sigma_{s + 1}x|^2 \ldots |\sigma_{s + t}x|^2 < \prod_{i = 1}^{s + t} C_{i} = C. 
	  \]
	  Значит, для некоторого $i$ мы имеем $x \cO_{K} = a_i \cO_{K}$. Положим $u = \frac{x}{a_i}$. Тогда $\Nm(u) = 1 \implies u \in \cO_{K}^{*}$. 

	  Так для каждого $\sigma _j$ мы находим свой $u$ (назовём его $u_j$). Проверим, что $\{ u_j \}$ подойдут.  Пусть $\tau = \sigma_i$,
	  \[
	   	|\tau u_j| = \frac{|\tau x|}{|\tau a_k},
	   \] 
	   докажем, что для всех $\tau \neq \sigma_j$ будет выполнено $|\tau u_j| < 1$ (это означает, что в соотвествующей координате будет знак минус). Ясно, что этого будет достаточно, так как сумма координат равна нулю. Рассмотрим два случая: 
	   \begin{itemize}
	   	\item Пусть $\tau \in \{ \sigma_1, \ldots, \sigma_{s} \}, \ \tau = \sigma_i$, тогда 
	   	\[
	   		|\tau u_j| = \frac{|\tau x|}{|\tau a_k|} < \frac{C_i}{\varepsilon} = 1, \text{ так как } i \neq j.
	   	\]
	   	\item Пусть $\tau \in \{ \sigma_{s + 1}, \ldots, \sigma_{s + t} \}$, тогда 
	   	\[
	   		|\tau u_j| = \frac{|\tau x|}{|\tau a_j|} < \frac{\sqrt{C_i}}{\sqrt{\varepsilon}} = 1.
	   	\]
	   \end{itemize}

	   Таким образом, мы показали, что $\Im{\ell}$~--- полная решетка в гипеплоскости, то есть $\Im{\ell} \cong \Z^{s + t - 1}$, откуда, как мы уже замечали в доказательстве слабой теоремы Дирихле о единицах~\ref{Weak_dirichlet_theorem}
	   \[
	   		\cO_{K}^{*} \cong \mu \oplus \Z^{s + t - 1}.
	   \]
	  
	   \end{proof}
	  

	   

	  \begin{statement}[ДЗ 11, задача 3]\label{prop-alpha} 
	  	Пусть $K = \Q(\alpha), \ \alpha^3 + a \alpha + \beta,$ где $a, b \in \Z$. Пусть $(p) = \fp_1 \fp_2 \fp_3$ и $\alpha \in \fp_1 \fp_2$. Тогда $\alpha \in \fp_3$.
	  \end{statement}
	  \begin{proof}
	  	Перепишем данное равенство, как $\alpha(\alpha^2 + a) = - b$ и возьмём норму от обеих частей: 
	  	\[
	  		\Nm(\alpha)\Nm(\alpha^2 + a) = - b^3 \implies \Nm(\alpha) = -b, \ \Nm(\alpha^2 + a) = b^2. 
	  	\]

	  	Далее разберём несколько случаев: 
	  	\begin{enumerate}
	  		\item Если $a \in p\Z$, несложно убедиться, что $p \mid b$ а тогда $\alpha^3 \in p\cO_{K} \subset \fp_{3} \implies \alpha \in \fp_{3}$, так как идеал простой. 

	  		\item Пусть теперь $a \notin p\Z$. Положим $\v_{p}(b) = n \ge 1$. Тогда $\alpha^2 + a \notin \fp_1, \notin \fp_{2}$, но с другой стороны $\Nm(\alpha^2 + a) \divby p \implies \alpha^2 + a \in \fp_{3}$. Заметим, что $\Nm(\fp_3) = p$, $\alpha^2 + a = \fp_{3} \fq$, откуда 
	  		\[
	  			b^2 = \Nm(\alpha^2 + a) = p^s \underbrace{\Nm(\fq)}_{\notdivby p}, \v_{p}(b) = 2n \implies s = 2n. 
	  		\]
	  		Тогда $\alpha(\alpha^2 + a) = -b$, откуда $\v_{\fp_3}(\alpha(\alpha^2 + a)) = 2n$, а с другой стороны
	  		\[
	  		  	\alpha(\alpha^2 + a) = p^n \cdot d, \ (d, p ) = 1 \implies \alpha(\alpha^2 + a) = \fp_1^n \cdot \fp_2^n \cdot \fp_{3}^n,
	  		  \]  
	  		  что даёт нам противоречие. 
	  		
	  	\end{enumerate}




	  	Итак, в качестве контрпримера мы будем рассматривать кубическое уравнение 
	  	\[
	  		3x_1^3 + 4y_1^3 + 5z_1^3 = 0.
	  	\]
	  	Сделаем такие замены переменных: 
	  	\[
	  		z = -z_1, \quad x = 2y_1, y = x_1.
	  	\]
	  	В результате мы получим уравнение 
	  	\[
	  		x^3 + 6y^3 = 10z^3.
	  	\]

	  	Наша задача будет состоять в том, чтобы доказать, что оно не имеет нетривиальных целых решений. Предположим противное и выберем решение с минимальным $|z|$. Заметим, что в таком случае мы работаем в кольце целых поля $\Q(\theta)$, где $\theta^3 = 6$ и тогда 
	  	\[
	  		\Nm(x + \theta y) = 10z^3.
	  	\]
	  	Положм $\alpha = x + \theta y$, $(\alpha) = \fp_1^m \fq$, где $\fp_1 \not\ \mid 2, \ \fp_1 \not\ \mid 5$, но $\fp_1 \mid p \neq 2, 5$.

	  	Далее рассмотрим два случая:

	  	\begin{enumerate}
	  		\item $\fq \not\ \mid p$: применяя норму к равенству выше, получаем 
	  		\[
	  			(10z^3) = \lr*{\Nm(\fp_1)}^m \Nm(q).
	  		\]
	  		В левую часть $p$ входит в кратной трём степени, посмотрим на правую часть. $\Nm(\fp_1)$~--- некоторая степень $p$ от единицы до тройки, пусть $\Nm{\fp_{1}} = p^s$. Тогда мы получаем, что $sm \divby 3$, откуда или $s$, или $p$ делится на 3. Пусть $s \divby 3$, тогда $\Nm{\fp_{1}} = p^3 \implies \fp_{1} = \fp$ (так как эта степень~---  степень инерции), но тогда $\alpha \divby p \implies p \mid x, \ p \mid y$ и мы сможем перейти к аналогичному решению с меньшим модулем, просто поделив на $p$. 

	  		Значит $m \divby 3 \implies$. 

	  		\item Пусть $(\fq, (p)) \neq 1 \implies (\alpha) = \fp_{1}^m \fp_{2} \fq'$ (где $\fp_{2}$~--- еще один простой идеал над $p$). Если $p = \fp_1 \fp_2 \fp_3$ и мы можем применить задачу~\ref{prop-alpha} для 
	  		\[
	  		 	\alpha\theta = (x + y\theta)\theta = x\theta + y\theta^2 \implies \Tr(\alpha\theta) = 0 \implies \alpha\theta \notin \Q.
	  		 \] 
	  		 Т.е. $\alpha$~--- порождающий элемент нашего расширения. Так вот, по предыдущей задаче $\alpha \theta \in (p)$.

	  		 В противном случае $(p) = \fp_1^2 \fp_2$ или $(p) = \fp_1 \fp_2$ или $(p) = \fp_1 \fp_2^2$. В любом из этих случаев мы получаем, что $\alpha^2 \divby p$, но в то же время 
	  		 \[
	  		 	\alpha^2  = (x + y\theta)^2 = x^2 + 2xy\theta + y^2\theta^2 \implies p \mid x, p \mid y
	  		 \]
	  		 и мы снова можем сделать спуск. 
	  	\end{enumerate}
	  	
	  	Теперь мы можем заключить, что $\alpha^3 = I^3 \cdot \fm$, где $\fm$~--- произведение максимальных идеалов над 2 и над 5.  

	  	Совершенно ясно, что $z$ нечётно. Посмотрим, какие идеалы висят над двойкой: по модулю два урванение имеет вид $x^3 - 6 = 0$, откуда над двойкой висит единственный максимальный идеал $\fp_{2} = (2, \theta) = (\theta - 2)$ и двойка является его кубом. В то же время, $\Nm(\theta - 2) = 2$, откуда  $\v_{\fp_{2}}(\alpha)= 1$. Теперь посмотрим на идеалы, висяшие над пятеркой: 
	  	\[
	  		x^3 - 6 = (x - 1)(x^2 + x + 1) \pmod{5} \implies \fp_{5} = (5, \theta - 1) = (\theta - 1), \quad \fp' = (5, \theta^2 + \theta + 1) = (\theta^2 + \theta + 1).
	  	\]
	  	$\Nm(\theta - 1) = 6$, а так как норма пятёрки~--- это $5^3$, мы имеем $\Nm(\theta^2 + \theta + 1) = 5^2$. 

	  	Значит, в $\alpha$ входит либо $\fp_{5}$, либо $\fp'_{5}$, но не одновременно, так как иначе $\alpha$ делится на 5 и можно сделать спуск. 

	  	Проанализируем по отдельности эти два случая: 
	  	\begin{enumerate}
	  		\item $(\alpha) = \fp_{2} \fp_{5}^{?} \cdot \ldots$, возьмём норму в этом равенстве: 
	  		\[
	  			10z^3 = \Nm((\alpha)) = \Nm(\fp_5)^{3s + 1} \cdot \ldots
	  		\]
	  		\item В этом случае $(\alpha) = \fp_2 \fp_5^{3r + 2} \cdot \ldots$. 
	  	\end{enumerate}

	  	Тогда $(\alpha) = \fp_2 \fp_5 J^3 = (\theta - 2)(\theta - 1)J^3 = (\theta - 2)(\theta^2 + \theta + 1)^2 J^3 = ((\theta - 2)(\theta - 1)t^3) = ((\theta - 2)(\theta^2 + \theta + 1)^2 t^3)$. Значит, идеалы отличаются на обратимый элемент, а из теоремы Дирихле о единицах, мы знаем все обратимые элементы, то есть 
  		\[
  			(\alpha) = \alpha_0 t_0^3, \quad \alpha_0 \in \{ (\theta - 2)(\theta - 1), (\theta - 2)(\theta^2 + \theta + 1)^2,  x_1, x_2, x_3, x_4 \},
  		\]

  		где $x_i$~--- результат домножения первых двух элементов на первую и вторую степени основной единицы $\varepsilon = 1 - 6\theta + 3\theta^2$ (кратную трём степень основной единицы мы можем заносить в $t_0^3$). Итак, пусть 
  		\[
  			\alpha = (\theta - 2)(\theta - 1)(u + v\theta + w\theta^2)^3 = (\theta^2 - 3\theta + 2)(u + v\theta + w\theta^2)^3 = x + y \theta.
  		\]
  		Посмотрим на это равенство по модулю 3 и приравнеяем коэффициенты при $\theta^2$: 
  		Скобка с кубом (так как $\theta^3 = 6$) по модулю 3 равна $u^3$, откуда при $\theta^2$ коэффициент по модулю три в левой части равенства будет равен $u^3$ в левой части и 0 в правой части. Значит, $u \equiv 0 \pmod{3}$. Тогда $(u + v\theta + w\theta^2)^3 \equiv 0 \pmod{3}$, откуда $\alpha \divby 3 \implies x, y \divby 3$ и мы можем сделать спуск. Так как $\varepsilon \equiv 1 \pmod{3}$, при домножении на $\varepsilon$ и $\varepsilon^2$ ничего существенно нового не происходит (а оставшийся случай разбирается аналогично). 

	  	


 	  \end{proof}

	  


 



