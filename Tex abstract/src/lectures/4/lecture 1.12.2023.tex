
	Вернёмся к доказательству сильной теоремы Дирихле о единицах. Ясно, что для этого нам достаточно доказать оценку $m \ge s + t - 1$, а это равносильно тому, что $\Im{\ell}$~--- полная решётка в гиперплоскости 

	\[
			x_1 + x_2 + \ldots + x_{s + t} = 0.
	\]

	Для этого необходимо найти систему из $(s + t - 1)$-го линейно независимого над $\R$ элемента $\ell\lr*{\cO_{K}^{*}}$. 

	Соотвественно, надо найти $s + t$ элементов $u_{1}, \ldots, u_{s + t} \in \cO_{K}^{*}$, которые дадут нам $s + t - 1$ линейно независимый над $\R$ вектор в образе. Мы постараемся найти такие $u$, что их образы имеют вид 
	\[
		u_1 \mapsto (+, -, - , \ldots, -), u_2 \mapsto (-, +, -, \ldots, -), \ldots u_n \mapsto (-, -, \ldots, +).
	\]

	Обозначение выше означает, что на соовтествующей координате стоит число соотвествующего знака. Покажем сначала, что такие векторы нам подойдут. Обозначим матрицу, полученную из первых $(s + t - 1)$ координат $\ell(u_i)$ где $i = 1, \ldots, s + t - 1$, как строк за $(a_{i j})$. Ясно, что достаточно доказать, что эта матрица имеет полный ранг. Заметим, что так каак сумма сумма всех элементов в строке и мы вычеркнули один отрицательный, в каждой усеченной строке сумма по строке будет положительной.  Докажем теперь такую лемму: 

	\begin{lemma} 
		Пусть $A \in \mathrm{M}_{m m}(\R)$ такая, что $\forall i  \ a_{ii} > 0$, $\forall i \neq j \ a_{i j } < 0, \ \forall i \ \sum_{j = 1}^{m} a_{i j} > 0$. Тогда $\rank{A} = m$. 
	\end{lemma}

	\begin{proof}
		 Предположим, что $\Ker{A} \neq \{ 0 \}$, то есть система
		 \[
		 	\begin{cases} a_{11} x_1 + \ldots + a_{1 m } x_m = 0 \\ \vdots \\ a_{m 1} x_1 + \ldots + a_{m m} x_m = 0 \end{cases}.
		 \]
		 имеет нетривиальное решение. 

		 В силу симметрии относительно переменных, мы можем полагать, что $x_1$~--- максимальная по модулю координата. Тогда 
		 \[
		 	 0 = |a_{11} x_1 + \ldots + a_{1m} x_m| \ge |a_{11} x_{1}| - |a_{1 2} x_2| - \ldots - |a_{1 m}||x_m| \ge |x_1|(a_{1 1} - |a_{1 2} | - \ldots - |a_{1 m }|) \ge 0,
		 \]
		 откуда $|x_1| = 0 \implies |x_i| = 0 \ \forall i$.
	\end{proof}

	Остаётся найти систему $u_1, \ldots, u_{s + t}$, которые в образе дадут нужные знаки координат. Пусть $n = s + 2t$, рассмотрим множество
	\[
	 	Y = (x_1, \ldots, x_{s}, y_1, z_1, \ldots y_t, z_t), \quad |x_i| < C_i \forall 1 \le i \le s, y_i^2 + z_i^2 < C_{s + i}.
	 \] 

	 Нетрудно проверить, что $Y$~--- ограниченное, выпуклое и центрально-симметричное. Кроме того, 
	 \[
	 	\Vol\lr*{Y} = 2^s \prod_{i = 1}^{s} C_i \cdot \pi^t \cdot \prod_{i = 1}^t C_{s + i} = 2^s \pi^t \cdot \prod_{i = 1}^{s + t} C_i. 
	 \]

	 Пусть $\Gamma$~--- полная решётка, $\Delta$~--- объем фундаментальной области. Тогда, если 
	 \[
	 	2^{s}\pi^{t} \prod_{i = 1}^{s + t} C_i > 2^n \Delta,
	 \]
	 то эта повернхность будет содержать точкк из решетки (по теореме о выпуклом теле). Заметим, что неравенство выше равносильно тоиу, что 
	 \[
	 	\prod_{i = 1}^{s + t}C_i >  \lr*{\frac{4}{\pi}}^t \cdot \Delta.
	 \]
	 Возьмём $C > \lr*{\frac{4}{\pi}}^t \Delta$. Рассмотрим все $a_i \cO_{K} \subset \cO_{K}\colon \Nm(a_i \cO_{K}) < C$. 
	 Пусть $\varepsilon = \min\lr*{|\sigma_i a_{j}|, |\sigma_{s + } a_{}j|^2}$ > 0.
	  Выберем некоторый $\sigma_{k}$ и зафиксируем его. Положим 
	  \[
	  	C_{i} = \begin{cases} \varepsilon, i \neq k \\  C \cdot \varepsilon^{-(s + t - 1)}, i = k\end{cases}.
	  \]

	  Тогда мы имеем 
	  \[
	  		\prod_{i} C_i > \lr*{\frac{4}{\pi}}^{t} \Delta.
	  \]
	  Заметим, что $\exists 0 \neq x \in \cO_{K}\colon$
	  \[
	  		|\sigma_{s}x| < C_1, \ldots, |\sigma_{s}x| < C_{s}, |\sigma_{s + 1}x|^2 < C_{s + 1}, \ldots, |\sigma_{s + t}x|^2 < C_{s + t}.
	  \]
	  Тогда оценка, нужная нам для теоремы Минковского будет выполнятся и $Y$ будет содержать ненулевую точку  $x$ решётки. Вычислим норму этого элемента: 
	  \[
	  	\Nm(x\cO_{K}) = |\Nm(x)| < \prod_{i} C_i  = C  \implies x\cO_{K} = a_i \cO_{K} \text{ для некоторого } i \implies u = \frac{x}{a} \in \cO_{K}^*.
	  \]

	  Так как мы для каждого $\sigma_k$ нашли такой обратимый элемент, проделывая такую операцию для каждого $\sigma_i \in \Gal(K/\Q)$, мы получим искомую систему из $s + t$ обратимых элементов.  Поймём, почему мы получили систему с нужной расстановкой знаков. Ясно, что так как там в координатах стоят логарифмы, достаточно сравнивать модули того, что под логарифмами,с единицами. 

	  Возьмём некоторыы $\tau \in \{ \sigma_1, \ldots \sigma_{s}, \sigma_{s + 1}, \ldots, \sigma_{s + t} \}.$ Тогда 
	  \[
	  	|\tau u| = \frac{|\tau x|}{|\tau a_i|} = 
	  \]


	  Проверим, что если $\tau  \neq \sigma_k$, то $|\tau u| < 1$ (этого достатчно, так как если мы покажем, что все координаты отрицательны, то, так как их сумма равна нулю, оставшаяся автоматически будет положительной). Если  $\tau \neq \sigma_k$ и $\tau$~--- вещественно, то 
	  \[
	  	|\tau u| = \frac{|\tau x|}{|\tau a_i|} < \frac{C_i}{\varepsilon} = 1.
	  \]
	  Аналогично будет разбираться случай, когда $\tau$~--- комплексное: 
	  \[
	  	|\tau u| < \frac{|\tau x|}{|\tau a_i|} < \frac{\sqrt{C_{s + i}}}{\sqrt{\varepsilon}} = 1.
	  \]

	  Значит, наша система будет действительно искомой: $+$ будет в $k$-й координате, а в остальных~--- минусы. 

	  Итак, таким образом, мы наконец доказали сильную теорему Дирихле о единицах: 

	  \begin{theorem}[Дирихле, о единицах, сильная форма]
	  Пусть $K/\Q$~--- конечное расширение, а числа $s$ и $t$, связанные с количеством вложений числового поля $K \to \Q^{alg}$ определены как в \hyperlink{real_and_complex_inclusions}{}. Тогда мультипликативная группа кольца целых числового поля $K$ имеет вид: 
	  	\[ \cO_{K}^{*} \cong \mu \oplus \Z^{s + t - 1}, \]
	  	где $\mu$~--- группа корней из единицы.
	  \end{theorem}

	  \subsection{Контпример к локально-глобальному принципу.}

	  Проделаем сначала некоторую подготовительную работу. 

	  \begin{statement}[ДЗ \ref{hw:11}, задача 3] 
	  	Пусть $K = \Q(\alpha), \ \alpha^3 + a \alpha + \beta,$ где $a, b \in \Z$. Пусть $(p) = \fp_1 \fp_2 \fp_3$ и $\alpha \in \fp_1 \fp_2$. Тогда $\alpha \in \fp_3$.
	  \end{statement}
	  \begin{proof}
	  	Перепишем данное равенство, как $\alpha(\alpha^2 + a) = - b$ и возьмём норму от обеих частей: 
	  	\[
	  		\Nm(\alpha)\Nm(\alpha^2 + a) = - b^3 \implies \Nm(\alpha) = -b, \ \Nm(\alpha^2 + a) = b^2. 
	  	\]

	  	Далее разберём несколько случаев: 
	  	\begin{enumerate}
	  		\item Если $a \in p\Z$, несложно убедиться, что $p \mid b$ а тогда $\alpha^3 \in p\cO_{K} \subset \fp_{3} \implies \alpha \in \fp_{3}$, так как идеал простой. 

	  		\item Пусть теперь $a \notin p\Z$. Положим $\v_{p}(b) = n \ge 1$. Тогда $\alpha^2 + a \notin \fp_1, \notin \fp_{2}$, но с другой стороны $\Nm(\alpha^2 + a) \divby p \implies \alpha^2 + a \in \fp_{3}$. Заметим, что $\Nm(\fp_3) = p$, $\alpha^2 + a = \fp_{3} \fq$, откуда 
	  		\[
	  			b^2 = \Nm(\alpha^2 + a) = p^s \underbrace{\Nm(\fq)}_{\notdivby p}, \v_{p}(b) = 2n \implies s = 2n. 
	  		\]
	  		Тогда $\alpha(\alpha^2 + a) = -b$, откуда $\v_{\fp_3}(\alpha(\alpha^2 + a)) = 2n$, а с другой стороны
	  		\[
	  		  	\alpha(\alpha^2 + a) = p^n \cdot d, \ (d, p ) = 1 \implies \alpha(\alpha^2 + a) = \fp_1^n \cdot \fp_2^n \cdot \fp_{3}^n,
	  		  \]  
	  		  что даёт нам противоречие. 
	  		
	  	\end{enumerate}

 	  \end{proof}

	  


 



