    \subsection{Подмногообразия и аналитические подмножества}
    
    Докажем теперь несколько классических теорем для случая комплексных многообразий. 
    
    \begin{theorem}[Об обратном отображении]
        Пусть $U, V$~--- открытые подмножества в $\C^n$, $0 \in U$ и $f\colon U \to V$~---
        такое голоморфное отображение, что матрица $\cJ(f) = \lr*{\partial f_i / \partial z_{j}}$ невырождена в $0$.
        
        Тогда отображение $f$ взаимно однозначно в окрестности точки $0$ и обратное отображение $f^{-1}$
        голоморфно в некоторой окрестности $f(0)$.
    \end{theorem}
    
    \begin{proof}
        Как мы уже отмечали в~\ref{rem1}, $\left\lvert \det \cJ_{\R}(f) \right\rvert = |\det{\cJ(f)}|^2 \neq 0$ в точке $0$, а
        значит, по обычной теореме об обратном отображении, функция $f$ имеет в окрестности точки $0$
        обратную $C^{\infty}(U, V)$ функцию $f^{-1}$. Заметим, что $f^{-1}(f(z)) = z$, так что,
        дифференцируя это равенство в нуле мы имеем
        \[ 0 = \frac{\partial}{\partial \overline{z_j}}(f^{-1}(f(z)))_{j} = \sum_{k} \frac{\partial f^{-1}_{j}}{\partial z_k}\frac{\partial f_k}{\partial \overline{z_i}} + \sum_{k} \frac{\partial f^{-1}_{j}}{\partial \overline{z_k}}\lr*{\frac{\partial \overline{f_k}}{\partial\overline{z_i}}} = \sum_{k} \frac{\partial f_{j}^{-1}}{\partial \overline{z_k}} \lr*{\frac{\partial \overline{f_k}}{\partial z_i}} \quad \forall i, j. \]
        Так как матрица $(\partial f_k / \partial z_j)$ была невырождена, отсюда следует, что
        $\partial f_j^{-1} / \partial \overline{z_k} = 0 \ \forall j, k$, что и означает голоморфность функции $f$.
    \end{proof}

    \begin{theorem}[О неявной функции]\label{Complex_implicit_function}
        Пусть заданы функции $f_1, \ldots, f_k \in \cO_n$, удовлетворяющие условию
        \[ \det{\lr*{\frac{\partial f_i}{\partial z_j}(0)}}_{1 \le i, \ j \le k} \neq 0. \]

        Тогда существуют такие функции $w_1, \ldots, w_k \in \cO_{n - k}$, что в окрестности точки $0 \in \C^n$
        \[ f_1(z) = \ldots f_k(z) = 0 \Leftrightarrow z_i = w_i(z_{k + 1}, \ldots, z_n), \ 1 \le i \le k.  \]

    \end{theorem}

    \begin{proof}
        Как обычно, по обычной теореме о неявной функции в случае $C^{\infty}$ существуют функции $\omega_{1}, \ldots, \omega_{k}$ с нужным свойством.
        Остается показать голоморфность. Это делается непосредственно вот таким стандартным вычислением вычислением:
        \[ 0 = \frac{\partial}{\partial \overline{z_{\alpha}} } (f_{j}(\omega(z), z)) = \ldots = \sum \frac{\partial \omega_{\ell}}{\partial \overline{z_{\alpha}}} \frac{\partial f_{j}}{\partial \omega_{\ell}} \Rightarrow \frac{\partial \omega_{\ell}}{\partial \overline{z_{\alpha}}} = 0 \ \forall \alpha, \ell, \]
    \end{proof}
    
    \begin{remark}
       Видимо почти всегда, когда мы хотим показать голоморфность, мы тупо считаем в локальных производных антиголоморфную производную.
    \end{remark}

    Теперь мы увидим, что комлексные многообразия в смысле их морфизмов таки имеют свою, отличную от вещественной, специфику:

    \begin{statement}
        Если $f\colon U \to V$~--- взаимно однозначное голоморфное отображение открытых множеств в $\C^n$, то
        $\det{\cJ(f)} \neq 0$, то есть $f^{-1}$ голоморфно.
    \end{statement}

    \begin{remark}
       Мы видели этот факт в обычном  комплексном анализе (доказывали, что производная однолистной функции не обнуляется).
    \end{remark}

    \begin{definition}
        \emph{Комплексным подмногообразием $S$} комплексного многообразия $M$ называется подмножество $S \subset M$,  которое локально задается либо как множество нулей
        совокупности голоморфных функций $f_{1}, \ldots f_{k}$ с условием $\rank{\cJ(f)} = k$, либо как образ открытого подмножества
        $U \subset \C^{n - k}$ при отображении $f\colon U \to M$ с условием $\rank{\cJ(f)} = n - k$.
    \end{definition}
    
    Эквивалентность этих определений следует из теоремы о неявной функции~\ref{Complex_implicit_function}.
    
    \begin{definition}
        Аналитическим подмножеством $V$ комплексного многообразия $M$ называется подмножество, являющееся локально множеством нулей конечного набора голоморфных функций.

       Точка $p \in V$ называется \emph{гладкой}\footnote{возможно, корректнее использовать слово регулярная?} точкой $V$, если $V$
        в некоторой её окрестности задаётся набором голоморфных функций $f_{1}, \ldots, f_{k}$, причем таким, что $\rank{\cJ(f)} = k$.

        Множество гладких точек $V$ обозначается $V^{*}$, а все точки из $V \setminus V^{*}$ называются \emph{особыми}.
        Они формируют \emph{множество особенностей} аналитического подмножества $V$, которое мы будем обозначать, как $V_{s}$.

        В частности, если $p$~--- точка аналитической гиперповерхности $V \subset M$, задаваемой в локальных координатах $z$ функцией $f$, определим \emph{кратность} $\mult_{p}(V)$, как
        порядок обращения $f$  в нуль в точке $p$, то есть наибольшее такое $m$, что
        \[ \frac{\partial^{k}f}{\partial z_{i_{1}} \ldots \partial z_{i_{k}}} = 0 \ \forall k \le m - 1. \]

    \end{definition}
    
    \begin{statement}
        Множество $V_{s}$ содержится в аналитическом подмножестве многообразия $M$, не совпадающем с $V$. 
    \end{statement}
    \begin{remark}
       А на самом деле, при аккуратном выборе функций, несложно показать, что $V_{s}$~--- аналитическое подмножество в $M$.
    \end{remark}   

    Запомним также полезный нам в будущем факт:

    \begin{statement}
        Аналитическое множество $V$ неприводимо тогда и только тогда, когда $V^{*}$ связно.
    \end{statement}

    \textcolor{red}{Тут было еще что-то про касательные конусы, пока что забьем на это, лень читать. }


    \subsection{Когомологии де Рама и Дольбо}

    Пусть $M$~--- гладкое многообразие. Обозначим за $A^{p}(M; \R)$
    пространство дифференциальных форм степени $p$ на $M$, а через $Z^{p}(M; \R)$
    подпространство замкнутых $p$-форм.

    Так как $d^2 = 0$, у нас есть (ко)цепной комплекс
    \[ A^{0}(M; \R) \to \ldots \to A^{p}(M; \R) \to A^{p + 1}(M; \R) \to \ldots \]
    а его группы когомологий называются группами \emph{когомологий де Рама} многообразия $M$.

    Иными словами, группы когомологий де Рама~--- это факторгруппы замкнутых форм по модулю точных
    \[ H_{\textrm{DR}}^{p}(M; \R)  = Z^{p}(M; \R) / d A^{p - 1}(M). \]

    Совершенно также мы можем рассматривать комплекснозначные формы и давать все соотвествующие определения (используя обозначения $ A^{p}(M)$ и аналогичные, то
    есть без коэффициентов):
    \[ H_{\mathrm{DR}}^{p}(M) = Z^{p}(M)/d A^{p - 1}(M) \]

    \begin{remark}
       Нетрудно заметить, что как и всегда с коэффициентами,
        \[ H^{p}_{\mathrm{DR}}(M) = H_{\mathrm{DR}}^{p}(M; \R) \otimes \C. \]
    \end{remark}

    Как мы заметили в самом первом параграфе, комплексифицированное кокасательное пространство раскладывается в голоморфную и антиголоморфную часть:
    \[ T^{*}_{\C, z}M = T_{z}^{*'} M \oplus T_{z}^{*''} M,  \]
    что дает нам разложение
    \[ \Lambda^n T^{*}_{\C, z}M = \bigoplus_{p + q = n}\lr*{ \Lambda^{p} T^{*'}_{z}(M) \otimes \Lambda^{q} T_{z}^{*''}(M)}, \]
    а это (по определению внешних форм) даёт нам
    \[ A^{n}(M) = \bigoplus_{p + q = n} A^{p, q}(M), \text{где }\]
    \[ A^{p, q}(M) = \{ \varphi \in A^{n}(M) \ \vert \ \varphi(z) \in \Lambda^{p}T_{z}^{*'}(M) \otimes \Lambda^q T_{z}^{*''}(M) \ \forall z \in M \}. \]

    Соотвественно, форму $\varphi \in A^{p, q}$ называют формой типа $(p, q)$. Обозначим за $\pi^{(p, q)}$ проекцию
    \[ A^{*}(M) \to A^{p, q}(M), \]
    так что для $\varphi \in A^{*}(M)$ имеем $\varphi = \sum \pi^{(p, q)}\varphi$.

    Если $\varphi A^{p, q}(M)$, то для любого $z \in M$
    \[ \mathrm{d}\varphi(z) \in \lr*{\Lambda^{p} T_{z}^{*'}M \otimes \Lambda^{q} T^{*''}_{z}M} \wedge T^{*}_{\C, z}M, \]
    \[ \mathrm{d}\varphi \in A^{p + 1, q}(M) \oplus A^{p, q + 1}(M). \]
    Определоим теперь для этих замечательных дифференциальных форма операторы
    \[ \overline{\partial}\colon A^{p, q}(M) \to A^{p, q + 1}, \quad \partial \colon A^{p, q}(M) \to A^{p + 1, q}(M) \]
    \[ \overline{\partial} = \pi^{(p, q + 1)} \circ \mathrm{d}, \quad \partial = \pi^{(p + 1, q)} \circ \mathrm{d}, \text{ то есть } d = \partial + \overline{\partial}. \]

    В локальных координатах $z = (z_{1}, \ldots, z_{n})$ форма $\varphi \in A^{n}(M)$ имеет тип  $(p, q)$, если она имеет
    представление в виде
    \[ \varphi(z) = \sum_{I, J} \varphi_{I, J}(z) \mathrm{d}z_{I} \wedge \mathrm{d}\overline{z}_{J} \]

    \begin{remark}
       Короче говоря, вся эта страшная белиберда была, чтоб сказать, что бывают не только голоморфные дифференциальные формы, но и такие, где
        один кусок голоморфный, а другой антиголоморфный.
    \end{remark}

   Дифференцировать эти формы можно так:
    \[ \overline{\partial}\varphi(z) = \sum_{I, J, j} - \frac{\partial}{\partial \overline{z_{j}}} \varphi_{I J}\]

    







