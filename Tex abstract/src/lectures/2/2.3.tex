
	\subsection{Дивизоры и линейные расслоения}

	\noindent\bf{Дивизоры.}

	Пусть $M$~--- $n$-мерное комплексное многообразие. Тогда любое аналитическое подмножество $V \subset M$ размерности $n - 1$ является аналитической гиперповерхностью, то есть в окрестности каждой точки $p \in V \subset M$ оно может быть задано как множество нулей  некоторой голоморфной функции $f$. Кроме того, ясно, что любая голморфная функция $g$, обращающаяся в нуль на $V$, делится на $f$ в окрестности точки $p$. 

	Пусть $V_1^*$~--- компонента связности $V^* = V \setminus V_s$, тогда $\overline{V_1^*}$~--- аналитическое подмножество в $M$. Соотвественно, $V$ единственным образом представляется в в виде объединения неприводимых аналитических гиперповерхностей 
	\[
		V = V_1 \cup \ldots \cup V_m \quad V_i = \overline{V_i^*}. 
	\]

	\begin{definition} 
		\emph{Дивизором} $D$ на многооборазии $M$ называется локально конечная формальная линейная комбинация  
		\[
			D = \sum_{i} a_i V_i
		\]

		неприводимых аналитических гиперповерхностей в $M$. 
	\end{definition}

	\begin{remark}
		Локальная конечность означает, что для любой точки $p \in M$ существует её окрестность, пересекающаяся лишь с конечным числом гиперповерхностей $V_i$, входящих в $D$. В случае копактного многообразия дивизоры образуют абелеву группу по сложению, которую мы будем обозначать $\Div(M)$.
	\end{remark}

	\begin{definition} 
		Дивизор $D = \sum a_i V_i$ называется \emph{эффективным}, если $a_i \ge 0 \ \forall i$. Обычно это обозначают, как $D \ge 0$. 
	\end{definition}

	\begin{remark}
		Если у нас есть аналитическая гиперповерзность $V$, неприводимые компоненты которой имеют вид $\{ V_i \}$, мы отождествляем её с дивизором $\sum V_i$. 
	\end{remark}

	\begin{definition} 
		Пусть $V \subset M$~--- неприводимая аналитическая гиперповерхность, а $f$~--- функция, локально определяющая $V$ вблизи $p$. Тогда для голоморфной $g$, заданной в окрестности $p$ определим $\ord_{V, p}(g)$, как наибольшее целое число $a$, при котором в кольце $\cO_{M, p}$  имеет место разложение $g = f^a h$.
	\end{definition}

	\begin{remark}
		Ясно, что это определение не зависит от точки $p$, так как взаимнопростые элементы $\cO_M$ отаются взаимнопротыми в близких локальных кольцах $\cO_{M, p}$. Ясно, что тогда порядок можно обозначать, как $\ord_{V}(g)$.
	\end{remark}

	Ясно, что порядок обладает таким вот свойством: 
	\[
		\ord_{V}(gh) = \ord_{V}(g) + \ord_{V}(h).
	\]

	\begin{definition} 
		Пусть $f$~--- мероморфная функция на $M$, записаная локально в виде $g/h$, где $(g, h) = 1$ и $g, h$ голоморфны. Тогда для неприводимой гиперповерхности $V$ положим 
		\[
			\ord_{V}(f) = \ord_{V}(g) - \ord_{V}(h).
		\]
	\end{definition}

	\begin{remark}
		Соотвественно, $f$ имеет нуль порядка $a$ на $V$, если $\ord_{V}(f) = a > 0$  и полюс порядка $a$ на $V$, если $\ord_{V}(f) = a < 0$.
	\end{remark}

	\begin{definition} 
		\emph{Дивизором $(f)$} мероморфной функции $f$ называют  формальную линейную комбинацию 
		\[
			(f) = \sum \ord_{V}(f) V
		\]

		Если $f$ локально представлена в виде $f = g/h$, то \emph{дивизором нулей} называют 
		\[
			(f)_{0} =\sum \ord_{V}(g) \cdot V,
		\]
		а \emph{дивизором полюсов} называют 
		\[
			(f)_{\infty}  =\sum \ord_{V}(h) \cdot V,
		\]
	\end{definition}

	\begin{remark}
		Если дивизоры нулей и полюсов корректно определны и $(g, h) = 1$, то 
		\[
			(f) = (f)_{0} - (f)_{\infty}.
		\]
	\end{remark}

	Про дивизоры, конечно же, можно говорить в терминах пучков. Пусть $\cM^{*}$~--- мультипликативный пучок мероморфных функций на $M$, $\cO^*$~--- его подпучок не обращающихся в нуль (обратимых) голоморфных функций. Тогда дивизор $D \in \Div(M)$~--- это просто глобальное сечение факторпучка $\cM^*/\cO^*$.

	Действительно, глобальное сечение $\{ f_{\alpha} \}$ факторпучка $\cM^{*}/\cO^{*}$ задаёься открытым покрытием $\{ U_{\alpha} \}$ многообразия $M$ и такими ненулевыми мероморфными функциями $f_{\alpha}$ на $U_{\alpha}$, что 
	\[
		f_{\alpha}/f^{\beta} \in \cO^{*}\lr*{U_{\alpha} \cap U_{\beta}}. 
	\]

	Тогда для любой гиперповерхности $V \subset M$ $\ord_{V}{f_{\alpha}} = \ord_{V}{f_{\beta}}$ и набор $\{ f_{\alpha} \}$ задаёт дивизор 
	\[
		D = \sum \ord_{V}(f_{\alpha}) \cdot V,
	\]
	где $\alpha$ для каждого $V$ выбран так, что $V \cap U_{\alpha} \neq \varnothing$. 


	Теперь рассмотрим дивизор $D = \sum a_i V_i$, по нему можно построить такое открытое покрытие $\{ U_{\alpha} \}$ многообразия $M$, что в каждом $U_{\alpha}$ все $V_i$ из $D$ локально определяются функциями $g_{i, \alpha} \in \cO(U_{\alpha})$. Тогда мы можем положить 
	\[
		f_{\alpha} = \prod_{i} g_{i, \alpha}^{a_i} \in \cM^{*}\lr*{U_{\alpha}}
	\]

	и получить глобальное сечение факторпучка $\cM^*/\cO^*$. Соотвественно, вот эти вот функции $\{ f_{\alpha} \}$ называют функциями, локально задающими дивизор. Отсюда мы имеет отождествеление 
	\[
		H^{0}\lr*{M, \cM^{*}/\cO^{*}} \cong \Div(M).
	\]

	
