    \subsection{Комплексные многообразия}

    \begin{definition}
        \emph{Комплексным многообразием} $M$ называется гладкое многообразие, допускающее
        такое открытое покрытие $\{ U_{\alpha} \}_{\alpha \in I}$ и такие координатные отображения
        $\varphi_{\alpha}\colon U_{\alpha} \to \C^n$, что все функции перехода $\varphi_{\alpha} \circ \varphi_{\beta}^{-1}$
        голоморфны на $\varphi_{\beta}(U_{\alpha} \cap U_{\beta})$.

        Функция $f$ на открытом подмножестве $U \subset M$ называется \emph{голоморфной}, если $\forall \alpha \in I$
        функция $f \circ \varphi^{-1}_{\alpha}$ голоморфна в $\varphi_{\alpha}(U_{\alpha} \cap U)$.

        Набор $z = (z_1, \ldots, z_n)$ функций на $U \subset M$ называется \emph{голоморфной системой координат}, если $\varphi_{\alpha} \circ z^{-1}$ и $z \circ \varphi^{-1}_{\alpha}$ голоморфны на $z(U \cap U_{\alpha})$ и $\varphi_{\alpha}(U \cap U_{\alpha})$
        для всех $\alpha$.

        Отображение $f\colon M \to N$, где $M$ и $N$~--- комплексные многообразия, называется \emph{голоморфным}, если в голоморфных локальных координатах оно
        задаётся голоморфными функциями.
    \end{definition}

    \begin{example}[Примеры комплексных многообразий]
        Приведём какие-нибудь примеры комплексных многообразий:

        \begin{enumerate}
            \item Одномерное комплексное многообразие называют \bf{римановой поверхностью}.

            \item $P\mathbb{C}^n  = (\C^{n + 1}\setminus \{ 0 \})/\{ z \sim \lambda z \} = \mathbb{P}^n$~--- комплексное проективное пространство.
            Это пространство компактно, так как есть непрерывное сюръективное отображение $S^n \subset \C^{n + 1} \to \mathbb{P}^n$.

            \item Пусть $\Lambda = \Z^{k} \subset \mathbb{C}^n$~--- дискретная решётка. Факторгруппа $\C^n/\Lambda$ обладает структурой
            комплексного многообразия, которую индуцирует проекция $\pi\colon \C^n \to \C^n/\Lambda$.
            Это многообразие компактно тогда и только тогда, когда $k = 2n$ и в этом случае $\C^n / \Lambda$ называется \bf{комплексным тором}.

            \item \textcolor{magenta}{Тут был еще пример, что при неразветвлённом накрытии структура комплексного многообразия наследуется, но я хз, что такое разветвлённое накрытие.}
        \end{enumerate}
    \end{example}

    \noindent\bf{Касательное пространство к комплексному многообразию.}\hypertarget{complex_tungent_space}{}

        Пусть $M$~--- комплексное многообразие, $p \in M$, а $z = (z_{1}, \ldots, z_n)$~--- система голоморфных координат в окрестности $p$.
        В случае комплексного многообразия имеются три различных понятия \emph{касательного пространства} к $M$ в точке $p \in M$.

        \begin{enumerate}
            \item Рассмотрим $M$, как вещественное $2n$-многообразие. Тогда $T_{\R, p}M$~--- пространство
            $\R$-линейных дифференцирований кольца $C^{\infty}(M, \R)$ (с носителем в окрестности $p$).
            Если мы представим голоморфные координаты в виде $z_j = x_j + i y_j$, то $T_{\R, p}M$ будет иметь базис
            $\{ \frac{\partial }{\partial x_j}, \ \frac{\partial}{\partial y_j}\}$, как векторное пространство над $\R$.

            \item Пространство  $T_{\R, p}M$ можно комплексифицировать при помощи расширения скаляров, то есть рассмотреть
            \[ T_{\C, p}M \eqdef T_{\R, p}M \otimes_{\R} \C. \]
            $T_{\C, p}M$ называют \emph{комплексифицированным касательным пространством} к $M$ в точке $p$.
            Его можно реализовать, как пространство $\C$-линейных дифференцирований кольца $C^{\infty}(M, \C)$ (опять же, фукнции с носителем в окрестности $p$).
            Соотвественно, там можно выбрать базис $\{ \frac{\partial}{\partial x_j}, \frac{\partial}{\partial y_j}\}$, а при замене базиса на комлпексные обозначения
            \[ \frac{\partial}{\partial z_j} = \frac{1}{2} \lr*{\frac{\partial}{\partial x_j} - i \frac{\partial}{\partial y_j}}, \ \frac{\partial}{\partial\overline{z_j}} = \frac{1}{2}\lr*{\frac{\partial}{\partial x_j} + i \frac{\partial}{\partial y_j}}. \]
            <<более стандартный>> базис $\{ \frac{\partial}{\partial z_j}, \ \frac{\partial}{\partial \overline{z_j}}\}$.

            \item Подпространство $T'_{p}M = \Span\{ \frac{\partial }{\partial z_j} \} \le T_{\C, p}M$ называется  \emph{голоморфным касательным пространством} к $M$
            в точке $p$. Оно может быть реализовано, как подпространство в $T_{\C, p}M$, состоящее из дифференцирований, обращающихся в ноль на антиголоморфных функциях (таких $f$, что $\overline{f}$~--- голоморфна).
            Соответственно, подпространство $T''_{p}M = \Span \{ \frac{\partial }{\partial \overline{z_j}}\}$ называется \emph{антиголоморфным касательным пространством} к $M$ в точке $p$.
            Ясно, что
            \[ T_{\C, p}M = T'_{p}M \oplus T''_{p}M. \]
        \end{enumerate}

        Заметим, что для  комплексных многообразий $M, N$ любое $f \in C^{\infty}(M, N)$ индуцирует линейное отображение
        \[ f_{*}\colon T_{\R, p}M \to T_{\R, f(p)}N \]
        а значит и линейное отображение
        \[ f_{*}\colon T_{\C, p}M \to T_{\C, f(p)}N, \]
        но не отображение $T'_{p}M \to T'_{f'(p)}N$ для всех $p \in M$.

    На самом деле, отображение $f\colon M \to N$ голоморфно тогда и только тогда, когда
    \[ f_{*}(T'_{p}M) \subset T'_{f(p)}N \quad \forall p \in M. \]
    То есть, когда голоморфное касательное пространство отображается в голоморфное.

    Заметим, что также, поскольку $T_{\C, p}M = T_{\R, p}M \otimes \C$, операция сопряжения, переводящая
    \[ \frac{\partial}{\partial z_j} \mapsto \frac{\partial}{\partial \overline{z_j}} \]
    корректно определена на $T_{\C, p}M$ и, как нетрудно заметить, $T''_{p}M = \overline{T'_{p}M}$. Отсюда следует, что проекция
    \[ T_{\R, p}M \to T_{\C, p}M \to T'_{p}M \]
    есть $\R$-линейный изоморфизм.
    
    Это обстоятельство позволяет заниматься геометрией исключительно в голоморфном касательном пространстве.
    
    \begin{example}
        Пусть $z(t) \colon [0, 1] \to \C$~--- гладкая кривая.  Тогда $z(t) = x(t) + iy(t)$ и в качестве касательной мы можем взять
        \[ x'(t) \frac{\partial}{\partial x} + y'(t) \frac{\partial}{\partial y} \text{ в } T_{\R}\C \text{, либо } z'(t) \frac{\partial}{\partial z} \text{ в } T'\C. \]
    \end{example}


    \begin{definition}
        Пусть теперь $M, N$~--- комплексные многообразия, $z = (z_1, \ldots, z_n)$~--- голомрфные координаты в окрестности
    точки $p \in M$, а $(w_1, \ldots, w_n)$~--- голоморфные координаты в окрестности точки $q = f(p)$, где $f\colon M \to N$~--- голоморфное отображение.
    В связи с различными понятиями касательных пространств, мы имеем и различные понятия \emph{якобиана} $f$.

    \begin{enumerate}
        \item Пусть $z_j = x_j + i y_j, \ w_{k} = u_{k} + i v_{k}$. Тогда в базисах
        $\{ \frac{\partial}{\partial x_j}, \frac{\partial}{\partial y_j}\}$ и $\{ \frac{\partial}{\partial u_k}, \frac{\partial}{\partial v_k}\}$ пространств
        $T_{\R, p}M$ и $T_{\R, q}N$ линейное отображение $f_{*}$ задаётся $2m \times 2n$-матрицей
        \[ \cJ_{\R}(f) = \begin{pmatrix} \frac{\partial u_k}{\partial x_j} & \frac{\partial u_k}{\partial y_j} \\ \frac{\partial v_k}{\partial x_j} & \frac{\partial v_k}{\partial y_j} \end{pmatrix}. \]

        В базисах $\{ \frac{\partial}{\partial z_j}, \frac{\partial}{\partial \overline{z_j}}\}$ и $\{ \frac{\partial}{\partial w_j}, \frac{\partial}{\partial \overline{w_k}}\}$ пространств
        $T_{\C, p}M$ и $T_{\C, q}N$ отображение $f_{*}$ задаётся матрицей
        \[ \cJ_{\C}(f) = \begin{pmatrix} \cJ(f) & 0 \\ 0 & \overline{\cJ(f)} \end{pmatrix}, \text{ где } \cJ(f) = \lr*{ \frac{\partial w_k}{\partial z_j} }_{k, j}. \]
    \end{enumerate}

    \end{definition}

    \begin{remark}\label{rem1}
       В частности, отметим, что $\rank \cJ_{\R}(f) = 2 \rank \cJ(f)$ и в случае $m = n$
       \[ \det\cJ_{\R}(f) = \det \cJ(f) \det \overline{\cJ(f)} = \left\lvert \det{\cJ(f)}\right\rvert^2 \ge 0, \]
        то есть голоморфные отображения \bf{сохраняют ориентацию}.
    \end{remark}

    Мы будем считать, что пространство $\C^n$ естественно ориентированно $2n$-формой
    \[ \lr*{\frac{i}{2}}^n (\mathrm{d}z_{1} \wedge \mathrm{d}\overline{z_1}) \wedge (\mathrm{d}z_{2} \wedge \mathrm{d}\overline{z_2}) \wedge \ldots \wedge (\mathrm{d}z_n \wedge \mathrm{d}\overline{z_n}) = \mathrm{d}x_1 \wedge \mathrm{d}y_1 \wedge \ldots \wedge \mathrm{d}x_n \wedge \mathrm{d}y_n. \]

    Ясно, что если $\varphi_{\alpha}\colon U_{\alpha} \to \C^n$ и $\varphi_{\beta}\colon U_{\beta} \to \C^n$~--- голоморфные координатные отображения на
    комплексном многообразии $M$, то прообразы при $\varphi_{\alpha}$ и $\varphi_{\beta}$
    естественной ориентации на  $\C^n$ согласованы на $U_{\alpha} \cap U_{\beta}$.

    Соотвественно, любое комплексное многообразие \bf{имеет естественную ориентацию}, которая сохраняется
    при голоморфных отображениях.

    \subsection{Векторные рассоления}

    \begin{definition} 
        Пусть $M$~--- гладкое многообразие. Комплексным $C^{\infty}$-расслоением на $M$
    называется семейство $\{ E_{X} \}_{x \in M}$ комплексных векторных пространств $E_{x}$, параметризованных точками многообразия $M$,
    со структурой $C^{\infty}$ многообразия на 
    \[
        E = \bigcup_{x \in M} E_{x}
    \]
    такой, что выполняются следующие условия: 
    \begin{enumerate}
        \item отображение проектирования $\pi\colon E \to M$, переводящее $E_x$ в $x$ принадлежит классу $C^{\infty}$.
        \item $\forall x_0 \in M$ найдутся открытое множество $U \subset M\colon U \ni x_0$  и диффеоморфизм 
        \[
            \varphi_{U}\colon \pi^{-1}(U) \to U \times \C^k,
        \]
        который отображает вектоорное пространство $E_x$ изоморфно на $\{ x \} \times \C^k$  для всех $x \in U$.
        Такое отображение $\varphi_{U}$ называется \emph{тривиализацей}. 
    \end{enumerate}
    
    Размерность слоёв $E_{x}$ расслоения $E$ называется \emph{рангом} $E$. Расслоение ранга 1 называется \emph{линейным}.
    \end{definition}

    \begin{remark}
        Для любой пары тривиализаций $\varphi_{U}, \varphi_{V}$ \emph{отображение} перехода $g_{uv}(x) = (\varphi_{U} \circ \varphi_{V}^{-1})\vert_{\{ x \} \times \C^k \}}\colon U \cap V \to \mathrm{GL}(k)$ принадлежит классу $C^{\infty}$. Кроме того, они удовлетворяют тождествам: 
        \[
            g_{UV}(x) \cdot g_{VU}(x) = I \quad \ \forall x \in U \cap V 
        \]
        \[
            g_{UV}(x) g_{VW}(X) \cdot g_{WU}(x) = I \quad \ \forall x \in U \cap V \cap W 
        \]
    \end{remark}

    Обратно, если задано открытое покрытие  $\cU = \{ U_{\alpha} \}$ многообразия $M$ и $C^{\infty}$ отображения  $g_{\alpha \beta} \colon U_{\alpha} \cap U_{\beta} \to \mathrm{GL}(k)$, удовлетворяющие тождествам выше, то найдётся едиснвтенное комплексное векторное расслоение $E \to M$ с такими функциями перехода. 

    Действительно, мы можем положить $E = \bigsqcup_{\alpha}\lr*{U_{\alpha} \times C^k}$, в котором мы отождествляем точки $(x, \lambda) \in U_{\beta} \times \C^k$ и $(x, \lambda g_{\alpha \beta}(x))$, а структура многообразия на $E$ определяется вложениями $U_{\alpha} \times \C^k \to E$. 

    Обычно операции над векторными пространствами переносятся и на векторные расслоения: 

    \begin{itemize}
        \item Если $E \to M$~--- векторное расслоение, то можно определить двойственное раслоение $E^* \to M$, взяв в качесвте слоёв $E^*_x \eqdef (E_x)^*$. Тривиализации $\varphi_u\colon E_U \to U \times \C^k$ (где $E_u = \pi^{-1}(U)$) индуцируют отображения 
        \[
            \varphi_U^*\colon E^*_U \to U \times (\C^k)^* \cong U \times \C^k,
        \]
        которые наделяют $E^*$ структурой многообразия. Эту конструкцию проще получить при помощи функций перехода: 
        $E^* \to M$ будет векторным расслоением с функциями перехода $j_{\alpha \beta}(x) = ^{i}g_{\alpha \beta}(x)^{-1}$.

        \item Пусть $E \to M$ и $F \to M$~--- комплексные векторные расслоения рангов $k$ и $\ell$ с функциями перехода $\{ g_{\alpha \beta}\}$ и $\{ h_{\alpha \beta} \}$. Тогда мы можем определить $E \oplus F$, как векторное расслоение, заданное функциями перехода 
        \[ j_{\alpha \beta} = \begin{pmatrix} g_{\alpha \beta}(x) & 0 \\ 0 & h_{\alpha \beta}(x) \end{pmatrix} \in \mathrm{GL}(\C^k \oplus \C^{\ell}). \]
        \item Также мы можем определить расслоение $E \otimes F$, как расслоение, заданное функциями перехода 
        \[
            j_{\alpha \beta}(x) = g_{\alpha \beta}(x) \otimes h_{\alpha \beta}(x) \in \mathrm{GL}\lr*{\C^k \otimes \C^{\ell}}.
        \]
        \item Аналогично, $\Lambda^{r}E$~--- векторное расслоение, заданное формулами 
        \[
            j_{\alpha \beta} = \Lambda^r(g_{\alpha \beta}(x)) \in \mathrm{GL}\lr*{\Lambda^r \C^k}.
        \]
        В частности, $\Lambda^k E$ будет линейным расслоением с функциями перехода 
        \[
            j_{\alpha \beta}(x) = \det{g_{\alpha \beta}(x)} \in \mathrm{GL}(1, \C) = \C^*.
        \]
    \end{itemize}

    Для векторных расслоений можно также определить подрасслоения и прообразы.\footnote{но делать этого мы пока что не будем.}

    \begin{definition} 
        Веторные расслоения $E \to M$ и $F \to M$ \emph{изоморфны}, если существует отображение $f\colon E \to F$ такое, что $f_{x}\colon E_{x} \to F_{x}$~--- изоморфизмы $\forall x \in M$.

        Векторное расслоение $E \to M$ называется \emph{тривиальным}, если оно изоморфно $M \times \C^k$.

        \emph{Сечением $\sigma$} векторного расслоения $E \xrightarrow{\pi} M$ над $U \subset M$ называется $C^{\infty}$ отображение 
        \[ \sigma\colon U \to E\colon \sigma(x) \in E_x \ \forall x \in U. \]

        \emph{Репером} для $E$ над $U \subset M$ называется набор $\sigma_1, \ldots, \sigma_k$ сечений $E$ над $U$ таких, что 
        $(\sigma_1(x), \ldots, \sigma_k(x))$ является базисом пространства $E_x \ \forall x \in U$.
    \end{definition}

    Репер для $E$ над $U$, по существу, то же самое, что тривиализация расслоения $E$ над $U$: при заданной тривиализации $\varphi_{U}\colon E_{U} \to U \times \C^k$, то сечения $\sigma_i(x) = \varphi_U^{-1}(x, e_i)$ образуют базис. И обратно, если задан репер $\sigma_1, \ldots, \sigma_k$, то можно определить тривиализацию $\varphi_U(\lambda) = (x, (\lambda_1, \ldots, \lambda_k))$ для $\lambda = \sum \lambda_i \sigma_i(x)$ в $E_x$. 

    Заметим, что при заданной тривиализации $\varphi_U$ расслоения $E$ над $U$ любое его сечение $\sigma$ можно единственным образом представить, как векторзначную $C^{\infty}$-функцию $f = (f_1, \ldots, f_k)$, раскладывая $\sigma(x)$ по базису: 
    \[
        \sigma(x) = \sum f_i(x) \sigma^{-1}_{U}(x, e_i).
    \]
    Если же $\varphi_{V}$~--- тривиализация расслоения $E$ над $V$ и $f' = (f_1', \ldots, f_k')$~--- соотвествуующие представления $\sigma\vert_{U \cap V}$, то 
    \[
        \sum f_i(x) \varphi_{U}^{-1}(x, e_i) = \sum f_i'(x) \varphi_{V}^{-1}(x, e_i),
    \]
    так что 
    \[ \sum f_i(x) e_i = \sum f_i'(x) \varphi_U \varphi_{V}^{-1}(x, e_i) \implies f = g_{UV} f'. \]

    Таким образом, при заданных тривиализациях 
    \[
        \{ \varphi_{\alpha}\colon E_{U_{\alpha}} \to U_{\alpha} \times \C^k \}
    \]
    сечения расслоения $E$ над $\bigcup U_{\alpha}$ в точности соотвествуют наборам 
    \[
        \{ f_{\alpha} = (f_{\alpha_1}, \ldots, f_{\alpha_k})\}_{\alpha}
    \]
    векторзначных $C^\infty$ функций, удовлетворяющих $f_\alpha = g_{\alpha \beta} f_{\beta}$.
    
    \begin{example}[Векторные расслоения]
        Рассмотрим некоторые базовые примеры векторных расслоений: 
        \begin{enumerate}
            \item \bf{Касательные и кокасательные расслоения:}\\
            \emph{Комплексным касательным расслоением} к комплексному многообразию $M$ мы будем называть 
            \[
                TM = \bigsqcup_{z \in M}T_{Z}M, \text{ где } 
            \]
            $T_{z}M$~--- комплексное касательное пространство к $M$ в точке $x$.     
            В расслоении $TM$ есть подрасслоения $T'M$ и $T''M$ определяющиеся \hyperlink{complex_tungent_space}{естественным образом}.
            
            \item \bf{Дифференциальные формы:}
            \begin{definition} 
                \emph{Дифференциальной формой} степени $k$ называется сечение расслоения $\Lambda^k (TM)^*$. Расслоение комплексных дифференциальных форма степени $k$ мы будет обозначать $\Omega^k_{\C}(M)$ или $\Omega^k_{\C, M}$. 

                Пусть $M$~--- вещественное многообразие. Тогда легко видеть, что если $f \in C^{k - 1}(M)$, то $\mathrm{d}f$~--- $C^{k - 1}$-гладкое сечение расслоения $\Omega^1_{\R}(M)$. Кроме того, нетрудно видеть, что если $x_1, \ldots, x_n$~--- локальные координаты в карте $U \subset M$, то $k$-формы $\mathrm{d}x_{I} = \mathrm{d}x_1 \vee \ldots \vee \mathrm{d}x_{i_k}$, $1 \le i_1 \le \ldots \le i_k \le n$ образуют базис слоя $\Omega^k_{\R}(X)$ в каждой точке отркытого множества $U$. В самом деле, локальные координаты $x_1, \ldots, x_n$ задают локальную тривиализацию каательного расслоения $TM$: соотвествующий локальный базис в слое задаётся в каждой точке дифференцированиями $\frac{\partial}{\partial x_i}\bigg\vert_{x}$. Тогда 1-формы $\mathrm{d} x_i$ образуют двойственный базис в расслоении $\Omega^1_{\R}(X)$.
            \end{definition}
        \end{enumerate}
    \end{example}