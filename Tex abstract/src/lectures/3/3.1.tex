
\subsection{Предварительные сведения и напоминания}

\begin{definition}
	Собственный идеал $I$ в кольце $R$ называется \emph{простым}, если $ab \in I \implies a \in I$ или $b \in I$.

	Собственный идеал $I$ в кольце $R$ называется \emph{максимальным}, если он не содержится ни в каком другом собственном идеале.
\end{definition}

\noindent\bf{Простейшие свойства:}

\begin{enumerate}
	\item Для любого собственного идеала существует максимальный идеал, содержащий его. 

	\item Любой максимальный идеал является простым. 

	\item Собственный идеал $I$ является простым тогда и только тогда, когда $R/I$~--- область целостности. 

	\item Собственный идеал $I$ является максимальным тогда и только тогда, когда $R/I$~--- поле. 
\end{enumerate}

\begin{definition} 
	Элементы $a$ и $b$ называются \emph{ассоциированными}, если $aR = bR$.

	Необратимый элемент $a \in R$ называется \emph{неприводимым}, если из равенства $a = bc$ следует, что или $b$ или $c$ ассоциирован с $a$.

	Элемент называется \emph{простым}, если главный идеал $(a)$  простой. 
\end{definition}

\begin{remark}
	Простой $\implies$ неприводимый. Обратное, вообще гооворя, неверно. 
\end{remark}

\begin{definition} 
	Кольцо $R$ называется \emph{нётеровым}, если оно удовлетворяет условию обрыва \bf{возрастающих} цепочек (ACC) для идеалов. 
	Модуль называется \emph{нётеровым}, если он удовлетворяет ACC для подмодулей. 
\end{definition}

\begin{lemma} 
	Следующие условия на кольцо $R$ эквивалентны:
	\begin{enumerate}
		\item $R$ нетерово. 

		\item Любой идеал в $R$ конечнопорожден. 

		\item Любой подмодуль коненчопрожденного $R$-модуля конечнопрожден. 

		\item Любой конечнопорожденный $R$-модуль нетеров. 
	\end{enumerate}
\end{lemma}

\begin{theorem}[Гильберта, о базисе]
	Кольцо многочленов от конечного числа переменных над нётеровым кольцом нётерово. 
	Иными словами, если $R$~--- нётерово кольцо, то любой идеал в кольцо $R[x_{1}, \ldots, x_{n}]$ порожден конечным числом многочленов. 
\end{theorem}


\subsection{Аффинные алгебраиические многообразия}

\bf{Я думаю, что как только я нормально послушаю курс алгебриаческой, этот параграф будет переписан. }

Пусть $F$~--- поле, $\AA^{n}_{F} = F^n$~--- аффинное пространство над ним. 

Пусть $J \subset A = F[t_{1}, \ldots, t_{n}]$, обозначим через $V(J)$ множество всех общих нулей всех многочленов из идеала $J$, 
то есть 

\[
	V(J) = \{ x \in \AA^{n}_{F} \ \vert \ f(x) = 0 \ \forall f \in J \}. 
\]

\begin{definition} 
	Пусть $I$~--- идеал в колцье $R$. \emph{Радикал} идеала $I$ определяется, как 
	\[
		\sqrt{I} \eqdef \{ f \in R \ \vert \ \exists n \in \N \colon f^n \in I \}.
	\]

	Идеал $I$ называется \emph{радикальным}, если он совпадает со своим радикалом. 
\end{definition}

\begin{remark}
	Другими словами, $I$~--- радикальный идеал $\Leftrightarrow R/I$~--- редуцированное кольцо (т.е. без нильпотентных элементов). 
\end{remark}

Несложно заметить, что $V(J) = V(AJ)$, где $AJ = \sum_{f \in J} Af$.  Действительно, если $f(x) = 0, g(x) = 0$, то
$\forall q, p \in F[t_1, \ldots, t_n]$ $fq + pg = 0 \Rightarrow V(J) = V(AJ)$. Соотвественно, так как $f^m(x) = 0 \implies f(x) = 0$, мы 
имеем $V(J) = V(\sqrt{AJ})$, а это говорит нам, что имеет смысл рассматривать только радикальные идеалы. 

\begin{definition}[Топология зарисского]
	Определим на $\AA^{n}_{F}$ \emph{топологию Зарисского}: набором замкнутых множеств будет
	 \[
	 \{ V(J) \subset \AA_{F}^{n} \ \vert \ J \text{~--- радикальный идеал в } F[t_{1}, \ldots, t_{n}]\}.
	 \]

	 Замкнутые подмножества $\AA_{F}^{n}$ в этой топологии называют \emph{аффинными алгебраическими многообразиями (affine algebraic variety)}.\footnote{вообще говоря, кажется, что это не вполне правильное определение, так как тут это просто алгебраическое множество, а вот аффинное многообразие~--- окольцованное пространство. Поговорим об этом позже. }
 \end{definition}

 \begin{remark}
 	Проверим, что это удовлетворяет аксиомам топологии: 

 	\begin{itemize}
		\item $V(1) = \varnothing$.
		\item $V(0) = \AA_{F}^{n}$.
		\item $V\lr*{\bigcup_{k} J_{k}} = \bigcap_{k} V(J_{k})$, то есть пересечение замкнутых замкнуто.
 	\end{itemize}
 \end{remark}

Для подмножества $X \subset \AA_{F}^{n}$ определим $I(X) = \{ f \in F[t_{1}, \ldots, t_{n}] \ \vert \ f(x) = 0 \ \forall x \in X \}$. Легко видеть, что $V(I(X)) = \Cl(X)$ в топологии Зарисского.  Совершенно ясно, что $I(X)$~--- идеал в кольце $F[t_{1}, \ldots, t_{n}]$.

\begin{definition} 
	\emph{Морфизмом} аффинных алгебраических многообразий $X \subset \AA^{n}_{f}, Y \subset \AA_{F}^{n}$ называется полиномиальное 
	отображение $X \to Y$.
\end{definition}

Аффинные многообразия с таким набором морфизмов образуют категорию $\Aff$.

\begin{definition} 
	Так как $\AA^{1}_{F} = F$, морфизмы $X \to \AA_{F}^{1}$~--- просто какие-то элементы $F[x_{1}, \ldots, x_{n}]$. Соотвественно, морфизмы $f$ и $g$ совпадают, если $f - g \in I(X)$, то есть $\Hom_{\Aff}(X, \AA_{F}^{1}) \cong F[t_{1}, \ldots, t_{n}]/I(X)$. Это кольцо называется \emph{аффинной алгерой} многообразия $X$ и обозначается $F[X]$.
\end{definition}

Так как $\Hom_{\Aff}(\_, \AA_{F}^{1})$ является контравариантным функтором, а кольцевые операции определяются на $\Hom_{\Aff}(X, \AA_{F}^{1} )$ естественным образом, отображение $X \mapsto F[X]$ определяет контравариантный функтор $\Aff \to F-\Alg_{fin.gen.}$~--- конечнопорожденные редуцированные алгебры. 

Построим функтор в обратную сторону. Рассмотрим $R \in F-\Alg_{fin.gen.}$ и выберем в ней набор образующих (то есть, выберем эпиморфизм $\pi_{R}\colon F[t_{1}, \ldots, t_{n}]$). Рассмотрим функтор $\cX = \Hom_{F-\Alg_{fin.gen.}}(_{-}, F)\colon F-\Alg_{fin.gen.} \to \Set$. 

Множество $\cX(A)$ мы можем отождествить с $\AA_{F^{n}}$ по формуле 
\[ \varphi \mapsto (\varphi(t_1), \ldots, \varphi(t_n)). \]

Таким образом, $\cX(R)$ вкладывается в $\AA_{F}^{n}$ при помощи отображения $\psi \mapsto \psi \circ \pi_{R}$. 
Кроме того, множество $\cX(R) = V(\Ker{\pi_{R}})$ является аффинным алгебраическим многообразием с аффинной алгеброй 
$F[t_{1}, \ldots, t_{n}]/I(V(\Ker{\pi_{R}}))$. Так мы имеем: 

\[
	\cX(F[X]) = \cX(A/I(X)) = V(I(X)) = X \quad F[X(R)] = A/I(V(\Ker{\pi_{R}})).
\]

Последняя алгебра изоморфна $R$ тогда и только тогда, когда $I(V(J)) = J$, где $R \cong A/J$.

\begin{theorem}[Теорема Гильберта о нулях]
	Пусть $F = F^{alg}$, $J \subset F[t_{1}, \ldots, t_{n}]$, а $f \in F[t_{1}, \ldots, t_{n}]$. Тогда 
	$f(V(J)) = 0 \Leftrightarrow f \in \sqrt{RJ}$. Иными словами, $f \in I(V(J)) \Leftrightarrow f \in \sqrt{RJ}$.
\end{theorem}

Другими словами, теорема Гильберта о нулях говорит нам, что над алгебраически замнутым полем $F$ аффинные алгебраические многообразия (замкнутые подмножества $\AA_{F}^{n}$) взаимно однозначно соотвествуют радикальным идеалам в $F[t_{1}, \ldots, t_{n}]$ и категории $\Aff$ и $F-\Alg_{fin.gen.}$ антиэквивалентны. 

Аналогичные рассуджения можно провести и для замкнутых подмножеств аффинного многообразия $X$ и радикальных идеалов его аффинной алгебры $F[X]$. При этом точкам аффинного многообразия $X$ соотвествуют максимальные идеалы $F[X]$, то есть, элементы $\Specm(F[X])$.


\subsection{Топология Зарисского на спектре кольца}

Пусть $R$~--- кольцо, $\Specm{R}$~--- его максимальный спектр (множество его максимальных идеалов). 
Зададим на $\Specm{R}$ набор замкнутых множеств 
\[
	\widetilde{V}(J) \eqdef \{ \fm \in \Specm{R} \ \vert \ \fm \supset J \}, \ J \subset R. 
\]


При таком определении топологии $X$ будет гомеоморфно $\Specm(F[X])$ (как мы и отмечали выше, точки соотвествуют максимальным идеалам).

В случае незамкнутого поля или бесконечнопорожденных алгебр правильно вместо максимального спектра рассматривать простой спектр. Топология Зарисского на нём определяется следующим образом; 
\[
	J \subset R, \quad V(J) \eqdef \{ \fp \in \Spec{R} \ \vert \ J \subset \fp \}.
\]

\subsection{Словарик алгебраической геометрии}



\begin{center}
	\begin{tabular}{ |c|c| }
		\hline
		Геометрия & Алгебра \\
		\hline
		Замкнутые подмножества $X$ & Идеалы в $F[X]$ \\
		Точки $X$ & Максимальные идеалы в $F[X]$ \\ 
		Неприводимые замкныте подмножества в $X$ & Простые идеалы в $F[X]$ \\ 
		will be upd & will be upd.\\
		\hline
	\end{tabular}
\end{center}

	




























