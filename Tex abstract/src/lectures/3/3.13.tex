	
	\subsection{Дедекиндовы кольца}

	\begin{statement}\label{Dedekind_domain} 
		Пусть $R$~--- нётерова одномерная область целостности. Тогда следующие условия эквивалентны: 
		\begin{enumerate}
			\item $R$ целозамкнуто. 
			\item Любой примарный идеал имеет вид $\fm^k$ для некоторого $\fm \in \Specm{R}$.
			\item $\forall \fm \in \Specm{R}$ кольцо $R_{\fm}$~--- кольцо дискретного нормирования. 
		\end{enumerate}
	\end{statement}

	\begin{proof}
		$(1) \Leftrightarrow (3)$ просто в силу того, что целозамкнутость~--- локальное свойство и теоремы~\ref{DVR_equiv}. Ну и, в силу того, что $R_{\fm}$~--- нётеровы одномерные локальные. 

		$(3) \implies (2):$ В таком кольце любой ненулевой примарный идеал $I$ является $\fm$-примарным, а такие однозначно соотвествуют примарным идеалам локализации $R_{\fm}$. Так как $R_{\fm}$~--- $\mathrm{DVR}$, там все примарные идеалы имеют вид $\fm^n R_{\fm}$ (так как $R_{\fm}$~--- локальное кольцо с единственным максимальным идеалом $\fm R_{\fm}$), а $\lambda_*$~--- биекция на множестве примарных идеалов, не пересекающихся с мультипликативным подмножеством (которое тут $R \setminus \fm$, да), мы имеем $\lambda_*(I) = \lambda_*(\fm^n) \implies I = \fm^n$. 

		$(2) \implies (3):$ Любой идеал в $R_{\fm}$ имеет примарное разложение $\implies$ является примарным. $\lambda^{*}(J)$~--- $\fm$-примарный  $\implies \lambda^*(J) = \fm^n \implies \lambda_{*}(\lambda^*(J)) = \lambda_*(\fm^n) = (\fm R_{\fm})^n$, откуда по теореме~\ref{DVR_equiv} $R_{\fm}$~--- кольцо дискретного нормирования.  
	\end{proof}

	\begin{definition} 
		Кольца, удовлетворяющие условию~\ref{Dedekind_domain} называют \emph{дедекиндовыми}. 
	\end{definition}

	\begin{theorem} 
		Пусть $Z$~--- Дедекиндово кольцо, $Q$~--- его поле частных, $F/Q$~--- конечное расщирение (полей), а $R = \Int_{F}{Z}$. Тогда $R$~--- дедекиндово. 
	\end{theorem}

	\begin{proof}
		Так как $R$~--- целое замыкание, $\dim{R} = 1$. Так как $F$~--- конечное расширение, $\forall \alpha \in F$ является корнем многочлена 
		\[
			\alpha^n + \frac{a_{n - 1}}{b_{n - 1}}\alpha^{n - 1} + \ldots + \frac{a_0}{b_0} = 0, \quad a_i, b_i \in \Z \implies b \alpha^n + c_{n - 1} \alpha^{n - 1} + \ldots + c_0 = 0 \implies (b\alpha)^n + d_{n - 1} (b\alpha)^{n - 1} + \ldots + d_0 = 0
		\]
		Значит, $b\alpha \in R$, откуда $\alpha \in \lr*{Z \setminus 0}^{-1}R$.

		Так как $F$~--- поле частных $R$, $R$ целозамкнуто. Для дедекиндовости нам не хватает Нётеровости. 

		Рассмотрим $F$, как векторное пространство над $Q$. Рассмотрим оператор 
		\[
			m_{\alpha} \in \End_{Q-\fm \fo \fd}\lr*{F}, \ m_{\alpha}(x) = \alpha x.
		\]

		\emph{Далее доказательство приводится только для случая сепарабельного расширения}. 
		Так вот, если расширение сепарабельно, $\exists \alpha\in F \colon \Tr{m_{\alpha}} \neq 0$. Рассмотрим невырожденную билинейную форму $B(x, y) = \Tr{m_{xy}}\colon F \times F \to Q$.

		Возьмём базис $u_1, \ldots, u_n$~--- базис $F$ над $Q$ (можно полагать, что $u_i \in R$) и $v_1, \ldots, v_n$~--- двойственный базис относительно $B$. Возьмём $x \in F$, тогда 
		\[
			x = \sum_{k = 1}^{n} B(x, u_k) v_k.
		\]
		$x \in R, \ u_k \in R$, тогда $x u_k \in R$, а значит, его минимальный многочлен над $Q$ имеет коэффициенты из $Z$ (\textcolor{red}{была такая теорема, надо найти и вставить ссылку}). В то же время ясно, что минимальный многочлен $x u_k$ равен минимальному многочлену эндоморфизма $m_{x u_k}$. Собственные числа $m_{x u_k}$~--- это корни минимального многочлена, а они являются целыми над $Z$, следовательно и их сумма (с учетом кратности)~--- целая над $Z$, а это в точности след. Значит, $R$~--- подмодуль конечнопорожденного $Z$-модуля, а значит, так как $Z$~--- дедекиндово, $R$ конечнопорождено, как $Z$-модуль $\implies R$~--- нётерово.  
	\end{proof}

	В случае $Z = \Z$, кольцо $R$ называется дедекиндовым кольцом \emph{арифметического типа} или \emph{кольцом целых числового поля}. В случае $Z = K[t]$ кольцо $R$ называется дедекиндовым кольцом \emph{функционального} типа. 

	\subsection{Hauptidealsatz}

	\begin{definition} 
		Пусть $I$~--- идеал. Тогда  его \emph{высота} $h(I)$~--- длина наибольшей цепочки вложенных в него простых идеалов. 
	\end{definition}

	\begin{theorem}[Крулль, о высоте]
	 	Пусть $x \in R$~--- нётерово коммутативное кольцо с единицей, $\fp$~--- минимальный простой идеал, содержащий $(x) = xR$. Тогда $h(\fp) \le 1$. 	 
	 \end{theorem}  
	 \begin{proof}
	 	Во-первых, условие теоремы располагает к замене $R$ на $R_{\fp}$, т.е. далее будем считать, что $R$~--- локально с единственным максимальным идеалом $\fp$. Так что $\fp$~--- единственный минимальный простой, содержащий $xR$, а $xR$~--- $\fp$-примарным, откуда $\dim\lr*{R/xR} = 0$. Значит, $R/xR$~--- нульмерное нётерово, то есть Артиново. Действительтно, $\sqrt{xR} = \fp \implies \exists n \in \N \colon \fp^n = xR$, откуда $(\fp/xR)^n = 0$. Значит, если $p' \in \Spec{R/xR}$, то 
	 	\[
	 		\lr*{\fp/xR}^n \subset \fp' \subset \fp/xR \implies \fp/xR \subset \fp' \subset \fp/xR.
	 	\]

	 	\begin{definition} 
	 		Пусть $\fq \in \Spec{R}, \ \lambda = \lambda_{\fq}\colon R \to R_{\fq}$. Тогда \emph{символическая степень} идеала $\fq$ используется, как 
	 		\[
	 			\fq^{(n)} \eqdef \lambda^*\lr*{\lambda_*(\fq^n)}.
	 		\]
	 	\end{definition}

	 	\begin{lemma} 
	 		Идеал $\fq^{(n)}$~--- примарный. 
	 	\end{lemma}

	 	\begin{proof}
	 		$\lambda_{*}(\fq) \in \Spec{R_{\fq}} \implies \lambda_{*}(\fq^n) = \lambda_{*}(\fq)^n$~--- примарный, а $\lambda^*$ отображает примарные в примарные.  
	 	\end{proof}

	 	Пусть $\overline{\cdot}\colon R \to R/xR$~--- канонический гомоморфизм. Рассмотрим в $R/xR$ такую убывающую цепочку идеалов: 
	 	\[
	 		\overline{q} \supset \overline{\fq^{(2)}} \supset \ldots \supset \overline{\fq}^{(n)} = \overline{\fq}^{n + 1},
	 	\]
	 	так как $R/xR$~--- артиново. Возьмём $q^{(n)} \ni z = y + xr, \ y \in \fq^{(n + 1)}, r \in R$. Тогда $xr \fq^{(n)}$ и $x \notin \fq = \sqrt{\fq^{(n)}}$. Тогда отсюда следует, что $r \in q^{(n)}$


		Теперь $\fq^{(n)} = \fq^{(n + 1)} + x \cdot \fq^{(n)}$. Тогда в $R/\fq^{(n + 1)}$ мы имеем $\widetilde{x} \widetilde{q^{(n)}} = \widetilde{\fq^{(n)}}, \ \widetilde{x} \in \Rad{R/\fq^{(n + 1)}}$ и тогда по лемме Накаямы $\fq^{(n)} = 0$. 

		Значит, $\fq^{(n)} = \fq^{(n + 1)} \implies \lambda^{*}\lr*{\lambda_{*}(\fq^n)} = \lambda^*(\lambda_*(\fq^{n + 1})) \implies \lambda_*(\fq)^n = \lambda_*(\fq)^n \cdot \lambda_*(\fq)$, а $\lambda_*(\fq) \subset \Rad(R_{\fq})$ и тогда опять же по лемме Накаямы мы имеем $\lambda^*(\fq^n) = 0$. 

		Значит, $\Spec{R_{\fq}} = \{ \fq \} \implies $ в $R$ нет простых, содержащихся в $\fq \implies h(\fq) = 0 \implies h(\fp) \le 1$.  
	 \end{proof}