	
	\subsection{Лемма Накаямы}

	Пусть $I \subset R$~--- идеал, $M$~--- конечнопорожденный $R$-модуль. 

	Из базового курса алгебры мы знаем такой факт: 

	\begin{theorem}[Гамильтона-Кэли]\label{standard_GK}
		Пусть $A \in \mathrm{M}_{n}(R)$, где $R$~--- коммутативное кольцо. Тогда $\chi_{A}(A) = 0$.
	\end{theorem}

	Докажем теперь некоторое его обобщение. 

	\begin{theorem}[Гамильтона-Кэли]
		Пусть $\varphi \in \End(M)$ такой, что $\Im{\varphi} \subset IM$. Тогда существует многочлен $p(t) = t^n + \alpha_{n - 1}t^{n - 1} + \ldots + \alpha_0$ такой что: 
		\begin{itemize}
			\item $\alpha_i \in I^{n - i}$.
			\item $p(\varphi) = 0$.
		\end{itemize}
	\end{theorem}

	\begin{proof}
		Пусть у модуля $M$ есть $n$ образующих, тогда есть сюръективное отображение $R^n \twoheadrightarrow M$ (а значит и $IR^n \twoheadrightarrow IM$) и вообще есть следующая коммутативная диаграмма: 
		\begin{center}
		\includegraphics{lectures/3/commutative diagramms/GK.pdf}
 		\end{center}

 		Верхняя стрелка $\psi$ есть из универсального свойства свободного модуля. Так как каждый базисный элемент переходит в элемент с коэффициентами из $I$, $\psi \in M_{n}(I)$. Положим $p = \chi_{\psi}$. Тогда, так как $f$~--- сюръективно, $\forall m \in M \ \exists x\colon f(x) = m$. Тогда:
 		\[
 			p(\varphi)(m) = p(\varphi)(f(x)) = p(\psi)(g(x)) = 0 \implies p(\varphi) = 0.
 		\]
	\end{proof}

	\begin{theorem}[Лемма Накаямы] 
		Пусть $M = IM$. Тогда $\exists a \in M\colon \forall m \in I \  am = m$
	\end{theorem}
	\begin{proof}
		$\id_{M}(M) = IM \implies $, а значит, по теореме Гамильтона-Кэли $\exists p(t) = t^n + \alpha_{n - 1}t^{n - 1} + \ldots + \alpha_0, \ \alpha_i \in I \colon p(\id_{M}) = 0$. Тогда 
		\[
			\id_{M}(1 + \alpha_{n - 1} + \ldots + \alpha_{0}) = 0 \implies \id_{M}(-(\alpha_{n - 1} + \ldots  + \alpha_0)) = 1.
		\]
		Тогда $a = -(\alpha_{n - 1} + \ldots  + \alpha_0)$ подходит. В самом деле, 
		\[
			am = \id_{M}(m) = m \quad \forall m \in M.
		\]
	\end{proof}

