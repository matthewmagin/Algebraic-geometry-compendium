	
	\section{Гауссовы суммы}

	\subsection{Общие сведения}

	\begin{definition} 
		Пусть $p$~--- простое число. \emph{Мультипликативными характерами} группы $\F_p$ мы будем называть гомоморфизмы $\F_{p} \to \C^*$. 
	\end{definition}

	Если мы выбрали некоторый первообразный корень $\omega$, порождающий $\F_p^*$, то характер задаётся тем, в какой первообразынй корень из 1 переходит $\omega$. 

	Через $\chi_0$ обозначим единичный характер, то есть такой, что $\chi(x) = 1 \ \forall x$. Пусть теперь $\chi$~-- произвольный характер,  $a \in \Z, a \notdivby p$, тогда определён $\chi(\overline{a})$. Доопределим его в нуле следующим образом: 
	\[
		\chi(\overline{0}) = \begin{cases} 0, & \chi \neq \chi_0 \\ 1, \chi = \chi_0 \end{cases}. 
	\]
	Теперь все наши характеры заданы на $\Z$ как $\chi(a) \eqdef \chi(\overline{a})$.

	Отметим для начала несколько самых простых свойств: 

	\begin{itemize}
		\item $\chi(1) = 1$,
		\item $\forall a \ \chi(a)$~--- корень $(p - 1)$-й степени из единицы, 
		\item $\chi(a^{-1}) = \overline{\chi(a)}$.
	\end{itemize}


	\begin{definition} 
		Пусть $\chi$~--- мультипликативный характер группы $\F_p$. \emph{Суммой Гаусса} соотвестувеющей характеру $\chi$ мы будем называть выражение вида 
		\[
			S(\chi, a) = \sum_{x = 0}^{p - 1} \chi(x) \zeta_p^{ax}. 
		\]
	\end{definition}

	\begin{remark}
		Видно, ччто $S(\chi, a)$ зависит лишь от $\chi$ и класса $\overline{a} \in \F_p$. 
	\end{remark}

	Сделаем сначала несколько наблюдений. 

	\begin{statement} 
		Для сумм Гаусса выполняется следующее соотношение 
		\[
		S(\chi, a) \cdot \chi(a) = \begin{cases} S(\chi, 1), & a \notdivby p, \ \chi \neq \chi_0 \\ 0, & a \notdivby p, \ \chi = \chi_0 \text{ или } a \divby p, \ \chi \neq \chi_0 \\ p, & a \divby p, \ \chi = \chi_0 \end{cases}.
	\]
	\end{statement}
	
	\begin{proof}
		Действительно, если $\chi \neq \chi_0$ и $a \notdivby p$, то
	\[
		\chi(a)S(\chi, a) = \sum_{x = 0}^{p - 1}\chi(ax) \zeta^{ax} = \sum_{t = 0}^{p - 1} \chi(t) \zeta^{t} = S(\chi, t).
	\]
	Если же $a \notdivby p, \ \chi = \chi_0$, то 
	\[
		S(\chi_0, a) = \sum_{x = 0}^{p - 1} \zeta^{ax} =  0.
	\]
	И, если же $a \divby p$, то мы имеем 
	\[
		S(\chi, 0) = \sum_{x = 0}^{p - 1} \chi(x) = 0,
	\]
	так как если мы обозначим эту сумму за $T$ и выберем такое $B \in \F_{p}^*$, что $\chi(b) \neq 1$, мы получим 
	\[
		\chi(b) \cdot T = \chi(b) \sum_{x = 0}^{p - 1} \chi(x) = \sum_{x = 0}^{p - 1} \chi(b x) = \sum_{t = 0}^{p - 1} \chi(t) = T \implies T = 0.
	\]
	\end{proof}
	
	Теперь вычислим модуль Гауссовой суммы. Во-первых, сразу очевидно, что 
	\[
		S(\chi_0, 1) \cdot \overline{S(\chi_0, 1)} = 0. 
	\]
	Вычислим $\sum_{a = 0}^{p - 1} S(\chi, a) \overline{S(\chi, a)}$ двумя способами. Во-первых если $a \neq 0$, то заметим, что по предложению выше 
	\[
		S(\chi, a) \cdot \overline{S(\chi, a)} = \chi(a^{-1}) \cdot \chi(a) \cdot S(\chi, 1) \overline{S(\chi, 1)} = |S(\chi, 1)|^2.
	\]
	Тогда мы получаем, что 
	\[
		\sum_{a = 0}^{p - 1} S(\chi, a) \overline{S(\chi, a)} = (p - 1)|S(\chi, 1)|^2,
	\]
	так как $S(\chi, 0) = 0$. С другой стороны, 
	\[
		S(\chi, a) \cdot \overline{S(\chi, a)} = \sum_{x} \sum_{y} \chi(x) \overline{\chi(y)} \zeta^{ax - ay}.
	\]

	Теперь вспомним, что 
	\[
		\sum_{t = 0}^{p - 1} \zeta^{at} = \begin{cases} p, & a \equiv 0 \pmod{p} \\ 0, & \text{ иначе. } \end{cases} \implies \frac{1}{p} \sum_{t = 0}^{p - 1} \zeta^{t(x - y)} = \delta_{x y}.
	\]
	Отсюда получаем, что 
	\[
		\sum_{a = 0}^{p - 1} S(\chi, a) \cdot \overline{S(\chi, a)} = \sum_{x = 0}^{p - 1} \sum_{y = 0}^{p - 1} \chi(x) \overline{\chi(y)} \zeta^{ax - ay}  \delta_{x y} p = (p - 1)p.
	\]
	Отсюда мы получили, что $(p - 1)|S(\chi, 1)|^2 = (p - 1)p$, откуда 
	\[
		|S(\chi, 1)| = \sqrt{p} \implies |S(\chi, a)| = \sqrt{p}.
	\]

	\begin{example}
		Рассмотрим теперь частный случай $p \neq 2$ и характеров 
		\[
			\chi(a) = \lr*{\frac{x}{a}}. 
		\]
		Тогда мы получаем, что 
		\[
			p = |S(\chi, 1)|^2 = \lr*{\sum_{x = 1}^{p - 1} \lr*{\frac{x}{p}}\zeta^x } \cdot \lr*{\sum_{y = 1}^{p - 1} \lr*{\frac{-y}{p}}\zeta^{-y} } = \lr*{\frac{-1}{p}} |S(\chi, 1)|^2.
		\]
		Соответственно, если $\lr*{\frac{-1}{p}} = 1$, то $\sqrt{p} \in \Q\lr*{\zeta_p}$. Если же $\lr*{\frac{-1}{p}} = -1$, то $\sqrt{-p} \in \Q(\zeta_p) \implies \sqrt{p} \in \Q(\zeta_{4p})$. 

		Если же $p = 2$, то $2\sqrt{-1} = (1 + \sqrt{-1})^2 \implies \sqrt{2} \in \Q(\zeta_{8p})$.

		Теперь рассмотрим произвольное натуральное $n$, разложим его в произведение простых: 
		\[
			n = p_1^{k_1} \cdot \ldots p_{s}^{k_s}, 
		\]
		а отсюда $\Q(\sqrt{n}) \subset \Q(\sqrt{p_1}, \ldots, \sqrt{p_s})$, а это расширение лежит в каком-то груговом расширении (по выкладке выше). 

		С другой же стороны, если $\Q(\sqrt{p}) \subset \Q(\zeta_n)$, то $n \divby p$ (так как простое число $p$ будет разветвлено в $\Q(\sqrt{p})$, значит будет разветвлено и в $\Q(\zeta_n)$, но, как мы видели в первой части курса, такого не происходит при $n \notdivby p$). 
	\end{example}

	\subsection{Количество решений уравнений над конечным полем}

	Пусть $F(x_1, \ldots, x_n) \in \Z[x_1, \ldots, x_n]$, $\zeta$~--- певообразные корень степени $p$ из единицы. Рассмотрим сумму 
	\[
		S = \sum_{(x_1, \ldots, x_n) \in \F_p^n} \sum_{x \in \F_p} \zeta^{x F(x_1, \ldots, x_n)}.
	\]
	Заметим, что 
	\[
		\sum_{x \in \F_p} \zeta^{x F(x_1, \ldots, x_n)} = \begin{cases} 0, & \text{ если } F(x_1, \ldots, x_n) \not\equiv 0 \pmod{p} \\ p, & \text{ если } F(x_1, \ldots, x_n) \not\equiv 0 \pmod{p} \end{cases}. 
	\]
	Соотвественно, отсюда получаем, что 

	\[
		S = \sum_{(x_1, \ldots, x_n) \in \F_p^n} \sum_{x \in \F_p} \zeta^{x F(x_1, \ldots, x_n)} = p \cdot N,
	\]
	где $N$~--- это количество решений уравнения $F(x_1, \ldots, x_n) = 0$ в $\F_{p}$.

	Рассмотрим теперь случай, когда $F$ имеет диагональный вид: 
	\[
		F(x_1, \ldots, x_n) = a_1 x_1^{r_1} + \ldots + a_n x_n^{r_n}, \quad a_i \in \Z, \ a_i \not\equiv 0 \pmod{p}. 
	\]
	Посчитаем сумму $S$ вторым способом: 
	\[
		S = \sum_{(x, x_1, \ldots, x_n) \in \F_p^{n + 1}} \zeta^{x\lr*{a_1 x_1^{r_1} + \ldots + a_n x_n^{r_n}}} = p^n + \sum_{x \neq 0} \lr*{ \sum_{x_1 = 0}^{p - 1} \zeta^{x a_1 x_1^{r_1}} }\lr*{ \sum_{x_2 = 0}^{p - 1} \zeta^{x a_2 x_2^{r_2}} } \ldots \lr*{ \sum_{x_n = 0}^{p - 1} \zeta^{x a_n x_n^{r_n}} }
	\]

	Посмотрим на отдельные сомножители: 
	\[
		\sum_{y = 0}^{p - 1} \zeta^{y^{r}} = \sum_{z = 0}^{p - 1} m(z) \zeta^{z},
	\]
	где $m(z)$~--- число решений уравнения $y^r = z$ над полем $\F_p$ или же число решений сравнения 
	\[
		y^r = z \pmod{p}
	\]
	относительно $y$. Ясно, что $m(0) = 1$. Найдём $m(z)$ явно при $z \not\equiv 0 \pmod{p}$. Выберем первообразный корень $g$ по модулю $p$, то есть $\F_{p} = \langle g \rangle$ и пусть 
	\[
		x = g^k \pmod{p},
	\]
	где показатель $k$ определён одозначно (по модулю $p - 1$, естественно). Пусть также $y = g^{u} \pmod{p}$, тогда 
	\[
		y^r = z \pmod{p} \Leftrightarrow ru \equiv k \pmod{p - 1}.
	\]

	Это сравнение имеет $d = (r, p - 1)$ решений по $u$ или не имеет ни одного решения, в зависимости от того, делится $k$ на $d$ или нет: 
	\[
		m(z) = \begin{cases} d, & k \equiv 0 \pmod{d} \\ 0, & k \not\equiv 0 \pmod{d} \end{cases}.
	\]

	Но, в аналитическом смысле такая формула не слишком удобна. Получим другую. Выберем некоторый первообразный корень степени $d$ из единицы и обозначим его за $\varepsilon$. Рассмотрим характеры 
	\[
		\chi_{s}(z) = e^{k s}, \quad s = 0, \ldots, d - 1
	\]
	где число $k$ для $z$ определяется как выше. Или, эквивалентно можно говорить, что мы задали отображение. 
	\[
		\chi_s \colon \F_p^* = \langle g \rangle \to \mu_{d}, \quad g \mapsto \varepsilon^s.
	\]
	Тогда видим, что если $k \equiv 0 \pmod d$, то каким бы ни было $s = 0, \ldots, s - 1$, мы имеем $\varepsilon^{k s} = 1$ и 
	\[
		\sum_{s = 0}^{d - 1} \chi_s(z) = d.
	\]
	Если же $k \not\equiv 0 \pmod{d}$, то так как $\varepsilon$~--- первообразный корень, $\varepsilon^{k} \not \equiv 1$, а значит, 
	\[
		\sum_{s = 0}^{d - 1} \chi_s(z) = \sum_{s = 0}^{d - 1} \varepsilon^{ks} = \frac{\varepsilon^{kd} - 1}{\varepsilon^k - 1} = 0.
	\]
	Итак, мы получили, что 
	\[
		m(z) = \sum_{s = 0}^{d - 1} \chi_s(z).
	\]

	Тогда мы получили, что 
	\[
		\sum_{y = 0}^{p - 1} \zeta^{a y^{r}} = \sum_{z = 0}^{p - 1} m(z) \zeta^{a z} = 1 + \sum_{z = 0}^{p - 1} \sum_{s = 0}^{d - 1} \chi_s(z) = \sum_{s = 0}^{d - 1} S(\chi_s, a) = \sum_{s = 1}^{d - 1} S(\chi_s, a). 
	\]

	Тогда мы можем оценить модуль этой суммы: 
	\[
		\left\lvert \sum_{y = 0}^{p - 1} \zeta^{a y^{r}} \right\rvert \le  (d - 1) |S(\chi_s, a)| = (d - 1)\sqrt{p}.
	\]
	Применим это к замечательной формуле 
	\[
		S = \sum_{(x, x_1, \ldots, x_n) \in \F_p^{n + 1}} \zeta^{x\lr*{a_1 x_1^{r_1} + \ldots + a_n x_n^{r_n}}} = p^n + \sum_{x \neq 0} \lr*{ \sum_{x_1 = 0}^{p - 1} \zeta^{x a_1 x_1^{r_1}} }\lr*{ \sum_{x_2 = 0}^{p - 1} \zeta^{x a_2 x_2^{r_2}} } \ldots \lr*{ \sum_{x_n = 0}^{p - 1} \zeta^{x a_n x_n^{r_n}} }	
	\]
	\[
		|S - p^n| \le (p - 1) \cdot p^{\frac{n}{2}} \cdot (d_1 - 1) \ldots (d_n - 1),
	\]
	где $d_i = (r_i, p_i - 1)$. 

	Значит, мы получили такое неравенство для количества решений: 
	\[
		|N - p^{n - 1}| \le (p - 1) \cdot p^{\frac{n}{2} - 1} \cdot (d_1 - 1) \ldots (d_n - 1).
	\]

	

 




	
	







	

