    \section{Локальные поля. Введение. }
    
	\subsection{Кольца дискретного нормирования}

	\begin{definition} 
		Пусть $F$~--- поле, $\v\colon F^{*} \to \Z$~--- эпиморфизм групп, то есть 
		\begin{itemize}
			\item $\v(xy) = \v(x) + \v(y)$.
			\item $\Im{\v} = \Z$ или, что то же самое, $\Im{\v} \ni 1$.
		\end{itemize}
		и, кроме того, $\v(x + y) \ge \min(\v(x), \v(y))$.

		Тогда $\v$ называют \emph{дискретным нормированием} на поле $F$. 
	\end{definition}

	\begin{remark}
		Дискретное нормирование обычно доопределяют на $0$, полагая, что $\v(0) = +\infty$. 
	\end{remark}

	\begin{example}
		Мы уже видели, что для конечного расширения $\Q$ можно определять нормирование следующим образом: пусть $\fp \in \Specm(\cO_{K})$, тогда мы можем определить нормирование на $K$ так: 
		\[
			a \in K \quad (a) = \fp^{k} \cdot \fq_{1}^{k_1} \cdot \ldots \cdot \fq_{m}^{k_m}, \ \fq_i \neq \fp \rightsquigarrow \v_{\fp}(a) \eqdef k.
		\]

		Нетрудно проверить, что это в самом деле нормирование. 

		Действительно, $\exists x \in \fp \setminus \fp^2$, а значит, $\v_{\fp}(x) = 1 \implies \Im{\v} = \Z$. Кроме того, если $(x) = \fp^m \cdot I, \ (y) = \fp^n \cdot J$ (не умаляя общности, $x, y \in \cO_{K}, \ m \le n$), то $x \in \fp^m, \ y \in \fp^n \implies x + y \in \fp^{m}$, откуда $\v_{\fp}(x + y) \ge m$. 
	\end{example}

	\begin{example}
		Пусть $F$~--- поле. Введём нормирование на $F(t)$, которое будет тривиальным на $F$. Пусть $p$~--- неприводимый унитарный многочлен. Тогда $\forall h \in F(t) \ \exists f, g \in F[t]\colon$

		\[
			h(t) = p^n(t) \cdot \frac{f(t)}{g(t)}, \quad (f, p) = (g, p) = (1).
		\]
		Тогда мы можем положить $\v_{p}(h) = n$.
	\end{example}

	\begin{definition} 
		Пусть на поле $K$ задано дискретное нормирование $\v$. Определим по нему \emph{кольцо дискретного нормирования}
		\[
			\cO_{\v} \eqdef \{ x \in K \ \vert \ \v(x) \ge 0 \}.
		\]
	\end{definition}

	Сразу можно заметить, что 
	\[
		\cO_{\v}^{*} = \{ x \in \cO_{\v} \ \vert \ \v(x) = 0 \}.
	\]
	Действительно, если $x$ обратим, то $\v(x) + \v(x^{-1}) = \v(1) = 0$, откуда $\v(x) = 0$. И, если же $\v(x) = 0$, то $v(x^{-1}) = -\v(x) = 0$, откуда $x^{-1} \in \cO_{\v}$. 

	Кольцо $\cO_{\v}$ является локальным с единственным максимальным идеалом 
	\[
		\fm_{\v} = \{ x \in \cO_{v} \ \vert \v(x) \ge 1 \}.
	\]

	Совершенно ясно, что это идеал. Более того, как мы видим, все элементы $\cO_{\v}$, не лежащие в $\fm_{\v}$, будут обратимы, откуда ясно, что это единственный максимальный идеал. 

	Кроме того, этот идеал является главным. Действительно, возьмём элемент $\pi \in \cO_{\v}$, такой, что $\v(\pi) = 1$, тогда $\fm_{\v} = (\pi)$. 

	В самом деле, включение $(\pi) \subset \fm_{\v}$ очевидно, докажем обратное. Возьмём $x \in \fm_{\v}, \ \v(x) = n > 0$.  Тогда 
	\[
		\v\lr*{\frac{x}{\pi^n}} = 0 \implies \frac{x}{\pi^n} \in \cO_{\v}^{*} \implies x \in \pi^n\cO_{v}.
	\]

	\begin{statement} 
		Кольцо дискретного нормирования $\cO_{\v}$ является локальным кольцом с единственным максимальным идеалом 
		\[
			\fm_{\v} = \{ x \in \cO_{v} \ \vert \v(x) \ge 1 \}.
		\]	
		Мультипликативная группа этого кольца имеет вид 
		\[
			\cO_{\v}^{*} = \{ x \in \cO_{\v} \ \vert \ \v(x) = 0 \}.
		\]	

	\end{statement}

	\begin{exercise}
		Любой идеал кольца $\cO_{\v}$ является степенью идеала $\fm_{\v}$.
	\end{exercise}

	Также легко видеть, что $\cO_{\v}$ целозамкнуто и $\dim{\cO_{\v}} = 1$, так как 
	\[
	  	\Spec{\cO_{\v}} = \{ (0), \fm_{\v} \}.
	  \]  

	Кроме того, очевидно, что кольцо $\cO_{\v}$ нётерово (более того, оно является областью главных идеалов). 


	
	Есть и обратное утверждение: 
	\begin{statement}\label{ant_2_prop_1} 
		Если кольцо $A$~--- нётерова локальная область целостности, в которой максимальный идеал главный, и $A$ не является полем, то $A$~--- кольцо дискретного нормирования\footnote{Вообще говоря, есть очень много характеризаций колец дискретного нормирования. Например, такое: нётеровы целозамкнутые локальные кольца размерности 1. }. 	
	\end{statement}
	\begin{proof}
		Покажем сначала, что любой идеал в кольце $A$ является степенью максимального идеала $\fm = (\pi)$. Действительно, рассмотрим идеал $I \subset \fm \subset A$. Выберем такую степень $k$, что $I \subset \fm^{k}$, но $I \not\subset \fm^{k + 1}$. Такая степень обязательно найдётся, так как 
	\[
		\bigcap_{k \in \N} \fm^k = 0,
	\]
	так как если $x$ лежит в этом пересечении, то $x = \pi x_1 = \pi^2 x_2 = \pi^3 x_3$, откуда мы мы получим возрастающую цепочку $(x) \subset (x_1) \subset (x_2) \subset \ldots$, которая обязана стабилизироваться в силу нётеровости. Тогда  $(x_n) = (x_{n + 1})$, откуда $ x_n = \pi x_{n + 1} = \pi a x_{n}$, откуда $x_n (1 - \pi a) = 0$, что противоречит целостности кольца $A$. Докажем, что тогда $I = (\pi^k)$. Совершенно ясно, что $(\pi^k) \subset I$,  а обратное включение также верно, так как $J = I/\pi^k$ содержит элементы, которые не делятся на $\pi$ (а так как все такие элементы обратимы), $J = A$, откуда $I = (\pi^k)$.

	Тогда, если $K$~--- поле частных кольца $A$, то ясно, как определить на нём нормирование: 
	\[
		x \in K^{*}, \ x = \frac{y}{z}, \ y, z \in A \rightsquigarrow (x) = (\pi^k), \ (y) = (\pi^m) \rightsquigarrow \v(x) = k - m.
	\]
	И, совершенно очевидно, что $\cO_{\v} = \{ x \in K \ \vert \ \v(x) \ge 0 \}$. 
	\end{proof}

	Часто встречается ситуация, когда нормирований на поле несколько. Например, на поле $\Q$ есть бесконечно много $p$-адических нормирований $\v_{p}$, связанных с простыми числами. Интересно узнать, насколько эти нормирования независимы. 

	\begin{lemma}\label{ant_2_lemma_1} 
		Пусть $\v_1, \v_2$~--- два нормирования на поле $K$, причем $\cO_{\v_1} \subset \cO_{\v_2}$. Тогда $\v_1 = \v_2$.
	\end{lemma}

	\begin{proof}
		\bf{1)} Докажем, что $\fm_{\v_2} \subset \fm_{\v_{1}}$. Возьмём $0 \neq x \in \fm_{\v_{2}}$, предположим, что $x \notin \fm_{\v_{1}}$. Тогда $\v_{1}(x) \le 0$, откуда $\v_{1}(x^{-1})\ge 0 \implies x^{-1} \in \cO_{\v_{1}} \subset \cO_{v_{2}}$, но это противоречит тому, что $x \in \fm_{\v_{2}}$.

		\bf{2)} Докажем, что $\fm_{\v_{1}} = \fm_{\v_2}$. Пусть $\fm_{\v_{1}} = (\pi_1), x \in \fm_{\v_{1}}$ тогда $x = \pi_1^n \cdot u$, где $u \in \cO_{\v_{1}}^{*}$. 

		Докажем, что $\pi_1 \in \fm_{\v_{2}}$. Если $\pi_1 \notin \fm_{\v_{2}}$, то $\pi_1 \in \cO_{\v_{2}}^{*}$, но тогда, так как $\cO_{\v_1}^* \subset \cO_{v_{2}}^{*}$, мы имеем $u \in \cO_{\v_2}^*$, откуда $x \in \cO_{\v_{2}}^*$. То есть, $\fm_{\v_1} \subset \cO_{\v_2}^*$, но при этом $\fm_{\v_{2}} \subset \fm_{v_{1}}$, что даёт нам противоречие. 

		\bf{3)} Докажем, что $\cO_{\v_{2}}^* \subset \cO_{\v_1}$. Пусть $x \in \cO_{\v_2}^*$, тогда, если $x \notin \cO_{\v_1}$, то $\v_1(x) < 0$, откуда $\v_1(x^{-1}) > 0 \implies x^{-1} \in \fm_{\v_1} = \fm_{v_2}$, откуда $\v_2(x^{-1}) > 0 \implies \v_{2}(x) < 0$, что противоречит тому, что $x \in \cO_{v_2}$. Тогда ясно, что $\cO_{\v_{1}} = \cO_{\v_2}$. 

		\bf{4)} Докажем, что $\v_1 = \v_2$. Так как $\cO_{\v_{1}} = \cO_{\v_2}$, $\cO_{\v_1}^* = \cO_{\v_2}^*$ и $\fm_{\v_1} = \fm_{\v_2}$. Ясно, что $\v_{1}(\pi_1) = \v_2(\pi_1) = 1$, а так как любой элемент представим в виде $\pi_1^n \cdot u,$ где $u$ обратим, нормирования совпадают на любом элементе.  
	\end{proof}

	\begin{definition} 
		Если $\v(x) > 0$, то $x$ называют \emph{нулём порядка $\v(x)$}  относительно $\v$.

		Если $\v(x) < 0$, то $x$ называют \emph{полюсом порядка $\v(x)$ относительно $\v$}.
	\end{definition}

	\begin{lemma}\label{ant_2_lemma_2}  
		Пусть $\v_1, \v_2$~--- два различных нормирования. Тогда $\exists x \in K^{*}$, который является нулём относительно $\v_1$ и полюсом относительно $\v_2$. 
	\end{lemma}
	\begin{proof}
		По лемме~\ref{ant_2_lemma_1} у нас нет включений между $\cO_{\v_1}$ и $\cO_{\v_2}$. Возьмём $y \in \cO_{\v_1} \setminus \cO_{\v_2}$ и $z \in \cO_{\v_2} \setminus \cO_{\v_1}$. Тогда нам подойдёт $x = \frac{y}{z}$. Действительно, 
		\[
			\v_1(x) = \underbrace{\v_1(y)}_{> 0} - \underbrace{\v_1(z)}_{< 0} > 0, \quad \v_2(x) = \underbrace{\v_2(y)}_{< 0} - \underbrace{\v_2(z)}_{> 0} > 0
		\]
	\end{proof}

	\begin{lemma}\label{ant_2_lemma_3} 
		Пусть $\v_1, \ldots, \v_n$~--- попарно различные нормирования. Тогда существует $x$ такой, что $\v_1(x) < 0$, $\v_2(x) > 0, \ldots, \v_n(x) < 0$.
	\end{lemma}
	\begin{proof}
		Докажем этот факт индукцией по $n$. В качестве базы нам подходит предыдущая лемма~\ref{ant_2_lemma_2}. 

		Теперь сделаем переход $n - 1 \mapsto n$. По предположению индукции мы можем найти $\widetilde{x}\colon \v_1\lr*{\widetilde{x}} < 0$, $\v_2\lr*{\widetilde{x}} > 0, \ldots, \v_{n - 1}\lr*{\widetilde{x}} < 0$.  По лемме~\ref{ant_2_lemma_2} мы можем найти $y$ такой, что $\v_1(y) > 0,$ а $\v_n(y) < 0$.  Рассмотрим элемент $\widetilde{x} + y^r$. Ясно, что $\v_1\lr*{\widetilde{x} + y^r} > 0$. Кроме того, ясно, что число $r$ можно взять достаточно большим, чтоб выполнялось неравенство 
		\[
			\forall i \quad 2 \le i \le n \quad \v_i\lr*{\widetilde{x} + y^r} < 0,
		\]
		тогда $\widetilde{x} + y^r$ нам подойдет.
	\end{proof}

	Теперь, по этой лемме выберем элемент $x \in K$ такой, что $\v_1(x) > 0, \ \forall i > 1 \ \v_i(x) < 0$. Рассмотрим 

	\[
		y = \frac{1}{1 + x^m}, \quad y - 1 = \frac{-x^m}{1 + x^m}
	\]

	\[
		\forall i > 1 \quad \v_i(1 + x^m) \le -m, \text{ так как } \v_i(x) \le -1 \implies \v_i(y) \ge m.
	\]
	Кроме того, $\v_1(y - 1) = m\v_1(x) - \v_1(1 + x^m) \ge m$. 

	Пусть $a_1, \ldots, a_n \in K$, рассмотрим $t = a_1 z_1 + \ldots + a_n z_n$, где $z_i$ такие, что 

	\[
		\begin{cases} \v_i(z_i - 1) \ge m \\ \v_j(z_i) \ge m \ \forall j \neq i. \end{cases}
	\]
	Ясно, что мы можем выбрать $z_i$  абсолютно также, как мы выбирали $y$. Тогда 
	\[
		\v_i(t - a_i) = \v_i(a_1 z_1 + \ldots + a_i(z_i - 1) + \ldots + a_n z_n) \ge m + \min(\v_i(a_1), \ldots, \v_i(a_n)).
	\]

	Таким образом, мы доказали такую теорему: 

	\begin{theorem}[Аппроксимационная теорема]\label{ant_2_thm_1} 
		Пусть $a_1, \ldots, a_n \in K, \ N > 0$, а $\v_1, \ldots, \v_n$~--- попарно различные нормирования. Тогда $\exists t \in K\colon v_i(t - a_i) > N$.
	\end{theorem}

	\begin{corollary}\label{ant_2_corol_1}
		Пусть $k_1, \ldots, k_n \in \Z$. Тогда существует $a \in K$ такой, что $\v_i(a) = k_i$.
	\end{corollary}
	\begin{proof}
		Выберем $a_1, \ldots, a_n$, что $\v_i(a_i) = k_i$, а $N$ достаточно большим (больше всех $k_i$). Тогда по аппроксимационной теореме $\exists a \in K \colon \v_i(a - a_i) > N$. 
		Но тогда 
		\[ 
			\v_i(a) = \v_i(a_i + (a - a_i)) = \min(\v_i(a_i), \v_i(a - a_i)) = v_i(a_i) = k_i.
		\]
	\end{proof}

	Аппроксимационную теорему можно интерпретировать и топологически. Во-первых, нормирование $\v$ определяет на поле $K$ неархимедову норму: возьмем $0 < c < 1$ и определим её, как 
	\[
		|x|_{\v} = \begin{cases} 0, \ x = 0 \\ c^{\v(x)}.  \end{cases}.
	\]
	По такой норме можно стандартным образом построить метрику $d_{\v}(x, y) = |x - y|_{\v}$ и тогда поле $K$ станет метрическим (в частности, топологическим) пространством. 

	\begin{corollary}\label{ant_2_corol_2}
		Рассмотрим все нормирования на поле $K$ и произведение $\prod_{\v} K $ с топологией произведения. Тогда аппроксимационная теорема говорит нам, что множество диагональных элементов $\{ (a, \ldots, a) \ \vert \ a \in K \}$ плотно в топологии произведения. 
		
	\end{corollary}

	\begin{example}
		Пусть $\Bbbk$~--- поле, $f \in \Bbbk[x, y]$~--- неприводимый многочлен. Тогда $A = \Bbbk[x, y]/(f)$~--- одномерное целостное кольцо. Пусть $f(x_0, y_0) = 0$, тогда $(f) \subset (x - x_0, y - y_0)$, причем $(x - x_0, y - y_0) = \fm$~--- максимальный идеал. Тогда $A_{\fm}$~--- одномерное целостное локальное нётерово кольцо. Пусть $(x_0, y_0)$~--- неособая точка ($f'_x(x_0, y_0) + f'_y(x_0, y_0) \neq 0$). По формуле Тейлора: 
		\begin{multline*}
			f(x, y) = f(x_0, y_0) + f'_x(x_0, y_0) (x - x_0) + f'_y(x_0, y_0) (y - y_0) + \ldots \equiv \\ \equiv f(x_0, y_0) + f'_x(x_0, y_0) (x - x_0) + f'_y(x_0, y_0) (y - y_0) \pmod{\fm^2}
		\end{multline*}
		Тогда в $A$ $f(x_0, y_0) + f'_x(x_0, y_0) (x - x_0) + f'_y(x_0, y_0) (y - y_0) \in \fm^2$. Значит, в фаторе по $\fm^2$ мы получим, что одна образующая идеала $\fm$ выражается через другую, то есть, что 
		\[
			\dim_{A/\fm}{\fm/\fm^2} = 1 \implies \fm \text{~--- главный. }
		\]
		Воспользуемся леммой Накаямы. Пусть $\fm/\fm^2 = (z)$, тогда $(\fm/(z))^2 = \fm/(z)$, откуда по лемме Накаямы~\ref{Nakayama} $\fm/(z) = 0$, то есть $\fm$ порождается элементом $(z)$. Значит, идеал $\fm$ в кольце $A_{\fm}$ является главным, откуда по теореме~\ref{ant_2_prop_1} кольцо  $A_{\fm}$~--- кольцо дискретного нормирования. 

		То есть, по неособой точке на кривой всегда можно построить кольцо дискретного нормирования. 
	\end{example}

	Мы в основном будем заниматься теми нормированиями, относительно которых поле $K$ является полным (как метрическое пространство). 

	

	\begin{definition} 
		Пусть $K$~--- поле, полное относительно $\v$. Тогда поле $\Bbbk = \cO_{\v}/\fm_{\v}$ называют \emph{полем вычетов} нормирования $\v$.
	\end{definition}

	Пусть $S \subset \cO_{\v}$ и $S$ состоит из представителей элементов поля вычетов и $0 \in S$. 

	\begin{theorem}\label{ant_2_thm_2}
		Пусть $0 \neq x \in K$, $\pi \in \cO_{\v}$~--- такой элемент, что $\fm_{\v} = (\pi)$. Тогда $x$ единственным образом раскладывается в ряд 
		\[
			x = a_n \pi^n + a_{n + 1} \pi^{n + 1} + \ldots,
		\]
		где число $n$ определено однозначно ($n = \v(x)$). 
	\end{theorem}
	\begin{proof}
		Заметим, что $\v\lr*{\frac{x}{\pi^n}} = 0$, то есть $x/\pi^n \in \cO_{\v} \setminus \fm_{\v}$. Тогда для него существует ненулевой  ненулевой представитель $a_n$ в поле вычетов. Иными словами, 
		\[
			\frac{x}{\pi^n} \equiv a_n \pmod{\fm_{\v}}. 
		\]
		Тогда $m = \v(x - a_n\pi^n) \ge n + 1$. Проделывая ту же самую процедуру для $x - a_n\pi^n$ мы получаем
		\[
		 	\v(x - a_n \pi^n + a_m \pi^m) \ge m + 1
		 \] 
		 и так далее. Так мы получаем разложение 
		 \[
		 	x = a_n \pi^n + a_{n + 1}\pi^{n + 1} + \ldots, \quad n = \v(x), \ a_i \in S.
		 \]

		 Теперь докажем единственность. Пусть 
		 \[
		 	x = a_n \pi^n + a_{n + 1}\pi^{n + 1} + \ldots = b_n \pi^n + b_{n + 1}\pi^{n + 1} + \ldots,
		 \]
		 тогда $(a_n - b_n)\pi^{n} \in  (\pi^{n + 1}) \implies a_n - b_n \in \fm_{\v}$, то есть 
		 $\overline{a_n} = \overline{b_n}$ в $\Bbbk = \cO_{\v}/\fm_{\v}$, но тогда, так как мы брали для каждого элемента поля вычетов единственный представитель, отсюда $a_n = b_n$. Аналогично мы получаем равенство всех коэффициентов. 
	\end{proof}

	\begin{homework}
		Задачи:
		\begin{enumerate}
			\item Опишите все нормирования поля $F(t)$, тривиальные на $F$.

			\item Рассмотрим фильтрацию $U = \cO_{\v} \supset U_1 \supset U_2 \supset \ldots \supset U_n \supset \ldots$, где 
			\[
				U_n = \{ x \in \cO_{\v} \ \vert \ \v(x - 1) \ge n \}. 
			\]
			\begin{enumerate}
				\item Покажите, что $U_n$~--- группа. 

				\item Докажите, что $U/U_1 \cong \Bbbk$. 

				\item Докажите, что $U_n/U_{n + 1} \cong \Bbbk$.
			\end{enumerate}

			\item Пусть $c \in \Z$, $p$~--- простое и $c \notdivby p$. Докажите, что последовательность $c^{p^n}$ сходится в поле $\Q_{p}$.

			\item Пусть $K = \Q_{p}$. Докажите, что $U_i \cong U$ при $i \ge 1$. 

			\item Докажите, что $\forall n \ |\Q_{p}/\Q_{p}^{*n}| < \infty$.

			\item Пусть $K$~--- полное поле и $\Char{\Bbbk} \not\ \mid m$. Тогда отображение $x \mapsto x^m$~--- изоморфизм $U_n \to U_n$.
		\end{enumerate}
	\end{homework}




	

	

	

	

	