    \subsection{Когомологии}

    Итак, рассмотрим цепной комплекс абелевых групп $(C_{\bullet}, \partial)$
    \[ \ldots \to C_{k} \to C_{k - 1} \to C_{k - 2} \to \ldots \]
    Тогда мы можем рассмотреть группы $C^{k} \eqdef \Hom\lr*{C_{k}, G}$, где $G$~--- фиксированная абелева группа.\footnote{В нашем, топологическом контексте, это группа коэффициентов.}
    Тогда мы получаем цепной комплекс
    \[ \ldots \leftarrow C^{k + 1} \xleftarrow{\delta} C^{k} \xleftarrow{\delta} C^{k - 1} \xleftarrow{\delta} \ldots \]
    Естественно, стрелки развернулись, так как мы подействовали на комплекс контравариантным функтором $\Hom(\_, G)$.
    Действие оператора $\delta$ определяется естественным образом:
    \[ \varphi \in C^{k}, \ \delta\varphi\colon C_{k + 1} \xrightarrow{\partial} C_{k} \xrightarrow{\varphi} G, \ \delta\varphi = \varphi \circ \partial. \]
    \begin{remark}
       Сразу же нетрудно заметить, что $\delta^2 = 0$, то есть построенный комплекс действительно будет комплексом. Действительно,
        \[ \delta_{k} \circ \delta_{k - 1}(\varphi(c)) = \delta_k(\varphi(\partial_{k - 1}c)) = \varphi(\partial_{k}\partial_{k - 1}c) = 0. \]
    \end{remark}

    \begin{definition}
        Группы гомологий коцепного комплекса $(C^{\bullet}, \delta) = (\Hom(C_{\bullet}), G), \delta)$ называют \emph{группами когомологий} комплекса $(C_{\bullet}, \partial)$ с коэффициентами в группе $G$ и обозначаются
    $H^{k}(C_{\bullet}; G)$. Как и в случае с гомологиями, $\Im{\delta_k}$ называют $k$-мерными кограницами, $\Ker{\delta_{k}}$~--- $k$-мерными коциклами, а  $C^k$~--- $k$-мерными коцепями.
    \end{definition}

    Таким образом, мы определили и \emph{сингулярные когомологии} пространства  $X$ (так как они строятся по сингулярным гомологиям).
    Заметим, что так как функтор $\Hom$ контравариантен, логично ожидать, что и когомологии будут контраваринатным функтором. Действительно,
    если $f\colon X \to Y$~--- непрерывное отображение, то у нас есть индуцированный морфизм
    \[ f_{*} \colon C_{k}(X) \to C_{k}(Y) \]
    и действием функтора $\Hom$ мы получаем индуцированный морфизм
    \[ f^{*}\colon C^{k}(Y) \to C^{k}(X) \]
    Действительно, действие будет работать так:
    \[ \varphi \in C^{k}(Y) f^{*}()\]

    


